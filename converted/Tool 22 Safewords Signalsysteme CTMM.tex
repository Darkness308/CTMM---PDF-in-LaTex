\hypertarget{tool-22-safe-words-signalsysteme-ctmm-modul}{%
\section{\texorpdfstring{\textbf{[Stop] TOOL 22 -- SAFE-WORDS \& SIGNALSYSTEME (CTMM-MODUL)}}{[Stop] TOOL 22 -- SAFE-WORDS \& SIGNALSYSTEME (CTMM-MODUL)}}\label{tool-22-safe-words-signalsysteme-ctmm-modul}}

\begin{quote}
[Brain] \textbf{\ul{Worum geht's hier -- für Freunde?}}\
Safe-Words sind vereinbarte Codes oder Zeichen, die sofort signalisieren:
\end{quote}

\begin{itemize}
\item
  \begin{quote}
  \textbf{„Ich kann nicht mehr``}
  \end{quote}
\item
  \begin{quote}
  \textbf{„Ich brauch Ruhe`` oder}
  \end{quote}
\item
  \begin{quote}
  \textbf{„Stopp -- das wird mir zu viel``}
  \end{quote}
\end{itemize}

\begin{quote}
Sie schützen vor Eskalation, Überforderung, Rückzug oder Missverständnissen -- ohne viele Worte.
\end{quote}

[Module] \textbf{Zentraler Bestandteil der Eskalationsprävention -- mit Symbol- und Notfallsystem}

\hypertarget{kapitelzuordnung-im-ctmm-system}{%
\subsection{\texorpdfstring{[Book] \textbf{\ul{KAPITELZUORDNUNG IM CTMM-SYSTEM}}}{[Book] KAPITELZUORDNUNG IM CTMM-SYSTEM}}\label{kapitelzuordnung-im-ctmm-system}}

\begin{itemize}
\tightlist
\item
  \texttt{Kap.\ }\texttt{2.2} $\rightarrow$ Reizschutz \& Vorbereitung im Alltag
\item
  \texttt{Kap.\ }\texttt{3.1\ --\ 3.5} $\rightarrow$ Eskalation, Rückzug, Notfallmaßnahmen
\item
  \texttt{Kap.\ }\texttt{4.1} $\rightarrow$ Verantwortung, respektvoller Umgang
\item
  \texttt{Kap.\ }\texttt{5.3} $\rightarrow$ Trigger-Management \& Frühwarnsystem
\end{itemize}

\begin{quote}
\textbf{[?] Aufgaben \& Trainings findest du im} \texttt{T}\texttt{rigger}\texttt{\ }\texttt{N}\texttt{otfallkarten}\textbf{,} \texttt{T}\texttt{ool}\texttt{\ }\texttt{26}\texttt{\ C}\texttt{o}\texttt{-}\texttt{regulation}\textbf{,} \texttt{B}\texttt{indungsleitfaden}\texttt{\ CTMM}
\end{quote}

\hypertarget{farbcode-systemnavigation}{%
\subsection{\texorpdfstring{[Art] \textbf{\ul{FARBCODE \& SYSTEMNAVIGATION}}}{[Art] FARBCODE \& SYSTEMNAVIGATION}}\label{farbcode-systemnavigation}}

\begin{longtable}[]{@{}
  >{\raggedright\arraybackslash}p{(\columnwidth - 4\tabcolsep) * \real{0.1245}}
  >{\raggedright\arraybackslash}p{(\columnwidth - 4\tabcolsep) * \real{0.4980}}
  >{\raggedright\arraybackslash}p{(\columnwidth - 4\tabcolsep) * \real{0.3776}}@{}}
\toprule\noalign{}
\begin{minipage}[b]{\linewidth}\raggedright
\emph{\textbf{Farbe}}
\end{minipage} & \begin{minipage}[b]{\linewidth}\raggedright
\emph{\textbf{Phase}}
\end{minipage} & \begin{minipage}[b]{\linewidth}\raggedright
\emph{\textbf{Verknüpfte Module}}
\end{minipage} \
\midrule\noalign{}
\endhead
\bottomrule\noalign{}
\endlastfoot
\textcolor{red}{$\bullet$} & Eskalation / Notfall & \texttt{T}\texttt{rigger}\texttt{\ }\texttt{N}\texttt{otfallkarten} \
\textcolor{orange}{$\bullet$} & Frühsignal / Ko-Regulation & \texttt{T}\texttt{ool}\texttt{\ }\texttt{26}\texttt{\ C}\texttt{o}\texttt{-}\texttt{regulation} \
\textcolor{green}{$\bullet$} & Alltag / Vereinbarung & \texttt{S}\texttt{afe}\texttt{\ }\texttt{rules}\texttt{\ WG-M}\texttt{odell} \
\end{longtable}

\hypertarget{safe-words-beispiele-eigene}{%
\subsection{\texorpdfstring{[?] \textbf{\ul{SAFE-WORDS (BEISPIELE + EIGENE)}}}{[?] SAFE-WORDS (BEISPIELE + EIGENE)}}\label{safe-words-beispiele-eigene}}

\begin{longtable}[]{@{}
  >{\raggedright\arraybackslash}p{(\columnwidth - 4\tabcolsep) * \real{0.3623}}
  >{\raggedright\arraybackslash}p{(\columnwidth - 4\tabcolsep) * \real{0.3044}}
  >{\raggedright\arraybackslash}p{(\columnwidth - 4\tabcolsep) * \real{0.3333}}@{}}
\toprule\noalign{}
\begin{minipage}[b]{\linewidth}\raggedright
\emph{\textbf{Safe-Word / Geste}}
\end{minipage} & \begin{minipage}[b]{\linewidth}\raggedright
\emph{\textbf{Bedeutung / Wirkung}}
\end{minipage} & \begin{minipage}[b]{\linewidth}\raggedright
\emph{\textbf{Wann einsetzen?}}
\end{minipage} \
\midrule\noalign{}
\endhead
\bottomrule\noalign{}
\endlastfoot
„Orange`` & Warnstufe -- ich werde gleich überfordert & Bei Stress, lautem Ton, innerem Rückzug \
„Kristall`` & Stopp -- bitte sofort aufhören & Bei Eskalation, Überforderung, Trigger \
[?][?] (Handzeichen, offen) & Ich will reden, aber schaff's nicht & Bei Freeze, Erstarrung, Nonverbales \
[Note]~ ~ ~ ~ ~ ~ ~ ~ ~ ~ ~ ~ ~ ~ ~ ~ ~ ~ ~Lagerfeuer & [Note] 5min in den Arm nehmen, ohne zu reden & [Note] Wenn ich Nähe brauche (Angst, Frieden in einem Moment) \
[Note] \_\_\_\_\_\_\_\_\_\_\_\_\_\_\_\_\_\_\_\_\_\_\_\_\_\_\_\_\_\_\_\_\_\_\_\_\_\_\_\_\_\_\_\_\_\_\_\_\_\_\_\_ & [Note] \_\_\_\_\_\_\_\_\_\_\_\_\_\_\_\_\_\_\_\_\_\_\_\_\_\_\_\_\_\_\_\_\_\_\_\_\_\_\_\_\_\_ & [Note] \_\_\_\_\_\_\_\_\_\_\_\_\_\_\_\_\_\_\_\_\_\_\_\_\_\_\_\_\_\_\_\_\_\_\_\_\_\_\_\_\_\_\_\_\_\_\_\_ \
\end{longtable}

\hypertarget{signalsysteme-zur-unterstuxfctzung}{%
\subsection{\texorpdfstring{\textbf{[Navigation] \ul{SIGNALSYSTEME ZUR UNTERSTÜTZUNG}}}{[Navigation] SIGNALSYSTEME ZUR UNTERSTÜTZUNG}}\label{signalsysteme-zur-unterstuxfctzung}}

\begin{itemize}
\tightlist
\item
  [?] \textbf{Symbolischer Gegenstand} (z.\,B. Stofftier, Stein, Karte)
\item
  [?] \textbf{Tagesanzeiger} (Magnet, Schild, Farbe auf Tür)
\item
  [?] \textbf{Lautstärke-Code} (Musikart, Kopfhörer sichtbar = „Bitte in
\end{itemize}

\begin{quote}
Ruhe lassen``)

\textbf{[?] Diese Zeichen können leise, sichtbar, intuitiv sein -- auch bei Disso oder Sprachverlust}
\end{quote}

\hypertarget{eigene-vereinbarungen}{%
\subsection{\texorpdfstring{\textbf{[?][?] \ul{EIGENE VEREINBARUNGEN}}}{[?][?] EIGENE VEREINBARUNGEN}}\label{eigene-vereinbarungen}}

\begin{longtable}[]{@{}
  >{\raggedright\arraybackslash}p{(\columnwidth - 4\tabcolsep) * \real{0.2631}}
  >{\raggedright\arraybackslash}p{(\columnwidth - 4\tabcolsep) * \real{0.3450}}
  >{\raggedright\arraybackslash}p{(\columnwidth - 4\tabcolsep) * \real{0.3919}}@{}}
\toprule\noalign{}
\begin{minipage}[b]{\linewidth}\raggedright
\emph{\textbf{Situation / Kontext}}
\end{minipage} & \begin{minipage}[b]{\linewidth}\raggedright
\emph{\textbf{Mein Code / Zeichen}}
\end{minipage} & \begin{minipage}[b]{\linewidth}\raggedright
\emph{\textbf{Was er bedeutet}}
\end{minipage} \
\midrule\noalign{}
\endhead
\bottomrule\noalign{}
\endlastfoot
[Note] \_\_\_\_\_\_\_\_\_\_\_\_\_\_\_\_\_\_\_\_\_\_\_\_\_\_\_\_\_\_\_\_\_\_\_\_ & [Note] \_\_\_\_\_\_\_\_\_\_\_\_\_\_\_\_\_\_\_\_\_\_\_\_\_\_\_\_\_\_\_\_\_\_\_\_\_\_\_\_\_\_\_\_\_\_\_\_\_\_ & [Note] \_\_\_\_\_\_\_\_\_\_\_\_\_\_\_\_\_\_\_\_\_\_\_\_\_\_\_\_\_\_\_\_\_\_\_\_\_\_\_\_\_\_\_\_\_\_\_\_\_\_\_\_\_\_\_\_ \
[Note] \_\_\_\_\_\_\_\_\_\_\_\_\_\_\_\_\_\_\_\_\_\_\_\_\_\_\_\_\_\_\_\_\_\_\_\_ & [Note] \_\_\_\_\_\_\_\_\_\_\_\_\_\_\_\_\_\_\_\_\_\_\_\_\_\_\_\_\_\_\_\_\_\_\_\_\_\_\_\_\_\_\_\_\_\_\_\_\_\_ & [Note] \_\_\_\_\_\_\_\_\_\_\_\_\_\_\_\_\_\_\_\_\_\_\_\_\_\_\_\_\_\_\_\_\_\_\_\_\_\_\_\_\_\_\_\_\_\_\_\_\_\_\_\_\_\_\_\_ \
[Note] \_\_\_\_\_\_\_\_\_\_\_\_\_\_\_\_\_\_\_\_\_\_\_\_\_\_\_\_\_\_\_\_\_\_\_\_ & [Note] \_\_\_\_\_\_\_\_\_\_\_\_\_\_\_\_\_\_\_\_\_\_\_\_\_\_\_\_\_\_\_\_\_\_\_\_\_\_\_\_\_\_\_\_\_\_\_\_\_\_ & [Note] \_\_\_\_\_\_\_\_\_\_\_\_\_\_\_\_\_\_\_\_\_\_\_\_\_\_\_\_\_\_\_\_\_\_\_\_\_\_\_\_\_\_\_\_\_\_\_\_\_\_\_\_\_\_\_\_ \
\end{longtable}

\hypertarget{ctmm-navigation}{%
\subsection{\texorpdfstring{[Navigation] \textbf{\ul{CTMM-NAVIGATION}}}{[Navigation] CTMM-NAVIGATION}}\label{ctmm-navigation}}

\begin{itemize}
\tightlist
\item
  \texttt{T}\texttt{rigger}\texttt{\ }\texttt{N}\texttt{otfallkarten} $\leftarrow$ Wenn Code ausgelöst wird (\texttt{Kap.}\texttt{\ }\texttt{3.2\ --\ 3.5})
\item
  \texttt{T}\texttt{ool}\texttt{\ }\texttt{26}\texttt{\ C}\texttt{o}\texttt{-}\texttt{regulation} $\leftarrow$ Reaktion auf Code durch Partner (\texttt{Kap.}\texttt{\ }\texttt{2.6})
\item
  \texttt{R}\texttt{itual}\texttt{\ W}\texttt{orkbook} $\leftarrow$ Rückkehr nach Stopp/Wiederanbindung (\texttt{Kap.}\texttt{\ }\texttt{3.4})
\item
  \texttt{B}\texttt{indungsleitfaden}\texttt{\ CTMM} $\leftarrow$ Warum Safewords so wichtig sind (\texttt{Kap.\ }\texttt{1}\texttt{\ }\texttt{\&}\texttt{\ }\texttt{4.1})
\end{itemize}

\begin{quote}
\textbf{[?] Safe-Words sind keine Schwäche -- sie sind Vertrauen in Form eines Codes.}
\end{quote}
