\hypertarget{tool-22-safe-words-signalsysteme-ctmm-modul}{\%
\section{\texorpdfstring{\textbf{🛑 TOOL 22 -- SAFE-WORDS \\& SIGNALSYSTEME (CTMM-MODUL)}}{🛑 TOOL 22 -- SAFE-WORDS \\& SIGNALSYSTEME (CTMM-MODUL)}}\label{tool-22-safe-words-signalsysteme-ctmm-modul}}

\begin{quote}
🧠 \textbf{\ul{Worum geht's hier -- für Freunde?}}\\
Safe-Words sind vereinbarte Codes oder Zeichen, die sofort signalisieren:
\end{quote}

\begin{itemize}
\item
  \begin{quote}
  \textbf{„Ich kann nicht mehr``}
  \end{quote}
\item
  \begin{quote}
  \textbf{„Ich brauch Ruhe`` oder}
  \end{quote}
\item
  \begin{quote}
  \textbf{„Stopp -- das wird mir zu viel``}
  \end{quote}
\end{itemize}

\begin{quote}
Sie schützen vor Eskalation, Überforderung, Rückzug oder Missverständnissen -- ohne viele Worte.
\end{quote}

🧩 \textbf{Zentraler Bestandteil der Eskalationsprävention -- mit Symbol- und Notfallsystem}

\hypertarget{kapitelzuordnung-im-ctmm-system}{\%
\subsection{\texorpdfstring{📘 \textbf{\ul{KAPITELZUORDNUNG IM CTMM-SYSTEM}}}{📘 KAPITELZUORDNUNG IM CTMM-SYSTEM}}\label{kapitelzuordnung-im-ctmm-system}}

\begin{itemize}
\tightlist
\item
  \texttt{Kap.\ }\texttt{2.2} → Reizschutz \\& Vorbereitung im Alltag
\item
  \texttt{Kap.\ }\texttt{3.1\ –\ 3.5} → Eskalation, Rückzug, Notfallmaßnahmen
\item
  \texttt{Kap.\ }\texttt{4.1} → Verantwortung, respektvoller Umgang
\item
  \texttt{Kap.\ }\texttt{5.3} → Trigger-Management \\& Frühwarnsystem
\end{itemize}

\begin{quote}
\textbf{📎 Aufgaben \\& Trainings findest du im} \texttt{T}\texttt{rigger}\texttt{\ }\texttt{N}\texttt{otfallkarten}\textbf{,} \texttt{T}\texttt{ool}\texttt{\ }\texttt{26}\texttt{\ C}\texttt{o}\texttt{-}\texttt{regulation}\textbf{,} \texttt{B}\texttt{indungsleitfaden}\texttt{\ CTMM}
\end{quote}

\hypertarget{farbcode-systemnavigation}{\%
\subsection{\texorpdfstring{🎨 \textbf{\ul{FARBCODE \\& SYSTEMNAVIGATION}}}{🎨 FARBCODE \\& SYSTEMNAVIGATION}}\label{farbcode-systemnavigation}}

\begin{longtable}[]{@{}
  >{\raggedright\arraybackslash}p{(\columnwidth - 4\tabcolsep) * \real{0.1245}}
  >{\raggedright\arraybackslash}p{(\columnwidth - 4\tabcolsep) * \real{0.4980}}
  >{\raggedright\arraybackslash}p{(\columnwidth - 4\tabcolsep) * \real{0.3776}}@{}}
\toprule\noalign{}
\begin{minipage}[b]{\linewidth}\raggedright
\emph{\textbf{Farbe}}
\end{minipage} \& \begin{minipage}[b]{\linewidth}\raggedright
\emph{\textbf{Phase}}
\end{minipage} \& \begin{minipage}[b]{\linewidth}\raggedright
\emph{\textbf{Verknüpfte Module}}
\end{minipage} \\
\midrule\noalign{}
\endhead
\bottomrule\noalign{}
\endlastfoot
🔴 \& Eskalation / Notfall \& \texttt{T}\texttt{rigger}\texttt{\ }\texttt{N}\texttt{otfallkarten} \\
🟠 \& Frühsignal / Ko-Regulation \& \texttt{T}\texttt{ool}\texttt{\ }\texttt{26}\texttt{\ C}\texttt{o}\texttt{-}\texttt{regulation} \\
🟢 \& Alltag / Vereinbarung \& \texttt{S}\texttt{afe}\texttt{\ }\texttt{rules}\texttt{\ WG-M}\texttt{odell} \\
\end{longtable}

\hypertarget{safe-words-beispiele-eigene}{\%
\subsection{\texorpdfstring{🔑 \textbf{\ul{SAFE-WORDS (BEISPIELE + EIGENE)}}}{🔑 SAFE-WORDS (BEISPIELE + EIGENE)}}\label{safe-words-beispiele-eigene}}

\begin{longtable}[]{@{}
  >{\raggedright\arraybackslash}p{(\columnwidth - 4\tabcolsep) * \real{0.3623}}
  >{\raggedright\arraybackslash}p{(\columnwidth - 4\tabcolsep) * \real{0.3044}}
  >{\raggedright\arraybackslash}p{(\columnwidth - 4\tabcolsep) * \real{0.3333}}@{}}
\toprule\noalign{}
\begin{minipage}[b]{\linewidth}\raggedright
\emph{\textbf{Safe-Word / Geste}}
\end{minipage} \& \begin{minipage}[b]{\linewidth}\raggedright
\emph{\textbf{Bedeutung / Wirkung}}
\end{minipage} \& \begin{minipage}[b]{\linewidth}\raggedright
\emph{\textbf{Wann einsetzen?}}
\end{minipage} \\
\midrule\noalign{}
\endhead
\bottomrule\noalign{}
\endlastfoot
„Orange`` \& Warnstufe -- ich werde gleich überfordert \& Bei Stress, lautem Ton, innerem Rückzug \\
„Kristall`` \& Stopp -- bitte sofort aufhören \& Bei Eskalation, Überforderung, Trigger \\
🖐️ (Handzeichen, offen) \& Ich will reden, aber schaff's nicht \& Bei Freeze, Erstarrung, Nonverbales \\
📝\textasciitilde{} \textasciitilde{} \textasciitilde{} \textasciitilde{} \textasciitilde{} \textasciitilde{} \textasciitilde{} \textasciitilde{} \textasciitilde{} \textasciitilde{} \textasciitilde{} \textasciitilde{} \textasciitilde{} \textasciitilde{} \textasciitilde{} \textasciitilde{} \textasciitilde{} \textasciitilde{} \textasciitilde{}Lagerfeuer \& 📝 5min in den Arm nehmen, ohne zu reden \& 📝 Wenn ich Nähe brauche (Angst, Frieden in einem Moment) \\
📝 \\_\\_\\_\\_\\_\\_\\_\\_\\_\\_\\_\\_\\_\\_\\_\\_\\_\\_\\_\\_\\_\\_\\_\\_\\_\\_\\_\\_\\_\\_\\_\\_\\_\\_\\_\\_\\_\\_\\_\\_\\_\\_\\_\\_\\_\\_\\_\\_\\_\\_\\_\\_ \& 📝 \\_\\_\\_\\_\\_\\_\\_\\_\\_\\_\\_\\_\\_\\_\\_\\_\\_\\_\\_\\_\\_\\_\\_\\_\\_\\_\\_\\_\\_\\_\\_\\_\\_\\_\\_\\_\\_\\_\\_\\_\\_\\_ \& 📝 \\_\\_\\_\\_\\_\\_\\_\\_\\_\\_\\_\\_\\_\\_\\_\\_\\_\\_\\_\\_\\_\\_\\_\\_\\_\\_\\_\\_\\_\\_\\_\\_\\_\\_\\_\\_\\_\\_\\_\\_\\_\\_\\_\\_\\_\\_\\_\\_ \\
\end{longtable}

\hypertarget{signalsysteme-zur-unterstuxfctzung}{\%
\subsection{\texorpdfstring{\textbf{🧭 \ul{SIGNALSYSTEME ZUR UNTERSTÜTZUNG}}}{🧭 SIGNALSYSTEME ZUR UNTERSTÜTZUNG}}\label{signalsysteme-zur-unterstuxfctzung}}

\begin{itemize}
\tightlist
\item
  🧺 \textbf{Symbolischer Gegenstand} (z.\,B. Stofftier, Stein, Karte)
\item
  🧼 \textbf{Tagesanzeiger} (Magnet, Schild, Farbe auf Tür)
\item
  🎧 \textbf{Lautstärke-Code} (Musikart, Kopfhörer sichtbar = „Bitte in
\end{itemize}

\begin{quote}
Ruhe lassen``)

\textbf{📎 Diese Zeichen können leise, sichtbar, intuitiv sein -- auch bei Disso oder Sprachverlust}
\end{quote}

\hypertarget{eigene-vereinbarungen}{\%
\subsection{\texorpdfstring{\textbf{✏️ \ul{EIGENE VEREINBARUNGEN}}}{✏️ EIGENE VEREINBARUNGEN}}\label{eigene-vereinbarungen}}

\begin{longtable}[]{@{}
  >{\raggedright\arraybackslash}p{(\columnwidth - 4\tabcolsep) * \real{0.2631}}
  >{\raggedright\arraybackslash}p{(\columnwidth - 4\tabcolsep) * \real{0.3450}}
  >{\raggedright\arraybackslash}p{(\columnwidth - 4\tabcolsep) * \real{0.3919}}@{}}
\toprule\noalign{}
\begin{minipage}[b]{\linewidth}\raggedright
\emph{\textbf{Situation / Kontext}}
\end{minipage} \& \begin{minipage}[b]{\linewidth}\raggedright
\emph{\textbf{Mein Code / Zeichen}}
\end{minipage} \& \begin{minipage}[b]{\linewidth}\raggedright
\emph{\textbf{Was er bedeutet}}
\end{minipage} \\
\midrule\noalign{}
\endhead
\bottomrule\noalign{}
\endlastfoot
📝 \\_\\_\\_\\_\\_\\_\\_\\_\\_\\_\\_\\_\\_\\_\\_\\_\\_\\_\\_\\_\\_\\_\\_\\_\\_\\_\\_\\_\\_\\_\\_\\_\\_\\_\\_\\_ \& 📝 \\_\\_\\_\\_\\_\\_\\_\\_\\_\\_\\_\\_\\_\\_\\_\\_\\_\\_\\_\\_\\_\\_\\_\\_\\_\\_\\_\\_\\_\\_\\_\\_\\_\\_\\_\\_\\_\\_\\_\\_\\_\\_\\_\\_\\_\\_\\_\\_\\_\\_ \& 📝 \\_\\_\\_\\_\\_\\_\\_\\_\\_\\_\\_\\_\\_\\_\\_\\_\\_\\_\\_\\_\\_\\_\\_\\_\\_\\_\\_\\_\\_\\_\\_\\_\\_\\_\\_\\_\\_\\_\\_\\_\\_\\_\\_\\_\\_\\_\\_\\_\\_\\_\\_\\_\\_\\_\\_\\_ \\
📝 \\_\\_\\_\\_\\_\\_\\_\\_\\_\\_\\_\\_\\_\\_\\_\\_\\_\\_\\_\\_\\_\\_\\_\\_\\_\\_\\_\\_\\_\\_\\_\\_\\_\\_\\_\\_ \& 📝 \\_\\_\\_\\_\\_\\_\\_\\_\\_\\_\\_\\_\\_\\_\\_\\_\\_\\_\\_\\_\\_\\_\\_\\_\\_\\_\\_\\_\\_\\_\\_\\_\\_\\_\\_\\_\\_\\_\\_\\_\\_\\_\\_\\_\\_\\_\\_\\_\\_\\_ \& 📝 \\_\\_\\_\\_\\_\\_\\_\\_\\_\\_\\_\\_\\_\\_\\_\\_\\_\\_\\_\\_\\_\\_\\_\\_\\_\\_\\_\\_\\_\\_\\_\\_\\_\\_\\_\\_\\_\\_\\_\\_\\_\\_\\_\\_\\_\\_\\_\\_\\_\\_\\_\\_\\_\\_\\_\\_ \\
📝 \\_\\_\\_\\_\\_\\_\\_\\_\\_\\_\\_\\_\\_\\_\\_\\_\\_\\_\\_\\_\\_\\_\\_\\_\\_\\_\\_\\_\\_\\_\\_\\_\\_\\_\\_\\_ \& 📝 \\_\\_\\_\\_\\_\\_\\_\\_\\_\\_\\_\\_\\_\\_\\_\\_\\_\\_\\_\\_\\_\\_\\_\\_\\_\\_\\_\\_\\_\\_\\_\\_\\_\\_\\_\\_\\_\\_\\_\\_\\_\\_\\_\\_\\_\\_\\_\\_\\_\\_ \& 📝 \\_\\_\\_\\_\\_\\_\\_\\_\\_\\_\\_\\_\\_\\_\\_\\_\\_\\_\\_\\_\\_\\_\\_\\_\\_\\_\\_\\_\\_\\_\\_\\_\\_\\_\\_\\_\\_\\_\\_\\_\\_\\_\\_\\_\\_\\_\\_\\_\\_\\_\\_\\_\\_\\_\\_\\_ \\
\end{longtable}

\hypertarget{ctmm-navigation}{\%
\subsection{\texorpdfstring{🧭 \textbf{\ul{CTMM-NAVIGATION}}}{🧭 CTMM-NAVIGATION}}\label{ctmm-navigation}}

\begin{itemize}
\tightlist
\item
  \texttt{T}\texttt{rigger}\texttt{\ }\texttt{N}\texttt{otfallkarten} ← Wenn Code ausgelöst wird (\texttt{Kap.}\texttt{\ }\texttt{3.2\ –\ 3.5})
\item
  \texttt{T}\texttt{ool}\texttt{\ }\texttt{26}\texttt{\ C}\texttt{o}\texttt{-}\texttt{regulation} ← Reaktion auf Code durch Partner (\texttt{Kap.}\texttt{\ }\texttt{2.6})
\item
  \texttt{R}\texttt{itual}\texttt{\ W}\texttt{orkbook} ← Rückkehr nach Stopp/Wiederanbindung (\texttt{Kap.}\texttt{\ }\texttt{3.4})
\item
  \texttt{B}\texttt{indungsleitfaden}\texttt{\ CTMM} ← Warum Safewords so wichtig sind (\texttt{Kap.\ }\texttt{1}\texttt{\ }\texttt{\\&}\texttt{\ }\texttt{4.1})
\end{itemize}

\begin{quote}
\textbf{📎 Safe-Words sind keine Schwäche -- sie sind Vertrauen in Form eines Codes.}
\end{quote}
