\% Bindungsleitfaden CTMM Übersicht .tex - Converted from Bindungsleitfaden CTMM Übersicht .docx
\% CTMM Therapy Material - Auto-generated by Conversion Pipeline

\section{Bindungsleitfaden Ctmm Übersicht }
\label{sec:bindungsleitfaden-ctmm-bersicht}

\section{\textbf{🏥 BINDUNGSLEITFADEN --}}
\subsection{\textbf{\textcolor{ctmmBlue}{CTMM}-ÜBERSICHT}}

\textcolor{ctmmBlue}{\faIcon{puzzle-piece}} \textit{*}Navigationshilfe für neurodiverse Beziehungsmuster -- als roter
Faden durch alle \textcolor{ctmmBlue}{CTMM}-Module\textit{*}

\subsection{\textbf{📘 [WAS IST BINDUNG IN \textcolor{ctmmBlue}{CTMM}?]}}

\begin{itemize}
\item   \textbf{Bindung heißt nicht:\textbf{ Kontrolle, Nähepflicht, dauerhafte
\end{itemize}

\begin{quote}
Verschmelzung
\end{quote}

\begin{itemize}
\item   \textbf{Bindung heißt:\textbf{ Vertrauen, Wiederkehr, Sicherheit trotz Rückzug
\item   CTMM nutzt Bindung als \textbf{dynamischen Prozess\textbf{ mit Werkzeugen,
\end{itemize}
Safe-Zonen und Gesprächsritualen

\subsection{\textbf{\textcolor{ctmmOrange}{\faIcon{compass}} [\textcolor{ctmmBlue}{CTMM}-MODULNAVIGATION -- BINDUNGSSYSTEM AUF EINEN]}}

\subsection{ \textbf{[BLICK]}}

--------------------------------------------------------------------------------------
\textbf{\textit{Bereich}}         \textbf{\textit{Modul(e)}}                           \textbf{\textit{Kapitel/Verweis}}
--------------------- ---------------------------------------- -----------------------
🔵 Grundlagen \\&       `B``indungsdynamik`` \textcolor{ctmmBlue}{CTMM}`               `Kap. ``1`

Definition

\textcolor{ctmmGreen}{\faIcon{circle}} Prävention \\&       `W``ertekompass`` ``vereint`,            `Kap. ``2.1 -- 2.6`
`S``afe`` ``rules`` WG``-M``odell`
Alltag

🟠 Eskalation         `T``rigger`` N``otfallkarten`,           `Kap. ``3.1 -- 3.5`
`R``itual`` W``orkbook`,
sichern               `T``ool`` ``26`` C``o``-R``egulation`

\textcolor{ctmmRed}{\faIcon{circle}} Notfallplanung     `T``rigger`` F``orschungstagebuch`,      `Kap. ``5.2 -- 5.5`
`K``risenprotokoll`
\\& Reaktion

🟣 Reflexion \\&        `T``rigger`` T``agebuch`` ``extended`,   `Kap. ``5.6`,
`V``ision`` ``board`` ``K``lartext`      `Tool ``27`
Fortschritt
--------------------------------------------------------------------------------------

\subsection{\textbf{💬 [SCHLÜSSELSÄTZE FÜR SICHERE BINDUNG]}}

\begin{itemize}
\item   „Ich ziehe mich zurück, aber ich komme wieder."
\item   „Meine Grenze ist nicht gegen dich, sondern für mich."
\item   „Wir reden darüber -- aber nicht jetzt, nicht in Panik."
\item   „Ich sehe dich, auch wenn ich mich gerade nicht spürbar zeige."
\end{itemize}

\subsection{\textbf{🔗 [ZUSAMMENSPIEL DER MODULE]}}

\begin{itemize}
\item   \textbf{Safe-Words ↔ Rückkehr-Rituale ↔ Notfallplan\textbf{ = Sicherheitskreis
\item   \textbf{Werte-Kompass ↔ WG-Regeln ↔ Tagesstruktur\textbf{ = Stabilitätsanker
\item   \textbf{Trigger-Tagebuch ↔ Vision-Board ↔ Gesprächsregeln\textbf{ =
\end{itemize}
Entwicklungskompass
-

\begin{quote}
\textbf{📎 Dieser Leitfaden ist Start- und Übersichtspunkt für jedes
\end{quote}
\begin{quote}
\textcolor{ctmmBlue}{CTMM}-Paar. Er ersetzt kein Gespräch -- aber strukturiert, was
\end{quote}
\begin{quote}
besprochen werden muss.\textbf{
\end{quote}

📤 *Empfohlen als Deckblatt deines \textcolor{ctmmBlue}{CTMM}-Ordners + Startseite deiner
digitalin*

\textit{HTML-Version}

\% End of Bindungsleitfaden CTMM Übersicht .tex
