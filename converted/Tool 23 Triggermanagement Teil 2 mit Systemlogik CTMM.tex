\hypertarget{tool-23-triggermanagement-mit-systemlogik-ctmm-modul}{\%
\section{\texorpdfstring{\textbf{⚙️ TOOL 23 -- TRIGGERMANAGEMENT MIT SYSTEMLOGIK (CTMM-MODUL)}}{⚙️ TOOL 23 -- TRIGGERMANAGEMENT MIT SYSTEMLOGIK (CTMM-MODUL)}}\label{tool-23-triggermanagement-mit-systemlogik-ctmm-modul}}

\emph{\ldots{} (Seiten 1--3 unverändert) \ldots{}}

\hypertarget{seite-4-triggerarten-kuxf6rperreaktionen-fruxfchwarnzeichen}{\%
\subsection{\texorpdfstring{\textbf{🧠 \ul{SEITE 4 -- TRIGGERARTEN, KÖRPERREAKTIONEN \\& FRÜHWARNZEICHEN}}}{🧠 SEITE 4 -- TRIGGERARTEN, KÖRPERREAKTIONEN \\& FRÜHWARNZEICHEN}}\label{seite-4-triggerarten-kuxf6rperreaktionen-fruxfchwarnzeichen}}

\begin{quote}
✏️ \textbf{Diese Seite hilft dir, Frühwarnzeichen besser zu lesen -- bei dir selbst oder beim Gegenüber. Sie kann als Check-in-Liste, Schulung oder Notfallblatt verwendet werden.}
\end{quote}

\hypertarget{huxe4ufige-triggerarten}{\%
\subsubsection{\texorpdfstring{\textbf{🚨 \ul{HÄUFIGE TRIGGERARTEN}}}{🚨 HÄUFIGE TRIGGERARTEN}}\label{huxe4ufige-triggerarten}}

\begin{longtable}[]{@{}
  >{\raggedright\arraybackslash}p{(\columnwidth - 4\tabcolsep) * \real{0.2411}}
  >{\raggedright\arraybackslash}p{(\columnwidth - 4\tabcolsep) * \real{0.4673}}
  >{\raggedright\arraybackslash}p{(\columnwidth - 4\tabcolsep) * \real{0.2916}}@{}}
\toprule\noalign{}
\begin{minipage}[b]{\linewidth}\raggedright
\emph{\textbf{Kategorie}}
\end{minipage} \& \begin{minipage}[b]{\linewidth}\raggedright
\emph{\textbf{Beispiele}}
\end{minipage} \& \begin{minipage}[b]{\linewidth}\raggedright
\emph{\textbf{Dein Bezug dazu}}
\end{minipage} \\
\midrule\noalign{}
\endhead
\bottomrule\noalign{}
\endlastfoot
\textbf{Ablehnung} \& Nicht gehört werden, ignoriert werden \& 📝 \\_\\_\\_\\_\\_\\_\\_\\_\\_\\_\\_\\_\\_\\_\\_\\_\\_\\_\\_\\_\\_\\_\\_\\_\\_\\_\\_\\_\\_\\_\\_\\_\\_\\_\\_\\_\\_\\_\\_\\_\\_\\_\\_\\_\\_\\_\\_\\_\\_\\_\\_\\_\\_\\_ \\
\textbf{Nähe \\& Distanz} \& Rückzug, Klammern, plötzliche Trennung \& 📝 \\_\\_\\_\\_\\_\\_\\_\\_\\_\\_\\_\\_\\_\\_\\_\\_\\_\\_\\_\\_\\_\\_\\_\\_\\_\\_\\_\\_\\_\\_\\_\\_\\_\\_\\_\\_\\_\\_\\_\\_\\_\\_\\_\\_\\_\\_\\_\\_\\_\\_\\_\\_\\_\\_ \\
\textbf{Kontrollverlust} \& Überraschung, Veränderung, Überforderung \& 📝 \\_\\_\\_\\_\\_\\_\\_\\_\\_\\_\\_\\_\\_\\_\\_\\_\\_\\_\\_\\_\\_\\_\\_\\_\\_\\_\\_\\_\\_\\_\\_\\_\\_\\_\\_\\_\\_\\_\\_\\_\\_\\_\\_\\_\\_\\_\\_\\_\\_\\_\\_\\_\\_\\_ \\
\textbf{Lautstärke} \& Blaulicht, Knall, laute Musik und Geräusche im Supermarkt/Kirmes \& 📝 \\_\\_\\_\\_\\_\\_\\_\\_\\_\\_\\_\\_\\_\\_\\_\\_\\_\\_\\_\\_\\_\\_\\_\\_\\_\\_\\_\\_\\_\\_\\_\\_\\_\\_\\_\\_\\_\\_\\_\\_\\_\\_\\_\\_\\_\\_\\_\\_\\_\\_\\_\\_\\_\\_ \\
\textbf{Misstrauen} \& Kontrolle, Nachfragen, „Glaubst du mir nicht?{}`` \& 📝 \\_\\_\\_\\_\\_\\_\\_\\_\\_\\_\\_\\_\\_\\_\\_\\_\\_\\_\\_\\_\\_\\_\\_\\_\\_\\_\\_\\_\\_\\_\\_\\_\\_\\_\\_\\_\\_\\_\\_\\_\\_\\_\\_\\_\\_\\_\\_\\_\\_\\_\\_\\_\\_\\_ \\
\textbf{Hilflosigkeit} \& Keine Lösung finden, Erwartungen nicht erfüllen können \& 📝 \\_\\_\\_\\_\\_\\_\\_\\_\\_\\_\\_\\_\\_\\_\\_\\_\\_\\_\\_\\_\\_\\_\\_\\_\\_\\_\\_\\_\\_\\_\\_\\_\\_\\_\\_\\_\\_\\_\\_\\_\\_\\_\\_\\_\\_\\_\\_\\_\\_\\_\\_\\_\\_\\_ \\
\textbf{Körperkontakt} \& Unerwartete Berührung, kein Raum mehr \& 📝 \\_\\_\\_\\_\\_\\_\\_\\_\\_\\_\\_\\_\\_\\_\\_\\_\\_\\_\\_\\_\\_\\_\\_\\_\\_\\_\\_\\_\\_\\_\\_\\_\\_\\_\\_\\_\\_\\_\\_\\_\\_\\_\\_\\_\\_\\_\\_\\_\\_\\_\\_\\_\\_\\_ \\
\end{longtable}

\hypertarget{kuxf6rperreaktionen-bei-stress-dissoziation}{\%
\subsubsection{\texorpdfstring{\textbf{🔍 \ul{KÖRPERREAKTIONEN BEI STRESS / DISSOZIATION}}}{🔍 KÖRPERREAKTIONEN BEI STRESS / DISSOZIATION}}\label{kuxf6rperreaktionen-bei-stress-dissoziation}}

\begin{longtable}[]{@{}
  >{\raggedright\arraybackslash}p{(\columnwidth - 4\tabcolsep) * \real{0.3165}}
  >{\raggedright\arraybackslash}p{(\columnwidth - 4\tabcolsep) * \real{0.4822}}
  >{\raggedright\arraybackslash}p{(\columnwidth - 4\tabcolsep) * \real{0.2013}}@{}}
\toprule\noalign{}
\begin{minipage}[b]{\linewidth}\raggedright
\emph{\textbf{Reaktion}}
\end{minipage} \& \begin{minipage}[b]{\linewidth}\raggedright
\emph{\textbf{Beschreibung oder Signal}}
\end{minipage} \& \begin{minipage}[b]{\linewidth}\raggedright
\emph{\textbf{Kennst du das bei dir?}}
\end{minipage} \\
\midrule\noalign{}
\endhead
\bottomrule\noalign{}
\endlastfoot
\textbf{Erstarren / Freeze} \& Körper wird ruhig, Sprachlosigkeit \& ✅ ☐ ❌ ☐ \\
\textbf{Zittern / Unruhe} \& Hände, Beine, innerlich vibrieren \& ✅ ☐ ❌ ☐ \\
\textbf{Hautempfindlichkeit} \& Kleidung, Geräusche unangenehm \& ✅ ☐ ❌ ☐ \\
\textbf{Verschwommene Wahrnehmung} \& Tunnelblick, Taubheit, Nebel \& ✅ ☐ ❌ ☐ \\
\textbf{Atemprobleme} \& Flaches Atmen, Engegefühl \& ✅ ☐ ❌ ☐ \\
\textbf{Zeitverlust / Filmriss} \& Erinnerungslücken, Blackout \& ✅ ☐ ❌ ☐ \\
\end{longtable}

\hypertarget{section}{\%
\subsubsection{}\label{section}}

\hypertarget{section-1}{\%
\subsubsection{}\label{section-1}}

\hypertarget{section-2}{\%
\subsubsection{}\label{section-2}}

\hypertarget{fruxfchwarnzeichen-bei-ihm-oder-ihr}{\%
\subsubsection{\texorpdfstring{\textbf{🧭 \ul{FRÜHWARNZEICHEN BEI IHM ODER IHR}}}{🧭 FRÜHWARNZEICHEN BEI IHM ODER IHR}}\label{fruxfchwarnzeichen-bei-ihm-oder-ihr}}

\begin{longtable}[]{@{}
  >{\raggedright\arraybackslash}p{(\columnwidth - 4\tabcolsep) * \real{0.3334}}
  >{\raggedright\arraybackslash}p{(\columnwidth - 4\tabcolsep) * \real{0.3333}}
  >{\raggedright\arraybackslash}p{(\columnwidth - 4\tabcolsep) * \real{0.3333}}@{}}
\toprule\noalign{}
\begin{minipage}[b]{\linewidth}\raggedright
\emph{\textbf{Zeichen / Handlung}}
\end{minipage} \& \begin{minipage}[b]{\linewidth}\raggedright
\emph{\textbf{Was könnte es bedeuten?}}
\end{minipage} \& \begin{minipage}[b]{\linewidth}\raggedright
\emph{\textbf{Reaktion / Hilfeidee}}
\end{minipage} \\
\midrule\noalign{}
\endhead
\bottomrule\noalign{}
\endlastfoot
\textbf{„Ich muss kurz raus.``} \& Überforderung, Rückzug \& 🧍 Raum geben, Timer setzen \\
\textbf{„Ist doch egal.``} \& Hilflosigkeit, Disso oder Wut \& 💬 Validieren, nicht diskutieren \\
\textbf{„Du verstehst mich nie.``} \& Schmerz, Bindungsverletzung \& 🧠 Wertespiegel, kein Gegenvorwurf \\
\textbf{Körperspannung, starre Haltung} \& Freeze, Angst, Schutzmechanismus \& 🎧 Safe-Zone aktivieren \\
\textbf{Blick bricht ab / wird starr} \& Disso, Rückzug aus dem Kontakt \& 🖐️ Nonverbale Geste + Stille \\
📝 \\_\\_\\_\\_\\_\\_\\_\\_\\_\\_\\_\\_\\_\\_\\_\\_\\_\\_\\_\\_\\_\\_\\_\\_\\_\\_\\_\\_\\_\\_\\_\\_\\_\\_\\_\\_\\_\\_\\_\\_\\_\\_\\_\\_\\_\\_\\_\\_\\_\\_\\_\\_\\_\\_\\_\\_\\_\\_\\_\\_ \& 📝 \\_\\_\\_\\_\\_\\_\\_\\_\\_\\_\\_\\_\\_\\_\\_\\_\\_\\_\\_\\_\\_\\_\\_\\_\\_\\_\\_\\_\\_\\_\\_\\_\\_\\_\\_\\_\\_\\_\\_\\_\\_\\_\\_\\_\\_\\_\\_\\_\\_\\_\\_\\_\\_\\_\\_\\_\\_\\_\\_\\_ \& 📝 \\_\\_\\_\\_\\_\\_\\_\\_\\_\\_\\_\\_\\_\\_\\_\\_\\_\\_\\_\\_\\_\\_\\_\\_\\_\\_\\_\\_\\_\\_\\_\\_\\_\\_\\_\\_\\_\\_\\_\\_\\_\\_\\_\\_\\_\\_\\_\\_\\_\\_\\_\\_\\_\\_\\_\\_\\_\\_\\_\\_ \\
\end{longtable}

\begin{quote}
\emph{\textbf{📎 Diese Seite eignet sich für Schulungen, Aushänge in der WG oder als Reminder im Alltag. Kann farblich angepasst oder mit Symbolkarten erweitert werden.}}
\end{quote}
