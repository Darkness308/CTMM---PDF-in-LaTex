\hypertarget{tool-26}\{\%
\section{\texorpdfstring{🧠 \textbf{TOOL 26:} }{🧠 TOOL 26: }}\label{tool-26}}

\hypertarget{ko-regulation-gemeinsame-stuxe4rkung-ctmm-modul}\{\%
\section{\texorpdfstring{\textbf{KO-REGULATION \\\& GEMEINSAME STÄRKUNG (CTMM-MODUL)}}\{KO-REGULATION \\\& GEMEINSAME STÄRKUNG (CTMM-MODUL)}}\label{ko-regulation-gemeinsame-stuxe4rkung-ctmm-modul}}

🧩 \emph{\textbf{Gemeinsame Selbstberuhigung bei Trigger, Überforderung \\\& Nähe-Distanz-Konflikten}}

\hypertarget{was-ist-ko-regulation}\{\%
\subsection{\texorpdfstring{🤝 \textbf{\ul{WAS IST KO-REGULATION?}}}{🤝 WAS IST KO-REGULATION?}}\label{was-ist-ko-regulation}}

Ko-Regulation bedeutet, dass zwei Menschen einander helfen, ihr Nervensystem zu stabilisieren. Nicht durch Worte allein -- sondern durch:

\begin{itemize}
\tightlist
\item
  Präsenz
\item
  Nähe ohne Druck
\item
  Gesten, Berührung, Wiederholung
\item
  Atem, Körpersprache, Resonanz
\end{itemize}

\hypertarget{wirkungszonen-der-ko-regulation}\{\%
\subsection{\texorpdfstring{💞 \textbf{\ul{WIRKUNGSZONEN DER KO-REGULATION}}}{💞 WIRKUNGSZONEN DER KO-REGULATION}}\label{wirkungszonen-der-ko-regulation}}

\begin{longtable}[]\{@{}
  >\{\raggedright\arraybackslash}p\{(\columnwidth - 2\tabcolsep) * \real{0.3165}}
  >\{\raggedright\arraybackslash}p\{(\columnwidth - 2\tabcolsep) * \real{0.6835}}@{}}
\toprule\noalign{}
\begin{minipage}[b]\{\linewidth}\raggedright
\emph{\textbf{Wirkungsebene}}
\end{minipage} \& \begin{minipage}[b]\{\linewidth}\raggedright
\emph{\textbf{Beispiel (ER \\\& SIE)}}
\end{minipage} \\
\midrule\noalign{}
\endhead
\bottomrule\noalign{}
\endlastfoot
🧠 \textbf{Neurophysiologie} \& Atem synchronisieren, ruhiger sprechen, Reiz abschirmen \\
💬 \textbf{Sprache} \& „Ich bin da. Du musst nichts erklären.`` \\
🖐️ \textbf{Körperkontakt} \& Hand auf Brust, Rücken streichen, keine Fixierung \\
⏳ \textbf{Zeit \\\& Rhythmus} \& 5-Minuten-Stille, kein Fragenhagel \\
🔄 \textbf{Rituale \\\& Symbole} \& Gemeinsames Licht, Musik, Symbolfoto, Ritualsatz \\
\end{longtable}

\hypertarget{wann-ist-ko-regulation-sinnvoll}\{\%
\subsection{\texorpdfstring{⚠️ \textbf{\ul{WANN IST KO-REGULATION SINNVOLL?}}}{⚠️ WANN IST KO-REGULATION SINNVOLL?}}\label{wann-ist-ko-regulation-sinnvoll}}

\begin{itemize}
\tightlist
\item
  Vor dem Eskalationspunkt (Frühwarnzeichen!)
\item
  Nach einem Streit -- ohne Schuldzuweisung
\item
  Wenn einer überfordert ist, der andere aber stabil
\item
  Nach Flashback, Dissoziation, Reizschub
\end{itemize}

\hypertarget{ko-regulation-vs.-rettung}\{\%
\subsection{\texorpdfstring{🔁 \textbf{\ul{KO-REGULATION VS. RETTUNG}}}{🔁 KO-REGULATION VS. RETTUNG}}\label{ko-regulation-vs.-rettung}}

\begin{longtable}[]\{@{}
  >\{\raggedright\arraybackslash}p\{(\columnwidth - 4\tabcolsep) * \real{0.2888}}
  >\{\raggedright\arraybackslash}p\{(\columnwidth - 4\tabcolsep) * \real{0.3019}}
  >\{\raggedright\arraybackslash}p\{(\columnwidth - 4\tabcolsep) * \real{0.4094}}@{}}
\toprule\noalign{}
\begin{minipage}[b]\{\linewidth}\raggedright
\emph{\textbf{Merkmal}}
\end{minipage} \& \begin{minipage}[b]\{\linewidth}\raggedright
\emph{\textbf{Ko-Regulation}}
\end{minipage} \& \begin{minipage}[b]\{\linewidth}\raggedright
\emph{\textbf{Retten / Kontrollieren}}
\end{minipage} \\
\midrule\noalign{}
\endhead
\bottomrule\noalign{}
\endlastfoot
\textbf{Verantwortung} \& geteilt \& asymmetrisch („ich rette dich``) \\
\textbf{Energie} \& wechselseitig regulierend \& erschöpfend für eine Seite \\
\textbf{Sprache} \& ruhig, absichtslos \& aktiv, fordernd, zielorientiert \\
\textbf{Berührung} \& angeboten, dosiert \& übergriffig, ohne Abstimmung \\
\textbf{Beziehungserfahrung} \& gemeinschaftlich, sicher \& regressiv, emotional unsicher \\
\end{longtable}

\hypertarget{ctmm-integration-navigation}\{\%
\subsection{\texorpdfstring{🧭 \textbf{\ul{CTMM-INTEGRATION \\\& NAVIGATION}}}\{🧭 CTMM-INTEGRATION \\\& NAVIGATION}}\label{ctmm-integration-navigation}}

\begin{itemize}
\tightlist
\item
  🟢 \texttt{Kap.\ }\texttt{2.6} -- Emotionale Präsenz, Beziehung als Team
\item
  🟠 \texttt{Kap.\ }\texttt{3.1\ –\ 3.5} -- Notfallstruktur, Rückkehrrituale
\item
  🟣 Tools: \texttt{Trigger-}\texttt{Tagebuch}, \texttt{Werte-Kompass}, \texttt{Bindungsdynamik}
\end{itemize}

✅ Geeignet für: Partnerübungen, Paartherapie, Buddy-Training

\begin{quote}
\textbf{📎 Dieses Tool stärkt das Teamgefühl -- nicht durch Gespräche, sondern durch Haltung, Timing \\\& Wiederholung}
\end{quote}
