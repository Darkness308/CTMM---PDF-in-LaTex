% CTMM Therapy Module: Tool 26 Co Regulation & Gemeinsame Stärkung Ctmm
% Converted from Word document: Tool 26 Co Regulation & Gemeinsame Stärkung CTMM.docx
% Auto-generated by document_converter.py

% Options for packages loaded elsewhere
\PassOptionsToPackage{unicode}{hyperref}
\PassOptionsToPackage{hyphens}{url}
%
\documentclass[
]{article}
\usepackage{amsmath,amssymb}
\usepackage{iftex}
\ifPDFTeX
  \usepackage[T1]{fontenc}
  \usepackage[utf8]{inputenc}
  \usepackage{textcomp} % provide euro and other symbols
\else % if luatex or xetex
  \usepackage{unicode-math} % this also loads fontspec
  \defaultfontfeatures{Scale=MatchLowercase}
  \defaultfontfeatures[\rmfamily]{Ligatures=TeX,Scale=1}
\fi
\usepackage{lmodern}
\ifPDFTeX\else
  % xetex/luatex font selection
\fi
% Use upquote if available, for straight quotes in verbatim environments
\IfFileExists{upquote.sty}{\usepackage{upquote}}{}
\IfFileExists{microtype.sty}{% use microtype if available
  \usepackage[]{microtype}
  \UseMicrotypeSet[protrusion]{basicmath} % disable protrusion for tt fonts
}{}
\makeatletter
\@ifundefined{KOMAClassName}{% if non-KOMA class
  \IfFileExists{parskip.sty}{%
    \usepackage{parskip}
  }{% else
    \setlength{\parindent}{0pt}
    \setlength{\parskip}{6pt plus 2pt minus 1pt}}
}{% if KOMA class
  \KOMAoptions{parskip=half}}
\makeatother
\usepackage{xcolor}
\usepackage{longtable,booktabs,array}
\usepackage{calc} % for calculating minipage widths
% Correct order of tables after \paragraph or \subparagraph
\usepackage{etoolbox}
\makeatletter
\patchcmd\longtable{\par}{\if@noskipsec\mbox{}\fi\par}{}{}
\makeatother
% Allow footnotes in longtable head/foot
\IfFileExists{footnotehyper.sty}{\usepackage{footnotehyper}}{\usepackage{footnote}}
\makesavenoteenv{longtable}
\ifLuaTeX
  \usepackage{luacolor}
  \usepackage[soul]{lua-ul}
\else
  \usepackage{soul}
\fi
\setlength{\emergencystretch}{3em} % prevent overfull lines
\providecommand{\tightlist}{%
  \setlength{\itemsep}{0pt}\setlength{\parskip}{0pt}}
\setcounter{secnumdepth}{-\maxdimen} % remove section numbering
\ifLuaTeX
  \usepackage{selnolig}  % disable illegal ligatures
\fi
\IfFileExists{bookmark.sty}{\usepackage{bookmark}}{\usepackage{hyperref}}
\IfFileExists{xurl.sty}{\usepackage{xurl}}{} % add URL line breaks if available
\urlstyle{same}
\hypersetup{
  hidelinks,
  pdfcreator={LaTeX via pandoc}}

\author{}
\date{}



\hypertarget{tool-26}{%
\section{\texorpdfstring{🧠 \textbf{TOOL 26:} }{🧠 TOOL 26: }}\label{tool-26}}

\hypertarget{ko-regulation-gemeinsame-stuxe4rkung-ctmm-modul}{%
\section{\texorpdfstring{\textbf{KO-REGULATION \& GEMEINSAME STÄRKUNG (CTMM-MODUL)}}{KO-REGULATION \& GEMEINSAME STÄRKUNG (CTMM-MODUL)}}\label{ko-regulation-gemeinsame-stuxe4rkung-ctmm-modul}}

🧩 \emph{\textbf{Gemeinsame Selbstberuhigung bei Trigger, Überforderung \& Nähe-Distanz-Konflikten}}

\hypertarget{was-ist-ko-regulation}{%
\subsection{\texorpdfstring{🤝 \textbf{\ul{WAS IST KO-REGULATION?}}}{🤝 WAS IST KO-REGULATION?}}\label{was-ist-ko-regulation}}

Ko-Regulation bedeutet, dass zwei Menschen einander helfen, ihr Nervensystem zu stabilisieren. Nicht durch Worte allein -- sondern durch:

\begin{itemize}
\tightlist
\item
  Präsenz
\item
  Nähe ohne Druck
\item
  Gesten, Berührung, Wiederholung
\item
  Atem, Körpersprache, Resonanz
\end{itemize}

\hypertarget{wirkungszonen-der-ko-regulation}{%
\subsection{\texorpdfstring{💞 \textbf{\ul{WIRKUNGSZONEN DER KO-REGULATION}}}{💞 WIRKUNGSZONEN DER KO-REGULATION}}\label{wirkungszonen-der-ko-regulation}}

\begin{longtable}[]{@{}
  >{\raggedright\arraybackslash}p{(\columnwidth - 2\tabcolsep) * \real{0.3165}}
  >{\raggedright\arraybackslash}p{(\columnwidth - 2\tabcolsep) * \real{0.6835}}@{}}
\toprule\noalign{}
\begin{minipage}[b]{\linewidth}\raggedright
\emph{\textbf{Wirkungsebene}}
\end{minipage} & \begin{minipage}[b]{\linewidth}\raggedright
\emph{\textbf{Beispiel (ER \& SIE)}}
\end{minipage} \\
\midrule\noalign{}
\endhead
\bottomrule\noalign{}
\endlastfoot
🧠 \textbf{Neurophysiologie} & Atem synchronisieren, ruhiger sprechen, Reiz abschirmen \\
💬 \textbf{Sprache} & „Ich bin da. Du musst nichts erklären.`` \\
🖐️ \textbf{Körperkontakt} & Hand auf Brust, Rücken streichen, keine Fixierung \\
⏳ \textbf{Zeit \& Rhythmus} & 5-Minuten-Stille, kein Fragenhagel \\
🔄 \textbf{Rituale \& Symbole} & Gemeinsames Licht, Musik, Symbolfoto, Ritualsatz \\
\end{longtable}

\hypertarget{wann-ist-ko-regulation-sinnvoll}{%
\subsection{\texorpdfstring{⚠️ \textbf{\ul{WANN IST KO-REGULATION SINNVOLL?}}}{⚠️ WANN IST KO-REGULATION SINNVOLL?}}\label{wann-ist-ko-regulation-sinnvoll}}

\begin{itemize}
\tightlist
\item
  Vor dem Eskalationspunkt (Frühwarnzeichen!)
\item
  Nach einem Streit -- ohne Schuldzuweisung
\item
  Wenn einer überfordert ist, der andere aber stabil
\item
  Nach Flashback, Dissoziation, Reizschub
\end{itemize}

\hypertarget{ko-regulation-vs.-rettung}{%
\subsection{\texorpdfstring{🔁 \textbf{\ul{KO-REGULATION VS. RETTUNG}}}{🔁 KO-REGULATION VS. RETTUNG}}\label{ko-regulation-vs.-rettung}}

\begin{longtable}[]{@{}
  >{\raggedright\arraybackslash}p{(\columnwidth - 4\tabcolsep) * \real{0.2888}}
  >{\raggedright\arraybackslash}p{(\columnwidth - 4\tabcolsep) * \real{0.3019}}
  >{\raggedright\arraybackslash}p{(\columnwidth - 4\tabcolsep) * \real{0.4094}}@{}}
\toprule\noalign{}
\begin{minipage}[b]{\linewidth}\raggedright
\emph{\textbf{Merkmal}}
\end{minipage} & \begin{minipage}[b]{\linewidth}\raggedright
\emph{\textbf{Ko-Regulation}}
\end{minipage} & \begin{minipage}[b]{\linewidth}\raggedright
\emph{\textbf{Retten / Kontrollieren}}
\end{minipage} \\
\midrule\noalign{}
\endhead
\bottomrule\noalign{}
\endlastfoot
\textbf{Verantwortung} & geteilt & asymmetrisch („ich rette dich``) \\
\textbf{Energie} & wechselseitig regulierend & erschöpfend für eine Seite \\
\textbf{Sprache} & ruhig, absichtslos & aktiv, fordernd, zielorientiert \\
\textbf{Berührung} & angeboten, dosiert & übergriffig, ohne Abstimmung \\
\textbf{Beziehungserfahrung} & gemeinschaftlich, sicher & regressiv, emotional unsicher \\
\end{longtable}

\hypertarget{ctmm-integration-navigation}{%
\subsection{\texorpdfstring{🧭 \textbf{\ul{CTMM-INTEGRATION \& NAVIGATION}}}{🧭 CTMM-INTEGRATION \& NAVIGATION}}\label{ctmm-integration-navigation}}

\begin{itemize}
\tightlist
\item
  🟢 \texttt{Kap.\ }\texttt{2.6} -- Emotionale Präsenz, Beziehung als Team
\item
  🟠 \texttt{Kap.\ }\texttt{3.1\ –\ 3.5} -- Notfallstruktur, Rückkehrrituale
\item
  🟣 Tools: \texttt{Trigger-}\texttt{Tagebuch}, \texttt{Werte-Kompass}, \texttt{Bindungsdynamik}
\end{itemize}

✅ Geeignet für: Partnerübungen, Paartherapie, Buddy-Training

\begin{quote}
\textbf{📎 Dieses Tool stärkt das Teamgefühl -- nicht durch Gespräche, sondern durch Haltung, Timing \& Wiederholung}
\end{quote}


