\% Bindungsdynamik CTMM Modul.tex - Converted from Bindungsdynamik CTMM Modul.docx
\% CTMM Therapy Material - Auto-generated by Conversion Pipeline

\section{Bindungsdynamik Ctmm Modul}
\label{sec:bindungsdynamik-ctmm-modul}

\section{🧾 \textbf{BINDUNGSDYNAMIK --}}
\subsection{\textbf{\textcolor{ctmmBlue}{CTMM}-MODUL}}

\textcolor{ctmmBlue}{\faPuzzlePiece} \textbf{Neurodiverse Paarkonflikte verstehen, strukturieren und
intervenieren -- ergänzt um die subjektive Sicht von ER \\& SIE\textbf{

\subsection{🔍 \textbf{[AUSGANGSLAGE: NEUROPSYCHOLOGISCHE PROFILBESONDERHEITEN]}}

\subsubsection{🔵 \textbf{ER (kPTBS, ASS, \textcolor{ctmmOrange}{ADHS}, Post-OP)}}

\begin{itemize}
\item   Rechter Temporallappen entfernt → visuelle, soziale \& räumliche
\end{itemize}

\begin{quote}
Desorientierung
\end{quote}

\begin{itemize}
\item   Amygdala fehlt → kaum Angst/Gefühlsregulation
\end{itemize}

\begin{quote}
möglich
\end{quote}

\begin{itemize}
\item   Hippocampus geschädigt → Kontextbindung gestört
\item   Prosopagnosie (Gesichtsblindheit) → Gedächtnislücken
\end{itemize}

\subsubsection{\textcolor{ctmmRed}{\faCircle} \textbf{SIE (\textcolor{ctmmPurple}{Borderline}, \textcolor{ctmmOrange}{ADHS}, ASS)}}

\begin{itemize}
\item   Emotionale Dysregulation, Impulsivität
\item   Komplexe Verlustangst, Nähe-Distanz-Konflikte
\item   Schwankende Bindungserfahrung (Idealisierung ↔ Abwertung)
\end{itemize}

\subsection{⚠️ \textbf{[DYNAMIK: TRIGGER-LOOP (Beidseitig selbstverstärkend)]}}

1️⃣ SIE: Verlustangst 			→ Klammern, Vorwürfe
2️⃣ ER: Reizüberflutung 			→ Rückzug
3️⃣ SIE: Reaktion als Ablehnung 	→ Eskalation
4️⃣ ER: Kompletter Rückzug, Notfallprotokoll

\textbf{\textcolor{ctmmYellow}{\faLightbulb} [Intervention:]} Safe-Word einsetzen \textbf{spätestens bei
Stufe 2!\textbf{

\subsection{\textcolor{ctmmPurple}{\faBrain} \textbf{[TYPISCHE INTERPRETATIONSFEHLER]}}

----------------------------------------------------------------------------------
\textbf{\textit{Verhalten ER}} \textbf{\textit{SIE              }}Neuro-Realität\textbf{\textit{    }}Lösung\textit{*}
Interpretation\textit{*}
------------------ ------------------- ----------------------- -------------------
Rückzug            Ablehnung, Ignoranz Reizschutzmechanismus   Ankündigung +
Rückkehrritual

Kühle Reaktion     Liebesentzug        Amygdala fehlt -- kein  Klartext: „Ich
Emotionsausdruck        fühle, aber anders"

Orientierungslos / Will weg            Temporallappen fehlt    GPS, feste
Schweigen                                                      Strukturen,
Begleitung

Vergesslichkeit /  Gleichgültigkeit    Hippocampusdefizit      External Memory
Verwirrung                                                     System
----------------------------------------------------------------------------------

\\#\\#

\subsection{\textcolor{ctmmBlue}{\faPuzzlePiece} \textbf{[WIE ER DIE DYNAMIK ERLEBT]}}

\begin{itemize}
\item   Ich ziehe mich nicht zurück, weil ich sie ablehne -- sondern weil
\end{itemize}
mein Gehirn sonst überlädt.
\begin{itemize}
\item   Ich vergesse Dinge nicht aus Gleichgültigkeit -- es ist wie ein
\end{itemize}
defektes Speichermodul.
\begin{itemize}
\item   Wenn sie wütend wird, schaltet mein System auf Notbetrieb. Ich höre
\end{itemize}
dann nur noch Flucht.
\begin{itemize}
\item   Ich brauche Ankündigung, Struktur, Wiederholung -- keine
\end{itemize}
Spontaneität unter Druck.

\subsubsection{✏️ \textbf{[Eigene Worte:]}}

\textcolor{ctmmBlue}{\faEdit}
\\\_\\\_\\\_\\\_\\\_\\\_\\\_\\\_\\\_\\\_\\\_\\\_\\\_\\\_\\\_\\\_\\\_\\\_\\\_\\\_\\\_\\\_\\\_\\\_\\\_\\\_\\\_\\\_\\\_\\\_\\\_\\\_\\\_\\\_\\\_\\\_\\\_\\\_\\\_\\\_\\\_\\\_\\\_\\\_\\\_\\\_\\\_\\\_\\\_\\\_\\\_\\\_\\\_\\\_\\\_\\\_\\\_\\\_\\\_\\\_\\\_\\\_\\\_\\\_\\\_\\\_\\\_\\\_\\\_\\\_\\\_\\\_\\\_\\\_

\textcolor{ctmmBlue}{\faEdit}
\\\_\\\_\\\_\\\_\\\_\\\_\\\_\\\_\\\_\\\_\\\_\\\_\\\_\\\_\\\_\\\_\\\_\\\_\\\_\\\_\\\_\\\_\\\_\\\_\\\_\\\_\\\_\\\_\\\_\\\_\\\_\\\_\\\_\\\_\\\_\\\_\\\_\\\_\\\_\\\_\\\_\\\_\\\_\\\_\\\_\\\_\\\_\\\_\\\_\\\_\\\_\\\_\\\_\\\_\\\_\\\_\\\_\\\_\\\_\\\_\\\_\\\_\\\_\\\_\\\_\\\_\\\_\\\_\\\_\\\_\\\_\\\_\\\_\\\_

\textcolor{ctmmBlue}{\faEdit}
\\\_\\\_\\\_\\\_\\\_\\\_\\\_\\\_\\\_\\\_\\\_\\\_\\\_\\\_\\\_\\\_\\\_\\\_\\\_\\\_\\\_\\\_\\\_\\\_\\\_\\\_\\\_\\\_\\\_\\\_\\\_\\\_\\\_\\\_\\\_\\\_\\\_\\\_\\\_\\\_\\\_\\\_\\\_\\\_\\\_\\\_\\\_\\\_\\\_\\\_\\\_\\\_\\\_\\\_\\\_\\\_\\\_\\\_\\\_\\\_\\\_\\\_\\\_\\\_\\\_\\\_\\\_\\\_\\\_\\\_\\\_\\\_\\\_\\\_\\\_\\\_\\\_\\\_\\\_\\\_\\\_\\\_\\\_\\\_\\\_\\\_\\\_\\\_\\\_\\\_\\\_\\\_\\\_\\\_\\\_\\\_\\\_\\\_\\\_\\\_\\\_\\\_\\\_\\\_

\subsection{\textcolor{ctmmBlue}{\faPuzzlePiece} \textbf{[WIE SIE DIE DYNAMIK ERLEBT]}}

\begin{itemize}
\item   Wenn er geht, fühle ich mich verlassen -- nicht weil ich klammern
\end{itemize}
will, sondern weil Panik aufsteigt.
\begin{itemize}
\item   Seine Kühle verletzt mich, auch wenn ich weiß, dass er nicht anders
\end{itemize}
kann.
\begin{itemize}
\item   Ich brauche Nähe als Sicherheit -- Rückzug fühlt sich wie
\end{itemize}
Liebesentzug an.
\begin{itemize}
\item   Ich brauche Wiederholung, Worte, Berührung -- nicht plötzliche
\end{itemize}
Stille.

\subsubsection{✏️ \textbf{[Eigene Worte:]}}

\textcolor{ctmmBlue}{\faEdit}
\\\_\\\_\\\_\\\_\\\_\\\_\\\_\\\_\\\_\\\_\\\_\\\_\\\_\\\_\\\_\\\_\\\_\\\_\\\_\\\_\\\_\\\_\\\_\\\_\\\_\\\_\\\_\\\_\\\_\\\_\\\_\\\_\\\_\\\_\\\_\\\_\\\_\\\_\\\_\\\_\\\_\\\_\\\_\\\_\\\_\\\_\\\_\\\_\\\_\\\_\\\_\\\_\\\_\\\_\\\_\\\_\\\_\\\_\\\_\\\_\\\_\\\_\\\_\\\_\\\_\\\_\\\_\\\_\\\_\\\_\\\_\\\_\\\_\\\_

\textcolor{ctmmBlue}{\faEdit}
\\\_\\\_\\\_\\\_\\\_\\\_\\\_\\\_\\\_\\\_\\\_\\\_\\\_\\\_\\\_\\\_\\\_\\\_\\\_\\\_\\\_\\\_\\\_\\\_\\\_\\\_\\\_\\\_\\\_\\\_\\\_\\\_\\\_\\\_\\\_\\\_\\\_\\\_\\\_\\\_\\\_\\\_\\\_\\\_\\\_\\\_\\\_\\\_\\\_\\\_\\\_\\\_\\\_\\\_\\\_\\\_\\\_\\\_\\\_\\\_\\\_\\\_\\\_\\\_\\\_\\\_\\\_\\\_\\\_\\\_\\\_\\\_\\\_\\\_

\textcolor{ctmmBlue}{\faEdit}
\\\_\\\_\\\_\\\_\\\_\\\_\\\_\\\_\\\_\\\_\\\_\\\_\\\_\\\_\\\_\\\_\\\_\\\_\\\_\\\_\\\_\\\_\\\_\\\_\\\_\\\_\\\_\\\_\\\_\\\_\\\_\\\_\\\_\\\_\\\_\\\_\\\_\\\_\\\_\\\_\\\_\\\_\\\_\\\_\\\_\\\_\\\_\\\_\\\_\\\_\\\_\\\_\\\_\\\_\\\_\\\_\\\_\\\_\\\_\\\_\\\_\\\_\\\_\\\_\\\_\\\_\\\_\\\_\\\_\\\_\\\_\\\_\\\_\\\_\\\_\\\_\\\_\\\_\\\_\\\_\\\_\\\_\\\_\\\_\\\_\\\_\\\_\\\_\\\_\\\_\\\_\\\_\\\_\\\_\\\_\\\_\\\_\\\_\\\_\\\_\\\_\\\_\\\_\\\_

\subsection{🔄 \textbf{[VERGLEICH: INNENWELT ER vs. SIE]}}

-------------------------------------------------------------------------
\textbf{\textit{Situation}}   \textbf{\textit{ER fühlt/erlebt...}}   \textit{*}SIE
denkt/interpretiert...\textit{*}
----------------- -------------------------- ----------------------------
\textbf{Rückzug nach    Ich muss raus, sonst       Er lässt mich wieder allein
Reiz\textbf{            Shutdown

\textbf{Erinnerung      Ich habe kein Bild im Kopf Ich bin ihm nicht wichtig
fehlt\textbf{

\textbf{Kühle Stimme}  Ich will keinen Fehler     Er liebt mich nicht
machen

\textbf{Kein            Ich kann nicht fokussieren Er vermeidet mich, schämt
Blickkontakt\textbf{                               sich
-------------------------------------------------------------------------

\\#\\#

\subsection{🔑 \textbf{[SAFE-WORD-SYSTEM (\textcolor{ctmmBlue}{CTMM}-Ebene Rot)]}}

-----------------------------------------------------------------------------------
\textbf{\textit{Safe-Word}}    \textbf{\textit{Bedeutung (ER)}} \textbf{\textit{Bedeutung (SIE)}}     \textbf{\textit{Reaktion}}
------------------ -------------------- ------------------------- -----------------
\textbf{"TimeOut"}      Reizüberflutung,     Panikphase → will Nähe,   Trennung + Signal
sofortiger Rückzug   braucht Halt

\textbf{"Lagerfeuer"}   Bin überfordert,     Ich habe Angst, bleib bei Körperkontakt
brauche Beruhigung   mir

\textbf{"Reset"}        Gespräch neustarten, Reset ohne Schuldfragen   Rückkehr-Ritual
keine Vorwürfe
-----------------------------------------------------------------------------------

✅ \textbf{Rückkehr ist Pflicht! (Rückkehrritual)} → Siehe `Kap. ``3.5`

\subsection{🤝 \textbf{[KO-REGULATION ALS BEZIEHUNGSPRINZIP]}}

\textbf{ER unterstützt SIE bei:}

\begin{itemize}
\item   Validierung: „Du bist wichtig, ich gehe nicht"
\item   Skills gemeinsam aktivieren → Atemübung, Reizumleitung
\item   Keine Diskussion bei Wut → klare Signale, Abstand, Wiederkehr
\end{itemize}

\textbf{SIE unterstützt ER bei:}

\begin{itemize}
\item   Navigation, Planung, Verankerung im Alltag (Buddy-Rolle)
\item   Reizreduktion durch Struktur, Zeitpuffer, Vorwarnung
\item   Hilft bei sozialen Situationen, übernimmt Kommunikation
\end{itemize}

\subsection{\textcolor{ctmmOrange}{\faCompass} \textbf{[\textcolor{ctmmBlue}{CTMM}-EINBINDUNG \\& NAVIGATION]}}

\begin{itemize}
\item   🔵 Grundlagen → `Kap. ``1`
\item   \textcolor{ctmmGreen}{\faCircle} Safe-Words, Rituale → `Kap. ``2.2 -- 2.6`
\item   \textcolor{ctmmRed}{\faCircle} Notfallstruktur → `Kap. ``3.1 -- 3.5`
\item   🟣 Tools \& Reflexion → `Tool 23`, `Tool 26`, `Kap. ``5.5`
\end{itemize}

\begin{quote}
\textit{*}Diese Datei ist Kernbestandteil der \textcolor{ctmmBlue}{CTMM}-Matrix:
\end{quote}
\begin{quote}
„Beziehungssicherung durch neuroadaptive Kommunikation"\textit{*}
\end{quote}

\% End of Bindungsdynamik CTMM Modul.tex
