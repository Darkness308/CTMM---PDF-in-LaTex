\hypertarget{tool-23-trigger-management-ctmm-modul}{%
\section{\texorpdfstring{\textbf{🔄 TOOL 23 -- TRIGGER MANAGEMENT (CTMM-MODUL)}}{🔄 TOOL 23 -- TRIGGER MANAGEMENT (CTMM-MODUL)}}\label{tool-23-trigger-management-ctmm-modul}

\begin{quote}
🧠 \textbf{\ul{Worum geht's hier -- für Freunde?}\\
Trigger verstehen heißt sich selbst verstehen -- der Schlüssel zur Selbstregulation. Ein Trigger ist ein Auslöser, der starke emotionale oder körperliche Reaktionen hervorruft. Im CTMM werden Trigger erkannt, verstanden und bearbeitet.
\end{quote}

🧩 \textbf{Systematischer Ansatz zur Trigger-Erkennung und Bewältigung mit interaktiven Elementen}

\hypertarget{kapitelzuordnung-im-ctmm-system}{%
\subsection{Kapitelzuordnung im CTMM-System}}\label{kapitelzuordnung-im-ctmm-system}

\begin{itemize}
\tightlist
\item
  \texttt{Kap.\ 2.3} → Trigger-Erkennung und erste Maßnahmen
\item
  \texttt{Kap.\ 3.1} → Akute Bewältigungsstrategien und Grounding
\item
  \texttt{Kap.\ 4.2} → Langfristige Trigger-Arbeit und Musteranalyse
\item
  \texttt{Kap.\ 5.2} → Trigger-Tagebuch und Dokumentation
\end{itemize}

\hypertarget{trigger-erkennungszeichen}{%
\subsection{1. Trigger Erkennungszeichen}}\label{trigger-erkennungszeichen}

\begin{center}
\begin{tabular}{|p{4cm}|p{5cm}|p{5cm}|}
\hline
\textbf{Reaktionsbereich} & \textbf{Typische Anzeichen} & \textbf{Intensitätsskala (1-10)} \\
\hline
\textbf{Körperliche Reaktionen} & Herzklopfen, Schwitzen, Anspannung, Übelkeit, Zittern & \rule{3cm}{0.4pt} \\
\hline
\textbf{Gedanken \& Erinnerungen} & Flashbacks, Grübeln, Katastrophendenken, Gedankenkreise & \rule{3cm}{0.4pt} \\
\hline
\textbf{Emotionale Zustände} & Angst, Wut, Traurigkeit, Hilflosigkeit, Panik & \rule{3cm}{0.4pt} \\
\hline
\textbf{Verhaltensänderungen} & Rückzug, Aggression, Erstarrung, Hypervigilanz & \rule{3cm}{0.4pt} \\
\hline
\end{tabular}
\end{center}

\hypertarget{bewaeltigungsstrategien-toolbox}{%
\subsection{2. Bewältigungsstrategien Toolbox}}\label{bewaeltigungsstrategien-toolbox}

\begin{center}
\begin{tabular}{|p{3cm}|p{4cm}|p{4cm}|p{3cm}|}
\hline
\textbf{Strategie} & \textbf{Anwendung} & \textbf{Wann einsetzen?} & \textbf{Wirksamkeit (✓)} \\
\hline
\textbf{Bewusstes Atmen} & 4-7-8 Technik, Bauchatmung & Bei Panik, Hyperventilation & \rule{1.5cm}{0.4pt} \\
\hline
\textbf{Grounding 5-4-3-2-1} & Sinne aktivieren, Realität verankern & Bei Dissoziation, Flashbacks & \rule{1.5cm}{0.4pt} \\
\hline
\textbf{Selbstfürsorge} & Pausen, Grenzen, Ressourcen aktivieren & Präventiv, bei Überlastung & \rule{1.5cm}{0.4pt} \\
\hline
\textbf{Soziale Unterstützung} & Safe Words, Vertrauensperson kontaktieren & Bei Eskalation, Überforderung & \rule{1.5cm}{0.4pt} \\
\hline
\textbf{Trigger-Tagebuch} & Muster erkennen, Fortschritte dokumentieren & Kontinuierlich zur Analyse & \rule{1.5cm}{0.4pt} \\
\hline
\end{tabular}
\end{center}

\hypertarget{persoenliche-trigger-analyse-arbeitsbereich}{%
\subsection{3. Persönliche Trigger-Analyse Arbeitsbereich}}\label{persoenliche-trigger-analyse-arbeitsbereich}

\textbf{Schritt 1: Aktuelle Trigger-Situation beschreiben}

\begin{itemize}
\tightlist
\item
  Was ist passiert? (Ort, Zeit, beteiligte Personen): \rule{8cm}{0.4pt}
\item
  Vorgeschichte/Kontext: \rule{8cm}{0.4pt}
\item
  Auslöser (spezifisch): \rule{8cm}{0.4pt}
\end{itemize}

\textbf{Schritt 2: Körperliche Reaktionen dokumentieren}

\begin{itemize}
\tightlist
\item
  Wie hat mein Körper reagiert? \rule{8cm}{0.4pt}
\item
  Herzschlag/Atmung: \rule{4cm}{0.4pt} Anspannung: \rule{4cm}{0.4pt}
\item
  Andere Körpersymptome: \rule{8cm}{0.4pt}
\end{itemize}

\textbf{Schritt 3: Gedanken und Gefühle erfassen}

\begin{itemize}
\tightlist
\item
  Welche Gedanken gingen mir durch den Kopf? \rule{8cm}{0.4pt}
\item
  Welche Gefühle entstanden? \rule{6cm}{0.4pt} Intensität (1-10): \rule{2cm}{0.4pt}
\item
  Erinnerungen/Assoziationen: \rule{8cm}{0.4pt}
\end{itemize}

\textbf{Schritt 4: Bewältigungsstrategien planen}

\begin{itemize}
\tightlist
\item
  Was könnte mir in Zukunft helfen? (konkrete Handlungen): \rule{8cm}{0.4pt}
\item
  Wen kann ich um Unterstützung bitten? \rule{8cm}{0.4pt}
\item
  Präventive Maßnahmen: \rule{8cm}{0.4pt}
\end{itemize}

\textbf{Schritt 5: Erfolg dokumentieren}

\begin{itemize}
\tightlist
\item
  Was habe ich diesmal bereits gut gemacht? \rule{8cm}{0.4pt}
\item
  Welche Strategien haben geholfen? \rule{8cm}{0.4pt}
\item
  Fortschritt im Vergleich zu früher: \rule{8cm}{0.4pt}
\end{itemize}

\hypertarget{trigger-muster-matrix}{%
\subsection{4. Trigger-Muster Matrix}}\label{trigger-muster-matrix}

\textbf{Persönliche Trigger-Muster erkennen und dokumentieren:}

\begin{center}
\begin{tabular}{|p{3cm}|p{3cm}|p{3cm}|p{4cm}|}
\hline
\textbf{Trigger-Typ} & \textbf{Häufigkeit} & \textbf{Intensität (1-10)} & \textbf{Bewältigungsstrategie} \\
\hline
Kritik/Vorwürfe & \rule{2cm}{0.4pt} & \rule{2cm}{0.4pt} & \rule{3cm}{0.4pt} \\
\hline
Unvorhergesehene Änderungen & \rule{2cm}{0.4pt} & \rule{2cm}{0.4pt} & \rule{3cm}{0.4pt} \\
\hline
Lautstärke/Geräusche & \rule{2cm}{0.4pt} & \rule{2cm}{0.4pt} & \rule{3cm}{0.4pt} \\
\hline
Nähe/Distanz-Konflikte & \rule{2cm}{0.4pt} & \rule{2cm}{0.4pt} & \rule{3cm}{0.4pt} \\
\hline
Überforderung & \rule{2cm}{0.4pt} & \rule{2cm}{0.4pt} & \rule{3cm}{0.4pt} \\
\hline
Individuelle Trigger: & & & \\
\rule{2.5cm}{0.4pt} & \rule{2cm}{0.4pt} & \rule{2cm}{0.4pt} & \rule{3cm}{0.4pt} \\
\hline
\rule{2.5cm}{0.4pt} & \rule{2cm}{0.4pt} & \rule{2cm}{0.4pt} & \rule{3cm}{0.4pt} \\
\hline
\end{tabular}
\end{center}

\hypertarget{notfall-trigger-protokoll}{%
\subsection{5. Notfall-Trigger Protokoll}}\label{notfall-trigger-protokoll}

\textbf{Bei akuten Triggern - sofort anwendbar:}

\begin{enumerate}
\def\labelenumi{\arabic{enumi}.}
\tightlist
\item
  \textbf{STOPP} -- bewusst wahrnehmen: ``Ich bin getriggert''
\item
  \textbf{ATMEN} -- 3x tief ein- und ausatmen
\item
  \textbf{GROUNDING} -- 5 Dinge sehen, 4 hören, 3 fühlen, 2 riechen, 1 schmecken
\item
  \textbf{SAFE-WORD} -- signalisieren, wenn andere anwesend
\item
  \textbf{SICHERHEIT} -- sicheren Ort aufsuchen oder schaffen
\item
  \textbf{RESSOURCE} -- Bewältigungsstrategie anwenden
\item
  \textbf{HILFE} -- Unterstützung holen, wenn nötig
\end{enumerate}

\hypertarget{navigation-und-verweise}{%
\subsection{Navigation und Verweise}}\label{navigation-und-verweise}

\begin{itemize}
\tightlist
\item
  ➡️ \textbf{Weiter zu:} Safe-Words \& Signalsysteme (Tool 22)
\item
  ⬅️ \textbf{Zurück zu:} Co-Regulation \& Gemeinsame Stärkung  
\item
  🔗 \textbf{Ergänzend:} Matching Matrix Trigger-Reaktion-Intervention
\item
  📖 \textbf{Vertiefung:} Arbeitsblatt Trigger-Forschungstagebuch
\end{itemize}

\textbf{🎯 Ziel:} Trigger als Informationsquelle verstehen und konstruktive Bewältigungsstrategien entwickeln, die zur Selbstregulation und Beziehungsstabilität beitragen.