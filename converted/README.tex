% Converted CTMM Therapy Document: README
% Generated by CTMM Document Conversion Pipeline
% Date: 2025-08-09 03:35:23

\section{README}
\label{sec:readme}

\hypertarget{ctmm-system}{%
\section{CTMM-System}\label{ctmm-system}}

Ein modulares LaTeX-Framework für Catch-Track-Map-Match Therapiematerialien.

\hypertarget{uxfcberblick}{%
\subsection{\textcolor{ctmmBlue}{\faBook~\1}}\label{uxfcberblick}}

Dieses Repository enthält ein vollständiges LaTeX-System zur Erstellung von CTMM-Therapiedokumenten, einschließlich: - Depression \& Stimmungstief Module - Trigger-Management - Bindungsdynamik - Formularelemente für therapeutische Dokumentation

\hypertarget{verwendung}{%
\subsection{\textcolor{ctmmBlue}{\faBook~\1}}\label{verwendung}}

\begin{enumerate}
\def\labelenumi{\arabic{enumi}.}
\tightlist
\item
  Klone das Repository
\item
  Kompiliere main.tex mit einem LaTeX-Editor
\item
  Oder öffne das Projekt in einem GitHub Codespace
\end{enumerate}

\hypertarget{struktur}{%
\subsection{\textcolor{ctmmBlue}{\faBook~\1}}\label{struktur}}

\begin{itemize}[label=\textcolor{ctmmOrange}{\faArrowRight}]
\tightlist
\item
  \texttt{/style/} - Design-Dateien und gemeinsam verwendete Komponenten
\item
  \texttt{/modules/} - Individuelle CTMM-Module als separate .tex-Dateien
\item
  \texttt{/assets/} - Diagramme und visuelle Elemente
\end{itemize}

\hypertarget{anforderungen}{%
\subsection{\textcolor{ctmmBlue}{\faBook~\1}}\label{anforderungen}}

\begin{itemize}[label=\textcolor{ctmmOrange}{\faArrowRight}]
\tightlist
\item
  LaTeX-Installation mit TikZ und hyperref
\item
  Oder GitHub Codespace (vorkonfiguriert)
\end{itemize}

\hypertarget{latex-hinweise-fuxfcr-entwickler}{%
\subsection{\textcolor{ctmmBlue}{\faBook~\1}}\label{latex-hinweise-fuxfcr-entwickler}}

\textbf{\textcolor{ctmmBlue}{\1}}

Das Projekt verfügt über ein automatisches Build-System (\texttt{ctmm\_build.py}), das folgende Funktionen bietet:

\hypertarget{automatisierte-build-pruxfcfung}{%
\subsubsection{\textcolor{ctmmGreen}{\faList~\1}}\label{automatisierte-build-pruxfcfung}}

\begin{Shaded}
\begin{Highlighting}\checkbox
\ExtensionTok{python3}\NormalTok{ ctmm\_build.py}
\end{Highlighting}
\end{Shaded}

Das Build-System: 1. \textbf{\textcolor{ctmmBlue}{\1}} nach allen \texttt{\textbackslash{}usepackage\{style/...\}} und \texttt{\textbackslash{}input\{modules/...\}} Befehlen 2. \textbf{\textcolor{ctmmBlue}{\1}} - erstellt minimale Templates für fehlende Dateien 3. \textbf{\textcolor{ctmmBlue}{\1}} - Build ohne Module zum Testen der Basis-Struktur 4. \textbf{\textcolor{ctmmBlue}{\1}} - mit allen Modulen 5. \textbf{\textcolor{ctmmBlue}{\1}} für neue Template-Dateien mit Hinweisen zur Vervollständigung

\hypertarget{modulare-test-strategie}{%
\subsubsection{\textcolor{ctmmGreen}{\faList~\1}}\label{modulare-test-strategie}}

\textbf{\textcolor{ctmmBlue}{\1}} - Jedes neue Modul wird automatisch erkannt und getestet - Fehlende Referenzen werden durch kommentierte Templates ersetzt (kein Dummy-Content) - Build bricht nicht mehr bei fehlenden Dateien ab - Templates enthalten sinnvolle Struktur mit \texttt{\textbackslash{}section} und Platzhaltern

\textbf{\textcolor{ctmmBlue}{\1}} Für granulare Modultests steht \texttt{build\_system.py} zur Verfügung:

\begin{Shaded}
\begin{Highlighting}\checkbox
\ExtensionTok{python3}\NormalTok{ build\_system.py }\AttributeTok{{-}{-}verbose}
\end{Highlighting}
\end{Shaded}

\begin{itemize}[label=\textcolor{ctmmOrange}{\faArrowRight}]
\tightlist
\item
  Testet Module schrittweise einzeln
\item
  Identifiziert problematische Module
\item
  Erstellt detaillierte Build-Reports
\item
  Protokolliert alle Operationen in \texttt{build\_system.log}
\end{itemize}

\hypertarget{github-workflow-integration}{%
\subsubsection{\textcolor{ctmmGreen}{\faList~\1}}\label{github-workflow-integration}}

Das GitHub Actions Workflow (\texttt{.github/workflows/latex-build.yml}) wurde korrigiert: - Referenziert nun korrekt \texttt{main.tex} (statt dem nicht existierenden \texttt{main\_final.tex}) - Lädt \texttt{main.pdf} als Artefakt hoch - Kann durch das Build-System bei Fehlern erweitert werden

\textbf{\textcolor{ctmmBlue}{\1}}

\begin{itemize}[label=\textcolor{ctmmOrange}{\faArrowRight}]
\tightlist
\item
  \textbf{\textcolor{ctmmBlue}{\1}}

  \begin{itemize}[label=\textcolor{ctmmOrange}{\faArrowRight}]
  \tightlist
  \item
    \texttt{\textbackslash{}usepackage\{...\}} darf nur in der Hauptdatei (z.B. \texttt{main.tex}) vor \texttt{\textbackslash{}begin\{document\}} stehen, niemals in Modulen oder nach \texttt{\textbackslash{}begin\{document\}}.
  \end{itemize}
\item
  \textbf{\textcolor{ctmmBlue}{\1}}

  \begin{itemize}[label=\textcolor{ctmmOrange}{\faArrowRight}]
  \item
    Definiere neue Makros (z.B. Checkboxen, Textfelder) zentral in der Präambel oder in einem Style-File, nicht in einzelnen Modulen.
  \item
    Beispiel für Checkboxen:

\begin{Shaded}
\begin{Highlighting}\checkbox
\CommentTok{\% In der Präambel:}
\BuiltInTok{\textbackslash{}usepackage}\NormalTok{\{}\ExtensionTok{amssymb}\NormalTok{\}}
\FunctionTok{\textbackslash{}newcommand}\NormalTok{\{}\ExtensionTok{\textbackslash{}checkbox}\NormalTok{\}\{}\SpecialStringTok{$}\SpecialCharTok{\textbackslash{}square}\SpecialStringTok{$}\NormalTok{\}}
\FunctionTok{\textbackslash{}newcommand}\NormalTok{\{}\ExtensionTok{\textbackslash{}checkedbox}\NormalTok{\}\{}\SpecialStringTok{$}\SpecialCharTok{\textbackslash{}blacksquare}\SpecialStringTok{$}\NormalTok{\}}
\end{Highlighting}
\end{Shaded}
  \item
    \textbf{\textcolor{ctmmBlue}{\1}} Verwende in Modulen und Tabellen ausschließlich die Makros \texttt{\textbackslash{}checkbox} und \texttt{\textbackslash{}checkedbox} für Checkboxen. Benutze niemals direkt \texttt{\textbackslash{}Box} oder \texttt{\textbackslash{}blacksquare}, da dies zu \texttt{Undefined\ control\ sequence}-Fehlern führen kann.
  \item
    Falls du einen solchen Fehler siehst, prüfe, ob irgendwo noch \texttt{\textbackslash{}Box} oder ähnliche Symbole direkt verwendet werden, und ersetze sie durch die Makros.
  \end{itemize}
\item
  \textbf{\textcolor{ctmmBlue}{\1}}

  \begin{itemize}[label=\textcolor{ctmmOrange}{\faArrowRight}]
  \tightlist
  \item
    Module sollten keine Pakete laden oder globale Makros definieren.
  \item
    Nur Inhalte und Befehle verwenden, die in der Präambel bereitgestellt werden.
  \end{itemize}
\item
  \textbf{\textcolor{ctmmBlue}{\1}}

  \begin{itemize}[label=\textcolor{ctmmOrange}{\faArrowRight}]
  \tightlist
  \item
    \texttt{Can\ be\ used\ only\ in\ preamble}: Ein Paket wurde im Fließtext geladen -- in die Präambel verschieben!
  \item
    \texttt{Undefined\ control\ sequence}: Ein Makro ist nicht definiert -- Definition prüfen oder in die Präambel verschieben.
  \item
    \texttt{Command\ ...\ already\ defined}: Ein Makro wurde doppelt definiert -- nur eine Definition behalten (am besten zentral).
  \end{itemize}
\end{itemize}

\hypertarget{vorgehen-bei-neuen-modulen}{%
\subsubsection{\textcolor{ctmmGreen}{\faList~\1}}\label{vorgehen-bei-neuen-modulen}}

\begin{enumerate}
\def\labelenumi{\arabic{enumi}.}
\item
  \textbf{\textcolor{ctmmBlue}{\1}}

\begin{Shaded}
\begin{Highlighting}\checkbox
\FunctionTok{\textbackslash{}input}\NormalTok{\{modules/mein{-}neues{-}modul\}}
\end{Highlighting}
\end{Shaded}
\item
  \textbf{\textcolor{ctmmBlue}{\1}}

\begin{Shaded}
\begin{Highlighting}\checkbox
\ExtensionTok{python3}\NormalTok{ ctmm\_build.py}
\end{Highlighting}
\end{Shaded}
\item
  \textbf{\textcolor{ctmmBlue}{\1}}

  \begin{itemize}[label=\textcolor{ctmmOrange}{\faArrowRight}]
  \tightlist
  \item
    \texttt{modules/mein-neues-modul.tex} mit Grundstruktur
  \item
    \texttt{modules/TODO\_mein-neues-modul.md} mit Aufgabenliste
  \end{itemize}
\item
  \textbf{\textcolor{ctmmBlue}{\1}} und TODO-Datei entfernen wenn fertig
\end{enumerate}

\textbf{\textcolor{ctmmBlue}{\1}} - Hinweise zu neuen Makros, Paketen oder typischen Stolperfallen hier dokumentieren.

\textbf{\textcolor{ctmmBlue}{\1}} Wenn du ein neues Modul schreibst, prüfe, ob du neue Pakete oder Makros brauchst -- und ergänze sie zentral, nicht im Modul selbst.
