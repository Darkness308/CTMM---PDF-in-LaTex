\% README.tex - Converted from README.md
\% CTMM Therapy Material - Auto-generated by Conversion Pipeline

\section{Readme}
\label{sec:readme}

\section{\textcolor{ctmmBlue}{CTMM}-System}

Ein modulares LaTeX-Framework für Catch-Track-Map-Match Therapiematerialien.

\subsection{Überblick}
Dieses Repository enthält ein vollständiges LaTeX-System zur Erstellung von \textcolor{ctmmBlue}{CTMM}-Therapiedokumenten, einschließlich:
\begin{itemize}
\item Depression \& Stimmungstief Module
\item Trigger-Management
\item Bindungsdynamik
\item Formularelemente für therapeutische Dokumentation
\end{itemize}

\subsection{Verwendung}
\begin{enumerate}
\item Klone das Repository
\item Kompiliere main.tex mit einem LaTeX-Editor
\item Oder öffne das Projekt in einem GitHub Codespace
\end{enumerate}

\subsection{Struktur}
\begin{itemize}
\item `/style/` - Design-Dateien und gemeinsam verwendete Komponenten
\item `/modules/` - Individuelle CTMM-Module als separate .tex-Dateien
\item `/assets/` - Diagramme und visuelle Elemente
\end{itemize}

\subsection{Anforderungen}
\begin{itemize}
\item LaTeX-Installation mit TikZ und hyperref
\item Oder GitHub Codespace (vorkonfiguriert)
\end{itemize}

\subsection{LaTeX-Hinweise für Entwickler}

\textbf{\textcolor{ctmmBlue}{CTMM} Build System:}

Das Projekt verfügt über ein automatisches Build-System (`ctmm\\_build.py`), das folgende Funktionen bietet:

\subsubsection{Automatisierte Build-Prüfung}
```bash
python3 ctmm\\_build.py
```

Das Build-System:
\begin{enumerate}
\item \textbf{Scannt main.tex\textbf{ nach allen `\usepackage{style/...}` und `\input{modules/...}` Befehlen
\item \textbf{Prüft Dateiexistenz\textbf{ - erstellt minimale Templates für fehlende Dateien
\item \textbf{Testet Grundgerüst\textbf{ - Build ohne Module zum Testen der Basis-Struktur
\item \textbf{Testet vollständigen Build\textbf{ - mit allen Modulen
\item \textbf{Erstellt TODO-Dateien\textbf{ für neue Template-Dateien mit Hinweisen zur Vervollständigung
\end{enumerate}

\subsubsection{Modulare Test-Strategie}

\textbf{Für Entwickler:}
\begin{itemize}
\item Jedes neue Modul wird automatisch erkannt und getestet
\item Fehlende Referenzen werden durch kommentierte Templates ersetzt (kein Dummy-Content)
\item Build bricht nicht mehr bei fehlenden Dateien ab
\item Templates enthalten sinnvolle Struktur mit `\section` und Platzhaltern
\end{itemize}

\textbf{Erweiterte Analyse:}
Für granulare Modultests steht `build\\_system.py` zur Verfügung:
```bash
python3 build\\_system.py --verbose
```
\begin{itemize}
\item Testet Module schrittweise einzeln
\item Identifiziert problematische Module
\item Erstellt detaillierte Build-Reports
\item Protokolliert alle Operationen in `build\_system.log`
\end{itemize}

\subsubsection{GitHub Workflow Integration}

Das GitHub Actions Workflow (`.github/workflows/latex-build.yml`) wurde korrigiert:
\begin{itemize}
\item Referenziert nun korrekt `main.tex` (statt dem nicht existierenden `main\_final.tex`)
\item Lädt `main.pdf` als Artefakt hoch
\item Kann durch das Build-System bei Fehlern erweitert werden
\end{itemize}

\textbf{Typische Fehlerquellen und Best Practices:}

\begin{itemize}
\item \textbf{Pakete immer in der Präambel laden:\textbf{
\item `\usepackage{...}` darf nur in der Hauptdatei (z.B. `main.tex`) vor `\begin{document}` stehen, niemals in Modulen oder nach `\begin{document}`.
\item \textbf{Makros und Befehle:\textbf{
\item Definiere neue Makros (z.B. Checkboxen, Textfelder) zentral in der Präambel oder in einem Style-File, nicht in einzelnen Modulen.
\item Beispiel für Checkboxen:
\end{itemize}
```tex
\\% In der Präambel:
\usepackage{amssymb}
\newcommand{\checkbox}{\\$\square\\$}
\newcommand{\checkedbox}{\\$\blacksquare\\$}
```
\begin{itemize}
\item \textbf{Wichtig:\textbf{ Verwende in Modulen und Tabellen ausschließlich die Makros `\checkbox` und `\checkedbox` für Checkboxen. Benutze niemals direkt `\Box` oder `\blacksquare`, da dies zu `Undefined control sequence`-Fehlern führen kann.
\item Falls du einen solchen Fehler siehst, prüfe, ob irgendwo noch `\Box` oder ähnliche Symbole direkt verwendet werden, und ersetze sie durch die Makros.
\item \textbf{Module:\textbf{
\item Module sollten keine Pakete laden oder globale Makros definieren.
\item Nur Inhalte und Befehle verwenden, die in der Präambel bereitgestellt werden.
\item \textbf{Fehlermeldungen:\textbf{
\item `Can be used only in preamble`: Ein Paket wurde im Fließtext geladen -- in die Präambel verschieben!
\item `Undefined control sequence`: Ein Makro ist nicht definiert -- Definition prüfen oder in die Präambel verschieben.
\item `Command ... already defined`: Ein Makro wurde doppelt definiert -- nur eine Definition behalten (am besten zentral).
\end{itemize}

\subsubsection{Vorgehen bei neuen Modulen}

\begin{enumerate}
\item \textbf{Referenz in main.tex hinzufügen:\textbf{
\end{enumerate}
```tex
\input{modules/mein-neues-modul}
```

\begin{enumerate}
\item \textbf{Build-System ausführen:\textbf{
\end{enumerate}
```bash
python3 ctmm\\_build.py
```

\begin{enumerate}
\item \textbf{Template wird automatisch erstellt:\textbf{
\begin{itemize}
\item `modules/mein-neues-modul.tex` mit Grundstruktur
\item `modules/TODO\_mein-neues-modul.md` mit Aufgabenliste
\end{itemize}
\end{enumerate}

\begin{enumerate}
\item \textbf{Inhalt ergänzen\textbf{ und TODO-Datei entfernen wenn fertig
\end{enumerate}

\textbf{README regelmäßig pflegen:}
\begin{itemize}
\item Hinweise zu neuen Makros, Paketen oder typischen Stolperfallen hier dokumentieren.
\end{itemize}

\textbf{Tipp:}
Wenn du ein neues Modul schreibst, prüfe, ob du neue Pakete oder Makros brauchst -- und ergänze sie zentral, nicht im Modul selbst.

\% End of README.tex
