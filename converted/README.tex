# CTMM Project Documentation

This repository contains comprehensive LaTeX-based therapeutic materials for the CTMM (Catch-Track-Map-Match) system, designed for creating professional therapy documents for neurodiverse couples dealing with mental health challenges.

## Overview

The CTMM system provides structured therapeutic materials for:
- Depression and mood disorders
- Trigger management
- Borderline Personality Disorder (BPD)
- ADHD, Autism Spectrum Disorder (ASD)
- Complex PTSD (CPTSD)
- Relationship dynamics and binding patterns

## Build System

This project features a comprehensive LaTeX workflow system with:
- **Enhanced build system** with multi-pass compilation and error handling
- **Document conversion pipeline** for Word/Markdown to LaTeX conversion
- **Advanced error analysis** and code optimization recommendations
- **Automated PDF verification** and quality assurance

### Quick Start

```bash
# Check build system and dependencies
python3 ctmm_build.py

# Build PDF with multi-pass compilation
make build

# Run detailed analysis
python3 build_system.py --verbose
```

## Repository Structure

```
├── main.tex                    # Main LaTeX document
├── style/                      # LaTeX style files
│   ├── ctmm-design.sty        # CTMM design elements
│   ├── form-elements.sty      # Interactive forms
│   └── ctmm-diagrams.sty      # Custom diagrams
├── modules/                    # Therapy modules
├── converted/                  # Converted documents
├── ctmm_build.py              # Main build system
├── build_system.py            # Advanced analysis
├── conversion_pipeline.py     # Document conversion
└── .github/workflows/         # CI/CD automation
```

## Document Conversion

The conversion pipeline supports transforming therapy materials from various formats:

```bash
# Convert single Markdown file
python3 conversion_pipeline.py --source document.md

# Batch convert directory
python3 conversion_pipeline.py --source docs/ --batch

# Create therapy tool template
python3 conversion_pipeline.py --tool "Tool Name" --tool-number 24
```

## Development Guidelines

### Adding New Modules

1. Reference in `main.tex`: `\input{modules/new-module}`
2. Run build system: `python3 ctmm_build.py`
3. Complete auto-generated template
4. Test: `make check`

### LaTeX Best Practices

- All `\usepackage` commands in preamble only
- Use predefined macros: `\checkbox`, `\checkedbox`
- CTMM colors: `ctmmBlue`, `ctmmOrange`, `ctmmGreen`, `ctmmPurple`
- German language content with proper encoding

## Quality Assurance

The build system provides comprehensive quality checks:
- **Multi-pass compilation** for cross-references
- **Error analysis** with actionable recommendations
- **PDF verification** and file integrity checks
- **Module isolation testing** to identify issues
- **Code optimization suggestions**

## Therapy Content

### Content Types
- **Arbeitsblätter** (Worksheets): Interactive self-reflection forms
- **Trigger Management**: Coping strategies and identification tools
- **Psychoeducation**: Mental health condition information
- **Relationship Tools**: Communication and binding pattern resources

### Sensitivity Notice
This repository contains mental health resources. Contributors should:
- Respect privacy (no personal information)
- Ensure clinical accuracy
- Maintain therapeutic, non-judgmental language
- Follow German therapeutic language conventions

## Technical Requirements

### Dependencies
- **LaTeX**: TeX Live with German language support
- **Python**: 3.6+ with chardet package
- **Optional**: Pandoc for Word document conversion

### Installation
```bash
# Ubuntu/Debian
sudo apt-get install texlive-latex-extra texlive-lang-german
pip install chardet

# Codespace (pre-configured)
make deps
```

## Contributing

1. Test builds before submitting: `make check`
2. Verify PDF output renders correctly
3. Use descriptive commit messages
4. Update documentation for new features
5. Maintain sensitivity to mental health contexts

## License

See LICENSE file for licensing information.

## Support

For issues with the build system or document conversion:
1. Check `build_report.md` for analysis
2. Review log files for detailed errors
3. Run verbose analysis: `python3 build_system.py --verbose`

---

**Note**: This is specialized therapeutic content requiring both LaTeX expertise and sensitivity to mental health contexts.