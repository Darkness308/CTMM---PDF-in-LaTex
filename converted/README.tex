\hypertarget{ctmm-system}{%
\section{CTMM-System}\label{ctmm-system}}

Ein modulares LaTeX-Framework für Catch-Track-Map-Match Therapiematerialien.

\hypertarget{ueberblick}{%
\subsection{Überblick}\label{ueberblick}}

Dieses Repository enthält ein vollständiges LaTeX-System zur Erstellung von CTMM-Therapiedokumenten, einschließlich:
- Depression \\& Stimmungstief Module
- Trigger-Management
- Bindungsdynamik
- Formularelemente für therapeutische Dokumentation

\hypertarget{verwendung}{%
\subsection{Verwendung}\label{verwendung}}

\begin{enumerate}
\def\labelenumi{\arabic{enumi}.}
\tightlist
\item
  Klone das Repository
\item
  Kompiliere main.tex mit einem LaTeX-Editor
\item
  Oder öffne das Projekt in einem GitHub Codespace
\end{enumerate}

\hypertarget{struktur}{%
\subsection{Struktur}\label{struktur}}

\begin{itemize}
\tightlist
\item
  \texttt{/style/} - Design-Dateien und gemeinsam verwendete Komponenten
\item
  \texttt{/modules/} - Individuelle CTMM-Module als separate .tex-Dateien
\item
  \texttt{/assets/} - Diagramme und visuelle Elemente
\end{itemize}

\hypertarget{anforderungen}{%
\subsection{Anforderungen}\label{anforderungen}}

\textbackslash{}begin\textbackslash{}{itemize\textbackslash{}}
\textbackslash{}tightlist
\textbackslash{}item
  LaTeX-Installation mit TikZ und hyperref
\textbackslash{}item
  Oder GitHub Codespace (vorkonfiguriert)
\textbackslash{}end\textbackslash{}{itemize\textbackslash{}}

\textbackslash{}hypertarget\textbackslash{}{latex-hinweise-fuxfcr-entwickler\textbackslash{}}\textbackslash{}{\textbackslash{}%
\textbackslash{}subsection\textbackslash{}{LaTeX-Hinweise für Entwickler\textbackslash{}}\textbackslash{}label\textbackslash{}{latex-hinweise-fuxfcr-entwickler\textbackslash{}}\textbackslash{}}

\textbackslash{}textbf\textbackslash{}{Typische Fehlerquellen und Best Practices:\textbackslash{}}

\textbackslash{}begin\textbackslash{}{itemize\textbackslash{}}
\textbackslash{}tightlist
\textbackslash{}item
  \textbackslash{}textbf\textbackslash{}{Pakete immer in der Präambel laden:\textbackslash{}}

  \textbackslash{}begin\textbackslash{}{itemize\textbackslash{}}
  \textbackslash{}tightlist
  \textbackslash{}item
    \textbackslash{}texttt\textbackslash{}{\textbackslash{}textbackslash\textbackslash{}{\textbackslash{}}usepackage\textbackslash{}\textbackslash{}{...\textbackslash{}\textbackslash{}}\textbackslash{}} darf nur in der Hauptdatei (z.B. \textbackslash{}texttt\textbackslash{}{main.tex\textbackslash{}}) vor \textbackslash{}texttt\textbackslash{}{\textbackslash{}textbackslash\textbackslash{}{\textbackslash{}}begin\textbackslash{}\textbackslash{}{document\textbackslash{}\textbackslash{}}\textbackslash{}} stehen, niemals in Modulen oder nach \textbackslash{}texttt\textbackslash{}{\textbackslash{}textbackslash\textbackslash{}{\textbackslash{}}begin\textbackslash{}\textbackslash{}{document\textbackslash{}\textbackslash{}}\textbackslash{}}.
  \textbackslash{}end\textbackslash{}{itemize\textbackslash{}}
\textbackslash{}item
  \textbackslash{}textbf\textbackslash{}{Makros und Befehle:\textbackslash{}}

  \textbackslash{}begin\textbackslash{}{itemize\textbackslash{}}
  \textbackslash{}item
    Definiere neue Makros (z.B. Checkboxen, Textfelder) zentral in der Präambel oder in einem Style-File, nicht in einzelnen Modulen.
  \textbackslash{}item
    Beispiel für Checkboxen:

\textbackslash{}begin\textbackslash{}{Shaded\textbackslash{}}
\textbackslash{}begin\textbackslash{}{Highlighting\textbackslash{}}[]
\textbackslash{}CommentTok\textbackslash{}{\textbackslash{}\textbackslash{}% In der Präambel:\textbackslash{}}
\textbackslash{}BuiltInTok\textbackslash{}{\textbackslash{}textbackslash\textbackslash{}{\textbackslash{}}usepackage\textbackslash{}}\textbackslash{}NormalTok\textbackslash{}{\textbackslash{}\textbackslash{}{\textbackslash{}}\textbackslash{}ExtensionTok\textbackslash{}{amssymb\textbackslash{}}\textbackslash{}NormalTok\textbackslash{}{\textbackslash{}\textbackslash{}}\textbackslash{}}
\textbackslash{}FunctionTok\textbackslash{}{\textbackslash{}textbackslash\textbackslash{}{\textbackslash{}}newcommand\textbackslash{}}\textbackslash{}NormalTok\textbackslash{}{\textbackslash{}\textbackslash{}{\textbackslash{}}\textbackslash{}ExtensionTok\textbackslash{}{\textbackslash{}textbackslash\textbackslash{}{\textbackslash{}}checkbox\textbackslash{}}\textbackslash{}NormalTok\textbackslash{}{\textbackslash{}\textbackslash{}}\textbackslash{}\textbackslash{}{\textbackslash{}}\textbackslash{}SpecialStringTok\textbackslash{}{\textbackslash{}$\textbackslash{}}\textbackslash{}SpecialCharTok\textbackslash{}{\textbackslash{}textbackslash\textbackslash{}{\textbackslash{}}square\textbackslash{}}\textbackslash{}SpecialStringTok\textbackslash{}{\textbackslash{}$\textbackslash{}}\textbackslash{}NormalTok\textbackslash{}{\textbackslash{}\textbackslash{}}\textbackslash{}}
\textbackslash{}FunctionTok\textbackslash{}{\textbackslash{}textbackslash\textbackslash{}{\textbackslash{}}newcommand\textbackslash{}}\textbackslash{}NormalTok\textbackslash{}{\textbackslash{}\textbackslash{}{\textbackslash{}}\textbackslash{}ExtensionTok\textbackslash{}{\textbackslash{}textbackslash\textbackslash{}{\textbackslash{}}checkedbox\textbackslash{}}\textbackslash{}NormalTok\textbackslash{}{\textbackslash{}\textbackslash{}}\textbackslash{}\textbackslash{}{\textbackslash{}}\textbackslash{}SpecialStringTok\textbackslash{}{\textbackslash{}$\textbackslash{}}\textbackslash{}SpecialCharTok\textbackslash{}{\textbackslash{}textbackslash\textbackslash{}{\textbackslash{}}blacksquare\textbackslash{}}\textbackslash{}SpecialStringTok\textbackslash{}{\textbackslash{}$\textbackslash{}}\textbackslash{}NormalTok\textbackslash{}{\textbackslash{}\textbackslash{}}\textbackslash{}}
\textbackslash{}end\textbackslash{}{Highlighting\textbackslash{}}
\textbackslash{}end\textbackslash{}{Shaded\textbackslash{}}
  \textbackslash{}item
    \textbackslash{}textbf\textbackslash{}{Wichtig:\textbackslash{}} Verwende in Modulen und Tabellen ausschließlich die Makros \textbackslash{}texttt\textbackslash{}{\textbackslash{}textbackslash\textbackslash{}{\textbackslash{}}checkbox\textbackslash{}} und \textbackslash{}texttt\textbackslash{}{\textbackslash{}textbackslash\textbackslash{}{\textbackslash{}}checkedbox\textbackslash{}} für Checkboxen. Benutze niemals direkt \textbackslash{}texttt\textbackslash{}{\textbackslash{}textbackslash\textbackslash{}{\textbackslash{}}Box\textbackslash{}} oder \textbackslash{}texttt\textbackslash{}{\textbackslash{}textbackslash\textbackslash{}{\textbackslash{}}blacksquare\textbackslash{}}, da dies zu \textbackslash{}texttt\textbackslash{}{Undefined\textbackslash{} control\textbackslash{} sequence\textbackslash{}}-Fehlern führen kann.
  \textbackslash{}item
    Falls du einen solchen Fehler siehst, prüfe, ob irgendwo noch \textbackslash{}texttt\textbackslash{}{\textbackslash{}textbackslash\textbackslash{}{\textbackslash{}}Box\textbackslash{}} oder ähnliche Symbole direkt verwendet werden, und ersetze sie durch die Makros.
  \textbackslash{}end\textbackslash{}{itemize\textbackslash{}}
\textbackslash{}item
  \textbackslash{}textbf\textbackslash{}{Module:\textbackslash{}}

  \textbackslash{}begin\textbackslash{}{itemize\textbackslash{}}
  \textbackslash{}tightlist
  \textbackslash{}item
    Module sollten keine Pakete laden oder globale Makros definieren.
  \textbackslash{}item
    Nur Inhalte und Befehle verwenden, die in der Präambel bereitgestellt werden.
  \textbackslash{}end\textbackslash{}{itemize\textbackslash{}}
\textbackslash{}item
  \textbackslash{}textbf\textbackslash{}{Fehlermeldungen:\textbackslash{}}

  \textbackslash{}begin\textbackslash{}{itemize\textbackslash{}}
  \textbackslash{}tightlist
  \textbackslash{}item
    \textbackslash{}texttt\textbackslash{}{Can\textbackslash{} be\textbackslash{} used\textbackslash{} only\textbackslash{} in\textbackslash{} preamble\textbackslash{}}: Ein Paket wurde im Fließtext geladen -- in die Präambel verschieben!
  \textbackslash{}item
    \textbackslash{}texttt\textbackslash{}{Undefined\textbackslash{} control\textbackslash{} sequence\textbackslash{}}: Ein Makro ist nicht definiert -- Definition prüfen oder in die Präambel verschieben.
  \textbackslash{}item
    \textbackslash{}texttt\textbackslash{}{Command\textbackslash{} ...\textbackslash{} already\textbackslash{} defined\textbackslash{}}: Ein Makro wurde doppelt definiert -- nur eine Definition behalten (am besten zentral).
  \textbackslash{}end\textbackslash{}{itemize\textbackslash{}}
\textbackslash{}item
  \textbackslash{}textbf\textbackslash{}{README regelmäßig pflegen:\textbackslash{}}

  \textbackslash{}begin\textbackslash{}{itemize\textbackslash{}}
  \textbackslash{}tightlist
  \textbackslash{}item
    Hinweise zu neuen Makros, Paketen oder typischen Stolperfallen hier dokumentieren.
  \textbackslash{}end\textbackslash{}{itemize\textbackslash{}}
\textbackslash{}end\textbackslash{}{itemize\textbackslash{}}

\textbackslash{}textbf\textbackslash{}{Tipp:\textbackslash{}}
Wenn du ein neues Modul schreibst, prüfe, ob du neue Pakete oder Makros brauchst -- und ergänze sie zentral, nicht im Modul selbst.
