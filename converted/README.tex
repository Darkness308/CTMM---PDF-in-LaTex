\hypertarget{ctmm-system}{\%
\section{CTMM-System}\label{ctmm-system}}

Ein modulares LaTeX-Framework für Catch-Track-Map-Match Therapiematerialien.

\hypertarget{uxfcberblick}{\%
\subsection{Überblick}\label{uxfcberblick}}

Dieses Repository enthält ein vollständiges LaTeX-System zur Erstellung von CTMM-Therapiedokumenten, einschließlich:
- Depression \\& Stimmungstief Module
- Trigger-Management
- Bindungsdynamik
- Formularelemente für therapeutische Dokumentation

\hypertarget{verwendung}{\%
\subsection{Verwendung}\label{verwendung}}

\begin{enumerate}
\def\labelenumi{\arabic{enumi}.}
\tightlist
\item
  Klone das Repository
\item
  Kompiliere main.tex mit einem LaTeX-Editor
\item
  Oder öffne das Projekt in einem GitHub Codespace
\end{enumerate}

\hypertarget{struktur}{\%
\subsection{Struktur}\label{struktur}}

\begin{itemize}
\tightlist
\item
  \texttt{/style/} - Design-Dateien und gemeinsam verwendete Komponenten
\item
  \texttt{/modules/} - Individuelle CTMM-Module als separate .tex-Dateien
\item
  \texttt{/assets/} - Diagramme und visuelle Elemente
\end{itemize}

\hypertarget{anforderungen}{\%
\subsection{Anforderungen}\label{anforderungen}}

\begin{itemize}
\tightlist
\item
  LaTeX-Installation mit TikZ und hyperref
\item
  Oder GitHub Codespace (vorkonfiguriert)
\end{itemize}

\hypertarget{latex-hinweise-fuxfcr-entwickler}{\%
\subsection{LaTeX-Hinweise für Entwickler}\label{latex-hinweise-fuxfcr-entwickler}}

\textbf{Typische Fehlerquellen und Best Practices:}

\begin{itemize}
\tightlist
\item
  \textbf{Pakete immer in der Präambel laden:}

  \begin{itemize}
  \tightlist
  \item
    \texttt{\textbackslash{}usepackage{...}} darf nur in der Hauptdatei (z.B. \texttt{main.tex}) vor \texttt{\textbackslash{}begin{document}} stehen, niemals in Modulen oder nach \texttt{\textbackslash{}begin{document}}.
  \end{itemize}
\item
  \textbf{Makros und Befehle:}

  \begin{itemize}
  \item
    Definiere neue Makros (z.B. Checkboxen, Textfelder) zentral in der Präambel oder in einem Style-File, nicht in einzelnen Modulen.
  \item
    Beispiel für Checkboxen:

\begin{Shaded}
\begin{Highlighting}[]
\CommentTok{\\% In der Präambel:}
\BuiltInTok{\textbackslash{}usepackage}\NormalTok{{}\ExtensionTok{amssymb}\NormalTok{}}
\FunctionTok{\textbackslash{}newcommand}\NormalTok{{}\ExtensionTok{\textbackslash{}checkbox}\NormalTok{}{}\SpecialStringTok{\$}\SpecialCharTok{\textbackslash{}square}\SpecialStringTok{\$}\NormalTok{}}
\FunctionTok{\textbackslash{}newcommand}\NormalTok{{}\ExtensionTok{\textbackslash{}checkedbox}\NormalTok{}{}\SpecialStringTok{\$}\SpecialCharTok{\textbackslash{}blacksquare}\SpecialStringTok{\$}\NormalTok{}}
\end{Highlighting}
\end{Shaded}
  \item
    \textbf{Wichtig:} Verwende in Modulen und Tabellen ausschließlich die Makros \texttt{\textbackslash{}checkbox} und \texttt{\textbackslash{}checkedbox} für Checkboxen. Benutze niemals direkt \texttt{\textbackslash{}Box} oder \texttt{\textbackslash{}blacksquare}, da dies zu \texttt{Undefined\ control\ sequence}-Fehlern führen kann.
  \item
    Falls du einen solchen Fehler siehst, prüfe, ob irgendwo noch \texttt{\textbackslash{}Box} oder ähnliche Symbole direkt verwendet werden, und ersetze sie durch die Makros.
  \end{itemize}
\item
  \textbf{Module:}

  \begin{itemize}
  \tightlist
  \item
    Module sollten keine Pakete laden oder globale Makros definieren.
  \item
    Nur Inhalte und Befehle verwenden, die in der Präambel bereitgestellt werden.
  \end{itemize}
\item
  \textbf{Fehlermeldungen:}

  \begin{itemize}
  \tightlist
  \item
    \texttt{Can\ be\ used\ only\ in\ preamble}: Ein Paket wurde im Fließtext geladen -- in die Präambel verschieben!
  \item
    \texttt{Undefined\ control\ sequence}: Ein Makro ist nicht definiert -- Definition prüfen oder in die Präambel verschieben.
  \item
    \texttt{Command\ ...\ already\ defined}: Ein Makro wurde doppelt definiert -- nur eine Definition behalten (am besten zentral).
  \end{itemize}
\item
  \textbf{README regelmäßig pflegen:}

  \begin{itemize}
  \tightlist
  \item
    Hinweise zu neuen Makros, Paketen oder typischen Stolperfallen hier dokumentieren.
  \end{itemize}
\end{itemize}

\textbf{Tipp:}
Wenn du ein neues Modul schreibst, prüfe, ob du neue Pakete oder Makros brauchst -- und ergänze sie zentral, nicht im Modul selbst.
