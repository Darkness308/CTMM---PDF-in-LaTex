\section{\textbf{🧩 MATCHING MATRIX: TRIGGER -- REAKTION -- INTERVENTION (CTMM)}
\label{matching-matrix-trigger-reaktion-intervention-ctmm}

\begin{quote}
🧠 \textbf{\ul{Worum geht's hier -- für Freunde?}\\
Dieses Modul hilft, typische Reiz-Reaktionsmuster in unserer Beziehung zu verstehen. Es ist wie ein Übersetzungsblatt -- was passiert in mir, in dir, und wie können wir hilfreich reagieren?
\end{quote}

🧩 \textbf{Dynamisches Tool zur Reflexion und Alltagssteuerung -- ergänzt das Trigger-Tagebuch \& die Ko-Regulation}

\subsection{Kapitelzuordnung im CTMM-System}\label{kapitelzuordnung-im-ctmm-system-matrix}

\begin{itemize}
\tightlist
\item
  \texttt{Kap.\ 1} → Grundlogik der Bindungsdynamik (Auslöser, Reaktion, Integration)
\item
  \texttt{Kap.\ 2.6} → Team-Regeln, Ko-Regulation, Nähe/Distanz
\item
  \texttt{Kap.\ 3.1\ –\ 3.5} → Eskalationssicherung, Rückzug, Intervention
\item
  \texttt{Kap.\ 5.2} → Trigger-Tagebuch, Matching-Auswertung, Reaktionslogik
\end{itemize}

\subsection{Farbcode \& Systemnavigation}\label{farbcode-systemnavigation}

\begin{center}
\begin{tabular}{|c|l|l|}
\hline
\textbf{Farbe} & \textbf{Phase} & \textbf{Modulkontext} \\
\hline
🔵 & Mustererkennung & \texttt{bindungsdynamik}, \texttt{Kap.\ 1} \\
\hline
🟠 & Reaktion \& Intervention & \texttt{tool\_co\_regulation} \\
\hline
🔴 & Eskalation / Rückzug & \texttt{trigger\_notfallkarten} \\
\hline
\end{tabular}
\end{center}

\subsection{Matching-Tabelle: Typische Szenarien}\label{matching-tabelle-typische-szenarien}

\begin{center}
\begin{tabular}{|p{3.5cm}|p{3cm}|p{3cm}|p{4.5cm}|}
\hline
\textbf{Trigger (Beispiel)} & \textbf{Reaktion ER} & \textbf{Reaktion SIE} & \textbf{Was hilft konkret (Aktion/Tool)} \\
\hline
\textbf{Vorwurf / Kritik} & Rückzug, Erstarren & Lautwerden, Rechtfertigung & 🔁 Ritualsatz, Ko-Regulation, Safe-Word \\
\hline
\textbf{Näheforderung ohne Warnung} & Reizüberflutung, Freeze & Impulsives Klammern & 🟠 Ankündigung, Rückzugsregel, Timer \\
\hline
\textbf{Unvorhersehbare Änderung} & Panik, Rigidität & Überforderung, Chaos & 🔵 Struktur, Vorankündigung, Plan B \\
\hline
\textbf{Überstimulation (Lärm)} & Shutdown, Wegducken & Stress, Ungeduld & 🔴 Rückzug, Noise-Cancelling, Pause \\
\hline
\textbf{Missverständnis} & Schweigen, Rückzug & Nachfragen, Klärung & 💬 Metakommunikation, Check-in \\
\hline
\textbf{Emotionale Überlastung} & Dissoziation & Hilflosigkeit & 🏠 Grounding, Safe Space, Verständnis \\
\hline
\end{tabular}
\end{center}

\subsection{Individuelle Trigger-Reaktions Matrix}\label{individuelle-trigger-reaktions-matrix}

\textbf{Persönliche Muster dokumentieren und verstehen:}

\begin{center}
\begin{tabular}{|p{3.5cm}|p{3cm}|p{3cm}|p{4.5cm}|}
\hline
\textbf{Unser spezifischer Trigger} & \textbf{ER: Typische Reaktion} & \textbf{SIE: Typische Reaktion} & \textbf{Hilfreiche Intervention} \\
\hline
\rule{3cm}{0.4pt} & \rule{2.5cm}{0.4pt} & \rule{2.5cm}{0.4pt} & \rule{4cm}{0.4pt} \\
\hline
\rule{3cm}{0.4pt} & \rule{2.5cm}{0.4pt} & \rule{2.5cm}{0.4pt} & \rule{4cm}{0.4pt} \\
\hline
\rule{3cm}{0.4pt} & \rule{2.5cm}{0.4pt} & \rule{2.5cm}{0.4pt} & \rule{4cm}{0.4pt} \\
\hline
\rule{3cm}{0.4pt} & \rule{2.5cm}{0.4pt} & \rule{2.5cm}{0.4pt} & \rule{4cm}{0.4pt} \\
\hline
\rule{3cm}{0.4pt} & \rule{2.5cm}{0.4pt} & \rule{2.5cm}{0.4pt} & \rule{4cm}{0.4pt} \\
\hline
\rule{3cm}{0.4pt} & \rule{2.5cm}{0.4pt} & \rule{2.5cm}{0.4pt} & \rule{4cm}{0.4pt} \\
\hline
\end{tabular}
\end{center}

\subsection{Eskalations-/Deeskalations-Matrix}\label{eskalations-deeskalations-matrix}

\textbf{Frühwarnzeichen erkennen und gegensteuern:}

\begin{center}
\begin{tabular}{|p{2.5cm}|p{3cm}|p{3cm}|p{3cm}|p{2.5cm}|}
\hline
\textbf{Eskalations-stufe} & \textbf{ER: Anzeichen} & \textbf{SIE: Anzeichen} & \textbf{Deeskalation} & \textbf{Zeitfenster} \\
\hline
🟢 \textbf{Früh} & Anspannung, weniger Worte & Schnelleres Sprechen & Aufmerksamkeit, Pause & 5-10 Min \\
\hline
🟡 \textbf{Mittel} & Rückzug beginnt & Lautstärke steigt & Safe-Word, Stopp-Regel & 2-5 Min \\
\hline
🟠 \textbf{Hoch} & Shutdown, Erstarrung & Emotionale Intensität & Räumliche Trennung & 1-2 Min \\
\hline
🔴 \textbf{Kritisch} & Dissoziation & Überwältigung & Notfallprotokoll & Sofort \\
\hline
\end{tabular}
\end{center}

\subsection{Interventions-Toolbox}\label{interventions-toolbox}

\textbf{Konkrete Handlungsoptionen nach Situationstyp:}

\begin{center}
\begin{tabular}{|p{3cm}|p{4cm}|p{4cm}|p{3cm}|}
\hline
\textbf{Situationstyp} & \textbf{Sofortmaßnahmen} & \textbf{Mittelfristige Strategien} & \textbf{Erfolgs-Check} \\
\hline
\textbf{Kommunikations-trigger} & Metakommunikation, ``Ich-Aussagen'' & Kommunikationsregeln erarbeiten & \rule{2.5cm}{0.4pt} \\
\hline
\textbf{Sensorische Überlastung} & Reizreduktion, Safe Space & Umgebung anpassen & \rule{2.5cm}{0.4pt} \\
\hline
\textbf{Nähe/Distanz Konflikt} & Bedürfnisse aussprechen & Rhythmus finden & \rule{2.5cm}{0.4pt} \\
\hline
\textbf{Unvorhersehbare Situationen} & Struktur schaffen, Plan B & Flexibilität trainieren & \rule{2.5cm}{0.4pt} \\
\hline
\textbf{Emotionale Dysregulation} & Grounding, Beruhigung & Regulationsstrategien & \rule{2.5cm}{0.4pt} \\
\hline
\end{tabular}
\end{center}

\subsection{Erfolgs-Tracking Matrix}\label{erfolgs-tracking-matrix}

\textbf{Fortschritte dokumentieren und Muster verstehen:}

\begin{center}
\begin{tabular}{|p{2cm}|p{3cm}|p{2cm}|p{3cm}|p{3cm}|}
\hline
\textbf{Datum} & \textbf{Trigger-Situation} & \textbf{Reaktion (1-10)} & \textbf{Angewandte Intervention} & \textbf{Ergebnis \& Lernen} \\
\hline
\rule{1.5cm}{0.4pt} & \rule{2.5cm}{0.4pt} & \rule{1.5cm}{0.4pt} & \rule{2.5cm}{0.4pt} & \rule{2.5cm}{0.4pt} \\
\hline
\rule{1.5cm}{0.4pt} & \rule{2.5cm}{0.4pt} & \rule{1.5cm}{0.4pt} & \rule{2.5cm}{0.4pt} & \rule{2.5cm}{0.4pt} \\
\hline
\rule{1.5cm}{0.4pt} & \rule{2.5cm}{0.4pt} & \rule{1.5cm}{0.4pt} & \rule{2.5cm}{0.4pt} & \rule{2.5cm}{0.4pt} \\
\hline
\rule{1.5cm}{0.4pt} & \rule{2.5cm}{0.4pt} & \rule{1.5cm}{0.4pt} & \rule{2.5cm}{0.4pt} & \rule{2.5cm}{0.4pt} \\
\hline
\rule{1.5cm}{0.4pt} & \rule{2.5cm}{0.4pt} & \rule{1.5cm}{0.4pt} & \rule{2.5cm}{0.4pt} & \rule{2.5cm}{0.4pt} \\
\hline
\end{tabular}
\end{center}

\subsection{Team-Reflexion und Anpassung}\label{team-reflexion-und-anpassung}

\textbf{Gemeinsame Auswertung (wöchentlich/monatlich):}

\begin{itemize}
\tightlist
\item
  \textbf{Welche Muster erkennen wir?} \rule{8cm}{0.4pt}
\item
  \textbf{Was funktioniert gut?} \rule{8cm}{0.4pt}
\item
  \textbf{Wo brauchen wir andere Strategien?} \rule{8cm}{0.4pt}
\item
  \textbf{Welche neuen Trigger sind aufgekommen?} \rule{8cm}{0.4pt}
\item
  \textbf{Wie können wir uns besser unterstützen?} \rule{8cm}{0.4pt}
\end{itemize}

\subsection{Notfall-Interventions-Protokoll}\label{notfall-interventions-protokoll}

\textbf{Bei akuten Krisen -- schnell verfügbare Strategien:}

\begin{center}
\begin{tabular}{|p{2cm}|p{4cm}|p{4cm}|p{3cm}|}
\hline
\textbf{Krisengrad} & \textbf{Erkennungszeichen} & \textbf{Sofort-Intervention} & \textbf{Follow-up} \\
\hline
🟡 \textbf{Leicht} & Stress, Unruhe & Atempause, Safe-Word & Check-in nach 30 Min \\
\hline
🟠 \textbf{Mittel} & Starke Anspannung & Räumliche Trennung & Gespräch nach 1-2h \\
\hline
🔴 \textbf{Schwer} & Dissoziation, Panik & Notfallkarten, Hilfe & Professionelle Unterstützung \\
\hline
\end{tabular}
\end{center}

\subsection{Anpassungs- und Lernzyklus}\label{anpassungs-und-lernzyklus}

\textbf{Kontinuierliche Verbesserung der Matching-Matrix:}

\begin{enumerate}
\def\labelenumi{\arabic{enumi}.}
\tightlist
\item
  \textbf{BEOBACHTEN} -- Trigger und Reaktionen dokumentieren
\item
  \textbf{VERSTEHEN} -- Muster und Zusammenhänge erkennen
\item
  \textbf{PLANEN} -- Neue Interventionsstrategien entwickeln
\item
  \textbf{AUSPROBIEREN} -- Strategien in realen Situationen testen
\item
  \textbf{BEWERTEN} -- Wirksamkeit und Anpassungsbedarf prüfen
\item
  \textbf{ANPASSEN} -- Matrix mit neuen Erkenntnissen aktualisieren
\end{enumerate}

\subsection{Navigation und Verweise}\label{navigation-und-verweise-matrix}

\begin{itemize}
\tightlist
\item
  ➡️ \textbf{Weiter zu:} Notfallkarten und Krisenprotokoll
\item
  ⬅️ \textbf{Zurück zu:} Matching Matrix Wochenlogik
\item
  🔗 \textbf{Ergänzend:} Tool 23 Trigger Management
\item
  📖 \textbf{Vertiefung:} Co-Regulation \& Gemeinsame Stärkung
\end{itemize}

\textbf{🎯 Ziel:} Trigger-Reaktions-Muster systematisch verstehen und durch passende Interventionen eine stabile, unterstützende Beziehungsdynamik entwickeln.