\% Matching Matrix Trigger Reaktion Intervention CTMM.tex - Converted from Matching Matrix Trigger Reaktion Intervention CTMM.docx
\% CTMM Therapy Material - Auto-generated by Conversion Pipeline

\section{Matching Matrix Trigger Reaktion Intervention Ctmm}
\label{sec:matching-matrix-trigger-reaktion-intervention-ctmm}

\section{\textcolor{ctmmBlue}{\faIcon{puzzle-piece}} \textbf{MATCHING-MATRIX}}
\subsection{\textbf{TRIGGER -- REAKTION -- INTERVENTION (\textcolor{ctmmBlue}{CTMM}-MODUL)}}

\begin{quote}
\textcolor{ctmmPurple}{\faIcon{brain}} \textbf{Worum geht's hier -- für Freunde?}\
\end{quote}
\begin{quote}
Dieses Modul hilft, typische Reiz-Reaktionsmuster in unserer Beziehung
\end{quote}
\begin{quote}
zu verstehen.\
\end{quote}
\begin{quote}
Es ist wie ein Übersetzungsblatt -- was passiert in mir, in dir, und
\end{quote}
\begin{quote}
wie können wir hilfreich reagieren?
\end{quote}

\textcolor{ctmmBlue}{\faIcon{puzzle-piece}} \textit{*}Dynamisches Tool zur Reflexion und Alltagssteuerung -- ergänzt das
\textcolor{ctmmRed}{Trigger}-Tagebuch \\& die Ko-Regulation\textit{*}

\\#\\#

\subsection{📘 \textbf{[KAPITELZUORDNUNG IM \textcolor{ctmmBlue}{CTMM}-SYSTEM]}}

\begin{itemize}
\item   `Kap. ``1` → Grundlogik der Bindungsdynamik (Auslöser, Reaktion,
\end{itemize}
Integration)
\begin{itemize}
\item   `Kap. ``2.6` → Team-Regeln, Ko-Regulation, Nähe/Distanz
\item   `Kap. ``3.1 -- 3.5` → Eskalationssicherung, Rückzug, Intervention
\item   `Kap. ``5.2` → Trigger-Tagebuch, Matching-Auswertung, Reaktionslogik
\end{itemize}

\\#\\#

\subsection{🎨 \textbf{[FARBCODE \\& SYSTEMNAVIGATION]}}

-------------------------------------------------------------------------------------
\textbf{\textit{Farbe}}   \textbf{\textit{Phase}}                     \textbf{\textit{Modulkontext}}
------------- ------------------------------- ---------------------------------------
🔵            Mustererkennung                 `Kap. ``1`, `bindungsdynamik`

🟠            Reaktion \\& Intervention         `T``ool`` ``26`` C``o`` R``egulation`

\textcolor{ctmmRed}{\faIcon{circle}}            Eskalation / Rückzug            `T``rigger`` N``otfallkarten`
-------------------------------------------------------------------------------------

\begin{quote}
\textit{*}📎 Die Farben helfen beim Wiedererkennen der Phasen im Alltag und
\end{quote}
\begin{quote}
bei der Navigation im Print- oder HTML-System\textit{*}
\end{quote}

\subsection{📊 \textbf{[MATCHING-TABELLE: TYPISCHE SZENARIEN]}}

-----------------------------------------------------------------------------------
\textbf{\textit{\textcolor{ctmmRed}{Trigger}            }}Reaktion ER\textbf{\textit{  }}Reaktion      \textit{*}Was hilft konkret
(Beispiel)\textbf{\textit{                            SIE}}           (Aktion/Tool)\textit{*}
--------------------- ------------------ ---------------- -------------------------
\textbf{Vorwurf / Kritik}  Rückzug, Erstarren Lautwerden,      🔁 Ritualsatz,
Rechtfertigung   Ko-Regulation, Safe-Word

\textbf{Näheforderung ohne  Reizüberflutung,   Impulsives       🟠 Ankündigung,
Warnung\textbf{             Freeze             Klammern         Rückzugsregel, Timer

\textbf{Kontrollverlust}   Frust, Bewegung,   Angst,           \textcolor{ctmmRed}{\faIcon{circle}} Strukturzettel,
Fluchtimpuls       Überforderung    Rückkehrritual,
Buddy-Call

\textbf{Unklarheit, was los Grübeln,           Drängen,         \textcolor{ctmmPurple}{\faIcon{brain}} Matching-Check-In,
ist\textbf{                 Vermeidung         wiederholen      \textcolor{ctmmBlue}{CTMM}-Visualisierung
-----------------------------------------------------------------------------------

\subsection{✏️ \textbf{[EIGENE BEOBACHTUNGEN]}}

-----------------------------------------------------------------------------------------------------------------------------------------------------------------------------------------------------------------------------------------------------------------------------------------------
\textbf{\textit{\textcolor{ctmmRed}{Trigger} (Kontext)}}                                            \textbf{\textit{Reaktion ER}}                                          \textbf{\textit{Reaktion SIE}}                                         \textbf{\textit{Guter Umgang / Was hilft}}
------------------------------------------------------------------ ---------------------------------------------------------- ---------------------------------------------------------- ------------------------------------------------------------------------------------------------------
\textcolor{ctmmBlue}{\faIcon{edit}}                                                                 \textcolor{ctmmBlue}{\faIcon{edit}}                                                         \textcolor{ctmmBlue}{\faIcon{edit}}                                                         \textcolor{ctmmBlue}{\faIcon{edit}}
\\\_\\\_\\\_\\\_\\\_\\\_\\\_\\\_\\\_\\\_\\\_\\\_\\\_\\\_\\\_\\\_\\\_\\\_\\\_\\\_\\\_\\\_\\\_\\\_\\\_\\\_\\\_\\\_\\\_\\\_\\\_\\\_   \\\_\\\_\\\_\\\_\\\_\\\_\\\_\\\_\\\_\\\_\\\_\\\_\\\_\\\_\\\_\\\_\\\_\\\_\\\_\\\_\\\_\\\_\\\_\\\_\\\_\\\_\\\_\\\_   \\\_\\\_\\\_\\\_\\\_\\\_\\\_\\\_\\\_\\\_\\\_\\\_\\\_\\\_\\\_\\\_\\\_\\\_\\\_\\\_\\\_\\\_\\\_\\\_\\\_\\\_\\\_\\\_   \\\_\\\_\\\_\\\_\\\_\\\_\\\_\\\_\\\_\\\_\\\_\\\_\\\_\\\_\\\_\\\_\\\_\\\_\\\_\\\_\\\_\\\_\\\_\\\_\\\_\\\_\\\_\\\_\\\_\\\_\\\_\\\_\\\_\\\_\\\_\\\_\\\_\\\_\\\_\\\_\\\_\\\_\\\_\\\_\\\_\\\_\\\_\\\_\\\_\\\_

\textcolor{ctmmBlue}{\faIcon{edit}}                                                                 \textcolor{ctmmBlue}{\faIcon{edit}}                                                         \textcolor{ctmmBlue}{\faIcon{edit}}                                                         \textcolor{ctmmBlue}{\faIcon{edit}}
\\\_\\\_\\\_\\\_\\\_\\\_\\\_\\\_\\\_\\\_\\\_\\\_\\\_\\\_\\\_\\\_\\\_\\\_\\\_\\\_\\\_\\\_\\\_\\\_\\\_\\\_\\\_\\\_\\\_\\\_\\\_\\\_   \\\_\\\_\\\_\\\_\\\_\\\_\\\_\\\_\\\_\\\_\\\_\\\_\\\_\\\_\\\_\\\_\\\_\\\_\\\_\\\_\\\_\\\_\\\_\\\_\\\_\\\_\\\_\\\_   \\\_\\\_\\\_\\\_\\\_\\\_\\\_\\\_\\\_\\\_\\\_\\\_\\\_\\\_\\\_\\\_\\\_\\\_\\\_\\\_\\\_\\\_\\\_\\\_\\\_\\\_\\\_\\\_   \\\_\\\_\\\_\\\_\\\_\\\_\\\_\\\_\\\_\\\_\\\_\\\_\\\_\\\_\\\_\\\_\\\_\\\_\\\_\\\_\\\_\\\_\\\_\\\_\\\_\\\_\\\_\\\_\\\_\\\_\\\_\\\_\\\_\\\_\\\_\\\_\\\_\\\_\\\_\\\_\\\_\\\_\\\_\\\_\\\_\\\_\\\_\\\_\\\_\\\_

\textcolor{ctmmBlue}{\faIcon{edit}}                                                                 \textcolor{ctmmBlue}{\faIcon{edit}}                                                         \textcolor{ctmmBlue}{\faIcon{edit}}                                                         \textcolor{ctmmBlue}{\faIcon{edit}}
\\\_\\\_\\\_\\\_\\\_\\\_\\\_\\\_\\\_\\\_\\\_\\\_\\\_\\\_\\\_\\\_\\\_\\\_\\\_\\\_\\\_\\\_\\\_\\\_\\\_\\\_\\\_\\\_\\\_\\\_\\\_\\\_   \\\_\\\_\\\_\\\_\\\_\\\_\\\_\\\_\\\_\\\_\\\_\\\_\\\_\\\_\\\_\\\_\\\_\\\_\\\_\\\_\\\_\\\_\\\_\\\_\\\_\\\_\\\_\\\_   \\\_\\\_\\\_\\\_\\\_\\\_\\\_\\\_\\\_\\\_\\\_\\\_\\\_\\\_\\\_\\\_\\\_\\\_\\\_\\\_\\\_\\\_\\\_\\\_\\\_\\\_\\\_\\\_   \\\_\\\_\\\_\\\_\\\_\\\_\\\_\\\_\\\_\\\_\\\_\\\_\\\_\\\_\\\_\\\_\\\_\\\_\\\_\\\_\\\_\\\_\\\_\\\_\\\_\\\_\\\_\\\_\\\_\\\_\\\_\\\_\\\_\\\_\\\_\\\_\\\_\\\_\\\_\\\_\\\_\\\_\\\_\\\_\\\_\\\_\\\_\\\_\\\_\\\_
-----------------------------------------------------------------------------------------------------------------------------------------------------------------------------------------------------------------------------------------------------------------------------------------------

\subsection{\textcolor{ctmmOrange}{\faIcon{compass}} \textbf{[\textcolor{ctmmBlue}{CTMM}-NAVIGATION]}}

\begin{itemize}
\item   `T``rigger`` F``orschungstagebuch` ← individuelle
\end{itemize}
Triggerdokumentation (`Kap. ``5.2`)
\begin{itemize}
\item   `T``ool`` ``23`` ``T``riggermanagement` ← Auslöser-Logik,
\end{itemize}
Eskalationszyklus (`Kap. ``5.3`)
\begin{itemize}
\item   `T``ool`` ``26`` C``o``-R``egulation` ← Partnerintervention, Haltung
\end{itemize}
in Echtzeit (`Kap. ``2.6`)
\begin{itemize}
\item   `R``itual`` W``orkbook` ← Rückkehrrituale nach Matching-Konflikt
\end{itemize}
(`Kap. ``3.4`)

\begin{quote}
\textit{*}📎 Diese Matching-Matrix ist besonders nützlich bei Paartherapie,
\end{quote}
\begin{quote}
Reflexion am Abend oder als Aushang für Buddy-Supervision.\textit{*}
\end{quote}

\% End of Matching Matrix Trigger Reaktion Intervention CTMM.tex
