\hypertarget{matching-matrix}{%
\section{\texorpdfstring{🧩 \textbf{MATCHING-MATRIX}}{🧩 MATCHING-MATRIX}}\label{matching-matrix}}

\hypertarget{trigger-reaktion-intervention-ctmm-modul}{%
\section{\texorpdfstring{\textbf{TRIGGER -- REAKTION -- INTERVENTION (CTMM-MODUL)}}{TRIGGER -- REAKTION -- INTERVENTION (CTMM-MODUL)}}\label{trigger-reaktion-intervention-ctmm-modul}}

\begin{quote}
🧠 \textbf{Worum geht's hier -- für Freunde?}\\
Dieses Modul hilft, typische Reiz-Reaktionsmuster in unserer Beziehung zu verstehen.\\
Es ist wie ein Übersetzungsblatt -- was passiert in mir, in dir, und wie können wir hilfreich reagieren?
\end{quote}

🧩 \emph{\textbf{Dynamisches Tool zur Reflexion und Alltagssteuerung -- ergänzt das Trigger-Tagebuch \\\& die Ko-Regulation}}

\hypertarget{section}{%
\subsection{}\label{section}}

\hypertarget{kapitelzuordnung-im-ctmm-system}{%
\subsection{\texorpdfstring{📘 \textbf{\ul{KAPITELZUORDNUNG IM CTMM-SYSTEM}}}{📘 KAPITELZUORDNUNG IM CTMM-SYSTEM}}\label{kapitelzuordnung-im-ctmm-system}}

\begin{itemize}
\tightlist
\item
  \texttt{Kap.\ 1} → Grundlogik der Bindungsdynamik (Auslöser, Reaktion, Integration)
\item
  \texttt{Kap.\ 2.6} → Team-Regeln, Ko-Regulation, Nähe/Distanz
\item
  \texttt{Kap.\ 3.1 -- 3.5} → Eskalationssicherung, Rückzug, Intervention
\item
  \texttt{Kap.\ 5.2} → Trigger-Tagebuch, Matching-Auswertung, Reaktionslogik
\end{itemize}

\hypertarget{section-1}{%
\subsection{}\label{section-1}}

\hypertarget{farbcode-systemnavigation}{%
\subsection{\texorpdfstring{🎨 \textbf{\ul{FARBCODE \\\& SYSTEMNAVIGATION}}}\{🎨 FARBCODE \\\& SYSTEMNAVIGATION}}\label{farbcode-systemnavigation}}

\begin{longtable}[]\{@{}
  >\{\raggedright\arraybackslash}p\{(\columnwidth - 4\tabcolsep) * \real{0.1237}}
  >\{\raggedright\arraybackslash}p\{(\columnwidth - 4\tabcolsep) * \real{0.4482}}
  >\{\raggedright\arraybackslash}p\{(\columnwidth - 4\tabcolsep) * \real{0.4281}}@{}}
\toprule\noalign{}
\begin{minipage}[b]\{\linewidth}\raggedright
\emph{\textbf{Farbe}}
\end{minipage} \& \begin{minipage}[b]\{\linewidth}\raggedright
\emph{\textbf{Phase}}
\end{minipage} \& \begin{minipage}[b]\{\linewidth}\raggedright
\emph{\textbf{Modulkontext}}
\end{minipage} \\
\midrule\noalign{}
\endhead
\bottomrule\noalign{}
\endlastfoot
🔵 \& Mustererkennung \& \texttt{Kap.\ }\texttt{1}, \texttt{bindungsdynamik} \\
🟠 \& Reaktion \\\& Intervention \& \texttt{T}\texttt{ool}\texttt{\ }\texttt{26}\texttt{\ C}\texttt{o}\texttt{\ R}\texttt{egulation} \\
🔴 \& Eskalation / Rückzug \& \texttt{T}\texttt{rigger}\texttt{\ N}\texttt{otfallkarten} \\
\end{longtable}

\begin{quote}
\emph{\textbf{📎 Die Farben helfen beim Wiedererkennen der Phasen im Alltag und bei der Navigation im Print- oder HTML-System}}
\end{quote}

\hypertarget{matching-tabelle-typische-szenarien}\{\%
\subsection{\texorpdfstring{📊 \textbf{\ul{MATCHING-TABELLE: TYPISCHE SZENARIEN}}}{📊 MATCHING-TABELLE: TYPISCHE SZENARIEN}}\label{matching-tabelle-typische-szenarien}}

\begin{longtable}[]\{@{}
  >\{\raggedright\arraybackslash}p\{(\columnwidth - 6\tabcolsep) * \real{0.2064}}
  >\{\raggedright\arraybackslash}p\{(\columnwidth - 6\tabcolsep) * \real{0.2301}}
  >\{\raggedright\arraybackslash}p\{(\columnwidth - 6\tabcolsep) * \real{0.2064}}
  >\{\raggedright\arraybackslash}p\{(\columnwidth - 6\tabcolsep) * \real{0.3571}}@{}}
\toprule\noalign{}
\begin{minipage}[b]\{\linewidth}\raggedright
\emph{\textbf{Trigger (Beispiel)}}
\end{minipage} \& \begin{minipage}[b]\{\linewidth}\raggedright
\emph{\textbf{Reaktion ER}}
\end{minipage} \& \begin{minipage}[b]\{\linewidth}\raggedright
\emph{\textbf{Reaktion SIE}}
\end{minipage} \& \begin{minipage}[b]\{\linewidth}\raggedright
\emph{\textbf{Was hilft konkret (Aktion/Tool)}}
\end{minipage} \\
\midrule\noalign{}
\endhead
\bottomrule\noalign{}
\endlastfoot
\textbf{Vorwurf / Kritik} \& Rückzug, Erstarren \& Lautwerden, Rechtfertigung \& 🔁 Ritualsatz, Ko-Regulation, Safe-Word \\
\textbf{Näheforderung ohne Warnung} \& Reizüberflutung, Freeze \& Impulsives Klammern \& 🟠 Ankündigung, Rückzugsregel, Timer \\
\textbf{Kontrollverlust} \& Frust, Bewegung, Fluchtimpuls \& Angst, Überforderung \& 🔴 Strukturzettel, Rückkehrritual, Buddy-Call \\
\textbf{Unklarheit, was los ist} \& Grübeln, Vermeidung \& Drängen, wiederholen \& 🧠 Matching-Check-In, CTMM-Visualisierung \\
\end{longtable}

\hypertarget{eigene-beobachtungen}\{\%
\subsection{\texorpdfstring{✏️ \textbf{\ul{EIGENE BEOBACHTUNGEN}} }{✏️ EIGENE BEOBACHTUNGEN }}\label{eigene-beobachtungen}}

\begin{longtable}[]\{@{}
  >\{\raggedright\arraybackslash}p\{(\columnwidth - 6\tabcolsep) * \real{0.2384}}
  >\{\raggedright\arraybackslash}p\{(\columnwidth - 6\tabcolsep) * \real{0.2077}}
  >\{\raggedright\arraybackslash}p\{(\columnwidth - 6\tabcolsep) * \real{0.2077}}
  >\{\raggedright\arraybackslash}p\{(\columnwidth - 6\tabcolsep) * \real{0.3462}}@{}}
\toprule\noalign{}
\begin{minipage}[b]\{\linewidth}\raggedright
\emph{\textbf{Trigger (Kontext)}}
\end{minipage} \& \begin{minipage}[b]\{\linewidth}\raggedright
\emph{\textbf{Reaktion ER}}
\end{minipage} \& \begin{minipage}[b]\{\linewidth}\raggedright
\emph{\textbf{Reaktion SIE}}
\end{minipage} \& \begin{minipage}[b]\{\linewidth}\raggedright
\emph{\textbf{Guter Umgang / Was hilft}}
\end{minipage} \\
\midrule\noalign{}
\endhead
\bottomrule\noalign{}
\endlastfoot
📝 \\_\\_\\_\\_\\_\\_\\_\\_\\_\\_\\_\\_\\_\\_\\_\\_\\_\\_\\_\\_\\_\\_\\_\\_\\_\\_\\_\\_\\_\\_\\_\\_ \& 📝 \\_\\_\\_\\_\\_\\_\\_\\_\\_\\_\\_\\_\\_\\_\\_\\_\\_\\_\\_\\_\\_\\_\\_\\_\\_\\_\\_\\_ \& 📝 \\_\\_\\_\\_\\_\\_\\_\\_\\_\\_\\_\\_\\_\\_\\_\\_\\_\\_\\_\\_\\_\\_\\_\\_\\_\\_\\_\\_ \& 📝 \\_\\_\\_\\_\\_\\_\\_\\_\\_\\_\\_\\_\\_\\_\\_\\_\\_\\_\\_\\_\\_\\_\\_\\_\\_\\_\\_\\_\\_\\_\\_\\_\\_\\_\\_\\_\\_\\_\\_\\_\\_\\_\\_\\_\\_\\_\\_\\_\\_\\_ \\
📝 \\_\\_\\_\\_\\_\\_\\_\\_\\_\\_\\_\\_\\_\\_\\_\\_\\_\\_\\_\\_\\_\\_\\_\\_\\_\\_\\_\\_\\_\\_\\_\\_ \& 📝 \\_\\_\\_\\_\\_\\_\\_\\_\\_\\_\\_\\_\\_\\_\\_\\_\\_\\_\\_\\_\\_\\_\\_\\_\\_\\_\\_\\_ \& 📝 \\_\\_\\_\\_\\_\\_\\_\\_\\_\\_\\_\\_\\_\\_\\_\\_\\_\\_\\_\\_\\_\\_\\_\\_\\_\\_\\_\\_ \& 📝 \\_\\_\\_\\_\\_\\_\\_\\_\\_\\_\\_\\_\\_\\_\\_\\_\\_\\_\\_\\_\\_\\_\\_\\_\\_\\_\\_\\_\\_\\_\\_\\_\\_\\_\\_\\_\\_\\_\\_\\_\\_\\_\\_\\_\\_\\_\\_\\_\\_\\_ \\
📝 \\_\\_\\_\\_\\_\\_\\_\\_\\_\\_\\_\\_\\_\\_\\_\\_\\_\\_\\_\\_\\_\\_\\_\\_\\_\\_\\_\\_\\_\\_\\_\\_ \& 📝 \\_\\_\\_\\_\\_\\_\\_\\_\\_\\_\\_\\_\\_\\_\\_\\_\\_\\_\\_\\_\\_\\_\\_\\_\\_\\_\\_\\_ \& 📝 \\_\\_\\_\\_\\_\\_\\_\\_\\_\\_\\_\\_\\_\\_\\_\\_\\_\\_\\_\\_\\_\\_\\_\\_\\_\\_\\_\\_ \& 📝 \\_\\_\\_\\_\\_\\_\\_\\_\\_\\_\\_\\_\\_\\_\\_\\_\\_\\_\\_\\_\\_\\_\\_\\_\\_\\_\\_\\_\\_\\_\\_\\_\\_\\_\\_\\_\\_\\_\\_\\_\\_\\_\\_\\_\\_\\_\\_\\_\\_\\_ \\
\end{longtable}

\hypertarget{ctmm-navigation}\{\%
\subsection{\texorpdfstring{🧭 \textbf{\ul{CTMM-NAVIGATION}}}{🧭 CTMM-NAVIGATION}}\label{ctmm-navigation}}

\begin{itemize}
\tightlist
\item
  \texttt{T}\texttt{rigger}\texttt{\ F}\texttt{orschungstagebuch} ← individuelle Triggerdokumentation (\texttt{Kap.\ }\texttt{5.2})
\item
  \texttt{T}\texttt{ool}\texttt{\ }\texttt{23}\texttt{\ }\texttt{T}\texttt{riggermanagement} ← Auslöser-Logik, Eskalationszyklus (\texttt{Kap.\ }\texttt{5.3})
\item
  \texttt{T}\texttt{ool}\texttt{\ }\texttt{26}\texttt{\ C}\texttt{o}\texttt{-R}\texttt{egulation} ← Partnerintervention, Haltung in Echtzeit (\texttt{Kap.\ }\texttt{2.6})
\item
  \texttt{R}\texttt{itual}\texttt{\ W}\texttt{orkbook} ← Rückkehrrituale nach Matching-Konflikt (\texttt{Kap.\ }\texttt{3.4})
\end{itemize}

\begin{quote}
\emph{\textbf{📎 Diese Matching-Matrix ist besonders nützlich bei Paartherapie, Reflexion am Abend oder als Aushang für Buddy-Supervision.}}
\end{quote}
