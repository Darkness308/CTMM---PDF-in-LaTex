\section{Bindungsleitfaden CTMM Übersicht }

\subsection{🏥 BINDUNGSLEITFADEN --}

\subsection{CTMM-ÜBERSICHT}

🧩 Navigationshilfe für neurodiverse Beziehungsmuster -- als roter Faden durch alle CTMM-Module

\subsection{📘 WAS IST BINDUNG IN CTMM?}

Bindung heißt nicht: Kontrolle, Nähepflicht, dauerhafte

Verschmelzung

Bindung heißt: Vertrauen, Wiederkehr, Sicherheit trotz Rückzug

CTMM nutzt Bindung als dynamischen Prozess mit Werkzeugen, Safe-Zonen und Gesprächsritualen

\subsection{🧭 CTMM-MODULNAVIGATION -- BINDUNGSSYSTEM AUF EINEN}

\subsection{BLICK}

Bereich

Modul(e)

Kapitel/Verweis

🔵 Grundlagen \textbackslash{}&

Definition

Bindungsdynamik CTMM

Kap. 1

🟢 Prävention \textbackslash{}&

Alltag

Wertekompass vereint, Safe rules WG-Modell

Kap. 2.1 -- 2.6

🟠 Eskalation

sichern

Trigger Notfallkarten, Ritual Workbook, Tool 26 Co-Regulation

Kap. 3.1 -- 3.5

🔴 Notfallplanung

\textbackslash{}& Reaktion

Trigger Forschungstagebuch, Krisenprotokoll

Kap. 5.2 -- 5.5

🟣 Reflexion \textbackslash{}&

Fortschritt

Trigger Tagebuch extended, Vision board Klartext

Kap. 5.6, Tool 27

\subsection{💬 SCHLÜSSELSÄTZE FÜR SICHERE BINDUNG}

"\'`Ich ziehe mich zurück, aber ich komme wieder.“

"\'`Meine Grenze ist nicht gegen dich, sondern für mich.“

"\'`Wir reden darüber -- aber nicht jetzt, nicht in Panik.“

"\'`Ich sehe dich, auch wenn ich mich gerade nicht spürbar zeige.“

\subsection{🔗 ZUSAMMENSPIEL DER MODULE}

Safe-Words ↔ Rückkehr-Rituale ↔ Notfallplan = Sicherheitskreis

Werte-Kompass ↔ WG-Regeln ↔ Tagesstruktur = Stabilitätsanker

Trigger-Tagebuch ↔ Vision-Board ↔ Gesprächsregeln = Entwicklungskompass

📎 Dieser Leitfaden ist Start- und Übersichtspunkt für jedes CTMM-Paar. Er ersetzt kein Gespräch -- aber strukturiert, was besprochen werden muss.

📤 Empfohlen als Deckblatt deines CTMM-Ordners + Startseite deiner digitalin

HTML-Version
