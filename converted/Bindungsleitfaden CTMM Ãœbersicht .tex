% CTMM Therapy Module: Bindungsleitfaden Ctmm Übersicht 
% Converted from Word document: Bindungsleitfaden CTMM Übersicht .docx
% Auto-generated by document_converter.py

% Options for packages loaded elsewhere
\PassOptionsToPackage{unicode}{hyperref}
\PassOptionsToPackage{hyphens}{url}
%
\documentclass[
]{article}
\usepackage{amsmath,amssymb}
\usepackage{iftex}
\ifPDFTeX
  \usepackage[T1]{fontenc}
  \usepackage[utf8]{inputenc}
  \usepackage{textcomp} % provide euro and other symbols
\else % if luatex or xetex
  \usepackage{unicode-math} % this also loads fontspec
  \defaultfontfeatures{Scale=MatchLowercase}
  \defaultfontfeatures[\rmfamily]{Ligatures=TeX,Scale=1}
\fi
\usepackage{lmodern}
\ifPDFTeX\else
  % xetex/luatex font selection
\fi
% Use upquote if available, for straight quotes in verbatim environments
\IfFileExists{upquote.sty}{\usepackage{upquote}}{}
\IfFileExists{microtype.sty}{% use microtype if available
  \usepackage[]{microtype}
  \UseMicrotypeSet[protrusion]{basicmath} % disable protrusion for tt fonts
}{}
\makeatletter
\@ifundefined{KOMAClassName}{% if non-KOMA class
  \IfFileExists{parskip.sty}{%
    \usepackage{parskip}
  }{% else
    \setlength{\parindent}{0pt}
    \setlength{\parskip}{6pt plus 2pt minus 1pt}}
}{% if KOMA class
  \KOMAoptions{parskip=half}}
\makeatother
\usepackage{xcolor}
\usepackage{longtable,booktabs,array}
\usepackage{calc} % for calculating minipage widths
% Correct order of tables after \paragraph or \subparagraph
\usepackage{etoolbox}
\makeatletter
\patchcmd\longtable{\par}{\if@noskipsec\mbox{}\fi\par}{}{}
\makeatother
% Allow footnotes in longtable head/foot
\IfFileExists{footnotehyper.sty}{\usepackage{footnotehyper}}{\usepackage{footnote}}
\makesavenoteenv{longtable}
\ifLuaTeX
  \usepackage{luacolor}
  \usepackage[soul]{lua-ul}
\else
  \usepackage{soul}
\fi
\setlength{\emergencystretch}{3em} % prevent overfull lines
\providecommand{\tightlist}{%
  \setlength{\itemsep}{0pt}\setlength{\parskip}{0pt}}
\setcounter{secnumdepth}{-\maxdimen} % remove section numbering
\ifLuaTeX
  \usepackage{selnolig}  % disable illegal ligatures
\fi
\IfFileExists{bookmark.sty}{\usepackage{bookmark}}{\usepackage{hyperref}}
\IfFileExists{xurl.sty}{\usepackage{xurl}}{} % add URL line breaks if available
\urlstyle{same}
\hypersetup{
  hidelinks,
  pdfcreator={LaTeX via pandoc}}

\author{}
\date{}



\hypertarget{bindungsleitfaden}{%
\section{\texorpdfstring{\textbf{🏥 BINDUNGSLEITFADEN --}}{🏥 BINDUNGSLEITFADEN --}}\label{bindungsleitfaden}}

\hypertarget{ctmm-uxfcbersicht}{%
\section{\texorpdfstring{\textbf{CTMM-ÜBERSICHT}}{CTMM-ÜBERSICHT}}\label{ctmm-uxfcbersicht}}

🧩 \emph{\textbf{Navigationshilfe für neurodiverse Beziehungsmuster -- als roter Faden durch alle CTMM-Module}}

\hypertarget{was-ist-bindung-in-ctmm}{%
\subsection{\texorpdfstring{\textbf{📘 \ul{WAS IST BINDUNG IN CTMM?}}}{📘 WAS IST BINDUNG IN CTMM?}}\label{was-ist-bindung-in-ctmm}}

\begin{itemize}
\tightlist
\item
  \textbf{Bindung heißt nicht:} Kontrolle, Nähepflicht, dauerhafte
\end{itemize}

\begin{quote}
Verschmelzung
\end{quote}

\begin{itemize}
\tightlist
\item
  \textbf{Bindung heißt:} Vertrauen, Wiederkehr, Sicherheit trotz Rückzug
\item
  CTMM nutzt Bindung als \textbf{dynamischen Prozess} mit Werkzeugen, Safe-Zonen und Gesprächsritualen
\end{itemize}

\hypertarget{ctmm-modulnavigation-bindungssystem-auf-einen}{%
\subsection{\texorpdfstring{\textbf{🧭 \ul{CTMM-MODULNAVIGATION -- BINDUNGSSYSTEM AUF EINEN}}}{🧭 CTMM-MODULNAVIGATION -- BINDUNGSSYSTEM AUF EINEN}}\label{ctmm-modulnavigation-bindungssystem-auf-einen}}

\hypertarget{blick}{%
\subsection{\texorpdfstring{ \textbf{\ul{BLICK}}}{ BLICK}}\label{blick}}

\begin{longtable}[]{@{}
  >{\raggedright\arraybackslash}p{(\columnwidth - 4\tabcolsep) * \real{0.3000}}
  >{\raggedright\arraybackslash}p{(\columnwidth - 4\tabcolsep) * \real{0.3889}}
  >{\raggedright\arraybackslash}p{(\columnwidth - 4\tabcolsep) * \real{0.3111}}@{}}
\toprule\noalign{}
\begin{minipage}[b]{\linewidth}\raggedright
\emph{\textbf{Bereich}}
\end{minipage} & \begin{minipage}[b]{\linewidth}\raggedright
\emph{\textbf{Modul(e)}}
\end{minipage} & \begin{minipage}[b]{\linewidth}\raggedright
\emph{\textbf{Kapitel/Verweis}}
\end{minipage} \\
\midrule\noalign{}
\endhead
\bottomrule\noalign{}
\endlastfoot
🔵 Grundlagen \&

Definition & \texttt{B}\texttt{indungsdynamik}\texttt{\ CTMM} & \texttt{Kap.\ }\texttt{1} \\
🟢 Prävention \&

Alltag & \texttt{W}\texttt{ertekompass}\texttt{\ }\texttt{vereint}, \texttt{S}\texttt{afe}\texttt{\ }\texttt{rules}\texttt{\ WG}\texttt{-M}\texttt{odell} & \texttt{Kap.\ }\texttt{2.1\ –\ 2.6} \\
🟠 Eskalation

sichern & \texttt{T}\texttt{rigger}\texttt{\ N}\texttt{otfallkarten}, \texttt{R}\texttt{itual}\texttt{\ W}\texttt{orkbook}, \texttt{T}\texttt{ool}\texttt{\ }\texttt{26}\texttt{\ C}\texttt{o}\texttt{-R}\texttt{egulation} & \texttt{Kap.\ }\texttt{3.1\ –\ 3.5} \\
🔴 Notfallplanung

\& Reaktion & \texttt{T}\texttt{rigger}\texttt{\ F}\texttt{orschungstagebuch}, \texttt{K}\texttt{risenprotokoll} & \texttt{Kap.\ }\texttt{5.2\ –\ 5.5} \\
🟣 Reflexion \&

Fortschritt & \texttt{T}\texttt{rigger}\texttt{\ T}\texttt{agebuch}\texttt{\ }\texttt{extended}, \texttt{V}\texttt{ision}\texttt{\ }\texttt{board}\texttt{\ }\texttt{K}\texttt{lartext} & \texttt{Kap.\ }\texttt{5.6}, \texttt{Tool\ }\texttt{27} \\
\end{longtable}

\hypertarget{schluxfcsselsuxe4tze-fuxfcr-sichere-bindung}{%
\subsection{\texorpdfstring{\textbf{💬 \ul{SCHLÜSSELSÄTZE FÜR SICHERE BINDUNG}}}{💬 SCHLÜSSELSÄTZE FÜR SICHERE BINDUNG}}\label{schluxfcsselsuxe4tze-fuxfcr-sichere-bindung}}

\begin{itemize}
\tightlist
\item
  „Ich ziehe mich zurück, aber ich komme wieder.``
\item
  „Meine Grenze ist nicht gegen dich, sondern für mich.``
\item
  „Wir reden darüber -- aber nicht jetzt, nicht in Panik.``
\item
  „Ich sehe dich, auch wenn ich mich gerade nicht spürbar zeige.``
\end{itemize}

\hypertarget{zusammenspiel-der-module}{%
\subsection{\texorpdfstring{\textbf{🔗 \ul{ZUSAMMENSPIEL DER MODULE}}}{🔗 ZUSAMMENSPIEL DER MODULE}}\label{zusammenspiel-der-module}}

\begin{itemize}
\tightlist
\item
  \textbf{Safe-Words ↔ Rückkehr-Rituale ↔ Notfallplan} = Sicherheitskreis
\item
  \textbf{Werte-Kompass ↔ WG-Regeln ↔ Tagesstruktur} = Stabilitätsanker
\item
  \textbf{Trigger-Tagebuch ↔ Vision-Board ↔ Gesprächsregeln} = Entwicklungskompass
\item
\end{itemize}

\begin{quote}
\textbf{📎 Dieser Leitfaden ist Start- und Übersichtspunkt für jedes CTMM-Paar. Er ersetzt kein Gespräch -- aber strukturiert, was besprochen werden muss.}
\end{quote}

📤 \emph{Empfohlen als Deckblatt deines CTMM-Ordners + Startseite deiner digitalin}

\emph{HTML-Version}


