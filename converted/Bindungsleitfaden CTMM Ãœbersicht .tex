\hypertarget{bindungsleitfaden}\{\%
\section{\texorpdfstring{\textbf{🏥 BINDUNGSLEITFADEN --}}{🏥 BINDUNGSLEITFADEN --}}\label{bindungsleitfaden}}

\hypertarget{ctmm-uxfcbersicht}\{\%
\section{\texorpdfstring{\textbf{CTMM-ÜBERSICHT}}{CTMM-ÜBERSICHT}}\label{ctmm-uxfcbersicht}}

🧩 \emph{\textbf{Navigationshilfe für neurodiverse Beziehungsmuster -- als roter Faden durch alle CTMM-Module}}

\hypertarget{was-ist-bindung-in-ctmm}\{\%
\subsection{\texorpdfstring{\textbf{📘 \ul{WAS IST BINDUNG IN CTMM?}}}{📘 WAS IST BINDUNG IN CTMM?}}\label{was-ist-bindung-in-ctmm}}

\begin{itemize}
\tightlist
\item
  \textbf{Bindung heißt nicht:} Kontrolle, Nähepflicht, dauerhafte
\end{itemize}

\begin{quote}
Verschmelzung
\end{quote}

\begin{itemize}
\tightlist
\item
  \textbf{Bindung heißt:} Vertrauen, Wiederkehr, Sicherheit trotz Rückzug
\item
  CTMM nutzt Bindung als \textbf{dynamischen Prozess} mit Werkzeugen, Safe-Zonen und Gesprächsritualen
\end{itemize}

\hypertarget{ctmm-modulnavigation-bindungssystem-auf-einen}\{\%
\subsection{\texorpdfstring{\textbf{🧭 \ul{CTMM-MODULNAVIGATION -- BINDUNGSSYSTEM AUF EINEN}}}{🧭 CTMM-MODULNAVIGATION -- BINDUNGSSYSTEM AUF EINEN}}\label{ctmm-modulnavigation-bindungssystem-auf-einen}}

\hypertarget{blick}\{\%
\subsection{\texorpdfstring{ \textbf{\ul{BLICK}}}{ BLICK}}\label{blick}}

\begin{longtable}[]\{@{}
  >\{\raggedright\arraybackslash}p\{(\columnwidth - 4\tabcolsep) * \real{0.3000}}
  >\{\raggedright\arraybackslash}p\{(\columnwidth - 4\tabcolsep) * \real{0.3889}}
  >\{\raggedright\arraybackslash}p\{(\columnwidth - 4\tabcolsep) * \real{0.3111}}@{}}
\toprule\noalign{}
\begin{minipage}[b]\{\linewidth}\raggedright
\emph{\textbf{Bereich}}
\end{minipage} \& \begin{minipage}[b]\{\linewidth}\raggedright
\emph{\textbf{Modul(e)}}
\end{minipage} \& \begin{minipage}[b]\{\linewidth}\raggedright
\emph{\textbf{Kapitel/Verweis}}
\end{minipage} \\
\midrule\noalign{}
\endhead
\bottomrule\noalign{}
\endlastfoot
🔵 Grundlagen \\\&

Definition \& \texttt{B}\texttt{indungsdynamik}\texttt{\ CTMM} \& \texttt{Kap.\ }\texttt{1} \\
🟢 Prävention \\\&

Alltag \& \texttt{W}\texttt{ertekompass}\texttt{\ }\texttt{vereint}, \texttt{S}\texttt{afe}\texttt{\ }\texttt{rules}\texttt{\ WG}\texttt{-M}\texttt{odell} \& \texttt{Kap.\ }\texttt{2.1\ –\ 2.6} \\
🟠 Eskalation

sichern \& \texttt{T}\texttt{rigger}\texttt{\ N}\texttt{otfallkarten}, \texttt{R}\texttt{itual}\texttt{\ W}\texttt{orkbook}, \texttt{T}\texttt{ool}\texttt{\ }\texttt{26}\texttt{\ C}\texttt{o}\texttt{-R}\texttt{egulation} \& \texttt{Kap.\ }\texttt{3.1\ –\ 3.5} \\
🔴 Notfallplanung

\\\& Reaktion \& \texttt{T}\texttt{rigger}\texttt{\ F}\texttt{orschungstagebuch}, \texttt{K}\texttt{risenprotokoll} \& \texttt{Kap.\ }\texttt{5.2\ –\ 5.5} \\
🟣 Reflexion \\\&

Fortschritt \& \texttt{T}\texttt{rigger}\texttt{\ T}\texttt{agebuch}\texttt{\ }\texttt{extended}, \texttt{V}\texttt{ision}\texttt{\ }\texttt{board}\texttt{\ }\texttt{K}\texttt{lartext} \& \texttt{Kap.\ }\texttt{5.6}, \texttt{Tool\ }\texttt{27} \\
\end{longtable}

\hypertarget{schluxfcsselsuxe4tze-fuxfcr-sichere-bindung}\{\%
\subsection{\texorpdfstring{\textbf{💬 \ul{SCHLÜSSELSÄTZE FÜR SICHERE BINDUNG}}}{💬 SCHLÜSSELSÄTZE FÜR SICHERE BINDUNG}}\label{schluxfcsselsuxe4tze-fuxfcr-sichere-bindung}}

\begin{itemize}
\tightlist
\item
  „Ich ziehe mich zurück, aber ich komme wieder.``
\item
  „Meine Grenze ist nicht gegen dich, sondern für mich.``
\item
  „Wir reden darüber -- aber nicht jetzt, nicht in Panik.``
\item
  „Ich sehe dich, auch wenn ich mich gerade nicht spürbar zeige.``
\end{itemize}

\hypertarget{zusammenspiel-der-module}\{\%
\subsection{\texorpdfstring{\textbf{🔗 \ul{ZUSAMMENSPIEL DER MODULE}}}{🔗 ZUSAMMENSPIEL DER MODULE}}\label{zusammenspiel-der-module}}

\begin{itemize}
\tightlist
\item
  \textbf{Safe-Words ↔ Rückkehr-Rituale ↔ Notfallplan} = Sicherheitskreis
\item
  \textbf{Werte-Kompass ↔ WG-Regeln ↔ Tagesstruktur} = Stabilitätsanker
\item
  \textbf{Trigger-Tagebuch ↔ Vision-Board ↔ Gesprächsregeln} = Entwicklungskompass
\item
\end{itemize}

\begin{quote}
\textbf{📎 Dieser Leitfaden ist Start- und Übersichtspunkt für jedes CTMM-Paar. Er ersetzt kein Gespräch -- aber strukturiert, was besprochen werden muss.}
\end{quote}

📤 \emph{Empfohlen als Deckblatt deines CTMM-Ordners + Startseite deiner digitalin}

\emph{HTML-Version}
