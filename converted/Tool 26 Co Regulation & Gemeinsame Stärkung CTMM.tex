\% Tool 26 Co Regulation \& Gemeinsame Stärkung CTMM.tex - Converted from Tool 26 Co Regulation \& Gemeinsame Stärkung CTMM.docx
\% CTMM Therapy Material - Auto-generated by Conversion Pipeline

\section{Tool 26 Co Regulation \& Gemeinsame Stärkung Ctmm}
\label{sec:tool-26-co-regulation-gemeinsame-st-rkung-ctmm}

\section{\textcolor{ctmmPurple}{\faIcon{brain}} \textbf{TOOL 26:}}
\subsection{\textbf{KO-REGULATION \\& GEMEINSAME STÄRKUNG (\textcolor{ctmmBlue}{CTMM}-MODUL)}}

\textcolor{ctmmBlue}{\faIcon{puzzle-piece}} \textit{*}Gemeinsame Selbstberuhigung bei \textcolor{ctmmRed}{Trigger}, Überforderung \\&
Nähe-Distanz-Konflikten\textit{*}

\subsection{🤝 \textbf{[WAS IST KO-REGULATION?]}}

Ko-Regulation bedeutet, dass zwei Menschen einander helfen, ihr
Nervensystem zu stabilisieren. Nicht durch Worte allein -- sondern
durch:

\begin{itemize}
\item   Präsenz
\item   Nähe ohne Druck
\item   Gesten, Berührung, Wiederholung
\item   Atem, Körpersprache, Resonanz
\end{itemize}

\subsection{💞 \textbf{[WIRKUNGSZONEN DER KO-REGULATION]}}

-----------------------------------------------------------------------
\textbf{\textit{Wirkungsebene}}    \textbf{\textit{Beispiel (ER \\& SIE)}}
---------------------- ------------------------------------------------
\textcolor{ctmmPurple}{\faIcon{brain}}                     Atem synchronisieren, ruhiger sprechen, Reiz
\textbf{Neurophysiologie}   abschirmen

💬 \textbf{Sprache}         „Ich bin da. Du musst nichts erklären."

🖐️ \textbf{Körperkontakt}   Hand auf Brust, Rücken streichen, keine
Fixierung

⏳ \textbf{Zeit \\& Rhythmus} 5-Minuten-Stille, kein Fragenhagel

🔄 \textbf{Rituale \\&         Gemeinsames Licht, Musik, Symbolfoto, Ritualsatz
Symbole\textbf{
-----------------------------------------------------------------------

\subsection{⚠️ \textbf{[WANN IST KO-REGULATION SINNVOLL?]}}

\begin{itemize}
\item   Vor dem Eskalationspunkt (Frühwarnzeichen!)
\item   Nach einem Streit -- ohne Schuldzuweisung
\item   Wenn einer überfordert ist, der andere aber stabil
\item   Nach Flashback, Dissoziation, Reizschub
\end{itemize}

\subsection{🔁 \textbf{[KO-REGULATION VS. RETTUNG]}}

-----------------------------------------------------------------------------
\textbf{\textit{Merkmal}}             \textbf{\textit{Ko-Regulation}}   \textbf{\textit{Retten / Kontrollieren}}
------------------------- --------------------- -----------------------------
\textbf{Verantwortung}         geteilt               asymmetrisch („ich rette
dich")

\textbf{Energie}               wechselseitig         erschöpfend für eine Seite
regulierend

\textbf{Sprache}               ruhig, absichtslos    aktiv, fordernd,
zielorientiert

\textbf{Berührung}             angeboten, dosiert    übergriffig, ohne Abstimmung

\textbf{Beziehungserfahrung}   gemeinschaftlich,     regressiv, emotional unsicher
sicher
-----------------------------------------------------------------------------

\subsection{\textcolor{ctmmOrange}{\faIcon{compass}} \textbf{[\textcolor{ctmmBlue}{CTMM}-INTEGRATION \\& NAVIGATION]}}

\begin{itemize}
\item   \textcolor{ctmmGreen}{\faIcon{circle}} `Kap. ``2.6` -- Emotionale Präsenz, Beziehung als Team
\item   🟠 `Kap. ``3.1 -- 3.5` -- Notfallstruktur, Rückkehrrituale
\item   🟣 Tools: `Trigger-``Tagebuch`, `Werte-Kompass`, `Bindungsdynamik`
\end{itemize}

✅ Geeignet für: Partnerübungen, Paartherapie, Buddy-Training

\begin{quote}
\textbf{📎 Dieses Tool stärkt das Teamgefühl -- nicht durch Gespräche,
\end{quote}
\begin{quote}
sondern durch Haltung, Timing \\& Wiederholung\textbf{
\end{quote}

\% End of Tool 26 Co Regulation \& Gemeinsame Stärkung CTMM.tex
