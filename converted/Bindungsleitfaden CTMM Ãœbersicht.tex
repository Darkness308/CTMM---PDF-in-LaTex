% Converted CTMM Therapy Document: Bindungsleitfaden CTMM Übersicht
% Generated by CTMM Document Conversion Pipeline
% Date: 2025-08-09 03:35:13

\section{Bindungsleitfaden CTMM Übersicht}
\label{sec:bindungsleitfaden-ctmm-übersicht}

\hypertarget{bindungsleitfaden}{%
\section[ 🏥 \ul{\textbf{\textcolor{ctmmBlue}{\1}}}]{\texorpdfstring{\protect\hypertarget{Xad3857b60a3d634da50c4438b62d043b84455a1}{}{} 🏥 \ul{\textbf{\textcolor{ctmmBlue}{\1}}}}{ 🏥 BINDUNGSLEITFADEN --}}\label{bindungsleitfaden}}

\hypertarget{ctmm-uxfcbersicht}{%
\section{\texorpdfstring{\ul{\textbf{\textcolor{ctmmBlue}{\1}}}}{CTMM-ÜBERSICHT}}\label{ctmm-uxfcbersicht}}

\hfill\break
\hfill\break

🧩 \textbf{\textcolor{ctmmBlue}{\1}}}

\hfill\break
\hfill\break

\hypertarget{was-ist-bindung-in-ctmm}{%
\subsection[📘 \ul{\textbf{\textcolor{ctmmBlue}{\1}}}]{\texorpdfstring{\protect\hypertarget{was-ist-bindung-in-ctmm}{}{}📘 \ul{\textbf{\textcolor{ctmmBlue}{\1}}}}{📘 WAS IST BINDUNG IN CTMM?}}\label{was-ist-bindung-in-ctmm}}

\begin{itemize}[label=\textcolor{ctmmOrange}{\faArrowRight}]
\tightlist
\item
  \textbf{\textcolor{ctmmBlue}{\1}} Kontrolle, Nähepflicht, dauerhafte
\end{itemize}

Verschmelzung

\begin{itemize}[label=\textcolor{ctmmOrange}{\faArrowRight}]
\item
  \textbf{\textcolor{ctmmBlue}{\1}} Vertrauen, Wiederkehr, Sicherheit trotz Rückzug
\item
  CTMM nutzt Bindung als \textbf{\textcolor{ctmmBlue}{\1}} mit Werkzeugen, Safe-Zonen und Gesprächsritualen
\end{itemize}

\hfill\break
\hfill\break

\hypertarget{ctmm-modulnavigation-bindungssystem-auf-einen}{%
\subsection[ 🧭 \ul{\textbf{\textcolor{ctmmBlue}{\1}}}]{\texorpdfstring{\protect\hypertarget{X7ea0137d93ca5196100bdbe5715b10158d1a8b4}{}{} 🧭 \ul{\textbf{\textcolor{ctmmBlue}{\1}}}}{ 🧭 CTMM-MODULNAVIGATION -- BINDUNGSSYSTEM AUF EINEN}}\label{ctmm-modulnavigation-bindungssystem-auf-einen}}

\hypertarget{blick}{%
\subsection{\textcolor{ctmmBlue}{\faBook~\1}}}}{BLICK}}\label{blick}}

\begin{longtable}\checkbox{@{}
  >{\raggedright\arraybackslash}p{(\columnwidth - 4\tabcolsep) * \real{0.2917}}
  >{\raggedright\arraybackslash}p{(\columnwidth - 4\tabcolsep) * \real{0.3889}}
  >{\raggedright\arraybackslash}p{(\columnwidth - 4\tabcolsep) * \real{0.3056}}@{}}
\toprule\noalign{}
\begin{minipage}[b]{\linewidth}\raggedright
\textbf{\textcolor{ctmmBlue}{\1}}}
\end{minipage} & \begin{minipage}[b]{\linewidth}\raggedright
\textbf{\textcolor{ctmmBlue}{\1}}}
\end{minipage} & \begin{minipage}[b]{\linewidth}\raggedright
\textbf{\textcolor{ctmmBlue}{\1}}}
\end{minipage} \\
\midrule\noalign{}
\endhead
\bottomrule\noalign{}
\endlastfoot
\begin{minipage}[t]{\linewidth}\raggedright
🔵 Grundlagen \&

Definition

\hfill\break
\strut
\end{minipage} & Bindungsdynamik CTMM & Kap. \textbf{\textcolor{ctmmBlue}{\1}} \\
\begin{minipage}[t]{\linewidth}\raggedright
🟢 Prävention \&

Alltag

\hfill\break
\strut
\end{minipage} & Wertekompass vereint, Safe rules WG-Modell & Kap. \textbf{\textcolor{ctmmBlue}{\1}} \\
🟠 Eskalation

sichern & \begin{minipage}[t]{\linewidth}\raggedright
Trigger Notfallkarten, Ritual Workbook, Tool 26 Co-Regulation

\hfill\break
\strut
\end{minipage} & Kap. \textbf{\textcolor{ctmmBlue}{\1}} \\
🔴 Notfallplanung

\& Reaktion & \begin{minipage}[t]{\linewidth}\raggedright
Trigger Forschungstagebuch, Krisenprotokoll

\hfill\break
\strut
\end{minipage} & Kap. \textbf{\textcolor{ctmmBlue}{\1}} \\
🟣 Reflexion \&

Fortschritt & Trigger Tagebuch extended, Vision board Klartext & Kap. \textbf{\textcolor{ctmmBlue}{\1}}, Tool \textbf{\textcolor{ctmmBlue}{\1}} \\
\end{longtable}

\hfill\break
\hfill\break

\hypertarget{schluxfcsselsuxe4tze-fuxfcr-sichere-bindung}{%
\subsection[ 💬 \ul{\textbf{\textcolor{ctmmBlue}{\1}}}]{\texorpdfstring{\protect\hypertarget{schluxfcsselsuxe4tze-fuxfcr-sichere-bindung}{}{} 💬 \ul{\textbf{\textcolor{ctmmBlue}{\1}}}}{ 💬 SCHLÜSSELSÄTZE FÜR SICHERE BINDUNG}}\label{schluxfcsselsuxe4tze-fuxfcr-sichere-bindung}}

\begin{itemize}[label=\textcolor{ctmmOrange}{\faArrowRight}]
\item
  „Ich ziehe mich zurück, aber ich komme wieder.''
\item
  „Meine Grenze ist nicht gegen dich, sondern für mich.''
\item
  „Wir reden darüber -- aber nicht jetzt, nicht in Panik.''
\item
  „Ich sehe dich, auch wenn ich mich gerade nicht spürbar zeige.''
\end{itemize}

\hfill\break
\hfill\break

\hypertarget{zusammenspiel-der-module}{%
\subsection[🔗 \ul{\textbf{\textcolor{ctmmBlue}{\1}}}]{\texorpdfstring{\protect\hypertarget{zusammenspiel-der-module}{}{}🔗 \ul{\textbf{\textcolor{ctmmBlue}{\1}}}}{🔗 ZUSAMMENSPIEL DER MODULE}}\label{zusammenspiel-der-module}}

\begin{itemize}[label=\textcolor{ctmmOrange}{\faArrowRight}]
\item
  \textbf{\textcolor{ctmmBlue}{\1}} \textbf{\textcolor{ctmmBlue}{\1}} \textbf{\textcolor{ctmmBlue}{\1}} \textbf{\textcolor{ctmmBlue}{\1}} \textbf{\textcolor{ctmmBlue}{\1}} = Sicherheitskreis
\item
  \textbf{\textcolor{ctmmBlue}{\1}} \textbf{\textcolor{ctmmBlue}{\1}} \textbf{\textcolor{ctmmBlue}{\1}} \textbf{\textcolor{ctmmBlue}{\1}} \textbf{\textcolor{ctmmBlue}{\1}} = Stabilitätsanker
\item
  \textbf{\textcolor{ctmmBlue}{\1}} \textbf{\textcolor{ctmmBlue}{\1}} \textbf{\textcolor{ctmmBlue}{\1}} \textbf{\textcolor{ctmmBlue}{\1}} \textbf{\textcolor{ctmmBlue}{\1}} = Entwicklungskompass
\item
\end{itemize}

📎 \textbf{\textcolor{ctmmBlue}{\1}}

\hfill\break
\hfill\break

📤 \textcolor{ctmmGreen}{\textit{\1}}

\textcolor{ctmmGreen}{\textit{\1}}
