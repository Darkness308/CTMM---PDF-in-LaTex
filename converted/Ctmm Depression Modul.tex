% CTMM Therapy Module: Ctmm Depression Modul
% Converted from Word document: Ctmm Depression Modul.docx
% Auto-generated by document_converter.py

% Options for packages loaded elsewhere
\PassOptionsToPackage{unicode}{hyperref}
\PassOptionsToPackage{hyphens}{url}
%
\documentclass[
]{article}
\usepackage{amsmath,amssymb}
\usepackage{iftex}
\ifPDFTeX
  \usepackage[T1]{fontenc}
  \usepackage[utf8]{inputenc}
  \usepackage{textcomp} % provide euro and other symbols
\else % if luatex or xetex
  \usepackage{unicode-math} % this also loads fontspec
  \defaultfontfeatures{Scale=MatchLowercase}
  \defaultfontfeatures[\rmfamily]{Ligatures=TeX,Scale=1}
\fi
\usepackage{lmodern}
\ifPDFTeX\else
  % xetex/luatex font selection
\fi
% Use upquote if available, for straight quotes in verbatim environments
\IfFileExists{upquote.sty}{\usepackage{upquote}}{}
\IfFileExists{microtype.sty}{% use microtype if available
  \usepackage[]{microtype}
  \UseMicrotypeSet[protrusion]{basicmath} % disable protrusion for tt fonts
}{}
\makeatletter
\@ifundefined{KOMAClassName}{% if non-KOMA class
  \IfFileExists{parskip.sty}{%
    \usepackage{parskip}
  }{% else
    \setlength{\parindent}{0pt}
    \setlength{\parskip}{6pt plus 2pt minus 1pt}}
}{% if KOMA class
  \KOMAoptions{parskip=half}}
\makeatother
\usepackage{xcolor}
\usepackage{longtable,booktabs,array}
\usepackage{calc} % for calculating minipage widths
% Correct order of tables after \paragraph or \subparagraph
\usepackage{etoolbox}
\makeatletter
\patchcmd\longtable{\par}{\if@noskipsec\mbox{}\fi\par}{}{}
\makeatother
% Allow footnotes in longtable head/foot
\IfFileExists{footnotehyper.sty}{\usepackage{footnotehyper}}{\usepackage{footnote}}
\makesavenoteenv{longtable}
\setlength{\emergencystretch}{3em} % prevent overfull lines
\providecommand{\tightlist}{%
  \setlength{\itemsep}{0pt}\setlength{\parskip}{0pt}}
\setcounter{secnumdepth}{-\maxdimen} % remove section numbering
\ifLuaTeX
  \usepackage{selnolig}  % disable illegal ligatures
\fi
\IfFileExists{bookmark.sty}{\usepackage{bookmark}}{\usepackage{hyperref}}
\IfFileExists{xurl.sty}{\usepackage{xurl}}{} % add URL line breaks if available
\urlstyle{same}
\hypersetup{
  hidelinks,
  pdfcreator={LaTeX via pandoc}}

\author{}
\date{}



\hypertarget{ctmm-modul-depression-stimmungstief-fruxfchwarnung-handlungssicherheit}{%
\section{🌧️ CTMM-MODUL: DEPRESSION \& STIMMUNGSTIEF -- FRÜHWARNUNG \& HANDLUNGSSICHERHEIT}\label{ctmm-modul-depression-stimmungstief-fruxfchwarnung-handlungssicherheit}}

\begin{quote}
🧠 \textbf{Worum geht's hier -- für Freunde \& Schüler?}\\
Depression wirkt leise, aber mächtig. Dieses Modul hilft dir (und deinem Umfeld), erste Anzeichen zu erkennen, Eskalationen vorzubeugen und gemeinsam handlungsfähig zu bleiben. Es geht nicht um Diagnose -- sondern um Sicherheit, Struktur und Mitgefühl.
\end{quote}

🧩 \emph{Verknüpfbar mit Tool 23 (Trigger), Tool 26 (Ko-Regulation), Matching-Tracker, Safe Words \& Rückkehrritualen}

\hypertarget{kapitelzuordnung-im-ctmm-system}{%
\subsection{📘 KAPITELZUORDNUNG IM CTMM-SYSTEM}\label{kapitelzuordnung-im-ctmm-system}}

\begin{itemize}
\tightlist
\item
  \texttt{Kap.\ 2.5} → Selbstwahrnehmung \& Antrieb
\item
  \texttt{Kap.\ 3.2} → Isolation \& Rückzug
\item
  \texttt{Kap.\ 4.4} → Überforderung, Erschöpfung \& Schutz
\item
  \texttt{Kap.\ 5.2/5.3} → Trigger-Frühzeichen \& Matching
\end{itemize}

\hypertarget{farbcode-systemnavigation}{%
\subsection{🎨 FARBCODE \& SYSTEMNAVIGATION}\label{farbcode-systemnavigation}}

\begin{longtable}[]{@{}
  >{\raggedright\arraybackslash}p{(\columnwidth - 4\tabcolsep) * \real{0.0913}}
  >{\raggedright\arraybackslash}p{(\columnwidth - 4\tabcolsep) * \real{0.2943}}
  >{\raggedright\arraybackslash}p{(\columnwidth - 4\tabcolsep) * \real{0.6143}}@{}}
\toprule\noalign{}
\begin{minipage}[b]{\linewidth}\raggedright
Farbe
\end{minipage} & \begin{minipage}[b]{\linewidth}\raggedright
Phase
\end{minipage} & \begin{minipage}[b]{\linewidth}\raggedright
Verknüpfte Module
\end{minipage} \\
\midrule\noalign{}
\endhead
\bottomrule\noalign{}
\endlastfoot
🔵 & Beobachtung & \texttt{tool\_23\_triggermanagement} \\
🔴 & Eskalation / Rückzug & \texttt{trigger\_notfallkarten} \\
🟠 & Stimmung stabilisieren & \texttt{tool\_24\_skills\_sie}, \texttt{tool\_21} \\
🟣 & Rückkehr \& Integration & \texttt{ritual\_workbook}, \texttt{bindungsleitfaden\_ctmm} \\
\end{longtable}

\hypertarget{fruxfchwarnzeichen-bei-depression}{%
\subsection{🧩 FRÜHWARNZEICHEN BEI DEPRESSION}\label{fruxfchwarnzeichen-bei-depression}}

\begin{longtable}[]{@{}
  >{\raggedright\arraybackslash}p{(\columnwidth - 2\tabcolsep) * \real{0.4722}}
  >{\raggedright\arraybackslash}p{(\columnwidth - 2\tabcolsep) * \real{0.5278}}@{}}
\toprule\noalign{}
\begin{minipage}[b]{\linewidth}\raggedright
Innerlich bei mir
\end{minipage} & \begin{minipage}[b]{\linewidth}\raggedright
Sichtbar für andere
\end{minipage} \\
\midrule\noalign{}
\endhead
\bottomrule\noalign{}
\endlastfoot
Ich will niemanden sehen & Rückzug, Zimmer bleibt dunkel \\
Ich empfinde nichts mehr & Keine Freude, kein Interesse \\
Alles wirkt anstrengend & Langsamer Gang, leise Stimme \\
Ich fühle mich wertlos & Selbstabwertung, Vermeidung \\
Ich denke, ich bin eine Last & Schuldgefühle, Isolation \\
📝 \_\_\_\_\_\_\_\_\_\_\_\_\_\_\_\_\_\_\_\_\_\_\_\_ & 📝 \_\_\_\_\_\_\_\_\_\_\_\_\_\_\_\_\_\_\_\_\_\_\_\_\_\_\_ \\
\end{longtable}

\begin{longtable}[]{@{}
  >{\raggedright\arraybackslash}p{(\columnwidth - 2\tabcolsep) * \real{0.4722}}
  >{\raggedright\arraybackslash}p{(\columnwidth - 2\tabcolsep) * \real{0.5278}}@{}}
\toprule\noalign{}
\begin{minipage}[b]{\linewidth}\raggedright
📝 \_\_\_\_\_\_\_\_\_\_\_\_\_\_\_\_\_\_\_\_\_\_\_\_
\end{minipage} & \begin{minipage}[b]{\linewidth}\raggedright
📝 \_\_\_\_\_\_\_\_\_\_\_\_\_\_\_\_\_\_\_\_\_\_\_\_\_\_\_
\end{minipage} \\
\midrule\noalign{}
\endhead
\bottomrule\noalign{}
\endlastfoot
\end{longtable}

\hypertarget{was-ich-dann-brauche-fuxfcr-mich-selbst-von-euch}{%
\subsection{🧭 WAS ICH DANN BRAUCHE (FÜR MICH SELBST / VON EUCH)}\label{was-ich-dann-brauche-fuxfcr-mich-selbst-von-euch}}

\begin{itemize}
\tightlist
\item
  🧠 „Sag mir kleine Schritte, keine Lösungen.``
\end{itemize}

\begin{itemize}
\tightlist
\item
  💬 „Sprich mit mir -- auch wenn ich schweige.``
\end{itemize}

\begin{itemize}
\tightlist
\item
  🎧 „Nimm meine Anzeichen ernst -- aber überforder mich nicht.``
\end{itemize}

\begin{itemize}
\tightlist
\item
  🧍 „Gib mir Raum ohne mich allein zu lassen.``
\end{itemize}

\begin{itemize}
\tightlist
\item
  📎 „Mach einen Notfall-Plan sichtbar: Was tun bei völliger Erschöpfung?{}``
\end{itemize}

\hypertarget{depression-notfallkasten-kompakt-fuxfcr-mich-und-euch}{%
\subsection{🧰 DEPRESSION-NOTFALLKASTEN (KOMPAKT, FÜR MICH UND EUCH)}\label{depression-notfallkasten-kompakt-fuxfcr-mich-und-euch}}

\begin{longtable}[]{@{}
  >{\raggedright\arraybackslash}p{(\columnwidth - 4\tabcolsep) * \real{0.2535}}
  >{\raggedright\arraybackslash}p{(\columnwidth - 4\tabcolsep) * \real{0.4463}}
  >{\raggedright\arraybackslash}p{(\columnwidth - 4\tabcolsep) * \real{0.3002}}@{}}
\toprule\noalign{}
\begin{minipage}[b]{\linewidth}\raggedright
Bereich
\end{minipage} & \begin{minipage}[b]{\linewidth}\raggedright
Was wirkt manchmal / oft
\end{minipage} & \begin{minipage}[b]{\linewidth}\raggedright
Was ist kontraproduktiv
\end{minipage} \\
\midrule\noalign{}
\endhead
\bottomrule\noalign{}
\endlastfoot
Kommunikation & 💬 Kurze Sätze, echtes Interesse & ❌ Ratschläge, Optimismusfloskeln \\
Körper / Aktivierung & 🚶 Bewegung, Licht, Lieblingslied & ❌ Druck, Sportbefehle \\
Rückzugsphasen & 🧍„Ich bin bei dir -- leise.`` & ❌ Kontaktabbruch \\
Beziehungssicherheit & 🧠 Erinner mich an unser Band / Vertrag & ❌ Schuldzuweisungen, „Reiß dich zusammen`` \\
Orientierung & 🗓️ Tagesstruktur gemeinsam checken & ❌ „Was willst du jetzt machen?{}`` \\
Symbolische Hilfe & 🔗 Safe-Word oder visuelles Zeichen einsetzen & ❌ „So schlimm kann's doch nicht sein`` \\
\end{longtable}

\begin{quote}
📎 Dieses Modul kann im Alltag oder in der Klinik verwendet werden. Es ist auch als Aushang, Printmodul oder Gesprächsstarter einsetzbar -- speziell in belasteten WG-Situationen oder Paarbeziehungen.
\end{quote}


