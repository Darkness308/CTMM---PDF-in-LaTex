\hypertarget{matching-matrix-wochenlogik}{%
\section{\texorpdfstring{\textbf{📊 MATCHING MATRIX WOCHENLOGIK}}{📊 MATCHING MATRIX WOCHENLOGIK}}\label{matching-matrix-wochenlogik}

\begin{quote}
🧠 \textbf{\ul{Worum geht's hier -- für Freunde?}\\
Die Wochenlogik hilft dabei, Beziehungsmuster über längere Zeiträume zu erkennen. Statt nur einzelne Trigger zu betrachten, schauen wir auf wiederkehrende Zyklen, Wochenmuster und Rhythmen in der Partnerschaft.
\end{quote}

🧩 \textbf{Systematische Wochenanalyse für neurodiverse Paare mit Fokus auf Mustererkennung}

\hypertarget{kapitelzuordnung-im-ctmm-system-wochenlogik}{%
\subsection{Kapitelzuordnung im CTMM-System}}\label{kapitelzuordnung-im-ctmm-system-wochenlogik}

\begin{itemize}
\tightlist
\item
  \texttt{Kap.\ 1.3} → Bindungsdynamik und Wochenmuster
\item
  \texttt{Kap.\ 2.6} → Team-Regeln und Rhythmusfindung
\item
  \texttt{Kap.\ 4.1} → Langfristige Beziehungsarbeit
\item
  \texttt{Kap.\ 5.1} → Dokumentation und Reflexionstools
\end{itemize}

\hypertarget{wochentage-energielevel-matrix}{%
\subsection{1. Wochentage-Energielevel Matrix}}\label{wochentage-energielevel-matrix}

\textbf{Individuelle Energiemuster erkennen und dokumentieren:}

\begin{center}
\begin{tabular}{|p{2cm}|p{2.5cm}|p{2.5cm}|p{3cm}|p{3cm}|}
\hline
\textbf{Wochentag} & \textbf{ER: Energie (1-10)} & \textbf{SIE: Energie (1-10)} & \textbf{Gemeinsame Aktivitäten} & \textbf{Herausforderungen} \\
\hline
Montag & \rule{2cm}{0.4pt} & \rule{2cm}{0.4pt} & \rule{2.5cm}{0.4pt} & \rule{2.5cm}{0.4pt} \\
\hline
Dienstag & \rule{2cm}{0.4pt} & \rule{2cm}{0.4pt} & \rule{2.5cm}{0.4pt} & \rule{2.5cm}{0.4pt} \\
\hline
Mittwoch & \rule{2cm}{0.4pt} & \rule{2cm}{0.4pt} & \rule{2.5cm}{0.4pt} & \rule{2.5cm}{0.4pt} \\
\hline
Donnerstag & \rule{2cm}{0.4pt} & \rule{2cm}{0.4pt} & \rule{2.5cm}{0.4pt} & \rule{2.5cm}{0.4pt} \\
\hline
Freitag & \rule{2cm}{0.4pt} & \rule{2cm}{0.4pt} & \rule{2.5cm}{0.4pt} & \rule{2.5cm}{0.4pt} \\
\hline
Samstag & \rule{2cm}{0.4pt} & \rule{2cm}{0.4pt} & \rule{2.5cm}{0.4pt} & \rule{2.5cm}{0.4pt} \\
\hline
Sonntag & \rule{2cm}{0.4pt} & \rule{2cm}{0.4pt} & \rule{2.5cm}{0.4pt} & \rule{2.5cm}{0.4pt} \\
\hline
\end{tabular}
\end{center}

\hypertarget{kommunikationsmuster-wochenanalyse}{%
\subsection{2. Kommunikationsmuster Wochenanalyse}}\label{kommunikationsmuster-wochenanalyse}

\textbf{Wann kommunizieren wir gut, wann schwierig?}

\begin{center}
\begin{tabular}{|p{2cm}|p{3cm}|p{3cm}|p{4cm}|p{2cm}|}
\hline
\textbf{Zeitraum} & \textbf{Positive Gespräche} & \textbf{Konflikte/Missverständnisse} & \textbf{Auslöser/Trigger} & \textbf{Qualität (1-10)} \\
\hline
Mo-Di Abend & \rule{2.5cm}{0.4pt} & \rule{2.5cm}{0.4pt} & \rule{3.5cm}{0.4pt} & \rule{1.5cm}{0.4pt} \\
\hline
Mi-Do Abend & \rule{2.5cm}{0.4pt} & \rule{2.5cm}{0.4pt} & \rule{3.5cm}{0.4pt} & \rule{1.5cm}{0.4pt} \\
\hline
Fr-Sa Abend & \rule{2.5cm}{0.4pt} & \rule{2.5cm}{0.4pt} & \rule{3.5cm}{0.4pt} & \rule{1.5cm}{0.4pt} \\
\hline
So-Mo Abend & \rule{2.5cm}{0.4pt} & \rule{2.5cm}{0.4pt} & \rule{3.5cm}{0.4pt} & \rule{1.5cm}{0.4pt} \\
\hline
Wochenende & \rule{2.5cm}{0.4pt} & \rule{2.5cm}{0.4pt} & \rule{3.5cm}{0.4pt} & \rule{1.5cm}{0.4pt} \\
\hline
\end{tabular}
\end{center}

\hypertarget{naehe-distanz-rhythmus}{%
\subsection{3. Nähe-Distanz-Rhythmus}}\label{naehe-distanz-rhythmus}

\textbf{Bedürfnisse nach Nähe und Rückzug im Wochenverlauf:}

\begin{center}
\begin{tabular}{|p{2.5cm}|p{2.5cm}|p{2.5cm}|p{2.5cm}|p{3cm}|}
\hline
\textbf{Tageszeit} & \textbf{ER: Nähebedürfnis} & \textbf{SIE: Nähebedürfnis} & \textbf{Gemeinsame Zeit} & \textbf{Rückzugspausen} \\
\hline
Morgens & Hoch / Mittel / Niedrig & Hoch / Mittel / Niedrig & \rule{2cm}{0.4pt} & \rule{2.5cm}{0.4pt} \\
\hline
Mittags & Hoch / Mittel / Niedrig & Hoch / Mittel / Niedrig & \rule{2cm}{0.4pt} & \rule{2.5cm}{0.4pt} \\
\hline
Nachmittags & Hoch / Mittel / Niedrig & Hoch / Mittel / Niedrig & \rule{2cm}{0.4pt} & \rule{2.5cm}{0.4pt} \\
\hline
Abends & Hoch / Mittel / Niedrig & Hoch / Mittel / Niedrig & \rule{2cm}{0.4pt} & \rule{2.5cm}{0.4pt} \\
\hline
Nachts & Hoch / Mittel / Niedrig & Hoch / Mittel / Niedrig & \rule{2cm}{0.4pt} & \rule{2.5cm}{0.4pt} \\
\hline
\end{tabular}
\end{center}

\hypertarget{stressfaktoren-und-unterstuetzung}{%
\subsection{4. Stressfaktoren und Unterstützung}}\label{stressfaktoren-und-unterstuetzung}

\textbf{Wöchentliche Belastungen und Ressourcen identifizieren:}

\begin{center}
\begin{tabular}{|p{3cm}|p{4cm}|p{4cm}|p{3cm}|}
\hline
\textbf{Bereich} & \textbf{Typische Stressfaktoren} & \textbf{Unterstützungsmöglichkeiten} & \textbf{Priorität (1-3)} \\
\hline
Arbeit/Beruf & \rule{3.5cm}{0.4pt} & \rule{3.5cm}{0.4pt} & \rule{2cm}{0.4pt} \\
\hline
Haushalt/Organisation & \rule{3.5cm}{0.4pt} & \rule{3.5cm}{0.4pt} & \rule{2cm}{0.4pt} \\
\hline
Soziale Kontakte & \rule{3.5cm}{0.4pt} & \rule{3.5cm}{0.4pt} & \rule{2cm}{0.4pt} \\
\hline
Familie/Verwandtschaft & \rule{3.5cm}{0.4pt} & \rule{3.5cm}{0.4pt} & \rule{2cm}{0.4pt} \\
\hline
Gesundheit/Therapie & \rule{3.5cm}{0.4pt} & \rule{3.5cm}{0.4pt} & \rule{2cm}{0.4pt} \\
\hline
Freizeit/Hobbys & \rule{3.5cm}{0.4pt} & \rule{3.5cm}{0.4pt} & \rule{2cm}{0.4pt} \\
\hline
\end{tabular}
\end{center}

\hypertarget{wochen-reflexion-und-planung}{%
\subsection{5. Wochen-Reflexion und Planung}}\label{wochen-reflexion-und-planung}

\textbf{Wochenrückblick (jeden Sonntag):}

\begin{itemize}
\tightlist
\item
  Was lief gut in unserer Beziehung diese Woche? \rule{8cm}{0.4pt}
\item
  Welche Herausforderungen gab es? \rule{8cm}{0.4pt}
\item
  Welche Muster erkennen wir? \rule{8cm}{0.4pt}
\item
  Was möchten wir nächste Woche anders machen? \rule{8cm}{0.4pt}
\end{itemize}

\textbf{Wochenplanung (jeden Montag):}

\begin{itemize}
\tightlist
\item
  Wann haben wir Zeit füreinander? \rule{8cm}{0.4pt}
\item
  Welche stressigen Termine stehen an? \rule{8cm}{0.4pt}
\item
  Wie können wir uns gegenseitig unterstützen? \rule{8cm}{0.4pt}
\item
  Welche Entspannungszeiten planen wir ein? \rule{8cm}{0.4pt}
\end{itemize}

\hypertarget{muster-erkennungs-matrix}{%
\subsection{6. Muster-Erkennungs Matrix}}\label{muster-erkennungs-matrix}

\textbf{Langfristige Beziehungsmuster identifizieren:}

\begin{center}
\begin{tabular}{|p{3cm}|p{3cm}|p{3cm}|p{4cm}|}
\hline
\textbf{Beobachtetes Muster} & \textbf{Häufigkeit} & \textbf{Auslöser} & \textbf{Intervention/Lösung} \\
\hline
Montagsmüdigkeit führt zu Distanz & wöchentlich / monatlich & \rule{2.5cm}{0.4pt} & \rule{3.5cm}{0.4pt} \\
\hline
Wochenendstress durch Aktivitäten & wöchentlich / monatlich & \rule{2.5cm}{0.4pt} & \rule{3.5cm}{0.4pt} \\
\hline
Mitte der Woche: gute Kommunikation & wöchentlich / monatlich & \rule{2.5cm}{0.4pt} & \rule{3.5cm}{0.4pt} \\
\hline
Freitagabend: Entspannung schwierig & wöchentlich / monatlich & \rule{2.5cm}{0.4pt} & \rule{3.5cm}{0.4pt} \\
\hline
Individuelles Muster: & & & \\
\rule{2.5cm}{0.4pt} & wöchentlich / monatlich & \rule{2.5cm}{0.4pt} & \rule{3.5cm}{0.4pt} \\
\hline
\rule{2.5cm}{0.4pt} & wöchentlich / monatlich & \rule{2.5cm}{0.4pt} & \rule{3.5cm}{0.4pt} \\
\hline
\end{tabular}
\end{center}

\hypertarget{wochenlogik-handlungsplan}{%
\subsection{7. Wochenlogik Handlungsplan}}\label{wochenlogik-handlungsplan}

\textbf{Basierend auf erkannten Mustern - konkrete Maßnahmen:}

\begin{enumerate}
\def\labelenumi{\arabic{enumi}.}
\tightlist
\item
  \textbf{Optimale Zeiten nutzen:} \rule{8cm}{0.4pt}
\item
  \textbf{Schwierige Zeiten vorbereiten:} \rule{8cm}{0.4pt}
\item
  \textbf{Regelmäßige Check-ins:} \rule{8cm}{0.4pt}
\item
  \textbf{Flexible Anpassungen:} \rule{8cm}{0.4pt}
\item
  \textbf{Notfallplan für stressige Wochen:} \rule{8cm}{0.4pt}
\end{enumerate}

\hypertarget{navigation-und-verweise-wochenlogik}{%
\subsection{Navigation und Verweise}}\label{navigation-und-verweise-wochenlogik}

\begin{itemize}
\tightlist
\item
  ➡️ \textbf{Weiter zu:} Matching Matrix Trigger-Reaktion-Intervention
\item
  ⬅️ \textbf{Zurück zu:} Bindungsleitfaden für neurodiverse Paare
\item
  🔗 \textbf{Ergänzend:} Co-Regulation \& Gemeinsame Stärkung
\item
  📖 \textbf{Vertiefung:} Therapiekoordination und Langzeitplanung
\end{itemize}

\textbf{🎯 Ziel:} Wöchentliche Rhythmen und Muster in der Beziehung erkennen, um präventiv zu handeln und die Partnerschaft nachhaltig zu stärken.