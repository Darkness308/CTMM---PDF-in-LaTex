% CTMM Therapy Module: Matching Matrix Wochenlogik
% Converted from Word document: Matching Matrix Wochenlogik.docx
% Auto-generated by document_converter.py

% Options for packages loaded elsewhere
\PassOptionsToPackage{unicode}{hyperref}
\PassOptionsToPackage{hyphens}{url}
%
\documentclass[
]{article}
\usepackage{amsmath,amssymb}
\usepackage{iftex}
\ifPDFTeX
  \usepackage[T1]{fontenc}
  \usepackage[utf8]{inputenc}
  \usepackage{textcomp} % provide euro and other symbols
\else % if luatex or xetex
  \usepackage{unicode-math} % this also loads fontspec
  \defaultfontfeatures{Scale=MatchLowercase}
  \defaultfontfeatures[\rmfamily]{Ligatures=TeX,Scale=1}
\fi
\usepackage{lmodern}
\ifPDFTeX\else
  % xetex/luatex font selection
\fi
% Use upquote if available, for straight quotes in verbatim environments
\IfFileExists{upquote.sty}{\usepackage{upquote}}{}
\IfFileExists{microtype.sty}{% use microtype if available
  \usepackage[]{microtype}
  \UseMicrotypeSet[protrusion]{basicmath} % disable protrusion for tt fonts
}{}
\makeatletter
\@ifundefined{KOMAClassName}{% if non-KOMA class
  \IfFileExists{parskip.sty}{%
    \usepackage{parskip}
  }{% else
    \setlength{\parindent}{0pt}
    \setlength{\parskip}{6pt plus 2pt minus 1pt}}
}{% if KOMA class
  \KOMAoptions{parskip=half}}
\makeatother
\usepackage{xcolor}
\usepackage{longtable,booktabs,array}
\usepackage{calc} % for calculating minipage widths
% Correct order of tables after \paragraph or \subparagraph
\usepackage{etoolbox}
\makeatletter
\patchcmd\longtable{\par}{\if@noskipsec\mbox{}\fi\par}{}{}
\makeatother
% Allow footnotes in longtable head/foot
\IfFileExists{footnotehyper.sty}{\usepackage{footnotehyper}}{\usepackage{footnote}}
\makesavenoteenv{longtable}
\setlength{\emergencystretch}{3em} % prevent overfull lines
\providecommand{\tightlist}{%
  \setlength{\itemsep}{0pt}\setlength{\parskip}{0pt}}
\setcounter{secnumdepth}{-\maxdimen} % remove section numbering
\ifLuaTeX
  \usepackage{selnolig}  % disable illegal ligatures
\fi
\IfFileExists{bookmark.sty}{\usepackage{bookmark}}{\usepackage{hyperref}}
\IfFileExists{xurl.sty}{\usepackage{xurl}}{} % add URL line breaks if available
\urlstyle{same}
\hypersetup{
  hidelinks,
  pdfcreator={LaTeX via pandoc}}

\author{}
\date{}



\hypertarget{matching---matrix-wochenlogik}{%
\section{\texorpdfstring{🧩 \textbf{MATCHING - MATRIX -- WOCHENLOGIK:}}{🧩 MATCHING - MATRIX -- WOCHENLOGIK:}}\label{matching---matrix-wochenlogik}}

\hypertarget{trigger-reaktion-hilfe}{%
\section{\texorpdfstring{\textbf{TRIGGER ↔ REAKTION ↔ HILFE}}{TRIGGER ↔ REAKTION ↔ HILFE}}\label{trigger-reaktion-hilfe}}

\hypertarget{ctmm-modul}{%
\section{\texorpdfstring{\textbf{(CTMM-MODUL)}}{(CTMM-MODUL)}}\label{ctmm-modul}}

\begin{quote}
🧠 \textbf{Worum geht's hier -- für Freunde ?}

Dieses Matching-Werkzeug ergänzt den Tracker.

Es zeigt typische Reiz-Reaktionsmuster über mehrere Tage.

Ziel ist, zu erkennen:
\end{quote}

\begin{itemize}
\item
  \begin{quote}
  \textbf{Wann wiederholt sich etwas?}
  \end{quote}
\item
  \begin{quote}
  \textbf{Wann helfen bestimmte Strategien besser?}
  \end{quote}
\item
  \begin{quote}
  \textbf{Wer braucht was -- wann?}
  \end{quote}
\end{itemize}

🧩 \textbf{Modul zur Wochenauswertung:}

\begin{itemize}
\item
  \textbf{Trigger erkennen}
\item
  \textbf{Reaktionen vergleichen}
\item
  \textbf{Muster entschlüsseln -- aus beiden Perspektiven}
\end{itemize}

\hypertarget{kapitelzuordnung-im-ctmm-system}{%
\subsection{\texorpdfstring{📘 \textbf{KAPITELZUORDNUNG IM CTMM-SYSTEM}}{📘 KAPITELZUORDNUNG IM CTMM-SYSTEM}}\label{kapitelzuordnung-im-ctmm-system}}

\begin{itemize}
\tightlist
\item
  \texttt{Kap.\ }\texttt{1} → Bindungslogik \& Systemverständnis
\item
  \texttt{Kap.\ }\texttt{2.6} → Ko-Regulation, Selbst-/Fremdwahrnehmung
\item
  \texttt{Kap.\ }\texttt{3.1\ –\ 3.4} → Reaktionsformen, Rückzug, Disso, Wut
\item
  \texttt{Kap.\ }\texttt{5.2\ –\ 5.3} → Triggeranalyse, Matching-Reaktionen
\end{itemize}

\hypertarget{section}{%
\subsection{}\label{section}}

\hypertarget{section-1}{%
\subsection{}\label{section-1}}

\hypertarget{farbcode-systemnavigation}{%
\subsection{\texorpdfstring{\textbf{🎨 FARBCODE \& SYSTEMNAVIGATION}}{🎨 FARBCODE \& SYSTEMNAVIGATION}}\label{farbcode-systemnavigation}}

\begin{longtable}[]{@{}
  >{\raggedright\arraybackslash}p{(\columnwidth - 4\tabcolsep) * \real{0.1073}}
  >{\raggedright\arraybackslash}p{(\columnwidth - 4\tabcolsep) * \real{0.4323}}
  >{\raggedright\arraybackslash}p{(\columnwidth - 4\tabcolsep) * \real{0.4605}}@{}}
\toprule\noalign{}
\begin{minipage}[b]{\linewidth}\raggedright
\textbf{Farbe}
\end{minipage} & \begin{minipage}[b]{\linewidth}\raggedright
\textbf{Phase}
\end{minipage} & \begin{minipage}[b]{\linewidth}\raggedright
\textbf{Kontextmodule}
\end{minipage} \\
\midrule\noalign{}
\endhead
\bottomrule\noalign{}
\endlastfoot
🔵 & Mustererkennung & \texttt{T}\texttt{rigger}\texttt{\ F}\texttt{orschungstagebuch} \\
🟠 & Paar-Reaktion/Matching & \texttt{T}\texttt{ool}\texttt{\ }\texttt{26}\texttt{\ C}\texttt{o}\texttt{-}\texttt{regulation} \\
🟣 & Wöchentliche Auswertung & \texttt{W}\texttt{erte}\texttt{k}\texttt{ompass}\texttt{\ }\texttt{vereint} \\
\end{longtable}

\begin{quote}
\textbf{📎 Diese Farben helfen beim Rückblick auf Paar-Muster, Teamdynamik und Wiederholungsschleifen}
\end{quote}

\hypertarget{matching-wochenplan-basis-matrix}{%
\subsection{\texorpdfstring{📆 \textbf{MATCHING-WOCHENPLAN (BASIS-MATRIX)}}{📆 MATCHING-WOCHENPLAN (BASIS-MATRIX)}}\label{matching-wochenplan-basis-matrix}}

\begin{longtable}[]{@{}
  >{\raggedright\arraybackslash}p{(\columnwidth - 10\tabcolsep) * \real{0.0295}}
  >{\raggedright\arraybackslash}p{(\columnwidth - 10\tabcolsep) * \real{0.2561}}
  >{\raggedright\arraybackslash}p{(\columnwidth - 10\tabcolsep) * \real{0.1608}}
  >{\raggedright\arraybackslash}p{(\columnwidth - 10\tabcolsep) * \real{0.1726}}
  >{\raggedright\arraybackslash}p{(\columnwidth - 10\tabcolsep) * \real{0.2084}}
  >{\raggedright\arraybackslash}p{(\columnwidth - 10\tabcolsep) * \real{0.1726}}@{}}
\toprule\noalign{}
\begin{minipage}[b]{\linewidth}\raggedright
\emph{\textbf{Tag}}
\end{minipage} & \begin{minipage}[b]{\linewidth}\raggedright
\emph{\textbf{Trigger (Wort / Aktion)}}
\end{minipage} & \begin{minipage}[b]{\linewidth}\raggedright
\emph{\textbf{Reaktion ER}}
\end{minipage} & \begin{minipage}[b]{\linewidth}\raggedright
\emph{\textbf{Reaktion SIE}}
\end{minipage} & \begin{minipage}[b]{\linewidth}\raggedright
\emph{\textbf{Was hätte geholfen?}}
\end{minipage} & \begin{minipage}[b]{\linewidth}\raggedright
\emph{\textbf{Wurde Tool genutzt?}}

\emph{\textbf{Ja Nein}}
\end{minipage} \\
\midrule\noalign{}
\endhead
\bottomrule\noalign{}
\endlastfoot
\textbf{Mo} & 📝 \_\_\_\_\_\_\_\_\_\_\_\_\_\_\_\_\_\_\_\_ & 📝 \_\_\_\_\_\_\_\_\_\_\_\_ & 📝 \_\_\_\_\_\_\_\_\_\_\_\_\_ & 📝 \_\_\_\_\_\_\_\_\_\_\_\_\_\_\_\_ & ✅ ☐ ❌ ☐ \\
\textbf{Di} & 📝 \_\_\_\_\_\_\_\_\_\_\_\_\_\_\_\_\_\_\_\_ & 📝 \_\_\_\_\_\_\_\_\_\_\_\_ & 📝 \_\_\_\_\_\_\_\_\_\_\_\_\_ & 📝 \_\_\_\_\_\_\_\_\_\_\_\_\_\_\_\_ & ✅ ☐ ❌ ☐ \\
\textbf{Mi} & 📝 \_\_\_\_\_\_\_\_\_\_\_\_\_\_\_\_\_\_\_\_ & 📝 \_\_\_\_\_\_\_\_\_\_\_\_ & 📝 \_\_\_\_\_\_\_\_\_\_\_\_\_ & 📝 \_\_\_\_\_\_\_\_\_\_\_\_\_\_\_\_ & ✅ ☐ ❌ ☐ \\
\textbf{Do} & 📝 \_\_\_\_\_\_\_\_\_\_\_\_\_\_\_\_\_\_\_\_ & 📝 \_\_\_\_\_\_\_\_\_\_\_\_ & 📝 \_\_\_\_\_\_\_\_\_\_\_\_\_ & 📝 \_\_\_\_\_\_\_\_\_\_\_\_\_\_\_\_ & ✅ ☐ ❌ ☐ \\
\textbf{Fr} & 📝 \_\_\_\_\_\_\_\_\_\_\_\_\_\_\_\_\_\_\_\_ & 📝 \_\_\_\_\_\_\_\_\_\_\_\_ & 📝 \_\_\_\_\_\_\_\_\_\_\_\_\_ & 📝 \_\_\_\_\_\_\_\_\_\_\_\_\_\_\_\_ & ✅ ☐ ❌ ☐ \\
\textbf{Sa} & 📝 \_\_\_\_\_\_\_\_\_\_\_\_\_\_\_\_\_\_\_\_ & 📝 \_\_\_\_\_\_\_\_\_\_\_\_ & 📝 \_\_\_\_\_\_\_\_\_\_\_\_\_ & 📝 \_\_\_\_\_\_\_\_\_\_\_\_\_\_\_\_ & ✅ ☐ ❌ ☐ \\
\textbf{So} & 📝 \_\_\_\_\_\_\_\_\_\_\_\_\_\_\_\_\_\_\_\_ & 📝 \_\_\_\_\_\_\_\_\_\_\_\_ & 📝 \_\_\_\_\_\_\_\_\_\_\_\_\_ & 📝 \_\_\_\_\_\_\_\_\_\_\_\_\_\_\_\_ & ✅ ☐ ❌ ☐ \\
\end{longtable}

✏️ \textbf{REFLEXION AM WOCHENENDE}

\hypertarget{was-war-huxe4ufig-wiederkehrend}{%
\subsubsection{\texorpdfstring{\textbf{Was war häufig? Wiederkehrend?}}{Was war häufig? Wiederkehrend?}}\label{was-war-huxe4ufig-wiederkehrend}}

🧠 \_\_\_\_\_\_\_\_\_\_\_\_\_\_\_\_\_\_\_\_\_\_\_\_\_\_\_\_\_\_\_\_\_\_\_\_\_\_\_\_\_\_\_\_\_\_\_\_\_\_\_\_\_\_\_\_\_\_\_\_\_\_\_\_\_\_\_\_\_\_\_\_\_\_\_\_\_\_\_\_\_\_\_\_\_\_\_\_\_\_\_\_ \_\_\_\_\_\_\_\_\_\_\_\_\_\_\_\_\_\_\_\_\_\_\_\_\_\_\_\_\_\_\_\_\_\_\_\_\_\_\_\_\_\_\_\_\_\_\_\_\_\_\_\_\_\_\_\_\_\_\_\_\_\_\_\_\_\_\_\_\_\_\_\_\_\_\_\_\_\_\_\_\_\_\_\_\_\_\_\_\_\_\_\_\_\_\_

\hypertarget{was-hat-uxfcberraschend-gut-geholfen}{%
\subsubsection{\texorpdfstring{\textbf{Was hat überraschend gut geholfen?}}{Was hat überraschend gut geholfen?}}\label{was-hat-uxfcberraschend-gut-geholfen}}

🧠 \_\_\_\_\_\_\_\_\_\_\_\_\_\_\_\_\_\_\_\_\_\_\_\_\_\_\_\_\_\_\_\_\_\_\_\_\_\_\_\_\_\_\_\_\_\_\_\_\_\_\_\_\_\_\_\_\_\_\_\_\_\_\_\_\_\_\_\_\_\_\_\_\_\_\_\_\_\_\_\_\_\_\_\_\_\_\_\_\_\_\_\_ \_\_\_\_\_\_\_\_\_\_\_\_\_\_\_\_\_\_\_\_\_\_\_\_\_\_\_\_\_\_\_\_\_\_\_\_\_\_\_\_\_\_\_\_\_\_\_\_\_\_\_\_\_\_\_\_\_\_\_\_\_\_\_\_\_\_\_\_\_\_\_\_\_\_\_\_\_\_\_\_\_\_\_\_\_\_\_\_\_\_\_\_\_\_\_

\hypertarget{gab-es-stille-phasen-unerkannte-trigger}{%
\subsubsection{\texorpdfstring{\textbf{Gab es stille Phasen, unerkannte Trigger?}}{Gab es stille Phasen, unerkannte Trigger?}}\label{gab-es-stille-phasen-unerkannte-trigger}}

🧠 \_\_\_\_\_\_\_\_\_\_\_\_\_\_\_\_\_\_\_\_\_\_\_\_\_\_\_\_\_\_\_\_\_\_\_\_\_\_\_\_\_\_\_\_\_\_\_\_\_\_\_\_\_\_\_\_\_\_\_\_\_\_\_\_\_\_\_\_\_\_\_\_\_\_\_\_\_\_\_\_\_\_\_\_\_\_\_\_\_\_\_\_ \_\_\_\_\_\_\_\_\_\_\_\_\_\_\_\_\_\_\_\_\_\_\_\_\_\_\_\_\_\_\_\_\_\_\_\_\_\_\_\_\_\_\_\_\_\_\_\_\_\_\_\_\_\_\_\_\_\_\_\_\_\_\_\_\_\_\_\_\_\_\_\_\_\_\_\_\_\_\_\_\_\_\_\_\_\_\_\_\_\_\_\_\_\_\_

\hypertarget{ctmm-navigation}{%
\subsection{\texorpdfstring{\textbf{🧭 CTMM-NAVIGATION}}{🧭 CTMM-NAVIGATION}}\label{ctmm-navigation}}

\begin{itemize}
\tightlist
\item
  \texttt{ctmm\_tages\_tracker} ← Vorarbeit für Matching-Linie (\texttt{Kap.\ 5.2})
\item
  \texttt{tool\_26\_ko\_regulation} ← Sicht von außen, Handlungsoptionen (\texttt{Kap.\ 2.6})
\item
  \texttt{ritual\_workbook} ← Rückkehrrituale bei Wiederholung (\texttt{Kap.\ 3.4})
\item
  \texttt{werte\_kompass\_vereint} ← Gab es Wertebruch? Neue Werte sichtbar? (\texttt{Kap.\ 4.1})
\item
\end{itemize}

\begin{quote}
\textbf{📎 Die Matching-Wochenmatrix ist ein zentrales Coaching- und Therapieinstrument --}

\textbf{gerade bei Paaren mit viel Wiederholungsstress.}
\end{quote}


