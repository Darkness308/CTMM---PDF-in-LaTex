\hypertarget{ctmm-system}{%
\section{CTMM-System}\label{ctmm-system}}

Ein modulares LaTeX-Framework für Catch-Track-Map-Match Therapiematerialien.

\hypertarget{ueberblick}{%
\subsection{Überblick}\label{ueberblick}}

Dieses Repository enthält ein vollständiges LaTeX-System zur Erstellung von CTMM-Therapiedokumenten, einschließlich:
\begin{itemize}
\item Depression \\& Stimmungstief Module
\item Trigger-Management
\item Bindungsdynamik
\item Formularelemente für therapeutische Dokumentation
\end{itemize}

\hypertarget{verwendung}{%
\subsection{Verwendung}\label{verwendung}}

\begin{enumerate}
\def\labelenumi{\arabic{enumi}.}
\tightlist
\item
  Klone das Repository
\item
  Kompiliere main.tex mit einem LaTeX-Editor
\item
  Oder öffne das Projekt in einem GitHub Codespace
\end{enumerate}

\hypertarget{struktur}{%
\subsection{Struktur}\label{struktur}}

\begin{itemize}
\tightlist
\item
  \texttt{/style/} - Design-Dateien und gemeinsam verwendete Komponenten
\item
  \texttt{/modules/} - Individuelle CTMM-Module als separate .tex-Dateien
\item
  \texttt{/assets/} - Diagramme und visuelle Elemente
\end{itemize}

\hypertarget{anforderungen}{%
\subsection{Anforderungen}\label{anforderungen}}

\begin{itemize}
\tightlist
\item
  LaTeX-Installation mit TikZ und hyperref
\item
  Oder GitHub Codespace (vorkonfiguriert)
\end{itemize}

\hypertarget{latex-hinweise-fuer-entwickler}{%
\subsection{LaTeX-Hinweise für Entwickler}\label{latex-hinweise-fuer-entwickler}}

\textbf{CTMM Build System:}

Das Projekt verfügt über ein automatisches Build-System (\texttt{ctmm\_build.py}), das folgende Funktionen bietet:

\hypertarget{automatisierte-build-pruefung}{%
\subsubsection{Automatisierte Build-Prüfung}\label{automatisierte-build-pruefung}}

\begin{verbatim}
python3 ctmm_build.py
\end{verbatim}

Das Build-System:
\begin{enumerate}
\def\labelenumi{\arabic{enumi}.}
\tightlist
\item
  \textbf{Scannt main.tex} nach allen \verb|\usepackage{style/...}| und \verb|\input{modules/...}| Befehlen
\item
  \textbf{Prüft Dateiexistenz} - erstellt minimale Templates für fehlende Dateien
\item
  \textbf{Testet Grundgerüst} - Build ohne Module zum Testen der Basis-Struktur
\item
  \textbf{Testet vollständigen Build} - mit allen Modulen
\item
  \textbf{Erstellt TODO-Dateien} für neue Template-Dateien mit Hinweisen zur Vervollständigung
\end{enumerate}

\hypertarget{modulare-test-strategie}{%
\subsubsection{Modulare Test-Strategie}\label{modulare-test-strategie}}

\textbf{Für Entwickler:}
\begin{itemize}
\tightlist
\item
  Jedes neue Modul wird automatisch erkannt und getestet
\item
  Fehlende Referenzen werden durch kommentierte Templates ersetzt (kein Dummy-Content)
\item
  Build bricht nicht mehr bei fehlenden Dateien ab
\item
  Templates enthalten sinnvolle Struktur mit \verb|\section| und Platzhaltern
\end{itemize}

\textbf{Erweiterte Analyse:}
Für granulare Modultests steht \texttt{build\_system.py} zur Verfügung:
\begin{verbatim}
python3 build_system.py --verbose
\end{verbatim}

\begin{itemize}
\tightlist
\item
  Testet Module schrittweise einzeln
\item
  Identifiziert problematische Module
\item
  Erstellt detaillierte Build-Reports
\item
  Protokolliert alle Operationen in \texttt{build\_system.log}
\end{itemize}

\hypertarget{github-workflow-integration}{%
\subsubsection{GitHub Workflow Integration}\label{github-workflow-integration}}

Das GitHub Actions Workflow (\texttt{.github/workflows/latex-build.yml}) wurde korrigiert:
\begin{itemize}
\tightlist
\item
  Referenziert nun korrekt \texttt{main.tex} (statt dem nicht existierenden \texttt{main\_final.tex})
\item
  Lädt \texttt{main.pdf} als Artefakt hoch
\item
  Kann durch das Build-System bei Fehlern erweitert werden
\end{itemize}

\textbf{Typische Fehlerquellen und Best Practices:}

\begin{itemize}
\tightlist
\item
  \textbf{Pakete immer in der Präambel laden:}
  \verb|\usepackage{...}| darf nur in der Hauptdatei (z.B. \texttt{main.tex}) vor \verb|\begin{document}| stehen, niemals in Modulen oder nach \verb|\begin{document}|.
\item
  \textbf{Makros und Befehle:}
  Definiere neue Makros (z.B. Checkboxen, Textfelder) zentral in der Präambel oder in einem Style-File, nicht in einzelnen Modulen.
\end{itemize}

Beispiel für Checkboxen:
\begin{verbatim}
% In der Präambel:
\usepackage{amssymb}
\newcommand{\checkbox}{$\square$}
\newcommand{\checkedbox}{$\blacksquare$}
\end{verbatim}

\textbf{Wichtig:} Verwende in Modulen und Tabellen ausschließlich die Makros \verb|\checkbox| und \verb|\checkedbox| für Checkboxen. Benutze niemals direkt \verb|\Box| oder \verb|\blacksquare|, da dies zu \texttt{Undefined control sequence}-Fehlern führen kann.

\begin{itemize}
\tightlist
\item
  \textbf{Module:}
  Module sollten keine Pakete laden oder globale Makros definieren.
  Nur Inhalte und Befehle verwenden, die in der Präambel bereitgestellt werden.
\item
  \textbf{Fehlermeldungen:}
  \begin{itemize}
  \tightlist
  \item
    \texttt{Can be used only in preamble}: Ein Paket wurde im Fließtext geladen -- in die Präambel verschieben!
  \item
    \texttt{Undefined control sequence}: Ein Makro ist nicht definiert -- Definition prüfen oder in die Präambel verschieben.
  \item
    \texttt{Command ... already defined}: Ein Makro wurde doppelt definiert -- nur eine Definition behalten (am besten zentral).
  \end{itemize}
\end{itemize}

\hypertarget{vorgehen-bei-neuen-modulen}{%
\subsubsection{Vorgehen bei neuen Modulen}\label{vorgehen-bei-neuen-modulen}}

\begin{enumerate}
\def\labelenumi{\arabic{enumi}.}
\tightlist
\item
  \textbf{Referenz in main.tex hinzufügen:}
\begin{verbatim}
\input{modules/mein-neues-modul}
\end{verbatim}
\item
  \textbf{Build-System ausführen:}
\begin{verbatim}
python3 ctmm_build.py
\end{verbatim}
\item
  \textbf{Template wird automatisch erstellt:}
  \begin{itemize}
  \tightlist
  \item
    \texttt{modules/mein-neues-modul.tex} mit Grundstruktur
  \item
    \texttt{modules/TODO\_mein-neues-modul.md} mit Aufgabenliste
  \end{itemize}
\item
  \textbf{Inhalt ergänzen} und TODO-Datei entfernen wenn fertig
\end{enumerate}

\textbf{README regelmäßig pflegen:}
Hinweise zu neuen Makros, Paketen oder typischen Stolperfallen hier dokumentieren.

\textbf{Tipp:}
Wenn du ein neues Modul schreibst, prüfe, ob du neue Pakete oder Makros brauchst -- und ergänze sie zentral, nicht im Modul selbst.