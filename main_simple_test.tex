\documentclass[a4paper,12pt]{article}

% Pakete
\usepackage[T1]{fontenc}
\usepackage[utf8]{inputenc}
\usepackage[ngerman]{babel}
\usepackage{geometry}
\usepackage{hyperref}
\usepackage{xcolor}
\usepackage{tcolorbox}
\usepackage{tabularx}
\usepackage{amssymb}

% CTMM Pakete (test version)
\usepackage{style/ctmm-design-test}
\usepackage{style/form-elements}
\usepackage{style/ctmm-diagrams}

% Dokumentenkonfiguration
\geometry{
  a4paper,
  margin=2.5cm,
  top=3cm,
  bottom=3cm
}

\hypersetup{
  colorlinks=false,
  pdfborder={0 0 0},
  pdftitle={CTMM-System - Test Build},
  pdfauthor={CTMM-Team},
  pdfsubject={Therapiematerialien mit ausfüllbaren Formularen},
  pdfkeywords={CTMM, Therapie, Trigger, Borderline, ADHS, ASS, KPTBS, interaktiv},
  pdfcreator={LaTeX mit hyperref},
  pdfproducer={pdfLaTeX},
  % Formular-spezifische Einstellungen
  bookmarksopen=true,
  bookmarksopenlevel=1,
  % Für bessere Formularkompatibilität
  pdfencoding=auto,
  unicode=true
}

\begin{document}

\title{%
  {\Huge\textcolor{ctmmBlue}{CTMM-System}}\\
  \vspace{0.5cm}
  {\Large\textcolor{ctmmOrange}{Catch-Track-Map-Match}}\\
  \vspace{0.3cm}
  {\large Interaktives Therapie-Workbook}
}

\author{Test Build}
\date{\today}

\maketitle

\section{Test Section}

This is a test build to verify the LaTeX system works.

\begin{ctmmBlueBox}{Test Box}
This is a test colored box.
\end{ctmmBlueBox}

\subsection{Form Elements Test}

Test text field: \ctmmTextField[5cm]{Default text}{testfield}

Test checkbox: \ctmmCheckBox[testcheck]{Test checkbox}

\end{document}