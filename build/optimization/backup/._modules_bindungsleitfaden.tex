% Bindungsleitfaden - CTMM Modul

\newpage
\section*{	extcolor{ctmmBlue}{\faHeart~Bindungsleitfaden für ER \& SIE}}
\label{sec:bindungsleitfaden}
\addcontentsline{toc}{section}{Bindungsleitfaden}

\begin{quote}
\textit{\textcolor{ctmmOrange}{Bindung ist die Basis für Sicherheit, Vertrauen und gemeinsames Wachstum.}}\\
\textbf{\textcolor{ctmmBlue}{Sichere Bindung als Fundament}}\\
Bindung beschreibt das emotionale Band zwischen Menschen, das Sicherheit, Vertrauen und Nähe ermöglicht. Im CTMM-Kontext ist Bindung die Basis für Entwicklung und Veränderung.
\end{quote}

\subsection*{\textcolor{ctmmBlue}{Bindungstypen}}

\begin{ctmmBlueBox}{Die vier Bindungsstile}
\begin{itemize}
  \item \textbf{Sichere Bindung}: Vertrauen, Nähe, Unterstützung
  \item \textbf{Unsichere Bindung}: Angst, Unsicherheit, Rückzug
  \item \textbf{Ambivalente Bindung}: Schwankend zwischen Nähe und Distanz
  \item \textbf{Desorganisierte Bindung}: Widersprüchliches Verhalten
\end{itemize}
\end{ctmmBlueBox}

\subsection*{\textcolor{ctmmBlue}{Bindung im Alltag}}

\begin{ctmmGreenBox}{Bindung stärken}
\begin{itemize}
  \item Verlässlichkeit zeigen
  \item Zuhören und Verständnis signalisieren
  \item Gemeinsame Rituale pflegen
  \item Gefühle benennen und annehmen
\end{itemize}
\end{ctmmGreenBox}

\subsection*{\textcolor{ctmmBlue}{Bindung und CTMM}}
Bindung ist die Grundlage für die Arbeit mit Triggern, Mustern und Veränderungen im CTMM-System. Ein sicherer Bindungsrahmen erleichtert die Anwendung der Tools und fördert nachhaltige Entwicklung.

\vspace{1cm}
\begin{center}
\textit{\textcolor{ctmmGreen}{\faChevronRight~Weiter zu} \ctmmRef{sec:triggermanagement}{Trigger-Management} | \textcolor{ctmmBlue}{\faChevronLeft~Zurück zu} \ctmmRef{sec:depression}{Depression-Modul}}
\end{center}
