% Triggermanagement - CTMM Modul (Tool 23)

\newpage
\section*{\textcolor{ctmmOrange}{\faExclamationCircle~Triggermanagement}}
\label{sec:triggermanagement}
\addcontentsline{toc}{section}{Triggermanagement}

\begin{quote}
\textit{\textcolor{ctmmOrange}{Trigger verstehen heißt sich selbst verstehen -- der Schlüssel zur Selbstregulation.}}\\
\textbf{\textcolor{ctmmOrange}{Was ist ein Trigger?}}\\
Ein Trigger ist ein Auslöser, der starke emotionale oder körperliche Reaktionen hervorruft. Im CTMM werden Trigger erkannt, verstanden und bearbeitet.
\end{quote}

\subsection*{\textcolor{ctmmOrange}{1. Trigger erkennen}}

\begin{ctmmOrangeBox}{Erkennungszeichen}
\begin{itemize}
  \item \textbf{Körperliche Reaktionen}: Herzklopfen, Schwitzen, Anspannung
  \item \textbf{Gedanken und Erinnerungen}: Flashbacks, Grübeln, Katastrophendenken  
  \item \textbf{Gefühle}: Angst, Wut, Traurigkeit, Hilflosigkeit
  \item \textbf{Verhaltensänderungen}: Rückzug, Aggression, Erstarrung
\end{itemize}
\end{ctmmOrangeBox}

\subsection*{\textcolor{ctmmOrange}{2. Umgang mit Triggern}}

\begin{ctmmGreenBox}{Bewältigungsstrategien}
\begin{itemize}
  \item \textbf{Bewusstes Atmen}: 4-7-8 Technik, Bauchatmung
  \item \textbf{Grounding}: 5-4-3-2-1 Methode (Sinne aktivieren)
  \item \textbf{Selbstfürsorge}: Pausen, Grenzen, Ressourcen aktivieren
  \item \textbf{Soziale Unterstützung}: Safe Words, Vertrauensperson kontaktieren
  \item \textbf{Trigger-Tagebuch}: Muster erkennen, Fortschritte dokumentieren
\end{itemize}
\end{ctmmGreenBox}

\subsection*{\textcolor{ctmmOrange}{3. CTMM-Tool 23: Triggermanagement}}

Das Tool unterstützt dabei, Trigger zu identifizieren, zu reflektieren und neue Handlungsoptionen zu entwickeln. Ziel ist es, die eigene Reaktion zu verstehen und zu steuern.

\subsection*{\textcolor{ctmmOrange}{4. Persönliche Trigger-Analyse}}

\begin{ctmmBlueBox}{Trigger-Reflexion Arbeitsbereich}
\textbf{Schritt 1: Aktuelle Trigger-Situation beschreiben}\\
\ctmmTextField[12cm]{Was ist passiert? (Ort, Zeit, beteiligte Personen)}{trigger_situation}

\vspace{0.5cm}
\textbf{Schritt 2: Körperliche Reaktionen dokumentieren}\\
\ctmmTextField[12cm]{Wie hat mein Körper reagiert? (Herzschlag, Atmung, Anspannung)}{trigger_koerper}

\vspace{0.5cm}
\textbf{Schritt 3: Gedanken und Gefühle erfassen}\\
\ctmmTextField[12cm]{Welche Gedanken gingen mir durch den Kopf?}{trigger_gedanken}\\[0.3cm]
\ctmmTextField[12cm]{Welche Gefühle entstanden? (Intensität 1-10)}{trigger_gefuehle}

\vspace{0.5cm}
\textbf{Schritt 4: Bewältigungsstrategien planen}\\
\ctmmTextField[12cm]{Was könnte mir in Zukunft helfen? (konkrete Handlungen)}{trigger_strategie}\\[0.3cm]
\ctmmTextField[12cm]{Wen kann ich um Unterstützung bitten?}{trigger_hilfe}

\vspace{0.5cm}
\textbf{Schritt 5: Erfolg dokumentieren}\\
\ctmmTextField[12cm]{Was habe ich diesmal bereits gut gemacht?}{trigger_erfolg}
\end{ctmmBlueBox}

\vspace{1cm}
\begin{center}
\textit{\textcolor{ctmmGreen}{\faChevronRight~Weiter zu} \ctmmRef{sec:notfallkarten}{Notfallkarten} | \textcolor{ctmmRed}{\faBookmark~Direkt zu} \ctmmRef{sec:5.2}{Trigger-Tagebuch}}
\end{center}
