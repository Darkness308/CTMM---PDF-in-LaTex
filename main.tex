\documentclass[a4paper,12pt]{article}

% Pakete
\usepackage[T1]{fontenc}
\usepackage[utf8]{inputenc}
\usepackage[ngerman]{babel}
\usepackage{geometry}
\usepackage{hyperref}
\usepackage{xcolor}
\usepackage{fontawesome5}
\usepackage{tcolorbox}
\usepackage{tabularx}
\usepackage{amssymb}

% CTMM Pakete
\usepackage{style/ctmm-design}
\usepackage{style/form-elements}
\usepackage{style/ctmm-diagrams}

% Dokumentenkonfiguration
\geometry{
  a4paper,
  margin=2.5cm,
  top=3cm,
  bottom=3cm
}

\hypersetup{
  colorlinks=false,
  pdfborder={0 0 0},
  pdftitle={CTMM-System},
  pdfauthor={CTMM-Team},
  pdfsubject={Therapiematerialien}
}


% Alternative mit Rahmen:
\newcommand{\ctmmTextBox}[2][3cm]{%
    \framebox[#1]{\rule{0pt}{#2}}%
}
\newcommand{\ctmmTextField}[1]{\underline{\hspace{#1}}}

\begin{document}

\title{%
  {\Huge\textcolor{ctmmBlue}{CTMM-System}}\\
  \vspace{0.5cm}
  {\Large\textcolor{ctmmOrange}{Catch-Track-Map-Match}}\\
  \vspace{0.5cm}
  {\normalsize\textcolor{ctmmGreen}{Therapiematerialien für neurodiverse Paare}}
}
\author{CTMM-Team}
\date{\today}
\maketitle

\tableofcontents
\newpage

\section*{\textcolor{ctmmBlue}{\faCompass~CTMM-System Übersicht}}
\addcontentsline{toc}{section}{CTMM-System Übersicht}

\begin{ctmmBlueBox}{Was ist CTMM?}
\textbf{CTMM} steht für \textbf{Catch-Track-Map-Match} -- ein System zur Bewältigung von Triggern und Beziehungsherausforderungen:

\begin{itemize}
    \item \textcolor{ctmmBlue}{\textbf{Catch:}} Trigger erkennen
    \item \textcolor{ctmmGreen}{\textbf{Track:}} Gefühl \& Situation verfolgen  
    \item \textcolor{ctmmOrange}{\textbf{Map:}} Muster verstehen
    \item \textcolor{ctmmPurple}{\textbf{Match:}} Handlung anpassen
\end{itemize}
\end{ctmmBlueBox}

% Module einbinden
\section*{\textcolor{ctmmBlue}{\faCloudRain~Depression \& Stimmungstief -- Frühwarnung \& Handlungssicherheit}}

\begin{quote}
\textbf{\textcolor{ctmmBlue}{Worum geht's hier -- für Freunde \& Schüler?}}\\
Depression wirkt leise, aber mächtig. Dieses Modul hilft dir (und deinem Umfeld), erste Anzeichen zu erkennen, Eskalationen vorzubeugen und gemeinsam handlungsfähig zu bleiben. Es geht nicht um Diagnose -- sondern um Sicherheit, Struktur und Mitgefühl.
\end{quote}

\textit{Verknüpfbar mit Tool 23 (Trigger), Tool 26 (Ko-Regulation), Matching-Tracker, Safe Words \& Rückkehrritualen}

\subsection*{\textcolor{ctmmBlue}{Kapitelzuordnung im CTMM-System}}

\begin{itemize}
  \item \texttt{Kap. 2.5} → Selbstwahrnehmung \& Antrieb
  \item \texttt{Kap. 3.2} → Isolation \& Rückzug
  \item \texttt{Kap. 4.4} → Überforderung, Erschöpfung \& Schutz
  \item \texttt{Kap. 5.2/5.3} → Trigger-Frühzeichen \& Matching
\end{itemize}

\subsection*{\textcolor{ctmmBlue}{Farbcode \& Systemnavigation}}

\begin{tabular}{|c|l|l|}
\hline
\textbf{Farbe} & \textbf{Phase} & \textbf{Verknüpfte Module} \\
\hline
\textcolor{ctmmBlue}{$\bullet$} & Beobachtung & \texttt{tool\_23\_triggermanagement} \\
\textcolor{ctmmRed}{$\bullet$} & Eskalation / Rückzug & \texttt{trigger\_notfallkarten} \\
\textcolor{ctmmOrange}{$\bullet$} & Stimmung stabilisieren & \texttt{tool\_24\_skills\_sie}, \texttt{tool\_21} \\
\textcolor{ctmmPurple}{$\bullet$} & Rückkehr \& Integration & \texttt{ritual\_workbook}, \texttt{bindungsleitfaden\_ctmm} \\
\hline
\end{tabular}

\subsection*{\textcolor{ctmmBlue}{Frühwarnzeichen bei Depression}}

\begin{tabular}{|p{7cm}|p{7cm}|}
\hline
\textbf{Innerlich bei mir} & \textbf{Sichtbar für andere} \\
\hline
Ich will niemanden sehen & Rückzug, Zimmer bleibt dunkel \\
Ich empfinde nichts mehr & Keine Freude, kein Interesse \\
Alles wirkt anstrengend & Langsamer Gang, leise Stimme \\
Ich fühle mich wertlos & Selbstabwertung, Vermeidung \\
Ich denke, ich bin eine Last & Schuldgefühle, Isolation \\
\hline
\end{tabular}

% Bindungsleitfaden - CTMM Modul

\newpage
\section*{\textcolor{ctmmBlue}{\faHeart~Bindungsleitfaden}}
\addcontentsline{toc}{section}{Bindungsleitfaden}

\begin{quote}
\textit{\textcolor{ctmmOrange}{Bindung ist die Basis für Sicherheit, Vertrauen und gemeinsames Wachstum.}}\\
\textbf{\textcolor{ctmmBlue}{Sichere Bindung als Fundament}}\\
Bindung beschreibt das emotionale Band zwischen Menschen, das Sicherheit, Vertrauen und Nähe ermöglicht. Im CTMM-Kontext ist Bindung die Basis für Entwicklung und Veränderung.
\end{quote}

\subsection*{\textcolor{ctmmBlue}{Bindungstypen}}

\begin{ctmmBlueBox}{Die vier Bindungsstile}
\begin{itemize}
  \item \textbf{Sichere Bindung}: Vertrauen, Nähe, Unterstützung
  \item \textbf{Unsichere Bindung}: Angst, Unsicherheit, Rückzug
  \item \textbf{Ambivalente Bindung}: Schwankend zwischen Nähe und Distanz
  \item \textbf{Desorganisierte Bindung}: Widersprüchliches Verhalten
\end{itemize}
\end{ctmmBlueBox}

\subsection*{\textcolor{ctmmBlue}{Bindung im Alltag}}

\begin{ctmmGreenBox}{Bindung stärken}
\begin{itemize}
  \item Verlässlichkeit zeigen
  \item Zuhören und Verständnis signalisieren
  \item Gemeinsame Rituale pflegen
  \item Gefühle benennen und annehmen
\end{itemize}
\end{ctmmGreenBox}

\subsection*{\textcolor{ctmmBlue}{Bindung und CTMM}}
Bindung ist die Grundlage für die Arbeit mit Triggern, Mustern und Veränderungen im CTMM-System. Ein sicherer Bindungsrahmen erleichtert die Anwendung der Tools und fördert nachhaltige Entwicklung.

% CTMM Triggermanagement Modul (Tool 23)
\section{Triggermanagement}
\begin{tcolorbox}[colback=ctmmOrange!5!white,colframe=ctmmOrange,title=Was ist ein Trigger?]
Ein Trigger ist ein Auslöser, der starke emotionale oder körperliche Reaktionen hervorruft. Im CTMM werden Trigger erkannt, verstanden und bearbeitet.
\end{tcolorbox}

\subsection{Trigger erkennen}
\begin{itemize}
  \item Körperliche Reaktionen (z.B. Herzklopfen, Schwitzen)
  \item Gedanken und Erinnerungen
  \item Gefühle (z.B. Angst, Wut, Traurigkeit)
\end{itemize}

\subsection{Umgang mit Triggern}
\begin{tcolorbox}[colback=ctmmGreen!5!white,colframe=ctmmGreen,title=Strategien]
\begin{itemize}
  \item Bewusstes Atmen
  \item Selbstfürsorge
  \item Soziale Unterstützung suchen
  \item Trigger-Tagebuch führen
\end{itemize}
\end{tcolorbox}

\subsection{CTMM-Tool 23: Triggermanagement}
Das Tool unterstützt dabei, Trigger zu identifizieren, zu reflektieren und neue Handlungsoptionen zu entwickeln. Ziel ist es, die eigene Reaktion zu verstehen und zu steuern.

\section*{\textcolor{ctmmBlue}{\faHeartbeat{}~Notfallkarten}}
\addcontentsline{toc}{section}{Notfallkarten}
\label{sec:notfallkarten}

\begin{ctmmBlueBox}{Notfallkarten}
\textbf{Name:} \ctmmTextField[4cm]{}{nk_name} \quad \textbf{Datum:} \ctmmTextField[4cm]{}{nk_date}\\
\vspace{0.5cm}
\textbf{Notfallkontakte:}\\
\ctmmTextArea[12cm]{2}{nk_contacts}{}\\
\vspace{0.5cm}
\textbf{Hinweise:}\\
\ctmmTextArea[12cm]{4}{nk_notes}{}
\end{ctmmBlueBox}

\vspace{1cm}
\begin{center}
\textit{\textcolor{ctmmGreen}{\faChevronRight~Weiter zu} \ctmmRef{sec:safewords}{Safe-Words} | \textcolor{ctmmBlue}{\faChevronLeft~Zurück zu} \ctmmRef{sec:triggermanagement}{Trigger-Management}}
\end{center}

\section*{\textcolor{ctmmRed}{\faStopCircle~Safe-Words \& Signalsysteme}}

\begin{quote}
\textbf{\textcolor{ctmmRed}{Worum geht's hier -- für Freunde?}}\\
Safe-Words sind vereinbarte Codes oder Zeichen, die sofort signalisieren: „Ich kann nicht mehr", „Ich brauch Ruhe" oder „Stopp -- das wird mir zu viel". Sie schützen vor Eskalation, Überforderung, Rückzug oder Missverständnissen -- ohne viele Worte.
\end{quote}

\textit{Zentraler Bestandteil der Eskalationsprävention -- mit Symbol- und Notfallsystem}

\subsection*{\textcolor{ctmmRed}{Safe-Words (Beispiele + Eigene)}}

\begin{ctmmRedBox}{Bewährte Safe-Word Beispiele}
\begin{center}
\begin{tabular}{|p{3cm}|p{5cm}|p{5cm}|}
\hline
\textbf{Safe-Word / Geste} & \textbf{Bedeutung / Wirkung} & \textbf{Wann einsetzen?} \\
\hline
„Orange" & Warnstufe -- ich werde gleich überfordert & Bei Stress, lautem Ton, innerem Rückzug \\
\hline
„Kristall" & Stopp -- bitte sofort aufhören & Bei Eskalation, Überforderung, Trigger \\
\hline
Handzeichen (offen) & Ich will reden, aber schaff's nicht & Bei Freeze, Erstarrung, Nonverbales \\
\hline
„Lagerfeuer" & 5min in den Arm nehmen, ohne zu reden & Wenn ich Nähe brauche (Angst, Frieden) \\
\hline
\end{tabular}
\end{center}
\end{ctmmRedBox}

\textbf{Eigene Safe-Words:}\\
\ctmmTextField[4cm]{Safe-Word 1:}{safeword1} \ctmmTextField[6cm]{Bedeutung:}{meaning1}\\
\ctmmTextField[4cm]{Safe-Word 2:}{safeword2} \ctmmTextField[6cm]{Bedeutung:}{meaning2}\\
\ctmmTextField[4cm]{Safe-Word 3:}{safeword3} \ctmmTextField[6cm]{Bedeutung:}{meaning3}

\subsection*{\textcolor{ctmmOrange}{Signalsysteme zur Unterstützung}}

\begin{ctmmOrangeBox}{Leise Zeichen für schwere Momente}
\begin{itemize}
  \item \textbf{Symbolischer Gegenstand} (z. B. Stofftier, Stein, Karte)
  \item \textbf{Tagesanzeiger} (Magnet, Schild, Farbe auf Tür)
  \item \textbf{Lautstärke-Code} (Musikart, Kopfhörer sichtbar = „Bitte in Ruhe lassen")
\end{itemize}

\textbf{Diese Zeichen können leise, sichtbar, intuitiv sein -- auch bei Disso oder Sprachverlust}
\end{ctmmOrangeBox}


\newpage
\section*{\textcolor{ctmmOrange}{\faCheckSquare~Interaktive Checklisten \& Bewertungen}}
\addcontentsline{toc}{section}{Interaktive Tools}
\label{sec:interactive}

\begin{quote}
\textit{\textcolor{ctmmOrange}{Selbstreflexion durch strukturierte Bewertung und Planung.}}\\
\textbf{\textcolor{ctmmOrange}{Messbare Fortschritte im CTMM-System}}\\
Diese Tools helfen dabei, den eigenen Fortschritt zu messen und konkrete nächste Schritte zu planen.
\end{quote}

\subsection*{\textcolor{ctmmOrange}{1. CTMM-Fortschritts-Check}}

\begin{ctmmOrangeBox}{Selbstbewertung (1-10 Punkte)}
\begin{tabular}{|p{6cm}|c|c|c|c|c|c|c|c|c|c|}
\hline
\textbf{CTMM-Bereich} & \textbf{1} & \textbf{2} & \textbf{3} & \textbf{4} & \textbf{5} & \textbf{6} & \textbf{7} & \textbf{8} & \textbf{9} & \textbf{10} \\
\hline
\textbf{Catch:} Trigger erkennen & $\square$ & $\square$ & $\square$ & $\square$ & $\square$ & $\square$ & $\square$ & $\square$ & $\square$ & $\square$ \\
\hline
\textbf{Track:} Gefühle verfolgen & $\square$ & $\square$ & $\square$ & $\square$ & $\square$ & $\square$ & $\square$ & $\square$ & $\square$ & $\square$ \\
\hline
\textbf{Map:} Muster verstehen & $\square$ & $\square$ & $\square$ & $\square$ & $\square$ & $\square$ & $\square$ & $\square$ & $\square$ & $\square$ \\
\hline
\textbf{Match:} Handlung anpassen & $\square$ & $\square$ & $\square$ & $\square$ & $\square$ & $\square$ & $\square$ & $\square$ & $\square$ & $\square$ \\
\hline
\end{tabular}
\end{ctmmOrangeBox}

\subsection*{\textcolor{ctmmOrange}{2. Wöchentlicher Reflexions-Check}}

\begin{ctmmBlueBox}{Woche vom: \ctmmTextField[6cm]{}{woche\_vom}}
    \textbf{Was lief gut diese Woche?}\\
\ctmmTextArea[12cm]{2}{woche\_gut}{}\\

    \textbf{Welche Herausforderungen gab es?}\\
\ctmmTextArea[12cm]{2}{woche\_herausforderungen}{}\\

    \textbf{Welche CTMM-Tools habe ich verwendet?}\\
\ctmmCheckBox{tools\_safewords}{Safe Words} \quad \ctmmCheckBox{tools\_notfallplan}{Notfallplan} \quad \ctmmCheckBox{tools\_trigger}{Triggermanagement} \quad \ctmmCheckBox{tools\_bindung}{Bindungsarbeit}\\

    \textbf{Nächste Woche möchte ich...}\\

\ctmmTextArea[12cm]{2}{woche\_naechste}{}\\
\end{ctmmBlueBox}

\subsection*{\textcolor{ctmmOrange}{3. Beziehungs-Dashboard}}



\begin{ctmmGreenBox}{Für Paare: Gemeinsame Bewertung}
\begin{tabularx}{\textwidth}{|X|c|c|X|}
\hline
\textbf{Bereich} & \textbf{Partner A} & \textbf{Partner B} & \textbf{Gemeinsame Ziele} \\
\hline
Kommunikation & \ctmmTextField[2cm]{}{bd_comm_a} & \ctmmTextField[2cm]{}{bd_comm_b} & \ctmmTextField[4cm]{}{bd_comm_ziele} \\
\hline
Konfliktlösung & \ctmmTextField[2cm]{}{bd_konflikt_a} & \ctmmTextField[2cm]{}{bd_konflikt_b} & \ctmmTextField[4cm]{}{bd_konflikt_ziele} \\
\hline
Intimität & \ctmmTextField[2cm]{}{bd_intim_a} & \ctmmTextField[2cm]{}{bd_intim_b} & \ctmmTextField[4cm]{}{bd_intim_ziele} \\
\hline
Alltagsorganisation & \ctmmTextField[2cm]{}{bd_alltag_a} & \ctmmTextField[2cm]{}{bd_alltag_b} & \ctmmTextField[4cm]{}{bd_alltag_ziele} \\

% QR-Code Integration für digitale Ressourcen

\newpage
\section*{\textcolor{ctmmPurple}{\faQrcode~Digitale Ressourcen}}
\addcontentsline{toc}{section}{Digitale Ressourcen}
\label{sec:qrcode}

\begin{quote}
\textit{\textcolor{ctmmOrange}{Verbindung von analog und digital -- das CTMM-System erweitern.}}\\
\textbf{\textcolor{ctmmPurple}{Online-Tools \& Apps für das CTMM-System}}\\
Scannen Sie die QR-Codes für zusätzliche digitale Hilfsmittel und Updates.
\end{quote}

\subsection*{\textcolor{ctmmPurple}{Online-Trigger-Tagebuch}}

\begin{ctmmBlueBox}{Digitales Tracking}
Führen Sie Ihr Trigger-Tagebuch digital und erhalten Sie automatische Muster-Analysen.

\begin{center}
\begin{tabular}{c}
\textbf{QR-Code: Trigger-App}\\
\rule{3cm}{3cm}\\
\small{Hier würde ein QR-Code zur}\\
\small{CTMM-Trigger-App erscheinen}
\end{tabular}
\end{center}
\end{ctmmBlueBox}

\subsection*{\textcolor{ctmmPurple}{CTMM-Notfall-App}}

\begin{ctmmRedBox}{Sofortige Hilfe}
Schneller Zugriff auf Ihre personalisierten Safe Words und Notfallkontakte.

\begin{center}
\begin{tabular}{c}
\textbf{QR-Code: Notfall-App}\\
\rule{3cm}{3cm}\\
\small{Hier würde ein QR-Code zur}\\
\small{CTMM-Notfall-App erscheinen}
\end{tabular}
\end{center}
\end{ctmmRedBox}

\subsection*{\textcolor{ctmmPurple}{Online-Community}}

\begin{ctmmGreenBox}{Austausch \& Support}
Vernetzen Sie sich mit anderen CTMM-Anwendern in einem sicheren Forum.

\begin{center}
\begin{tabular}{c}
\textbf{QR-Code: Community}\\
\rule{3cm}{3cm}\\
\small{Hier würde ein QR-Code zum}\\
\small{CTMM-Community-Forum erscheinen}
\end{tabular}
\end{center}
\end{ctmmGreenBox}

\subsection*{\textcolor{ctmmPurple}{Video-Tutorials}}

\begin{ctmmOrangeBox}{Lernen \& Vertiefen}
Schauen Sie sich praktische Anwendungsvideos für alle CTMM-Tools an.

\begin{center}
\begin{tabular}{|c|c|}
\hline
\textbf{Video-Thema} & \textbf{QR-Code / Link} \\
\hline
\textbf{4-7-8 Atemtechnik} & \href{https://youtube.com/watch?v=YRPh_GaiL8s}{\textcolor{ctmmBlue}{\faYoutube~3 Min Anleitung}} \\
\hline
\textbf{5-4-3-2-1 Grounding} & \href{https://youtube.com/watch?v=utUVx0ayoYw}{\textcolor{ctmmBlue}{\faYoutube~Grounding-Übung}} \\
\hline
\textbf{DBT Skills Demo} & \href{https://youtube.com/watch?v=q15eTySnWxc}{\textcolor{ctmmBlue}{\faYoutube~Skills Training}} \\
\hline
\textbf{Trigger-Management} & \href{https://youtube.com/watch?v=Mz3Mi_OZYno}{\textcolor{ctmmBlue}{\faYoutube~PTSD Coping}} \\
\hline
\textbf{Paartherapie-Kommunikation} & \href{https://youtube.com/watch?v=2s9ACDMcpjA}{\textcolor{ctmmBlue}{\faYoutube~Gottman-Methode}} \\
\hline
\end{tabular}
\end{center}

\vspace{0.5cm}
\textbf{Empfohlene Kanäle:}
\begin{itemize}
    \item \textcolor{ctmmGreen}{\faYoutube~Therapy in a Nutshell} - Trauma \& DBT Skills
    \item \textcolor{ctmmBlue}{\faYoutube~Kati Morton} - Mental Health Education  
    \item \textcolor{ctmmOrange}{\faYoutube~The School of Life} - Relationship Skills
    \item \textcolor{ctmmPurple}{\faYoutube~Marsha Linehan} - Original DBT Videos
\end{itemize}
\end{ctmmOrangeBox}

\vspace{1cm}
\begin{center}
\textit{\textcolor{ctmmPurple}{\faChevronRight~Weiter zu} \ctmmRef{sec:feedback}{Selbstreflexions-System} | \textcolor{ctmmGreen}{\faHome~Zurück zu} \ctmmRef{sec:navigation}{Navigation}}
\end{center}


\end{document}