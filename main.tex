\documentclass[a4paper,12pt]{article}

% Pakete
\usepackage[T1]{fontenc}
\usepackage[utf8]{inputenc}
\usepackage[ngerman]{babel}
\usepackage{geometry}
\usepackage{hyperref}
\usepackage{xcolor}
\usepackage{fontawesome5}
\usepackage{tcolorbox}
\usepackage{tabularx}
\usepackage{amssymb}

% CTMM Pakete
\usepackage{style/ctmm-design}
\usepackage{style/form-elements}
\usepackage{style/ctmm-diagrams}

% Dokumentenkonfiguration
\geometry{
  a4paper,
  margin=2.5cm,
  top=3cm,
  bottom=3cm
}

\hypersetup{
  colorlinks=false,
  pdfborder={0 0 0},
  pdftitle={CTMM-System - Interaktives Therapie-Workbook},
  pdfauthor={CTMM-Team},
  pdfsubject={Therapiematerialien mit ausfüllbaren Formularen},
  pdfkeywords={CTMM, Therapie, Trigger, Borderline, ADHS, ASS, KPTBS, interaktiv},
  pdfcreator={LaTeX mit hyperref},
  pdfproducer={pdfLaTeX},
  % Formular-spezifische Einstellungen
  bookmarksopen=true,
  bookmarksopenlevel=1,
  % Für bessere Formularkompatibilität
  pdfencoding=auto,
  unicode=true
}

\begin{document}

\title{%
  {\Huge\textcolor{ctmmBlue}{CTMM-System}}\\
  \vspace{0.5cm}
  {\Large\textcolor{ctmmOrange}{Catch-Track-Map-Match}}\\
  \vspace{0.5cm}
  {\normalsize\textcolor{ctmmGreen}{Therapiematerialien für neurodiverse Paare}}
}
\author{CTMM-Team}
\date{\today}
\maketitle

\tableofcontents
\newpage

\section*{\textcolor{ctmmBlue}{\faCompass~CTMM-System Übersicht}}
\addcontentsline{toc}{section}{CTMM-System Übersicht}

\begin{ctmmBlueBox}{Was ist CTMM?}
\textbf{CTMM} steht für \textbf{Catch-Track-Map-Match} -- ein System zur Bewältigung von Triggern und Beziehungsherausforderungen:

\begin{itemize}
    \item \textcolor{ctmmBlue}{\textbf{Catch:}} Trigger erkennen
    \item \textcolor{ctmmGreen}{\textbf{Track:}} Gefühl \& Situation verfolgen  
    \item \textcolor{ctmmOrange}{\textbf{Map:}} Muster verstehen
    \item \textcolor{ctmmPurple}{\textbf{Match:}} Handlung anpassen
\end{itemize}
\end{ctmmBlueBox}

% Module einbinden
% Navigation und Querverweise für CTMM-System
\section*{\textcolor{ctmmBlue}{\faMap~Navigations-System}}
\addcontentsline{toc}{section}{Navigations-System}
\label{sec:navigation}

\begin{ctmmBlueBox}{Orientierung im CTMM-System}
\textbf{Farbkodierung für schnelle Navigation:}

\begin{itemize}
    \item \textcolor{ctmmBlue}{\textbf{BLAU - Grundlagen}} - Warum wir uns triggern
    \item \textcolor{ctmmGreen}{\textbf{GRÜN - Tägliche Tools}} - Skills und Routinen  
    \item \textcolor{ctmmRed}{\textbf{ROT - Notfall-Guide}} - Krisenintervention
    \item \textcolor{ctmmYellow}{\textbf{GELB - Support}} - Freunde und Familie
    \item \textcolor{ctmmPurple}{\textbf{LILA - Arbeitsblätter}} - Tracking und Reflexion
\end{itemize}
\end{ctmmBlueBox}

\vspace{0.5cm}

\subsection*{\faChevronRight~Schnell-Navigation}

\begin{ctmmGreenBox}{GRÜN: Tägliche Tools - Jeden Tag nutzen}
\begin{itemize}
    \item \ctmmRef{sec:5.1}{Täglicher Check-In} - Morgens und abends
    \item \ctmmRef{sec:safewords}{Safe-Words System} - Bei Überforderung
    \item \ctmmRef{sec:triggermanagement}{Trigger-Management} - Präventiv
\end{itemize}
\end{ctmmGreenBox}

\begin{ctmmRedBox}{ROT: Notfall-Protokolle - In Krisen}
\begin{itemize}
    \item \ctmmRef{sec:notfallkarten}{Notfallkarten} - Sofort verfügbar
    \item \ctmmRef{sec:5.2}{Trigger-Tagebuch} - Nach der Krise
    \item \ctmmRef{sec:5.3}{Depression-Monitor} - Wöchentlich
\end{itemize}
\end{ctmmRedBox}

\begin{ctmmPurpleBox}{LILA: Reflexion \& Wachstum - Regelmäßig}
\begin{itemize}
    \item \ctmmRef{sec:feedback}{Selbstreflexions-System} - Monatlich
    \item \ctmmRef{sec:fortschritt}{Fortschrittsmessung} - Kontinuierlich
    \item \ctmmRef{sec:erfolge}{Erfolgs-Bibliothek} - Motivation
\end{itemize}
\end{ctmmPurpleBox}

\vspace{0.5cm}

\subsection*{\faQuestionCircle~Häufige Situationen}

\begin{center}
\begin{tabular}{p{5cm}p{8cm}}
\textbf{Situation} & \textbf{Gehe zu} \\
\hline
Überforderung spürbar & $\rightarrow$ \ctmmRef{sec:safewords}{Safe-Words} \\
Nach einem Streit & $\rightarrow$ \ctmmRef{sec:5.2}{Trigger-Tagebuch} \\
Schlechte Schlafqualität & $\rightarrow$ \ctmmRef{sec:5.3}{Depression-Monitor} \\
Erfolg feiern & $\rightarrow$ \ctmmRef{sec:erfolge}{Erfolgs-Bibliothek} \\
System anpassen & $\rightarrow$ \ctmmRef{sec:feedback}{Selbstreflexion} \\
Therapeuten koordinieren & $\rightarrow$ \ctmmRef{sec:therapiekoordination}{Therapie-Team} \\
Morgen-Routine & $\rightarrow$ \ctmmRef{sec:5.1}{Täglicher Check-In} \\
Krise eskaliert & $\rightarrow$ \ctmmRef{sec:notfallkarten}{Notfallkarten} \\
\end{tabular}
\end{center}

\vspace{0.5cm}

\subsection*{\faClock~Nach Tageszeit}

\begin{ctmmGreenBox}{Morgens (7-10 Uhr)}
\begin{enumerate}
    \item \ctmmRef{sec:5.1}{Täglicher Check-In} - Status beider Partner
    \item Medikamente-Check und Tagesplanung
    \item Support-Person informieren bei schwierigen Terminen
\end{enumerate}
\end{ctmmGreenBox}

\begin{ctmmYellowBox}{Abends (19-22 Uhr)}
\begin{enumerate}
    \item \ctmmRef{sec:5.1}{Abend-Reflexion} - Tag verarbeiten
    \item Bei Triggern: \ctmmRef{sec:5.2}{Trigger-Tagebuch} ausfüllen
    \item Morgen vorbereiten, Rituale pflegen
\end{enumerate}
\end{ctmmYellowBox}

\begin{ctmmRedBox}{Wöchentlich (Sonntags)}
\begin{enumerate}
    \item \ctmmRef{sec:5.3}{Depression-Monitor} auswerten
    \item \ctmmRef{sec:feedback}{Wochenreflexion} durchführen
    \item Erfolge der Woche dokumentieren
\end{enumerate}
\end{ctmmRedBox}

\vspace{0.5cm}

\textit{Tipp: Nutzen Sie die Farbkodierung, um schnell das richtige Werkzeug zu finden!}

\section*{\textcolor{ctmmBlue}{\faCloudRain~Depression \& Stimmungstief -- Frühwarnung \& Handlungssicherheit}}

\begin{quote}
\textbf{\textcolor{ctmmBlue}{Worum geht's hier -- für Freunde \& Schüler?}}\\
Depression wirkt leise, aber mächtig. Dieses Modul hilft dir (und deinem Umfeld), erste Anzeichen zu erkennen, Eskalationen vorzubeugen und gemeinsam handlungsfähig zu bleiben. Es geht nicht um Diagnose -- sondern um Sicherheit, Struktur und Mitgefühl.
\end{quote}

\textit{Verknüpfbar mit Tool 23 (Trigger), Tool 26 (Ko-Regulation), Matching-Tracker, Safe Words \& Rückkehrritualen}

\subsection*{\textcolor{ctmmBlue}{Kapitelzuordnung im CTMM-System}}

\begin{itemize}
  \item \texttt{Kap. 2.5} → Selbstwahrnehmung \& Antrieb
  \item \texttt{Kap. 3.2} → Isolation \& Rückzug
  \item \texttt{Kap. 4.4} → Überforderung, Erschöpfung \& Schutz
  \item \texttt{Kap. 5.2/5.3} → Trigger-Frühzeichen \& Matching
\end{itemize}

\subsection*{\textcolor{ctmmBlue}{Farbcode \& Systemnavigation}}

\begin{tabular}{|c|l|l|}
\hline
\textbf{Farbe} & \textbf{Phase} & \textbf{Verknüpfte Module} \\
\hline
\textcolor{ctmmBlue}{$\bullet$} & Beobachtung & \texttt{tool\_23\_triggermanagement} \\
\textcolor{ctmmRed}{$\bullet$} & Eskalation / Rückzug & \texttt{trigger\_notfallkarten} \\
\textcolor{ctmmOrange}{$\bullet$} & Stimmung stabilisieren & \texttt{tool\_24\_skills\_sie}, \texttt{tool\_21} \\
\textcolor{ctmmPurple}{$\bullet$} & Rückkehr \& Integration & \texttt{ritual\_workbook}, \texttt{bindungsleitfaden\_ctmm} \\
\hline
\end{tabular}

\subsection*{\textcolor{ctmmBlue}{Frühwarnzeichen bei Depression}}

\begin{tabular}{|p{7cm}|p{7cm}|}
\hline
\textbf{Innerlich bei mir} & \textbf{Sichtbar für andere} \\
\hline
Ich will niemanden sehen & Rückzug, Zimmer bleibt dunkel \\
Ich empfinde nichts mehr & Keine Freude, kein Interesse \\
Alles wirkt anstrengend & Langsamer Gang, leise Stimme \\
Ich fühle mich wertlos & Selbstabwertung, Vermeidung \\
Ich denke, ich bin eine Last & Schuldgefühle, Isolation \\
\hline
\end{tabular}

% Bindungsleitfaden - CTMM Modul

\newpage
\section*{\textcolor{ctmmBlue}{\faHeart~Bindungsleitfaden}}
\addcontentsline{toc}{section}{Bindungsleitfaden}

\begin{quote}
\textit{\textcolor{ctmmOrange}{Bindung ist die Basis für Sicherheit, Vertrauen und gemeinsames Wachstum.}}\\
\textbf{\textcolor{ctmmBlue}{Sichere Bindung als Fundament}}\\
Bindung beschreibt das emotionale Band zwischen Menschen, das Sicherheit, Vertrauen und Nähe ermöglicht. Im CTMM-Kontext ist Bindung die Basis für Entwicklung und Veränderung.
\end{quote}

\subsection*{\textcolor{ctmmBlue}{Bindungstypen}}

\begin{ctmmBlueBox}{Die vier Bindungsstile}
\begin{itemize}
  \item \textbf{Sichere Bindung}: Vertrauen, Nähe, Unterstützung
  \item \textbf{Unsichere Bindung}: Angst, Unsicherheit, Rückzug
  \item \textbf{Ambivalente Bindung}: Schwankend zwischen Nähe und Distanz
  \item \textbf{Desorganisierte Bindung}: Widersprüchliches Verhalten
\end{itemize}
\end{ctmmBlueBox}

\subsection*{\textcolor{ctmmBlue}{Bindung im Alltag}}

\begin{ctmmGreenBox}{Bindung stärken}
\begin{itemize}
  \item Verlässlichkeit zeigen
  \item Zuhören und Verständnis signalisieren
  \item Gemeinsame Rituale pflegen
  \item Gefühle benennen und annehmen
\end{itemize}
\end{ctmmGreenBox}

\subsection*{\textcolor{ctmmBlue}{Bindung und CTMM}}
Bindung ist die Grundlage für die Arbeit mit Triggern, Mustern und Veränderungen im CTMM-System. Ein sicherer Bindungsrahmen erleichtert die Anwendung der Tools und fördert nachhaltige Entwicklung.

% CTMM Triggermanagement Modul (Tool 23)
\section{Triggermanagement}
\begin{tcolorbox}[colback=ctmmOrange!5!white,colframe=ctmmOrange,title=Was ist ein Trigger?]
Ein Trigger ist ein Auslöser, der starke emotionale oder körperliche Reaktionen hervorruft. Im CTMM werden Trigger erkannt, verstanden und bearbeitet.
\end{tcolorbox}

\subsection{Trigger erkennen}
\begin{itemize}
  \item Körperliche Reaktionen (z.B. Herzklopfen, Schwitzen)
  \item Gedanken und Erinnerungen
  \item Gefühle (z.B. Angst, Wut, Traurigkeit)
\end{itemize}

\subsection{Umgang mit Triggern}
\begin{tcolorbox}[colback=ctmmGreen!5!white,colframe=ctmmGreen,title=Strategien]
\begin{itemize}
  \item Bewusstes Atmen
  \item Selbstfürsorge
  \item Soziale Unterstützung suchen
  \item Trigger-Tagebuch führen
\end{itemize}
\end{tcolorbox}

\subsection{CTMM-Tool 23: Triggermanagement}
Das Tool unterstützt dabei, Trigger zu identifizieren, zu reflektieren und neue Handlungsoptionen zu entwickeln. Ziel ist es, die eigene Reaktion zu verstehen und zu steuern.

\section*{\textcolor{ctmmBlue}{\faHeartbeat{}~Notfallkarten}}
\addcontentsline{toc}{section}{Notfallkarten}
\label{sec:notfallkarten}

\begin{ctmmBlueBox}{Notfallkarten}
\textbf{Name:} \ctmmTextField[4cm]{}{nk_name} \quad \textbf{Datum:} \ctmmTextField[4cm]{}{nk_date}\\
\vspace{0.5cm}
\textbf{Notfallkontakte:}\\
\ctmmTextArea[12cm]{2}{nk_contacts}{}\\
\vspace{0.5cm}
\textbf{Hinweise:}\\
\ctmmTextArea[12cm]{4}{nk_notes}{}
\end{ctmmBlueBox}

\vspace{1cm}
\begin{center}
\textit{\textcolor{ctmmGreen}{\faChevronRight~Weiter zu} \ctmmRef{sec:safewords}{Safe-Words} | \textcolor{ctmmBlue}{\faChevronLeft~Zurück zu} \ctmmRef{sec:triggermanagement}{Trigger-Management}}
\end{center}

\section*{\textcolor{ctmmRed}{\faStopCircle~Safe-Words \& Signalsysteme}}

\begin{quote}
\textbf{\textcolor{ctmmRed}{Worum geht's hier -- für Freunde?}}\\
Safe-Words sind vereinbarte Codes oder Zeichen, die sofort signalisieren: „Ich kann nicht mehr", „Ich brauch Ruhe" oder „Stopp -- das wird mir zu viel". Sie schützen vor Eskalation, Überforderung, Rückzug oder Missverständnissen -- ohne viele Worte.
\end{quote}

\textit{Zentraler Bestandteil der Eskalationsprävention -- mit Symbol- und Notfallsystem}

\subsection*{\textcolor{ctmmRed}{Safe-Words (Beispiele + Eigene)}}

\begin{ctmmRedBox}{Bewährte Safe-Word Beispiele}
\begin{center}
\begin{tabular}{|p{3cm}|p{5cm}|p{5cm}|}
\hline
\textbf{Safe-Word / Geste} & \textbf{Bedeutung / Wirkung} & \textbf{Wann einsetzen?} \\
\hline
„Orange" & Warnstufe -- ich werde gleich überfordert & Bei Stress, lautem Ton, innerem Rückzug \\
\hline
„Kristall" & Stopp -- bitte sofort aufhören & Bei Eskalation, Überforderung, Trigger \\
\hline
Handzeichen (offen) & Ich will reden, aber schaff's nicht & Bei Freeze, Erstarrung, Nonverbales \\
\hline
„Lagerfeuer" & 5min in den Arm nehmen, ohne zu reden & Wenn ich Nähe brauche (Angst, Frieden) \\
\hline
\end{tabular}
\end{center}
\end{ctmmRedBox}

\textbf{Eigene Safe-Words:}\\
\ctmmTextField[4cm]{Safe-Word 1:}{safeword1} \ctmmTextField[6cm]{Bedeutung:}{meaning1}\\
\ctmmTextField[4cm]{Safe-Word 2:}{safeword2} \ctmmTextField[6cm]{Bedeutung:}{meaning2}\\
\ctmmTextField[4cm]{Safe-Word 3:}{safeword3} \ctmmTextField[6cm]{Bedeutung:}{meaning3}

\subsection*{\textcolor{ctmmOrange}{Signalsysteme zur Unterstützung}}

\begin{ctmmOrangeBox}{Leise Zeichen für schwere Momente}
\begin{itemize}
  \item \textbf{Symbolischer Gegenstand} (z. B. Stofftier, Stein, Karte)
  \item \textbf{Tagesanzeiger} (Magnet, Schild, Farbe auf Tür)
  \item \textbf{Lautstärke-Code} (Musikart, Kopfhörer sichtbar = „Bitte in Ruhe lassen")
\end{itemize}

\textbf{Diese Zeichen können leise, sichtbar, intuitiv sein -- auch bei Disso oder Sprachverlust}
\end{ctmmOrangeBox}


\newpage
\section*{\textcolor{ctmmOrange}{\faCheckSquare~Interaktive Checklisten \& Bewertungen}}
\addcontentsline{toc}{section}{Interaktive Tools}
\label{sec:interactive}

\begin{quote}
\textit{\textcolor{ctmmOrange}{Selbstreflexion durch strukturierte Bewertung und Planung.}}\\
\textbf{\textcolor{ctmmOrange}{Messbare Fortschritte im CTMM-System}}\\
Diese Tools helfen dabei, den eigenen Fortschritt zu messen und konkrete nächste Schritte zu planen.
\end{quote}

\subsection*{\textcolor{ctmmOrange}{1. CTMM-Fortschritts-Check}}

\begin{ctmmOrangeBox}{Selbstbewertung (1-10 Punkte)}
\begin{tabular}{|p{6cm}|c|c|c|c|c|c|c|c|c|c|}
\hline
\textbf{CTMM-Bereich} & \textbf{1} & \textbf{2} & \textbf{3} & \textbf{4} & \textbf{5} & \textbf{6} & \textbf{7} & \textbf{8} & \textbf{9} & \textbf{10} \\
\hline
\textbf{Catch:} Trigger erkennen & $\square$ & $\square$ & $\square$ & $\square$ & $\square$ & $\square$ & $\square$ & $\square$ & $\square$ & $\square$ \\
\hline
\textbf{Track:} Gefühle verfolgen & $\square$ & $\square$ & $\square$ & $\square$ & $\square$ & $\square$ & $\square$ & $\square$ & $\square$ & $\square$ \\
\hline
\textbf{Map:} Muster verstehen & $\square$ & $\square$ & $\square$ & $\square$ & $\square$ & $\square$ & $\square$ & $\square$ & $\square$ & $\square$ \\
\hline
\textbf{Match:} Handlung anpassen & $\square$ & $\square$ & $\square$ & $\square$ & $\square$ & $\square$ & $\square$ & $\square$ & $\square$ & $\square$ \\
\hline
\end{tabular}
\end{ctmmOrangeBox}

\subsection*{\textcolor{ctmmOrange}{2. Wöchentlicher Reflexions-Check}}

\begin{ctmmBlueBox}{Woche vom: \ctmmTextField[6cm]{}{woche\_vom}}
    \textbf{Was lief gut diese Woche?}\\
\ctmmTextArea[12cm]{2}{woche\_gut}{}\\

    \textbf{Welche Herausforderungen gab es?}\\
\ctmmTextArea[12cm]{2}{woche\_herausforderungen}{}\\

    \textbf{Welche CTMM-Tools habe ich verwendet?}\\
\ctmmCheckBox{tools\_safewords}{Safe Words} \quad \ctmmCheckBox{tools\_notfallplan}{Notfallplan} \quad \ctmmCheckBox{tools\_trigger}{Triggermanagement} \quad \ctmmCheckBox{tools\_bindung}{Bindungsarbeit}\\

    \textbf{Nächste Woche möchte ich...}\\

\ctmmTextArea[12cm]{2}{woche\_naechste}{}\\
\end{ctmmBlueBox}

\subsection*{\textcolor{ctmmOrange}{3. Beziehungs-Dashboard}}



\begin{ctmmGreenBox}{Für Paare: Gemeinsame Bewertung}
\begin{tabularx}{\textwidth}{|X|c|c|X|}
\hline
\textbf{Bereich} & \textbf{Partner A} & \textbf{Partner B} & \textbf{Gemeinsame Ziele} \\
\hline
Kommunikation & \ctmmTextField[2cm]{}{bd_comm_a} & \ctmmTextField[2cm]{}{bd_comm_b} & \ctmmTextField[4cm]{}{bd_comm_ziele} \\
\hline
Konfliktlösung & \ctmmTextField[2cm]{}{bd_konflikt_a} & \ctmmTextField[2cm]{}{bd_konflikt_b} & \ctmmTextField[4cm]{}{bd_konflikt_ziele} \\
\hline
Intimität & \ctmmTextField[2cm]{}{bd_intim_a} & \ctmmTextField[2cm]{}{bd_intim_b} & \ctmmTextField[4cm]{}{bd_intim_ziele} \\
\hline
Alltagsorganisation & \ctmmTextField[2cm]{}{bd_alltag_a} & \ctmmTextField[2cm]{}{bd_alltag_b} & \ctmmTextField[4cm]{}{bd_alltag_ziele} \\

% QR-Code Integration für digitale Ressourcen

\newpage
\section*{\textcolor{ctmmPurple}{\faQrcode~Digitale Ressourcen}}
\addcontentsline{toc}{section}{Digitale Ressourcen}
\label{sec:qrcode}

\begin{quote}
\textit{\textcolor{ctmmOrange}{Verbindung von analog und digital -- das CTMM-System erweitern.}}\\
\textbf{\textcolor{ctmmPurple}{Online-Tools \& Apps für das CTMM-System}}\\
Scannen Sie die QR-Codes für zusätzliche digitale Hilfsmittel und Updates.
\end{quote}

\subsection*{\textcolor{ctmmPurple}{Online-Trigger-Tagebuch}}

\begin{ctmmBlueBox}{Digitales Tracking}
Führen Sie Ihr Trigger-Tagebuch digital und erhalten Sie automatische Muster-Analysen.

\begin{center}
\begin{tabular}{c}
\textbf{QR-Code: Trigger-App}\\
\rule{3cm}{3cm}\\
\small{Hier würde ein QR-Code zur}\\
\small{CTMM-Trigger-App erscheinen}
\end{tabular}
\end{center}
\end{ctmmBlueBox}

\subsection*{\textcolor{ctmmPurple}{CTMM-Notfall-App}}

\begin{ctmmRedBox}{Sofortige Hilfe}
Schneller Zugriff auf Ihre personalisierten Safe Words und Notfallkontakte.

\begin{center}
\begin{tabular}{c}
\textbf{QR-Code: Notfall-App}\\
\rule{3cm}{3cm}\\
\small{Hier würde ein QR-Code zur}\\
\small{CTMM-Notfall-App erscheinen}
\end{tabular}
\end{center}
\end{ctmmRedBox}

\subsection*{\textcolor{ctmmPurple}{Online-Community}}

\begin{ctmmGreenBox}{Austausch \& Support}
Vernetzen Sie sich mit anderen CTMM-Anwendern in einem sicheren Forum.

\begin{center}
\begin{tabular}{c}
\textbf{QR-Code: Community}\\
\rule{3cm}{3cm}\\
\small{Hier würde ein QR-Code zum}\\
\small{CTMM-Community-Forum erscheinen}
\end{tabular}
\end{center}
\end{ctmmGreenBox}

\subsection*{\textcolor{ctmmPurple}{Video-Tutorials}}

\begin{ctmmOrangeBox}{Lernen \& Vertiefen}
Schauen Sie sich praktische Anwendungsvideos für alle CTMM-Tools an.

\begin{center}
\begin{tabular}{|c|c|}
\hline
\textbf{Video-Thema} & \textbf{QR-Code / Link} \\
\hline
\textbf{4-7-8 Atemtechnik} & \href{https://youtube.com/watch?v=YRPh_GaiL8s}{\textcolor{ctmmBlue}{\faYoutube~3 Min Anleitung}} \\
\hline
\textbf{5-4-3-2-1 Grounding} & \href{https://youtube.com/watch?v=utUVx0ayoYw}{\textcolor{ctmmBlue}{\faYoutube~Grounding-Übung}} \\
\hline
\textbf{DBT Skills Demo} & \href{https://youtube.com/watch?v=q15eTySnWxc}{\textcolor{ctmmBlue}{\faYoutube~Skills Training}} \\
\hline
\textbf{Trigger-Management} & \href{https://youtube.com/watch?v=Mz3Mi_OZYno}{\textcolor{ctmmBlue}{\faYoutube~PTSD Coping}} \\
\hline
\textbf{Paartherapie-Kommunikation} & \href{https://youtube.com/watch?v=2s9ACDMcpjA}{\textcolor{ctmmBlue}{\faYoutube~Gottman-Methode}} \\
\hline
\end{tabular}
\end{center}

\vspace{0.5cm}
\textbf{Empfohlene Kanäle:}
\begin{itemize}
    \item \textcolor{ctmmGreen}{\faYoutube~Therapy in a Nutshell} - Trauma \& DBT Skills
    \item \textcolor{ctmmBlue}{\faYoutube~Kati Morton} - Mental Health Education  
    \item \textcolor{ctmmOrange}{\faYoutube~The School of Life} - Relationship Skills
    \item \textcolor{ctmmPurple}{\faYoutube~Marsha Linehan} - Original DBT Videos
\end{itemize}
\end{ctmmOrangeBox}

\vspace{1cm}
\begin{center}
\textit{\textcolor{ctmmPurple}{\faChevronRight~Weiter zu} \ctmmRef{sec:feedback}{Selbstreflexions-System} | \textcolor{ctmmGreen}{\faHome~Zurück zu} \ctmmRef{sec:navigation}{Navigation}}
\end{center}

% Therapiekoordination - Multidisziplinäres Treatment

\newpage
\section*{\textcolor{ctmmPurple}{\faUserMd~Therapie-Koordination}}
\addcontentsline{toc}{section}{Therapie-Koordination}
\label{sec:therapiekoordination}

\begin{quote}
\textit{\textcolor{ctmmPurple}{Koordination verschiedener Therapeuten für optimale Behandlung.}}\\
\textbf{\textcolor{ctmmPurple}{Multidisziplinärer Ansatz}}\\
Komplexe Fälle erfordern verschiedene Experten: Psychotherapeut, Neurologe, DBT-Spezialist, Trauma-Experte. Dieses Modul hilft bei der Koordination.
\end{quote}

\subsection*{\textcolor{ctmmPurple}{Mein Behandlungsteam}}

\begin{ctmmPurpleBox}{Therapeuten-Übersicht}
\begin{tabularx}{\textwidth}{|X|X|X|}
\hline
\textbf{Therapeut} & \textbf{Kontakt} & \textbf{Fokus} \\
\hline
\textbf{Psychotherapeut:} & \ctmmTextField[4cm]{}{therapist_psycho_contact} & PTBS, Borderline, ADHS \\
\hline
\textbf{Neurologe:} & \ctmmTextField[4cm]{}{therapist_neuro_contact} & Epilepsie, post-OP Betreuung \\
\hline
\textbf{DBT-Spezialist:} & \ctmmTextField[4cm]{}{therapist_dbt_contact} & Skills-Training, Emotionsregulation \\
\hline
\textbf{Trauma-Experte:} & \ctmmTextField[4cm]{}{therapist_trauma_contact} & KPTBS, Dissoziation \\
\hline
\textbf{Hausarzt:} & \ctmmTextField[4cm]{}{therapist_gp_contact} & Medikation, Koordination \\
\hline
\end{tabularx}
\end{ctmmPurpleBox}

\subsection*{\textcolor{ctmmPurple}{Aktuelle Behandlungsphase}}

\begin{ctmmBlueBox}{Dreiphasen-Modell (Trauma)}
\textbf{Phase 1 - Stabilisierung:} \\
\ctmmCheckBox{phase1_active}{Aktiv} \quad \ctmmCheckBox{phase1_completed}{Abgeschlossen} \\
\textbf{Schwerpunkt:} Achtsamkeit, Skills, Sicherheit \\[0.3cm]

\textbf{Phase 2 - Traumakonfrontation:} \\
\ctmmCheckBox{phase2_prep}{Vorbereitung} \quad \ctmmCheckBox{phase2_active}{Aktiv} \quad \ctmmCheckBox{phase2_completed}{Abgeschlossen} \\
\textbf{Schwerpunkt:} Verarbeitung traumatischer Erinnerungen \\[0.3cm]

\textbf{Phase 3 - Integration:} \\
\ctmmCheckBox{phase3_prep}{Vorbereitung} \quad \ctmmCheckBox{phase3_active}{Aktiv} \\
\textbf{Schwerpunkt:} Alltags-Integration, Zukunftsplanung
\end{ctmmBlueBox}

\subsection*{\textcolor{ctmmPurple}{DBT-Module-Fortschritt}}

\begin{ctmmGreenBox}{Skills-Training Status}
\begin{tabular}{|l|c|c|c|}
\hline
\textbf{DBT-Modul} & \textbf{Begonnen} & \textbf{In Arbeit} & \textbf{Abgeschlossen} \\
\hline
Achtsamkeit & \ctmmCheckBox{dbt_mindful_start}{} & \ctmmCheckBox{dbt_mindful_prog}{} & \ctmmCheckBox{dbt_mindful_done}{} \\
\hline
Stresstoleranz & \ctmmCheckBox{dbt_stress_start}{} & \ctmmCheckBox{dbt_stress_prog}{} & \ctmmCheckBox{dbt_stress_done}{} \\
\hline
Emotionsregulation & \ctmmCheckBox{dbt_emotion_start}{} & \ctmmCheckBox{dbt_emotion_prog}{} & \ctmmCheckBox{dbt_emotion_done}{} \\
\hline
Zwischenmenschliche Fertigkeiten (ZMF) & \ctmmCheckBox{dbt_zmf_start}{} & \ctmmCheckBox{dbt_zmf_prog}{} & \ctmmCheckBox{dbt_zmf_done}{} \\
\hline
\end{tabular}
\end{ctmmGreenBox}

\subsection*{\textcolor{ctmmPurple}{Neurologische Überwachung}}

\begin{ctmmOrangeBox}{Post-OP Monitoring (Hippocampus/Amygdala)}
\textbf{Letzte Kontrolle:} \ctmmTextField[4cm]{}{neuro_last_check} \\[0.3cm]
\textbf{Epilepsie-Medikation:} \ctmmTextField[6cm]{}{neuro_medication} \\[0.3cm]
\textbf{Neuropsychologische Tests:} \\
\ctmmCheckBox{neuro_memory}{Gedächtnistest} \quad \ctmmCheckBox{neuro_emotion}{Emotionsregulation} \\
\ctmmCheckBox{neuro_cognition}{Kognitive Funktion} \quad \ctmmCheckBox{neuro_other}{Andere:} \ctmmTextField[3cm]{}{neuro_other_text}
\end{ctmmOrangeBox}

\subsection*{\textcolor{ctmmPurple}{Therapie-Synchronisation}}

\textbf{Nächste Termine koordinieren:} \\
\ctmmTextArea[12cm]{2}{next_appointments}{}\\[0.3cm]

\textbf{Informationen zwischen Therapeuten teilen:} \\
\ctmmTextArea[12cm]{2}{info_sharing}{}\\[0.3cm]

\textbf{CTMM-Tools mit Therapeuten besprechen:} \\
\ctmmCheckBox{discuss_safewords}{Safe-Words System} \quad \ctmmCheckBox{discuss_triggers}{Trigger-Management} \\
\ctmmCheckBox{discuss_monitoring}{Depression-Monitor} \quad \ctmmCheckBox{discuss_progress}{Fortschrittsmessung}

\vspace{1cm}
\begin{center}
\textit{\textcolor{ctmmGreen}{\faChevronRight~Weiter zu} \ctmmRef{sec:qrcode}{QR-Code Integration} | \textcolor{ctmmPurple}{\faChevronLeft~Zurück zu} \ctmmRef{sec:feedback}{Selbstreflexion}}
\end{center}

% Feedback und Selbstreflexions-System
\newpage
\section*{\textcolor{ctmmPurple}{\faChartLine~Selbstreflexions-System}}
\addcontentsline{toc}{section}{Selbstreflexions-System}
\label{sec:feedback}

\begin{ctmmPurpleBox}{\faSync~Kontinuierliche Verbesserung}
\textit{Dieses System hilft uns, Fortschritte zu erfassen, Erfolge zu würdigen und Verbesserungsbereiche zu identifizieren. Es wächst mit uns mit.}
\end{ctmmPurpleBox}

\subsection*{\faCalendar~Tägliche Mikro-Reflexion (2 Minuten)}
\textit{Nach dem Abend-Check-In → \ctmmRef{sec:5.1}{\faEdit~Kap.5.1}}

\textbf{Drei einfache Fragen:}\\[0.3cm]
1. \textbf{Werkzeug-Check:} Welches Tool/Skill haben wir heute benutzt?\\
\ctmmTextField[10cm]{}{daily_tool}\\[0.3cm]

2. \textbf{Wirksamkeit:} Hat es geholfen? (1-10)\\
\ctmmTextField[3cm]{}{daily_effectiveness}\\[0.3cm]

3. \textbf{Anpassung:} Eine kleine Verbesserung für morgen?\\
\ctmmTextField[10cm]{}{daily_improvement}\\[0.5cm]

\subsection*{\faCalendar~Wöchentliche Mini-Retrospektive}
\textit{Sonntag beim Depression-Monitor → \ctmmRef{sec:5.3}{\faEdit~Kap.5.3}}

\begin{ctmmYellowBox}{Wochenreflexion}
\textbf{Genutzte Tools diese Woche:}\\
\ctmmCheckBox{weekly_safewords}{Safe-Words} \quad 	extbf{Skills:} \ctmmCheckBox{weekly_skills}{Neue Skills} \quad 	extbf{Rituale:} \ctmmCheckBox{weekly_rituals}{Gemeinsame Zeit}\\[0.3cm]

\textbf{Was hat gut funktioniert?}\\
\ctmmTextArea[12cm]{2}{week_success}{}\\[0.3cm]

\textbf{Was verbessern wir nächste Woche?}\\
\ctmmTextField[10cm]{}{week_improvement}\\[0.3cm]

\textbf{Ein Erfolg dieser Woche (egal wie klein):}\\
\ctmmTextField[10cm]{}{week_win}
\end{ctmmYellowBox}

\subsection*{\faChartLine~Fortschrittsmessung}

\begin{center}
\begin{tabularx}{\textwidth}{|X|X|X|X|}
\hline
\textbf{Metrik} & \textbf{Letzter Monat} & \textbf{Dieser Monat} & \textbf{Trend} \\
\hline
Safe-Word Nutzung/Woche & \ctmmTextField[2cm]{}{metric_safeword_last} & \ctmmTextField[2cm]{}{metric_safeword_now} & \ctmmCheckBox[trend_safeword_up]{↑} \ctmmCheckBox[trend_safeword_down]{↓} \\
\hline
Krisenzeit (Minuten) & \ctmmTextField[2cm]{}{metric_crisis_last} & \ctmmTextField[2cm]{}{metric_crisis_now} & \ctmmCheckBox{trend_crisis_up}{↑} \ctmmCheckBox{trend_crisis_down}{↓} \\
\hline
Schlafqualität (1-10) & \ctmmTextField[2cm]{}{metric_sleep_last} & \ctmmTextField[2cm]{}{metric_sleep_now} & \ctmmCheckBox{trend_sleep_up}{↑} \ctmmCheckBox{trend_sleep_down}{↓} \\
\hline
Gemeinsame Aktivitäten/Woche & \ctmmTextField[2cm]{}{metric_activities_last} & \ctmmTextField[2cm]{}{metric_activities_now} & \ctmmCheckBox{trend_activities_up}{↑} \ctmmCheckBox{trend_activities_down}{↓} \\
\hline
\end{tabularx}
\end{center}

\subsection*{\faTrophy~Erfolgs-Bibliothek}
\label{sec:erfolge}

\begin{ctmmGreenBox}{\faBullseye~Beweise sammeln}
\textbf{Datum:} \ctmmTextField[3cm]{}{success_date} \quad \textbf{Kategorie:} \ctmmCheckBox[cat_micro]{Mikro} \ctmmCheckBox[cat_breakthrough]{Durchbruch} \ctmmCheckBox[cat_surprise]{Überraschung}\\[0.3cm]

\textbf{Was ist passiert? (Konkrete Details)}\\
\ctmmTextArea[12cm]{3}{success_description}{}\\[0.3cm]

\textbf{Vorher vs. Nachher:}\\
\textbf{Früher:} \ctmmTextField[5cm]{}{success_before} \textbf{Heute:} \ctmmTextField[5cm]{}{success_after}\\[0.3cm]

\textbf{Gefühl und Bedeutung:}\\
\ctmmTextArea[12cm]{2}{success_meaning}{}\\[0.3cm]

\textbf{Was war entscheidend für diesen Erfolg?}\\
\ctmmTextField[10cm]{}{success_key_factor}
\end{ctmmGreenBox}

\subsection*{\faLightbulb~System-Anpassungen}

\begin{ctmmOrangeBox}{\faWrench~Unser System entwickelt sich weiter}
\textbf{Dokument-Feedback markieren:}\\
\faThumbtack~\textbf{Unklar:} \ctmmTextField[8cm]{}{feedback_unclear}\\
\faTimes~\textbf{Funktioniert nicht:} \ctmmTextField[8cm]{}{feedback_broken}\\
\faPlus~\textbf{Fehlt:} \ctmmTextField[8cm]{}{feedback_missing}\\
\faStar~\textbf{Besonders hilfreich:} \ctmmTextField[8cm]{}{feedback_helpful}\\[0.3cm]

\textbf{Neue Idee zum Testen:}\\
\ctmmTextArea[12cm]{2}{new_idea}{}\\[0.3cm]

\textbf{Gewünschte Änderung:}\\
\ctmmTextField[10cm]{}{desired_change}
\end{ctmmOrangeBox}

\vspace{1cm}
\begin{center}
\textit{\textcolor{ctmmGreen}{\faChevronRight~Weiter zu} \ctmmRef{sec:5.1}{Arbeitsblätter} | \textcolor{ctmmBlue}{\faHome~Zurück zu} \ctmmRef{sec:navigation}{Navigation}}
\end{center}


\newpage
\part*{KAPITEL 5: ARBEITSBLÄTTER}
\addcontentsline{toc}{part}{KAPITEL 5: ARBEITSBLÄTTER}

% Arbeitsblatt 5.1: Täglicher Check-In
\newpage
\section*{\textcolor{ctmmGreen}{\faCalendar~5.1 TÄGLICHER CHECK-IN}}
\label{sec:5.1}
\addcontentsline{toc}{section}{5.1 Täglicher Check-In}

\begin{ctmmGreenBox}{\faCircle~Jeden Morgen und Abend ausfüllen}
\textit{Kurze Selbsteinschätzung für bessere Selbstwahrnehmung und Kommunikation}
\end{ctmmGreenBox}

\vspace{0.5cm}

\subsection*{\faCalendar~Grunddaten}
\textbf{Datum:} \ctmmTextField[5.5cm]{}{date} \quad \textbf{Zeit:} \ctmmTextField[2.5cm]{}{time}\\[0.2cm]
\textbf{Check:} \ctmmCheckBox{morning}{Morgen} \ctmmCheckBox{evening}{Abend}

\vspace{0.5cm}

\subsection*{\faUsers~Beide Partner}

\begin{center}
\begin{tabularx}{\textwidth}{|l|X|X|}
\hline
\textbf{Bereich} & \textbf{ER} & \textbf{SIE} \\
\hline
\textbf{Stimmung (1-10):} & \ctmmTextField[3cm]{}{er_mood} & \ctmmTextField[3cm]{}{sie\_mm \\
\hline
\textbf{Schlafstunden:} & \ctmmTextField[3cm]{}{er_sleep} & \ctmmTextField[3cm]{}{sie\_mm \\
\hline
\textbf{Medikamente:} & \ctmmCheckBox{er_meds_yes}{OK} \ctmmCheckBox{er_meds_forgotten}{Vergessen} & \ctmmCheckBox{sie_meds_yes}{OK} \ctmmCheckBox{sie_meds_forgotten}{Vergessen} \\
\hline
\end{tabularx}
\end{center}

\vspace{0.7cm}
\subsection*{\faClipboard~Tagesplanung}
\textbf{Schwierige Termine heute:} \\
\ctmmTextField[12cm]{}{appointments}

\textbf{Support-Person verfügbar:} \\
\ctmmCheckBox{support_yes}{Ja, Name:} \ctmmTextField[4cm]{}{support\_mm \\
\ctmmCheckBox{support_no}{Nein, Backup:} \ctmmTextField[4cm]{}{support\_mm

\textbf{Safe-Words erklärt:} \\
\ctmmCheckBox{safewords_explained_guests}{Allen Anwesenden (Gäste/Familie)} \\
\ctmmCheckBox{safewords_explained_not_needed}{Nicht nötig (nur wir zwei)}

\vspace{0.7cm}
\subsection*{\faMoon~Abend-Reflexion}
\textbf{Was war heute gut:} \\
\ctmmTextField[12cm]{}{evening\_mm

\textbf{Was war schwierig:} \\
\ctmmTextField[12cm]{}{evening\_mm

\textbf{Safe-Word benutzt:} \ctmmCheckBox{safeword_no}{Nein} \ctmmCheckBox{safeword_yes}{Ja, welches:} \ctmmTextField[4cm]{}{safeword\_mm

\textbf{Morgen wichtig:} \\
\ctmmTextField[12cm]{}{tomorrow\_mm

\vspace{1cm}
\begin{center}
\textit{\textcolor{ctmmPurple}{\faChevronRight~Weiter zu} \ctmmRef{sec:5.2}{Trigger-Tagebuch} | \textcolor{ctmmGreen}{\faChevronLeft~Zurück zu} \ctmmRef{sec:navigation}{Navigation}}
\end{center}

% Arbeitsblatt 5.2: Trigger-Tagebuch
\newpage
\section*{\faBolt~5.2 TRIGGER-TAGEBUCH}
\label{sec:5.2}
\addcontentsline{toc}{section}{5.2 Trigger-Tagebuch}

\begin{ctmmYellowBox}{\faLightbulb~Tooltip}
Auch kleine Trigger zählen! Jedes Ausfüllen hilft uns beide besser zu verstehen.
\end{ctmmYellowBox}

\vspace{0.5cm}
\textbf{Datum/Zeit:} \ctmmTextField[8cm]{}{trigger_datetime}
\vspace{0.5cm}

\subsection*{\faUserFriends~TRIGGER-SITUATION}
\textit{Einfach ankreuzen was passt - kombinieren ist ok!}

\textbf{Ich war zusammen mit:} \\
\ctmmCheckBox[with_partner]{Partner} \quad
\ctmmCheckBox[with_friends]{Freunden} \quad
\ctmmCheckBox[with_family]{Familie} \quad
\ctmmCheckBox[with_others]{Anderen:} \ctmmTextField[3cm]{}{with_others_text}

\vspace{0.5cm}
\textbf{Triggermoment war:} \\
\ctmmCheckBox[trigger_sound]{Geräusch (laut/plötzlich)} \quad
\ctmmCheckBox[trigger_words]{Bestimmte Wörter/Sätze} \\
\ctmmCheckBox[trigger_tone]{Tonfall/Betonung} \quad
\ctmmCheckBox[trigger_bodylanguage]{Körpersprache} \\
\ctmmCheckBox[trigger_crowd]{Menschenmenge} \quad
\ctmmCheckBox[trigger_memory]{Erinnerung/Flashback} \\
\ctmmCheckBox[trigger_change]{Planänderung} \quad
\ctmmCheckBox[trigger_overwhelm]{Reizüberflutung} \\
\ctmmCheckBox[trigger_other]{Andere:} \ctmmTextField[3cm]{}{trigger_other_text}

\vspace{0.5cm}
\textbf{Beschreibe die Situation:} \\
\textit{Schüler-Tipp: "Es war wie..." oder "Es fühlte sich an wie..."} \\
\textit{Beispiel: "Wir waren im Supermarkt, als eine Kiste umfiel. Es war wie ein Blitz durch meinen Körper..."} \\
\ctmmTextArea[12cm]{2}{situation_desc}{}

\vspace{0.5cm}
\textbf{Welche Gefühle hat das bei mir ausgelöst:} \\
\textit{Du darfst mehrere ankreuzen - Gefühle sind oft gemischt!} \\
\ctmmCheckBox[feeling_angst]{Angst} \quad
\ctmmCheckBox[feeling_wut]{Wut} \quad
\ctmmCheckBox[feeling_trauer]{Trauer} \quad
\ctmmCheckBox[feeling_verlassenheit]{Verlassenheit} \\
\ctmmCheckBox[feeling_ueberforderung]{Überforderung} \quad
\ctmmCheckBox[feeling_hilflosigkeit]{Hilflosigkeit} \quad
\ctmmCheckBox[feeling_scham]{Scham} \quad
\ctmmCheckBox[feeling_verwirrung]{Verwirrung} \\
\ctmmCheckBox[feeling_other]{Andere:} \ctmmTextField[3cm]{}{feeling_other_text}

\subsection*{\faSync~REAKTIONEN}
\textit{Ehrlich sein hilft - wir urteilen nicht!}

\textbf{ER:} \ctmmCheckBox[reaktion_er_rueckzug]{Rückzug} \quad \ctmmCheckBox[reaktion_er_shutdown]{Shutdown} \quad \ctmmCheckBox[reaktion_er_panik]{Panik} \quad \ctmmCheckBox[reaktion_er_verwirrt]{Verwirrt} \\
\textbf{SIE:} \ctmmCheckBox[reaktion_sie_vorwuerfe]{Vorwürfe} \quad \ctmmCheckBox[reaktion_sie_klammern]{Klammern} \quad \ctmmCheckBox[reaktion_sie_weinen]{Weinen} \quad \ctmmCheckBox[reaktion_sie_wut]{Wut}

\vspace{0.5cm}
\textbf{Stress-Level (1-10):} \\
\textit{1=total entspannt, 10=extremer Notfall} \\
ER: \ctmmTextField[2cm]{}{stress_er} \quad SIE: \ctmmTextField[2cm]{}{stress_sie}

\subsection*{\faTools~WAS HALF}
\ctmmCheckBox[half_safeword]{Safe-Word benutzt:} \ctmmTextField[5cm]{}{half_safeword_text} \\
\ctmmCheckBox[half_skills]{Skills angewendet:} \ctmmTextField[5cm]{}{half_skills_text} \\
\ctmmCheckBox[half_friends]{Freunde geholt:} \ctmmTextField[5cm]{}{half_friends_text} \\
\ctmmCheckBox[half_pause]{Pause gemacht:} \ctmmTextField[5cm]{}{half_pause_text} \\
\ctmmCheckBox[half_other]{Andere:} \ctmmTextField[5cm]{}{half_other_text}

\subsection*{\faChartLine~LERNPUNKT}
\textbf{Nächstes Mal besser machen:} \\
\ctmmTextArea[12cm]{2}{learning_next_time}{}

\vspace{1cm}
\begin{center}
\textit{\textcolor{ctmmPurple}{\faChevronRight~Weiter zu} \ctmmRef{sec:5.3}{Depression-Monitor} | \textcolor{ctmmPurple}{\faChevronLeft~Zurück zu} \ctmmRef{sec:5.1}{Täglicher Check-In}}
\end{center}

% Arbeitsblatt 5.3: Depression-Monitoring
\newpage
\section*{\faChartLine~5.3 DEPRESSION-MONITORING}
\label{sec:5.3}
\addcontentsline{toc}{section}{5.3 Depression-Monitoring}

\begin{ctmmYellowBox}{\faLightbulb~Tooltip}
Dieses Blatt hilft uns, Muster in Stimmung und Energie zu erkennen. Sei ehrlich, es gibt keine 'richtigen' oder 'falschen' Antworten.
\end{ctmmYellowBox}

\vspace{0.5cm}
\textit{Fülle diese Tabelle bitte täglich aus, idealerweise immer zur gleichen Zeit. Das dauert weniger als 2 Minuten.}

\begin{center}
\renewcommand{\arraystretch}{1.8}
\begin{tabularx}{\textwidth}{|l|c|c|c|c|X|}
\hline
\textbf{Tag} & \textbf{Datum} & \textbf{Stimmung} (1-10) & \textbf{Energie} (1-10) & \textbf{Selbstfürsorge?} & \textbf{Notizen / Besonderheiten} \\
\hline
\textbf{Mo} & \dailyInput & \dailyInput & \dailyInput & \dailyInput & \\
\hline
\textbf{Di} & \dailyInput & \dailyInput & \dailyInput & \dailyInput & \\
\hline
\textbf{Mi} & \dailyInput & \dailyInput & \dailyInput & \dailyInput & \\
\hline
\textbf{Do} & \dailyInput & \dailyInput & \dailyInput & \dailyInput & \\
\hline
\textbf{Fr} & \dailyInput & \dailyInput & \dailyInput & \dailyInput & \\
\hline
\textbf{Sa} & \dailyInput & \dailyInput & \dailyInput & \dailyInput & \\
\hline
\textbf{So} & \dailyInput & \dailyInput & \dailyInput & \dailyInput & \\
\hline
\hline
\textbf{Mo} & \dailyInput & \dailyInput & \dailyInput & \dailyInput & \\
\hline
\textbf{Di} & \dailyInput & \dailyInput & \dailyInput & \dailyInput & \\
\hline
\textbf{Mi} & \dailyInput & \dailyInput & \dailyInput & \dailyInput & \\
\hline
\textbf{Do} & \dailyInput & \dailyInput & \dailyInput & \dailyInput & \\
\hline
\textbf{Fr} & \dailyInput & \dailyInput & \dailyInput & \dailyInput & \\
\hline
\textbf{Sa} & \dailyInput & \dailyInput & \dailyInput & \dailyInput & \\
\hline
\textbf{So} & \dailyInput & \dailyInput & \dailyInput & \dailyInput & \\
\hline
\end{tabularx}
\end{center}

\vspace{1cm}
\subsection*{\faQuestionCircle~AUSWERTUNG NACH 2 WOCHEN}
\textit{Nimm dir einen Moment Zeit, um die Tabelle zu betrachten. Was fällt dir auf?}

\begin{ctmmBlueBox}{Reflexionsfragen}
\begin{itemize}
    \item Was fällt mir auf, wenn ich die letzten zwei Wochen betrachte? Gibt es ein Muster?
    \item Gab es Tage, an denen meine Stimmung oder Energie besonders niedrig war? Was war an diesen Tagen los? (Siehe Trigger-Tagebuch)
    \item Welche Selbstfürsorge-Aktivitäten haben mir geholfen, auch nur ein bisschen?
    \item Was ist der eine kleine Schritt, den ich nächste Woche tun kann, um meine Energie oder Stimmung zu stabilisieren?
\end{itemize}
\end{ctmmBlueBox}

\vspace{1cm}
\begin{center}
\textit{\textcolor{ctmmPurple}{\faChevronRight~Weiter zu} \ctmmRef{sec:demo-interactive}{Interaktive Demo} | \textcolor{ctmmBlue}{\faHeart~Hilfe bei} \ctmmRef{sec:depression}{Depression-Modul}}
\end{center}

% Demo für interaktive PDF-Formulare
\newpage
\section*{\textcolor{ctmmBlue}{\faEdit~Demo: Interaktive PDF-Formulare}}
\label{sec:demo-interactive}
\addcontentsline{toc}{section}{Demo: Interaktive Formulare}

\begin{ctmmBlueBox}{Testen Sie die interaktiven Felder}
\textbf{Hinweis:} Diese Felder können direkt im PDF ausgefüllt werden. Die Eingaben werden lokal in der PDF-Datei gespeichert.
\end{ctmmBlueBox}

\vspace{1cm}

\subsection*{Textfelder}
\textbf{Name:} \ctmmTextField[6cm]{}{demo_name} \\[0.5cm]
\textbf{E-Mail:} \ctmmTextField[8cm]{}{demo_email} \\[0.5cm]
\textbf{Telefon:} \ctmmTextField[4cm]{}{demo_phone}

\vspace{1cm}

\subsection*{Checkboxen}
\textbf{Hobbys:} \\[0.3cm]
\ctmmCheckBox[hobby_reading]{Lesen} \quad
\ctmmCheckBox[hobby_music]{Musik} \quad
\ctmmCheckBox[hobby_sports]{Sport} \\[0.3cm]
\ctmmCheckBox[hobby_cooking]{Kochen} \quad
\ctmmCheckBox[hobby_travel]{Reisen} \quad
\ctmmCheckBox[hobby_gaming]{Gaming}

\vspace{1cm}

\subsection*{Mehrzeilige Textfelder}
\textbf{Beschreiben Sie Ihren Tag:} \\[0.3cm]
\ctmmTextArea[12cm]{3}{demo_day_description}{}

\vspace{0.5cm}

\textbf{Ziele für morgen:} \\[0.3cm]
\ctmmTextArea[12cm]{2}{demo_tomorrow_goals}{}

\vspace{1cm}

\begin{ctmmYellowBox}{\faInfoCircle~Funktionen}
\begin{itemize}
    \item \textbf{Ausfüllen:} Klicken Sie in die Felder und geben Sie Text ein
    \item \textbf{Checkboxen:} Klicken Sie die Kästchen an/ab
    \item \textbf{Speichern:} Strg+S speichert Ihre Eingaben in der PDF
    \item \textbf{Drucken:} Die Felder werden mit den Eingaben gedruckt
    \item \textbf{Reset:} Formular → Alle Felder löschen (im PDF-Viewer)
\end{itemize}
\end{ctmmYellowBox}

\vspace{1cm}

\begin{ctmmGreenBox}{Vorteile für das CTMM-System}
\begin{itemize}
    \item \textbf{Digitale Dokumentation:} Alle Eingaben direkt im PDF
    \item \textbf{Datenschutz:} Daten bleiben lokal bei Ihnen
    \item \textbf{Archivierung:} Gespeicherte PDFs für Verlaufsdokumentation
    \item \textbf{Therapeutische Auswertung:} LLMs können die Daten analysieren
    \item \textbf{Flexibilität:} Online ausfüllen oder ausdrucken
\end{itemize}
\end{ctmmGreenBox}


\end{document}