\documentclass[a4paper,12pt]{article} % <-- HINZUGEFÜGT

% Pakete
% \usepackage[utf8]{inputenc} % Nicht mehr nötig mit aktuellem pdfLaTeX
\usepackage{lmodern} % Schriftartenproblem beheben
\usepackage{microtype} % Verbesserte Schriftbildqualität und weniger Font-Warnungen
\usepackage{textcomp} % Zusätzliche Symbole und Font-Kompatibilität
\usepackage[ngerman]{babel}
\usepackage{geometry}

% CTMM Pakete
\usepackage{style/ctmm-config}
\usepackage{xcolor}
\usepackage{fontawesome5}
\usepackage{tcolorbox}
\usepackage{tabularx}
\usepackage{amssymb}
\usepackage{amsmath}
\usepackage{style/ctmm-design}
\usepackage{style/ctmm-navigation}
\usepackage{style/ctmm-form-elements} % Globale Dummy-Formularelemente
% hyperref MUSS als letztes geladen werden (vor bookmark!)
\usepackage{hyperref}
\usepackage{bookmark} % Optional: bessere Lesezeichen

% Dokumentenkonfiguration
\geometry{
  a4paper,
  margin=2.5cm,
  top=3cm,
  bottom=3cm
}

\hypersetup{
  % pdfborder={0 0 0}, % Optional: kann entfernt werden, falls Fehler
  pdftitle={CTMM-System - Interaktives Therapie-Workbook},
  pdfauthor={CTMM-Team},
  pdfsubject={Therapiematerialien mit ausfüllbaren Formularen},
  pdfkeywords={CTMM, Therapie, Trigger, Borderline, ADHS, ASS, KPTBS, interaktiv},
  pdfcreator={LaTeX mit hyperref},
  pdfproducer={pdfLaTeX},
  bookmarksopen=true,
  bookmarksopenlevel=1
}

\hbadness=10000
\vbadness=10000
\begin{document}

\title{%
  {\Huge\textcolor{ctmmBlue}{CTMM-System}}\\
  \vspace{0.5cm}
  {\Large\textcolor{ctmmOrange}{Catch-Track-Map-Match}}\\
  \vspace{0.5cm}
  {\normalsize\textcolor{ctmmGreen}{Therapiematerialien für neurodiverse Paare}}
}
\author{CTMM-Team}
\date{\today}
\maketitle

\tableofcontents
\newpage

\section*{\texorpdfstring{\textcolor{ctmmBlue}{\faCompass~CTMM-System Übersicht}}{CTMM-System Übersicht}}
\addcontentsline{toc}{section}{CTMM-System Übersicht}

\begin{ctmmBlueBox}{Was ist CTMM?}
\textbf{CTMM} steht für \textbf{Catch-Track-Map-Match} -- ein System zur Bewältigung von Triggern und Beziehungsherausforderungen:

\begin{itemize}
    \item \textcolor{ctmmBlue}{\textbf{Catch:}} Trigger erkennen
    \item \textcolor{ctmmGreen}{\textbf{Track:}} Gefühl \& Situation verfolgen  
    \item \textcolor{ctmmOrange}{\textbf{Map:}} Muster verstehen
    \item \textcolor{ctmmPurple}{\textbf{Match:}} Handlung anpassen
\end{itemize}
\end{ctmmBlueBox}

% Module einbinden
% Navigation und Querverweise für CTMM-System
\section*{\textcolor{ctmmBlue}{\faMap~Navigations-System}}
\addcontentsline{toc}{section}{Navigations-System}
\label{sec:navigation}

\begin{ctmmBlueBox}{Orientierung im CTMM-System}
\textbf{Farbkodierung für schnelle Navigation:}

\begin{itemize}
    \item \textcolor{ctmmBlue}{\textbf{BLAU - Grundlagen}} - Warum wir uns triggern
    \item \textcolor{ctmmGreen}{\textbf{GRÜN - Tägliche Tools}} - Skills und Routinen  
    \item \textcolor{ctmmRed}{\textbf{ROT - Notfall-Guide}} - Krisenintervention
    \item \textcolor{ctmmYellow}{\textbf{GELB - Support}} - Freunde und Familie
    \item \textcolor{ctmmPurple}{\textbf{LILA - Arbeitsblätter}} - Tracking und Reflexion
\end{itemize}
\end{ctmmBlueBox}

\vspace{0.5cm}

\subsection*{\faChevronRight~Schnell-Navigation}

\begin{ctmmGreenBox}{GRÜN: Tägliche Tools - Jeden Tag nutzen}
\begin{itemize}
    \item \ctmmRef{sec:5.1}{Täglicher Check-In} - Morgens und abends
    \item \ctmmRef{sec:safewords}{Safe-Words System} - Bei Überforderung
    \item \ctmmRef{sec:triggermanagement}{Trigger-Management} - Präventiv
\end{itemize}
\end{ctmmGreenBox}

\begin{ctmmRedBox}{ROT: Notfall-Protokolle - In Krisen}
\begin{itemize}
    \item \ctmmRef{sec:notfallkarten}{Notfallkarten} - Sofort verfügbar
    \item \ctmmRef{sec:5.2}{Trigger-Tagebuch} - Nach der Krise
    \item \ctmmRef{sec:5.3}{Depression-Monitor} - Wöchentlich
\end{itemize}
\end{ctmmRedBox}

\begin{ctmmPurpleBox}{LILA: Reflexion \& Wachstum - Regelmäßig}
\begin{itemize}
    \item \ctmmRef{sec:feedback}{Selbstreflexions-System} - Monatlich
    \item \ctmmRef{sec:fortschritt}{Fortschrittsmessung} - Kontinuierlich
    \item \ctmmRef{sec:erfolge}{Erfolgs-Bibliothek} - Motivation
\end{itemize}
\end{ctmmPurpleBox}

\vspace{0.5cm}

\subsection*{\faQuestionCircle~Häufige Situationen}

\begin{center}
\begin{tabular}{p{5cm}p{8cm}}
\textbf{Situation} & \textbf{Gehe zu} \\
\hline
Überforderung spürbar & $\rightarrow$ \ctmmRef{sec:safewords}{Safe-Words} \\
Nach einem Streit & $\rightarrow$ \ctmmRef{sec:5.2}{Trigger-Tagebuch} \\
Schlechte Schlafqualität & $\rightarrow$ \ctmmRef{sec:5.3}{Depression-Monitor} \\
Erfolg feiern & $\rightarrow$ \ctmmRef{sec:erfolge}{Erfolgs-Bibliothek} \\
System anpassen & $\rightarrow$ \ctmmRef{sec:feedback}{Selbstreflexion} \\
Therapeuten koordinieren & $\rightarrow$ \ctmmRef{sec:therapiekoordination}{Therapie-Team} \\
Morgen-Routine & $\rightarrow$ \ctmmRef{sec:5.1}{Täglicher Check-In} \\
Krise eskaliert & $\rightarrow$ \ctmmRef{sec:notfallkarten}{Notfallkarten} \\
\end{tabular}
\end{center}

\vspace{0.5cm}

\subsection*{\faClock~Nach Tageszeit}

\begin{ctmmGreenBox}{Morgens (7-10 Uhr)}
\begin{enumerate}
    \item \ctmmRef{sec:5.1}{Täglicher Check-In} - Status beider Partner
    \item Medikamente-Check und Tagesplanung
    \item Support-Person informieren bei schwierigen Terminen
\end{enumerate}
\end{ctmmGreenBox}

\begin{ctmmYellowBox}{Abends (19-22 Uhr)}
\begin{enumerate}
    \item \ctmmRef{sec:5.1}{Abend-Reflexion} - Tag verarbeiten
    \item Bei Triggern: \ctmmRef{sec:5.2}{Trigger-Tagebuch} ausfüllen
    \item Morgen vorbereiten, Rituale pflegen
\end{enumerate}
\end{ctmmYellowBox}

\begin{ctmmRedBox}{Wöchentlich (Sonntags)}
\begin{enumerate}
    \item \ctmmRef{sec:5.3}{Depression-Monitor} auswerten
    \item \ctmmRef{sec:feedback}{Wochenreflexion} durchführen
    \item Erfolge der Woche dokumentieren
\end{enumerate}
\end{ctmmRedBox}

\vspace{0.5cm}

\textit{Tipp: Nutzen Sie die Farbkodierung, um schnell das richtige Werkzeug zu finden!}

\section*{\textcolor{ctmmBlue}{\faCloudRain~Depression \& Stimmungstief -- Frühwarnung \& Handlungssicherheit}}

\begin{quote}
\textbf{\textcolor{ctmmBlue}{Worum geht's hier -- für Freunde \& Schüler?}}\\
Depression wirkt leise, aber mächtig. Dieses Modul hilft dir (und deinem Umfeld), erste Anzeichen zu erkennen, Eskalationen vorzubeugen und gemeinsam handlungsfähig zu bleiben. Es geht nicht um Diagnose -- sondern um Sicherheit, Struktur und Mitgefühl.
\end{quote}

\textit{Verknüpfbar mit Tool 23 (Trigger), Tool 26 (Ko-Regulation), Matching-Tracker, Safe Words \& Rückkehrritualen}

\subsection*{\textcolor{ctmmBlue}{Kapitelzuordnung im CTMM-System}}

\begin{itemize}
  \item \texttt{Kap. 2.5} → Selbstwahrnehmung \& Antrieb
  \item \texttt{Kap. 3.2} → Isolation \& Rückzug
  \item \texttt{Kap. 4.4} → Überforderung, Erschöpfung \& Schutz
  \item \texttt{Kap. 5.2/5.3} → Trigger-Frühzeichen \& Matching
\end{itemize}

\subsection*{\textcolor{ctmmBlue}{Farbcode \& Systemnavigation}}

\begin{tabular}{|c|l|l|}
\hline
\textbf{Farbe} & \textbf{Phase} & \textbf{Verknüpfte Module} \\
\hline
\textcolor{ctmmBlue}{$\bullet$} & Beobachtung & \texttt{tool\_23\_triggermanagement} \\
\textcolor{ctmmRed}{$\bullet$} & Eskalation / Rückzug & \texttt{trigger\_notfallkarten} \\
\textcolor{ctmmOrange}{$\bullet$} & Stimmung stabilisieren & \texttt{tool\_24\_skills\_sie}, \texttt{tool\_21} \\
\textcolor{ctmmPurple}{$\bullet$} & Rückkehr \& Integration & \texttt{ritual\_workbook}, \texttt{bindungsleitfaden\_ctmm} \\
\hline
\end{tabular}

\subsection*{\textcolor{ctmmBlue}{Frühwarnzeichen bei Depression}}

\begin{tabular}{|p{7cm}|p{7cm}|}
\hline
\textbf{Innerlich bei mir} & \textbf{Sichtbar für andere} \\
\hline
Ich will niemanden sehen & Rückzug, Zimmer bleibt dunkel \\
Ich empfinde nichts mehr & Keine Freude, kein Interesse \\
Alles wirkt anstrengend & Langsamer Gang, leise Stimme \\
Ich fühle mich wertlos & Selbstabwertung, Vermeidung \\
Ich denke, ich bin eine Last & Schuldgefühle, Isolation \\
\hline
\end{tabular}

% Bindungsleitfaden - CTMM Modul

\newpage
\section*{\textcolor{ctmmBlue}{\faHeart~Bindungsleitfaden}}
\addcontentsline{toc}{section}{Bindungsleitfaden}

\begin{quote}
\textit{\textcolor{ctmmOrange}{Bindung ist die Basis für Sicherheit, Vertrauen und gemeinsames Wachstum.}}\\
\textbf{\textcolor{ctmmBlue}{Sichere Bindung als Fundament}}\\
Bindung beschreibt das emotionale Band zwischen Menschen, das Sicherheit, Vertrauen und Nähe ermöglicht. Im CTMM-Kontext ist Bindung die Basis für Entwicklung und Veränderung.
\end{quote}

\subsection*{\textcolor{ctmmBlue}{Bindungstypen}}

\begin{ctmmBlueBox}{Die vier Bindungsstile}
\begin{itemize}
  \item \textbf{Sichere Bindung}: Vertrauen, Nähe, Unterstützung
  \item \textbf{Unsichere Bindung}: Angst, Unsicherheit, Rückzug
  \item \textbf{Ambivalente Bindung}: Schwankend zwischen Nähe und Distanz
  \item \textbf{Desorganisierte Bindung}: Widersprüchliches Verhalten
\end{itemize}
\end{ctmmBlueBox}

\subsection*{\textcolor{ctmmBlue}{Bindung im Alltag}}

\begin{ctmmGreenBox}{Bindung stärken}
\begin{itemize}
  \item Verlässlichkeit zeigen
  \item Zuhören und Verständnis signalisieren
  \item Gemeinsame Rituale pflegen
  \item Gefühle benennen und annehmen
\end{itemize}
\end{ctmmGreenBox}

\subsection*{\textcolor{ctmmBlue}{Bindung und CTMM}}
Bindung ist die Grundlage für die Arbeit mit Triggern, Mustern und Veränderungen im CTMM-System. Ein sicherer Bindungsrahmen erleichtert die Anwendung der Tools und fördert nachhaltige Entwicklung.

% ================================================================
% CTMM Module: Co-Regulation & Gemeinsame Stärkung
% Basierend auf: Tool 26 Co Regulation & Gemeinsame Stärkung CTMM.docx
% ================================================================

\section{Co-Regulation \& Gemeinsame Stärkung}
\label{sec:co-regulation}

\begin{ctmmBlueBox}[title=Tool 26: Co-Regulation System]

\subsection{Was ist Co-Regulation?}

Co-Regulation beschreibt die Fähigkeit, emotionale Zustände zwischen Partnern zu synchronisieren und sich gegenseitig zu regulieren. Es ist ein fundamentaler Baustein stabiler Beziehungen.

\vspace{0.5em}

\textbf{Grundprinzipien:}
\begin{itemize}
    \item \textcolor{ctmmBlue}{\textbf{Emotionale Abstimmung:}} Bewusste Wahrnehmung der Gefühlslage des Partners
    \item \textcolor{ctmmGreen}{\textbf{Synchronisation:}} Angleichung von Atmung, Körperhaltung und Energie
    \item \textcolor{ctmmOrange}{\textbf{Unterstützung:}} Aktive Hilfe bei der emotionalen Regulation
\end{itemize}

\end{ctmmBlueBox}

\subsection{Co-Regulation Techniken}

\begin{ctmmGreenBox}[title=Praktische Übungen]

\subsubsection{1. Atemkopplung}
\begin{enumerate}
    \item Setzt euch bequem gegenüber
    \item Achtet auf den Atemrhythmus des Partners
    \item Passt euren Atem langsam an
    \item Haltet die Synchronisation 2-3 Minuten
\end{enumerate}

\textbf{Reflexion:} \\
\ctmmTextArea[14cm]{3}{coregulation_breath_reflection}{}

\subsubsection{2. Emotionales Spiegeln}
\begin{itemize}
    \item \textbf{Wahrnehmen:} "Ich sehe, dass du angespannt bist"
    \item \textbf{Verstehen:} "Das muss schwer für dich sein"  
    \item \textbf{Unterstützen:} "Wie kann ich dir helfen?"
\end{itemize}

\textbf{Erfahrungen notieren:} \\
\ctmmTextArea[14cm]{4}{coregulation_mirroring_experience}{}

\end{ctmmGreenBox}

\subsection{Gemeinsame Stärkungsrituale}

\begin{ctmmOrangeBox}[title=Tägliche Verbindungsrituale]

\subsubsection{Morgenritual}
\ctmmCheckBox{morning_checkin}{Check-In Gespräch (5 min)}\\
\ctmmCheckBox{morning_touch}{Körperliche Nähe (Umarmung, Hand halten)}\\
\ctmmCheckBox{morning_intention}{Gemeinsame Tagesintention setzen}\\

\subsubsection{Abendritual}  
\ctmmCheckBox{evening_reflection}{Tagesreflexion gemeinsam}\\
\ctmmCheckBox{evening_gratitude}{3 Dinge wertschätzen}\\
\ctmmCheckBox{evening_connection}{Bewusste Verbindung (ohne Ablenkung)}\\

\textbf{Welche Rituale funktionieren für uns?} \\
\ctmmTextArea[14cm]{3}{coregulation_rituals_feedback}{}

\end{ctmmOrangeBox}

\subsection{Notfall Co-Regulation}

\begin{ctmmRedBox}[title=\textcolor{white}{Akute Regulationshilfe}]

\textcolor{white}{\textbf{Bei Überforderung des Partners:}}

\begin{enumerate}
    \item[\textcolor{white}{1.}] \textcolor{white}{\textbf{STOPP} - Eigene Reaktion regulieren}
    \item[\textcolor{white}{2.}] \textcolor{white}{\textbf{ATMEN} - Ruhige, tiefe Atmung vorleben}
    \item[\textcolor{white}{3.}] \textcolor{white}{\textbf{NÄHE} - Körperliche Präsenz anbieten}
    \item[\textcolor{white}{4.}] \textcolor{white}{\textbf{FÜHREN} - Ruhige Stimme, klare Anweisungen}
\end{enumerate}

\textcolor{white}{\textbf{Notfall-Sätze:}}
\begin{itemize}
    \item[\textcolor{white}{•}] \textcolor{white}{\glqq Atme mit mir zusammen\grqq}
    \item[\textcolor{white}{•}] \textcolor{white}{\glqq Du bist sicher, ich bin hier\grqq}
    \item[\textcolor{white}{•}] \textcolor{white}{\glqq Wir schaffen das zusammen\grqq}
\end{itemize}

\end{ctmmRedBox}

\subsection{Co-Regulation Tracker}

\begin{ctmmGreenBox}[title=Täglicher Co-Regulation Tracker]

\textbf{Datum:} \ctmmDate{coregulation} \quad \textbf{Zeit:} \ctmmTime{coregulation}\\[0.5em]

\textbf{Gemeinsame Aktivitäten heute:} \\
\ctmmCheckBox{coregulation_breath}{Atemkopplung} \quad
\ctmmCheckBox{coregulation_mirror}{Körperspiegelung} \quad
\ctmmCheckBox{coregulation_ritual}{Gemeinsame Rituale}\\[0.5em]

\textbf{Qualität der Co-Regulation (1-10):} \\
\ctmmEmotionScale{Co-Regulation Qualität}{coregulation_quality}\\[0.5em]

\textbf{Reflexion:} \\
\ctmmTextArea[14cm]{3}{coregulation_daily_reflection}{}

\end{ctmmGreenBox}

\subsection{Langfristige Entwicklung}

\begin{ctmmPurpleBox}[title=Wöchentliche Reflexion]

\textbf{Woche vom:} \ctmmTextField[4cm]{coregulation_week_date}{} \\[0.5em]

\textbf{Co-Regulation Erfolge:} \\
\ctmmTextArea[14cm]{3}{coregulation_successes}{}

\textbf{Herausforderungen:} \\  
\ctmmTextArea[14cm]{3}{coregulation_challenges}{}

\textbf{Nächste Schritte:} \\
\ctmmTextArea[14cm]{2}{coregulation_next_steps}{}

\textbf{Bewertung der Woche (1-10):} \\
\ctmmEmotionScale{Gesamtbewertung}{coregulation_week_rating}

\end{ctmmPurpleBox}

% ================================================================
% Integration in bestehende CTMM Module
% ================================================================

\subsection{Verbindung zu anderen Tools}

\begin{itemize}
    \item \textbf{Tool 22 (Safewords):} Co-Regulation als Alternative zu Stopp-Signalen
    \item \textbf{Tool 23 (Triggermanagement):} Gemeinsame Trigger-Regulation  
    \item \textbf{Bindungsleitfaden:} Co-Regulation stärkt sichere Bindung
    \item \textbf{Krisenprotokoll:} Co-Regulation in Krisensituationen
\end{itemize}

\textbf{Querverweise:}
\begin{itemize}
    \item Siehe \hyperref[sec:safewords]{Safewords \& Signalsysteme} (Seite \pageref{sec:safewords})
    \item Siehe \hyperref[sec:triggermanagement]{Triggermanagement} (Seite \pageref{sec:triggermanagement})
    \item Siehe \hyperref[sec:bindung]{Bindungsleitfaden} (Seite \pageref{sec:bindung})
\end{itemize}

\newpage

% ================================================================
% CTMM Module: Krisenprotokoll Ausfüllen
% Basierend auf: 5.5 Krisenprotokoll Ausfüllen.docx
% ================================================================

\section{Krisenprotokoll - Strukturierte Krisenbewältigung}
\label{sec:krisenprotokoll}

\begin{ctmmRedBox}[title=\textcolor{white}{CTMM Krisenprotokoll v5.5}]

\textcolor{white}{\textbf{Wichtig:} Dieses Protokoll dient der strukturierten Dokumentation und Nachbereitung von Krisensituationen. Es ersetzt NICHT professionelle Hilfe in akuten Notfällen.}

\vspace{0.5em}

\textcolor{white}{\textbf{Notfallnummern:}}
\begin{itemize}
    \item[\textcolor{white}{•}] \textcolor{white}{Rettungsdienst: 112}
    \item[\textcolor{white}{•}] \textcolor{white}{Krisentelefon: 0800 111 0 111}  
    \item[\textcolor{white}{•}] \textcolor{white}{Therapeut/in: \TextField[name=crisis_therapist_number,width=6cm,height=1em,bordercolor={1 1 1},backgroundcolor={0.9 0.9 0.9}]{}}
\end{itemize}

\end{ctmmRedBox}

\subsection{Sofortmaßnahmen in der Krise}

\begin{ctmmOrangeBox}[title=Erste Schritte]

\textbf{1. Sicherheit schaffen}
\begin{itemize}
    \item \ctmmCheckBox{crisis_safety_self}{Eigene Sicherheit gewährleistet}
    \item \ctmmCheckBox{crisis_safety_partner}{Sicherheit des Partners gewährleistet}
    \item \ctmmCheckBox{crisis_safety_environment}{Sichere Umgebung geschaffen}
\end{itemize}

\textbf{2. Safe-Words einsetzen}
\begin{itemize}
    \item \ctmmCheckBox{crisis_safeword_anker}{"Anker!" verwendet}
    \item \ctmmCheckBox{crisis_safeword_reset}{"Reset!" verwendet}  
    \item \ctmmCheckBox{crisis_safeword_eiszeit}{"Eiszeit!" verwendet}
    \item \ctmmCheckBox{crisis_safeword_other}{Anderes: \ctmmTextField[4cm]{crisis_safeword_custom}{}}
\end{itemize}

\textbf{3. Grundlegende Deeskalation}
\begin{itemize}
    \item \ctmmCheckBox{crisis_deescalation_breathing}{Atemtechnik angewendet}
    \item \ctmmCheckBox{crisis_deescalation_grounding}{Grounding-Technik verwendet}
    \item \ctmmCheckBox{crisis_deescalation_separation}{Räumliche Trennung vorgenommen}
    \item \ctmmCheckBox{crisis_deescalation_timeout}{Time-Out vereinbart}
\end{itemize}

\end{ctmmOrangeBox}

\subsection{Dokumentation der Krisensituation}

\begin{ctmmGreenBox}[title=Krisendokumentation]

\textbf{Datum und Zeit:} \\
\ctmmDateField{crisis_date}{Datum}{TT.MM.JJJJ} \quad \ctmmTimeField{crisis_time}{Beginn der Krise}

\textbf{Dauer der Krise:} \\
\ctmmTextField[4cm]{crisis_duration}{}{Minuten/Stunden}

\textbf{Beteiligte Personen:} \\
\ctmmCheckBox{crisis_involved_self}{Ich} \quad
\ctmmCheckBox{crisis_involved_partner}{Partner/in} \quad  
\ctmmCheckBox{crisis_involved_others}{Andere: \ctmmTextField[6cm]{crisis_involved_other_names}{}}

\textbf{Ort der Krise:} \\
\ctmmTextField[10cm]{crisis_location}{}{}

\textbf{Auslöser/Trigger:} \\
\ctmmTextArea[14cm]{4}{crisis_trigger}{Was hat die Krise ausgelöst?}

\textbf{Verlauf der Krise:} \\
\ctmmTextArea[14cm]{5}{crisis_course}{Beschreibung des Krisenverlaufs}

\end{ctmmGreenBox}

\subsection{Intensitätsskala und Auswirkungen}

\begin{ctmmBlueBox}[title=Bewertung der Krisensituation]

\textbf{Krisenlevel (1-10):} \\
\ctmmEmotionScale{Gesamtintensität der Krise}{crisis_intensity}

\vspace{0.5em}

\textbf{Emotionale Belastung:} \\[0.5em]

\textbf{Eigene Belastung (1-10):} \\
\ctmmEmotionScale{Meine Belastung}{crisis_own_stress}

\vspace{0.5em}

\textbf{Partner Belastung (1-10):} \\
\ctmmEmotionScale{Partner Belastung}{crisis_partner_stress}

\textbf{Körperliche Symptome:} \\
\begin{itemize}
    \item \ctmmCheckBox{crisis_symptoms_trembling}{Zittern}
    \item \ctmmCheckBox{crisis_symptoms_sweating}{Schwitzen}
    \item \ctmmCheckBox{crisis_symptoms_heartbeat}{Herzrasen}
    \item \ctmmCheckBox{crisis_symptoms_breathing}{Atemnot}
    \item \ctmmCheckBox{crisis_symptoms_tension}{Muskelverspannung}
    \item \ctmmCheckBox{crisis_symptoms_other}{Andere: \ctmmTextField[6cm]{crisis_symptoms_other_desc}{}}
\end{itemize}

\end{ctmmBlueBox}

\subsection{Angewandte Strategien und Hilfsmaßnahmen}

\begin{ctmmPurpleBox}[title=Eingesetzte Tools und Techniken]

\textbf{CTMM-Tools verwendet:} \\
\begin{itemize}
    \item \ctmmCheckBox{crisis_tools_breathing}{5-4-3-2-1 Grounding}
    \item \ctmmCheckBox{crisis_tools_safewords}{Safeword-System}
    \item \ctmmCheckBox{crisis_tools_coregulation}{Co-Regulation}
    \item \ctmmCheckBox{crisis_tools_trigger}{Triggermanagement}
    \item \ctmmCheckBox{crisis_tools_binding}{Bindungstools}
\end{itemize}

\textbf{Externe Hilfe:} \\
\begin{itemize}
    \item \ctmmCheckBox{crisis_help_emergency}{Notdienst kontaktiert}
    \item \ctmmCheckBox{crisis_help_therapist}{Therapeut/in kontaktiert}
    \item \ctmmCheckBox{crisis_help_crisis_line}{Krisentelefon genutzt}
    \item \ctmmCheckBox{crisis_help_family}{Familie/Freunde kontaktiert}
    \item \ctmmCheckBox{crisis_help_none}{Keine externe Hilfe nötig}
\end{itemize}

\textbf{Was hat geholfen?} \\
\ctmmTextArea[14cm]{4}{crisis_what_helped}{Welche Maßnahmen waren erfolgreich?}

\textbf{Was hat nicht geholfen?} \\
\ctmmTextArea[14cm]{3}{crisis_what_not_helped}{Welche Strategien waren ineffektiv?}

\end{ctmmPurpleBox}

\subsection{Nachbereitung und Reflexion}

\begin{ctmmGreenBox}[title=Post-Krisen Analyse]

\textbf{Zeitpunkt der Nachbereitung:} \\
\ctmmDateField{crisis_debrief_date}{Datum der Nachbereitung}{TT.MM.JJJJ} \quad \ctmmTimeField{crisis_debrief_time}{Zeit}

\textbf{Emotionaler Zustand nach der Krise:} \\
\ctmmEmotionScale{Befinden nach der Krise}{crisis_after_mood}

\textbf{Erkenntnisse und Learnings:} \\
\ctmmTextArea[14cm]{4}{crisis_learnings}{Was haben wir aus dieser Krise gelernt?}

\textbf{Trigger-Analyse:} \\
\ctmmTextArea[14cm]{3}{crisis_trigger_analysis}{Wie können wir ähnliche Trigger in Zukunft vermeiden?}

\textbf{Präventionsmaßnahmen:} \\
\ctmmTextArea[14cm]{3}{crisis_prevention}{Welche Maßnahmen setzen wir zur Prävention um?}

\end{ctmmGreenBox}

\subsection{Actionplan für die Zukunft}

\begin{ctmmBlueBox}[title=Verbesserungsmaßnahmen]

\textbf{Sofort umsetzbar (nächste 24h):} \\
\ctmmTextArea[14cm]{2}{crisis_action_immediate}{}

\textbf{Kurzfristig (nächste Woche):} \\  
\ctmmTextArea[14cm]{2}{crisis_action_short}{}

\textbf{Mittelfristig (nächster Monat):} \\
\ctmmTextArea[14cm]{2}{crisis_action_medium}{}

\textbf{Langfristig (nächste 3 Monate):} \\
\ctmmTextArea[14cm]{2}{crisis_action_long}{}

\textbf{Follow-up Termine vereinbart:} \\
\begin{itemize}
    \item \ctmmCheckBox{crisis_followup_therapist}{Therapeut/in: \ctmmTextField[4cm]{crisis_followup_therapist_date}{}}
    \item \ctmmCheckBox{crisis_followup_partner}{Partner-Gespräch: \ctmmTextField[4cm]{crisis_followup_partner_date}{}}
    \item \ctmmCheckBox{crisis_followup_review}{Protokoll-Review: \ctmmTextField[4cm]{crisis_followup_review_date}{}}
\end{itemize}

\end{ctmmBlueBox}

\subsection{Unterschriften und Bestätigung}

\begin{ctmmOrangeBox}[title=Dokumentation abgeschlossen]

\textbf{Ausgefüllt von:} \\
\ctmmTextField[6cm]{crisis_completed_by}{}{Name}

\textbf{Datum der Fertigstellung:} \\
\ctmmDateField{crisis_completion_date}{Fertigstellung}{TT.MM.JJJJ}

\textbf{Bestätigung Partner/in:} \\
\ctmmTextField[6cm]{crisis_partner_confirmation}{}{Name und Datum}

\textbf{Zusätzliche Bemerkungen:} \\
\ctmmTextArea[14cm]{2}{crisis_additional_notes}{}

\end{ctmmOrangeBox}

% ================================================================
% Querverweise zu anderen Modulen
% ================================================================

\subsection{Verbindung zu anderen CTMM-Tools}

\textbf{Vor der Krise:}
\begin{itemize}
    \item \hyperref[sec:triggermanagement]{Triggermanagement} - Prävention
    \item \hyperref[sec:safewords]{Safeword-System} - Früherkennung
    \item \hyperref[sec:co-regulation]{Co-Regulation} - Stabilisierung
\end{itemize}

\textbf{Nach der Krise:}
\begin{itemize}
    \item \hyperref[sec:trigger-forschungstagebuch]{Trigger-Forschungstagebuch} - Analyse
    \item \hyperref[sec:bindung]{Bindungsleitfaden} - Reparatur
    \item \hyperref[sec:depression]{Depression-Monitoring} - Überwachung
\end{itemize}

\newpage

% ================================================================
% CTMM Module: DBT Emotionsregulation & Interpersonelle Skills
% DEAR MAN, GIVE, und weitere DBT-Techniken für neurodiverse Paare
% ================================================================

\section{DBT-Skills für Emotionsregulation}
\label{sec:dbt-skills}

\begin{ctmmBlueBox}[title=Dialektisch-Behaviorale Therapie (DBT) Skills]

\textbf{DBT-Skills} sind bewährte Techniken zur Emotionsregulation und Verbesserung zwischenmenschlicher Beziehungen. Sie sind besonders hilfreich für neurodiverse Paare.

\vspace{0.5em}

\textbf{Die vier DBT-Module:}
\begin{itemize}
    \item \textcolor{ctmmBlue}{\textbf{Achtsamkeit:}} Bewusstsein für den gegenwärtigen Moment
    \item \textcolor{ctmmGreen}{\textbf{Emotionsregulation:}} Umgang mit intensiven Gefühlen
    \item \textcolor{ctmmOrange}{\textbf{Zwischenmenschliche Fertigkeiten:}} Kommunikation und Beziehungen
    \item \textcolor{ctmmPurple}{\textbf{Stresstoleranz:}} Krisenüberstehung ohne Verschlimmerung
\end{itemize}

\end{ctmmBlueBox}

\subsection{DEAR MAN - Effektive Kommunikation}

\begin{ctmmGreenBox}[title=DEAR MAN Technik]

\textbf{DEAR MAN} ist eine strukturierte Methode für schwierige Gespräche und Bitten.

\vspace{0.5em}

\begin{tabular}{|l|l|p{8cm}|}
\hline
\textbf{Buchstabe} & \textbf{Bedeutung} & \textbf{Beschreibung} \\
\hline
\textcolor{ctmmBlue}{\textbf{D}} & \textbf{Describe} & Beschreibe die Situation objektiv, ohne Bewertung \\
\hline
\textcolor{ctmmGreen}{\textbf{E}} & \textbf{Express} & Drücke deine Gefühle und Meinungen aus \\
\hline
\textcolor{ctmmOrange}{\textbf{A}} & \textbf{Assert} & Formuliere klar, was du möchtest \\
\hline
\textcolor{ctmmPurple}{\textbf{R}} & \textbf{Reinforce} & Erkläre die positiven Konsequenzen \\
\hline
\textcolor{ctmmRed}{\textbf{M}} & \textbf{Mindful} & Bleibe fokussiert auf dein Ziel \\
\hline
\textcolor{ctmmBlue}{\textbf{A}} & \textbf{Appear confident} & Tritt selbstbewusst auf \\
\hline
\textcolor{ctmmGreen}{\textbf{N}} & \textbf{Negotiate} & Sei bereit zu Kompromissen \\
\hline
\end{tabular}

\end{ctmmGreenBox}

\subsubsection{DEAR MAN Praxis-Übung}

\begin{ctmmOrangeBox}[title=DEAR MAN Arbeitsblatt]

\textbf{Situation, die ich ansprechen möchte:} \\
\TextField[name=dearman_situation,width=14cm,height=2em,multiline=true,bordercolor=ctmmOrange,backgroundcolor=ctmmOrange!5]{}

\vspace{0.5em}

\textbf{D - Describe (Objektive Beschreibung):} \\
\TextField[name=dearman_describe,width=14cm,height=3em,multiline=true,bordercolor=ctmmBlue,backgroundcolor=ctmmBlue!5]{}

\textbf{E - Express (Gefühle ausdrücken):} \\
\TextField[name=dearman_express,width=14cm,height=2em,multiline=true,bordercolor=ctmmGreen,backgroundcolor=ctmmGreen!5]{}

\textbf{A - Assert (Klare Bitte):} \\
\TextField[name=dearman_assert,width=14cm,height=2em,multiline=true,bordercolor=ctmmOrange,backgroundcolor=ctmmOrange!5]{}

\textbf{R - Reinforce (Positive Konsequenzen):} \\
\TextField[name=dearman_reinforce,width=14cm,height=2em,multiline=true,bordercolor=ctmmPurple,backgroundcolor=ctmmPurple!5]{}

\textbf{Reflexion nach dem Gespräch:} \\
\TextField[name=dearman_reflection,width=14cm,height=3em,multiline=true,bordercolor=ctmmGreen,backgroundcolor=ctmmGreen!5]{}

\end{ctmmOrangeBox}

\subsection{GIVE - Beziehungen stärken}

\begin{ctmmPurpleBox}[title=GIVE Technik]

\textbf{GIVE} hilft dabei, Beziehungen zu erhalten und zu stärken, während man für sich selbst einsteht.

\vspace{0.5em}

\begin{tabular}{|l|l|p{8cm}|}
\hline
\textbf{Buchstabe} & \textbf{Bedeutung} & \textbf{Beschreibung} \\
\hline
\textcolor{ctmmGreen}{\textbf{G}} & \textbf{Gentle} & Sei sanft, vermeide Angriffe und Bedrohungen \\
\hline
\textcolor{ctmmBlue}{\textbf{I}} & \textbf{Interested} & Zeige echtes Interesse an der anderen Person \\
\hline
\textcolor{ctmmOrange}{\textbf{V}} & \textbf{Validate} & Validiere die Gefühle und Sichtweise des anderen \\
\hline
\textcolor{ctmmPurple}{\textbf{E}} & \textbf{Easy manner} & Bleibe entspannt und verwende Humor wenn angemessen \\
\hline
\end{tabular}

\end{ctmmPurpleBox}

\subsubsection{GIVE Praxis-Übung}

\begin{ctmmBlueBox}[title=GIVE Reflexionsbogen]

\textbf{Letzte schwierige Unterhaltung mit Partner/in:} \\
\TextField[name=give_conversation,width=14cm,height=2em,multiline=true,bordercolor=ctmmBlue,backgroundcolor=ctmmBlue!5]{}

\vspace{0.5em}

\textbf{G - War ich gentle (sanft)?} \\
\CheckBox[name=give_gentle_yes,width=1em,height=1em]{} Ja, ich war respektvoll \\
\CheckBox[name=give_gentle_no,width=1em,height=1em]{} Nein, ich war zu hart \\
\textbf{Verbesserung:} \TextField[name=give_gentle_improve,width=10cm,height=1.5em,bordercolor=ctmmGreen,backgroundcolor=ctmmGreen!5]{}

\textbf{I - War ich interested (interessiert)?} \\
\CheckBox[name=give_interested_yes,width=1em,height=1em]{} Ja, ich habe zugehört \\
\CheckBox[name=give_interested_no,width=1em,height=1em]{} Nein, ich war zu selbstfokussiert \\
\textbf{Verbesserung:} \TextField[name=give_interested_improve,width=10cm,height=1.5em,bordercolor=ctmmBlue,backgroundcolor=ctmmBlue!5]{}

\textbf{V - Habe ich validated (validiert)?} \\
\CheckBox[name=give_validate_yes,width=1em,height=1em]{} Ja, ich habe Verständnis gezeigt \\
\CheckBox[name=give_validate_no,width=1em,height=1em]{} Nein, ich habe abgewertet \\
\textbf{Verbesserung:} \TextField[name=give_validate_improve,width=10cm,height=1.5em,bordercolor=ctmmOrange,backgroundcolor=ctmmOrange!5]{}

\textbf{E - War meine Manner easy (entspannt)?} \\
\CheckBox[name=give_easy_yes,width=1em,height=1em]{} Ja, ich war locker \\
\CheckBox[name=give_easy_no,width=1em,height=1em]{} Nein, ich war zu angespannt \\
\textbf{Verbesserung:} \TextField[name=give_easy_improve,width=10cm,height=1.5em,bordercolor=ctmmPurple,backgroundcolor=ctmmPurple!5]{}

\end{ctmmBlueBox}

\subsection{PLEASE - Emotionale Vulnerabilität reduzieren}

\begin{ctmmGreenBox}[title=PLEASE Skills für emotionale Balance]

\textbf{PLEASE} hilft dabei, die emotionale Vulnerabilität zu reduzieren und die Widerstandsfähigkeit zu erhöhen.

\vspace{0.5em}

\begin{tabular}{|l|l|p{7cm}|}
\hline
\textbf{P} & \textbf{Treat PhysicaL illness} & Körperliche Krankheiten behandeln \\
\hline
\textbf{L} & \textbf{Balance Eating} & Ausgewogene Ernährung \\
\hline
\textbf{E} & \textbf{Avoid mood-Altering substances} & Suchtmittel vermeiden \\
\hline
\textbf{A} & \textbf{Balance Sleep} & Gesunden Schlaf fördern \\
\hline
\textbf{S} & \textbf{Get Exercise} & Regelmäßig bewegen \\
\hline
\textbf{E} & \textbf{Build mastery} & Kompetenzgefühl entwickeln \\
\hline
\end{tabular}

\end{ctmmGreenBox}

\subsubsection{PLEASE Wochenplanung}

\begin{ctmmOrangeBox}[title=Wöchentlicher PLEASE Tracker]

\textbf{Woche vom:} \TextField[name=please_week_date,width=4cm,height=1em,bordercolor=ctmmOrange,backgroundcolor=ctmmOrange!5]{}

\vspace{0.5em}

\begin{tabular}{|l|c|c|c|c|c|c|c|}
\hline
\textbf{PLEASE Element} & \textbf{Mo} & \textbf{Di} & \textbf{Mi} & \textbf{Do} & \textbf{Fr} & \textbf{Sa} & \textbf{So} \\
\hline
\textbf{Körperliche Gesundheit} & 
\CheckBox[name=please_physical_mo,width=1em,height=1em]{} & 
\CheckBox[name=please_physical_di,width=1em,height=1em]{} & 
\CheckBox[name=please_physical_mi,width=1em,height=1em]{} & 
\CheckBox[name=please_physical_do,width=1em,height=1em]{} & 
\CheckBox[name=please_physical_fr,width=1em,height=1em]{} & 
\CheckBox[name=please_physical_sa,width=1em,height=1em]{} & 
\CheckBox[name=please_physical_so,width=1em,height=1em]{} \\
\hline
\textbf{Ausgewogene Ernährung} & 
\CheckBox[name=please_eating_mo,width=1em,height=1em]{} & 
\CheckBox[name=please_eating_di,width=1em,height=1em]{} & 
\CheckBox[name=please_eating_mi,width=1em,height=1em]{} & 
\CheckBox[name=please_eating_do,width=1em,height=1em]{} & 
\CheckBox[name=please_eating_fr,width=1em,height=1em]{} & 
\CheckBox[name=please_eating_sa,width=1em,height=1em]{} & 
\CheckBox[name=please_eating_so,width=1em,height=1em]{} \\
\hline
\textbf{Substanzen vermieden} & 
\CheckBox[name=please_substances_mo,width=1em,height=1em]{} & 
\CheckBox[name=please_substances_di,width=1em,height=1em]{} & 
\CheckBox[name=please_substances_mi,width=1em,height=1em]{} & 
\CheckBox[name=please_substances_do,width=1em,height=1em]{} & 
\CheckBox[name=please_substances_fr,width=1em,height=1em]{} & 
\CheckBox[name=please_substances_sa,width=1em,height=1em]{} & 
\CheckBox[name=please_substances_so,width=1em,height=1em]{} \\
\hline
\textbf{Guter Schlaf} & 
\CheckBox[name=please_sleep_mo,width=1em,height=1em]{} & 
\CheckBox[name=please_sleep_di,width=1em,height=1em]{} & 
\CheckBox[name=please_sleep_mi,width=1em,height=1em]{} & 
\CheckBox[name=please_sleep_do,width=1em,height=1em]{} & 
\CheckBox[name=please_sleep_fr,width=1em,height=1em]{} & 
\CheckBox[name=please_sleep_sa,width=1em,height=1em]{} & 
\CheckBox[name=please_sleep_so,width=1em,height=1em]{} \\
\hline
\textbf{Bewegung/Sport} & 
\CheckBox[name=please_exercise_mo,width=1em,height=1em]{} & 
\CheckBox[name=please_exercise_di,width=1em,height=1em]{} & 
\CheckBox[name=please_exercise_mi,width=1em,height=1em]{} & 
\CheckBox[name=please_exercise_do,width=1em,height=1em]{} & 
\CheckBox[name=please_exercise_fr,width=1em,height=1em]{} & 
\CheckBox[name=please_exercise_sa,width=1em,height=1em]{} & 
\CheckBox[name=please_exercise_so,width=1em,height=1em]{} \\
\hline
\textbf{Mastery Aktivität} & 
\CheckBox[name=please_mastery_mo,width=1em,height=1em]{} & 
\CheckBox[name=please_mastery_di,width=1em,height=1em]{} & 
\CheckBox[name=please_mastery_mi,width=1em,height=1em]{} & 
\CheckBox[name=please_mastery_do,width=1em,height=1em]{} & 
\CheckBox[name=please_mastery_fr,width=1em,height=1em]{} & 
\CheckBox[name=please_mastery_sa,width=1em,height=1em]{} & 
\CheckBox[name=please_mastery_so,width=1em,height=1em]{} \\
\hline
\end{tabular}

\vspace{0.5em}

\textbf{Reflexion der Woche:} \\
\TextField[name=please_weekly_reflection,width=14cm,height=3em,multiline=true,bordercolor=ctmmGreen,backgroundcolor=ctmmGreen!5]{}

\end{ctmmOrangeBox}

\subsection{TIPP - Krisenbewältigung}

\begin{ctmmRedBox}[title=\textcolor{white}{TIPP - Akute Krisenbewältigung}]

\textcolor{white}{\textbf{TIPP} sind Notfall-Skills für intensive emotionale Krisen, die sofort angewendet werden können.}

\vspace{0.5em}

\begin{tabular}{|l|p{10cm}|}
\hline
\textcolor{white}{\textbf{T}} & \textcolor{white}{\textbf{Temperature} - Körpertemperatur verändern (kaltes Wasser, Eiswürfel)} \\
\hline
\textcolor{white}{\textbf{I}} & \textcolor{white}{\textbf{Intense exercise} - Intensive Bewegung (Liegestütze, Laufen)} \\
\hline
\textcolor{white}{\textbf{P}} & \textcolor{white}{\textbf{Paced breathing} - Kontrollierte Atmung (4-7-8 Technik)} \\
\hline
\textcolor{white}{\textbf{P}} & \textcolor{white}{\textbf{Progressive muscle relaxation} - Muskelentspannung} \\
\hline
\end{tabular}

\end{ctmmRedBox}

\subsubsection{TIPP Notfall-Plan}

\begin{ctmmOrangeBox}[title=Persönlicher TIPP Notfallplan]

\textbf{Meine TIPP-Strategien für Krisen:}

\textbf{T - Temperature (Verfügbare Optionen):} \\
\CheckBox[name=tipp_cold_water,width=1em,height=1em]{} Kaltes Wasser über Handgelenke \\
\CheckBox[name=tipp_ice_cubes,width=1em,height=1em]{} Eiswürfel in den Händen halten \\
\CheckBox[name=tipp_cold_shower,width=1em,height=1em]{} Kalte Dusche \\
\textbf{Andere:} \TextField[name=tipp_temperature_other,width=8cm,height=1em,bordercolor=ctmmBlue,backgroundcolor=ctmmBlue!5]{}

\textbf{I - Intense exercise (Möglichkeiten):} \\
\CheckBox[name=tipp_pushups,width=1em,height=1em]{} Liegestütze \\
\CheckBox[name=tipp_jumping_jacks,width=1em,height=1em]{} Hampelmänner \\
\CheckBox[name=tipp_stairs,width=1em,height=1em]{} Treppensteigen \\
\textbf{Andere:} \TextField[name=tipp_exercise_other,width=8cm,height=1em,bordercolor=ctmmGreen,backgroundcolor=ctmmGreen!5]{}

\textbf{P - Paced breathing (Technik):} \\
\CheckBox[name=tipp_478_breathing,width=1em,height=1em]{} 4-7-8 Atmung \\
\CheckBox[name=tipp_box_breathing,width=1em,height=1em]{} Box-Atmung (4-4-4-4) \\
\CheckBox[name=tipp_belly_breathing,width=1em,height=1em]{} Bauchatmung \\

\textbf{P - Progressive relaxation (Bereiche):} \\
\CheckBox[name=tipp_shoulders,width=1em,height=1em]{} Schultern anspannen/entspannen \\
\CheckBox[name=tipp_hands,width=1em,height=1em]{} Hände zur Faust/öffnen \\
\CheckBox[name=tipp_face,width=1em,height=1em]{} Gesichtsmuskeln \\

\end{ctmmOrangeBox}

\subsection{Verbindung zu CTMM-System}

\begin{ctmmPurpleBox}[title=DBT-Skills im CTMM-Kontext]

\textbf{Integration der DBT-Skills in das CTMM-System:}

\begin{itemize}
    \item \textbf{CATCH:} DBT-Achtsamkeit hilft beim frühzeitigen Erkennen von Triggern
    \item \textbf{TRACK:} PLEASE-Skills verbessern die Selbstwahrnehmung
    \item \textbf{MAP:} DEAR MAN strukturiert die Kommunikation über Muster
    \item \textbf{MATCH:} GIVE und TIPP bieten konkrete Handlungsoptionen
\end{itemize}

\textbf{Querverweise:}
\begin{itemize}
    \item Siehe \hyperref[sec:triggermanagement]{Triggermanagement} für CATCH-Integration
    \item Siehe \hyperref[sec:co-regulation]{Co-Regulation} für GIVE-Anwendung
    \item Siehe \hyperref[sec:krisenprotokoll]{Krisenprotokoll} für TIPP-Einsatz
\end{itemize}

\end{ctmmPurpleBox}

\newpage

% CTMM Triggermanagement Modul (Tool 23)
\section{Triggermanagement}
\begin{tcolorbox}[colback=ctmmOrange!5!white,colframe=ctmmOrange,title=Was ist ein Trigger?]
Ein Trigger ist ein Auslöser, der starke emotionale oder körperliche Reaktionen hervorruft. Im CTMM werden Trigger erkannt, verstanden und bearbeitet.
\end{tcolorbox}

\subsection{Trigger erkennen}
\begin{itemize}
  \item Körperliche Reaktionen (z.B. Herzklopfen, Schwitzen)
  \item Gedanken und Erinnerungen
  \item Gefühle (z.B. Angst, Wut, Traurigkeit)
\end{itemize}

\subsection{Umgang mit Triggern}
\begin{tcolorbox}[colback=ctmmGreen!5!white,colframe=ctmmGreen,title=Strategien]
\begin{itemize}
  \item Bewusstes Atmen
  \item Selbstfürsorge
  \item Soziale Unterstützung suchen
  \item Trigger-Tagebuch führen
\end{itemize}
\end{tcolorbox}

\subsection{CTMM-Tool 23: Triggermanagement}
Das Tool unterstützt dabei, Trigger zu identifizieren, zu reflektieren und neue Handlungsoptionen zu entwickeln. Ziel ist es, die eigene Reaktion zu verstehen und zu steuern.

\section*{\textcolor{ctmmBlue}{\faHeartbeat{}~Notfallkarten}}
\addcontentsline{toc}{section}{Notfallkarten}
\label{sec:notfallkarten}

\begin{ctmmBlueBox}{Notfallkarten}
\textbf{Name:} \ctmmTextField[4cm]{}{nk_name} \quad \textbf{Datum:} \ctmmTextField[4cm]{}{nk_date}\\
\vspace{0.5cm}
\textbf{Notfallkontakte:}\\
\ctmmTextArea[12cm]{2}{nk_contacts}{}\\
\vspace{0.5cm}
\textbf{Hinweise:}\\
\ctmmTextArea[12cm]{4}{nk_notes}{}
\end{ctmmBlueBox}

\vspace{1cm}
\begin{center}
\textit{\textcolor{ctmmGreen}{\faChevronRight~Weiter zu} \ctmmRef{sec:safewords}{Safe-Words} | \textcolor{ctmmBlue}{\faChevronLeft~Zurück zu} \ctmmRef{sec:triggermanagement}{Trigger-Management}}
\end{center}

\section*{\textcolor{ctmmRed}{\faStopCircle~Safe-Words \& Signalsysteme}}

\begin{quote}
\textbf{\textcolor{ctmmRed}{Worum geht's hier -- für Freunde?}}\\
Safe-Words sind vereinbarte Codes oder Zeichen, die sofort signalisieren: „Ich kann nicht mehr", „Ich brauch Ruhe" oder „Stopp -- das wird mir zu viel". Sie schützen vor Eskalation, Überforderung, Rückzug oder Missverständnissen -- ohne viele Worte.
\end{quote}

\textit{Zentraler Bestandteil der Eskalationsprävention -- mit Symbol- und Notfallsystem}

\subsection*{\textcolor{ctmmRed}{Safe-Words (Beispiele + Eigene)}}

\begin{ctmmRedBox}{Bewährte Safe-Word Beispiele}
\begin{center}
\begin{tabular}{|p{3cm}|p{5cm}|p{5cm}|}
\hline
\textbf{Safe-Word / Geste} & \textbf{Bedeutung / Wirkung} & \textbf{Wann einsetzen?} \\
\hline
„Orange" & Warnstufe -- ich werde gleich überfordert & Bei Stress, lautem Ton, innerem Rückzug \\
\hline
„Kristall" & Stopp -- bitte sofort aufhören & Bei Eskalation, Überforderung, Trigger \\
\hline
Handzeichen (offen) & Ich will reden, aber schaff's nicht & Bei Freeze, Erstarrung, Nonverbales \\
\hline
„Lagerfeuer" & 5min in den Arm nehmen, ohne zu reden & Wenn ich Nähe brauche (Angst, Frieden) \\
\hline
\end{tabular}
\end{center}
\end{ctmmRedBox}

\textbf{Eigene Safe-Words:}\\
\ctmmTextField[4cm]{Safe-Word 1:}{safeword1} \ctmmTextField[6cm]{Bedeutung:}{meaning1}\\
\ctmmTextField[4cm]{Safe-Word 2:}{safeword2} \ctmmTextField[6cm]{Bedeutung:}{meaning2}\\
\ctmmTextField[4cm]{Safe-Word 3:}{safeword3} \ctmmTextField[6cm]{Bedeutung:}{meaning3}

\subsection*{\textcolor{ctmmOrange}{Signalsysteme zur Unterstützung}}

\begin{ctmmOrangeBox}{Leise Zeichen für schwere Momente}
\begin{itemize}
  \item \textbf{Symbolischer Gegenstand} (z. B. Stofftier, Stein, Karte)
  \item \textbf{Tagesanzeiger} (Magnet, Schild, Farbe auf Tür)
  \item \textbf{Lautstärke-Code} (Musikart, Kopfhörer sichtbar = „Bitte in Ruhe lassen")
\end{itemize}

\textbf{Diese Zeichen können leise, sichtbar, intuitiv sein -- auch bei Disso oder Sprachverlust}
\end{ctmmOrangeBox}


\newpage
\section*{\textcolor{ctmmOrange}{\faCheckSquare~Interaktive Checklisten \& Bewertungen}}
\addcontentsline{toc}{section}{Interaktive Tools}
\label{sec:interactive}

\begin{quote}
\textit{\textcolor{ctmmOrange}{Selbstreflexion durch strukturierte Bewertung und Planung.}}\\
\textbf{\textcolor{ctmmOrange}{Messbare Fortschritte im CTMM-System}}\\
Diese Tools helfen dabei, den eigenen Fortschritt zu messen und konkrete nächste Schritte zu planen.
\end{quote}

\subsection*{\textcolor{ctmmOrange}{1. CTMM-Fortschritts-Check}}

\begin{ctmmOrangeBox}{Selbstbewertung (1-10 Punkte)}
\begin{tabular}{|p{6cm}|c|c|c|c|c|c|c|c|c|c|}
\hline
\textbf{CTMM-Bereich} & \textbf{1} & \textbf{2} & \textbf{3} & \textbf{4} & \textbf{5} & \textbf{6} & \textbf{7} & \textbf{8} & \textbf{9} & \textbf{10} \\
\hline
\textbf{Catch:} Trigger erkennen & $\square$ & $\square$ & $\square$ & $\square$ & $\square$ & $\square$ & $\square$ & $\square$ & $\square$ & $\square$ \\
\hline
\textbf{Track:} Gefühle verfolgen & $\square$ & $\square$ & $\square$ & $\square$ & $\square$ & $\square$ & $\square$ & $\square$ & $\square$ & $\square$ \\
\hline
\textbf{Map:} Muster verstehen & $\square$ & $\square$ & $\square$ & $\square$ & $\square$ & $\square$ & $\square$ & $\square$ & $\square$ & $\square$ \\
\hline
\textbf{Match:} Handlung anpassen & $\square$ & $\square$ & $\square$ & $\square$ & $\square$ & $\square$ & $\square$ & $\square$ & $\square$ & $\square$ \\
\hline
\end{tabular}
\end{ctmmOrangeBox}

\subsection*{\textcolor{ctmmOrange}{2. Wöchentlicher Reflexions-Check}}

\begin{ctmmBlueBox}{Woche vom: \ctmmTextField[6cm]{}{woche\_vom}}
    \textbf{Was lief gut diese Woche?}\\
\ctmmTextArea[12cm]{2}{woche\_gut}{}\\

    \textbf{Welche Herausforderungen gab es?}\\
\ctmmTextArea[12cm]{2}{woche\_herausforderungen}{}\\

    \textbf{Welche CTMM-Tools habe ich verwendet?}\\
\ctmmCheckBox{tools\_safewords}{Safe Words} \quad \ctmmCheckBox{tools\_notfallplan}{Notfallplan} \quad \ctmmCheckBox{tools\_trigger}{Triggermanagement} \quad \ctmmCheckBox{tools\_bindung}{Bindungsarbeit}\\

    \textbf{Nächste Woche möchte ich...}\\

\ctmmTextArea[12cm]{2}{woche\_naechste}{}\\
\end{ctmmBlueBox}

\subsection*{\textcolor{ctmmOrange}{3. Beziehungs-Dashboard}}



\begin{ctmmGreenBox}{Für Paare: Gemeinsame Bewertung}
\begin{tabularx}{\textwidth}{|X|c|c|X|}
\hline
\textbf{Bereich} & \textbf{Partner A} & \textbf{Partner B} & \textbf{Gemeinsame Ziele} \\
\hline
Kommunikation & \ctmmTextField[2cm]{}{bd_comm_a} & \ctmmTextField[2cm]{}{bd_comm_b} & \ctmmTextField[4cm]{}{bd_comm_ziele} \\
\hline
Konfliktlösung & \ctmmTextField[2cm]{}{bd_konflikt_a} & \ctmmTextField[2cm]{}{bd_konflikt_b} & \ctmmTextField[4cm]{}{bd_konflikt_ziele} \\
\hline
Intimität & \ctmmTextField[2cm]{}{bd_intim_a} & \ctmmTextField[2cm]{}{bd_intim_b} & \ctmmTextField[4cm]{}{bd_intim_ziele} \\
\hline
Alltagsorganisation & \ctmmTextField[2cm]{}{bd_alltag_a} & \ctmmTextField[2cm]{}{bd_alltag_b} & \ctmmTextField[4cm]{}{bd_alltag_ziele} \\

% =====================================================
% CTMM Interactive Diagram Module
% Purpose: Kombiniert statisc\vspace{1cm}
\begin{center}
\textit{\textcolor{ctmmBlue}{\faChevronRight~Zurück zu} \ctmmRef{sec:navigation}{Navigation} | \textcolor{ctmmGreen}{\faChevronRight~Weiter zu} \ctmmRef{sec:feedback}{Selbstreflexion}}
\end{center}iagramme mit interaktiven PDF-Elementen
% Author: CTMM-Team Enhancement
% =====================================================

\section*{\textcolor{ctmmBlue}{\faChartBar~Interaktive Diagramm-Elemente}}
\addcontentsline{toc}{section}{Interaktive Diagramm-Elemente}
\label{sec:interactive-diagrams}

\begin{ctmmBlueBox}{\faLightbulb~Konzept}
Diese Diagramme sind \textbf{statisch visualisiert}, aber \textbf{interaktiv nutzbar} durch PDF-Formularfelder. Sie zeigen therapeutische Konzepte und ermöglichen gleichzeitig direkte Eingaben.
\end{ctmmBlueBox}

\subsection*{\faSync~Trigger-Zyklus mit Tracking}

\begin{center}
\begin{tcolorbox}[colback=white,colframe=ctmmBlue,width=14cm]
\centering
\textbf{CTMM-Trigger-Management-Zyklus}\\[0.5cm]

% Statisches Diagramm
\textcolor{ctmmBlue}{\Large\faEye} \textbf{ERKENNEN} 
$\xrightarrow{\text{führt zu}}$ 
\textcolor{ctmmGreen}{\Large\faShield} \textbf{VORBEUGEN} 
$\xrightarrow{\text{führt zu}}$ 
\textcolor{ctmmOrange}{\Large\faExclamationTriangle} \textbf{REAGIEREN} 
$\xrightarrow{\text{führt zu}}$ 
\textcolor{ctmmPurple}{\Large\faBrain} \textbf{LERNEN}\\[0.3cm]

\begin{small}
$\downarrow$ \textbf{Interaktive Eingaben zu jeder Phase} $\downarrow$
\end{small}
\end{tcolorbox}
\end{center}

\begin{ctmmGreenBox}[title=\faEye~Phase 1: ERKENNEN - Interaktive Eingabe]
\textbf{Heute erkannter Trigger:} \ctmmTextField[8cm]{}{cycle_trigger_today}\\[0.2cm]
\textbf{Frühwarnzeichen bemerkt:} \ctmmCheckBox{cycle_warning_body}{Körperlich} \ctmmCheckBox{cycle_warning_emotion}{Emotional} \ctmmCheckBox{cycle_warning_thought}{Gedanken}\\[0.2cm]
\textbf{Intensität (1-10):} \ctmmTextField[2cm]{}{cycle_intensity}
\end{ctmmGreenBox}

\begin{ctmmOrangeBox}[title=\faShield~Phase 2: VORBEUGEN - Was wurde eingesetzt?]
\textbf{Verwendete Strategien:}\\
\ctmmCheckBox{cycle_prev_breathing}{Atemtechnik} \quad \ctmmCheckBox{cycle_prev_safeword}{Safe-Word} \quad \ctmmCheckBox{cycle_prev_pause}{Pause} \quad \ctmmCheckBox{cycle_prev_space}{Raum schaffen}\\[0.2cm]
\textbf{Wirksamkeit (1-10):} \ctmmTextField[2cm]{}{cycle_prevention_effect}
\end{ctmmOrangeBox}

\begin{ctmmRedBox}[title=\faExclamationTriangle~Phase 3: REAGIEREN - Krisenmanagement]
\textcolor{white}{\textbf{Eskalation eingetreten:}} \textcolor{white}{\ctmmCheckBox{cycle_escalation}{Ja}}\\[0.2cm]
\textcolor{white}{\textbf{Angewendete Notfall-Tools:}}\\
\textcolor{white}{\ctmmCheckBox{cycle_emergency_grounding}{5-4-3-2-1 Grounding} \quad \ctmmCheckBox{cycle_emergency_separation}{Räumliche Trennung} \quad \ctmmCheckBox{cycle_emergency_help}{Externe Hilfe}}\\[0.2cm]
\textcolor{white}{\textbf{Dauer der Krise (Minuten):}} \textcolor{white}{\ctmmTextField[3cm]{}{cycle_crisis_duration}}
\end{ctmmRedBox}

\begin{ctmmPurpleBox}[title=\faBrain~Phase 4: LERNEN - Reflexion & Integration]
\textbf{Was haben wir gelernt?}\\
\ctmmTextArea[12cm]{2}{cycle_learning}{}\\[0.3cm]
\textbf{Verbesserung für nächstes Mal:}\\
\ctmmTextField[10cm]{}{cycle_improvement}\\[0.3cm]
\textbf{Erfolg bewerten (1-10):} \ctmmTextField[2cm]{}{cycle_success_rating}
\end{ctmmPurpleBox}

\subsection*{\faChartLine~Stimmungs-Ampel mit Eingabe}

\begin{center}
\begin{tcolorbox}[colback=white,colframe=ctmmGray,width=12cm]
\centering
\textbf{Aktuelle Stimmungszone}\\[0.5cm]

% Statische Ampel-Visualisierung
\begin{tabular}{|c|c|c|}
\hline
\cellcolor{ctmmGreen!30}\textcolor{ctmmGreen}{\Large\faSmile} & \cellcolor{ctmmOrange!30}\textcolor{ctmmOrange}{\Large\faMeh} & \cellcolor{ctmmRed!30}\textcolor{ctmmRed}{\Large\faFrown} \\
\textbf{STABIL} & \textbf{ACHTUNG} & \textbf{KRISE} \\
7-10 Punkte & 4-6 Punkte & 1-3 Punkte \\
\hline
\end{tabular}\\[0.5cm]

\textbf{Meine aktuelle Zone:} \ctmmCheckBox{mood_green}{Grün} \ctmmCheckBox{mood_orange}{Orange} \ctmmCheckBox{mood_red}{Rot}\\[0.3cm]
\textbf{Genaue Bewertung (1-10):} \ctmmTextField[2cm]{}{mood_exact}\\[0.3cm]
\textbf{Grund für diese Bewertung:} \ctmmTextField[8cm]{}{mood_reason}
\end{tcolorbox}
\end{center}

\subsection*{\faUsers~Partner-Dynamik-Tracker}

\begin{center}
\begin{tcolorbox}[colback=ctmmPurple!10,colframe=ctmmPurple,width=14cm]
\centering
\textbf{Ko-Regulations-System}\\[0.5cm]

% Statisches Beziehungsdiagramm
\textcolor{ctmmBlue}{\Large\faUser} \textbf{Partner A} 
$\xrightleftharpoons[\text{Ko-Regulation}]{\text{Trigger-Antwort}}$ 
\textcolor{ctmmGreen}{\Large\faUser} \textbf{Partner B}\\[0.5cm]

$\downarrow$ \textbf{Interaktive Bewertung der heutigen Dynamik} $\downarrow$
\end{tcolorbox}
\end{center}

\begin{ctmmYellowBox}[title=Heutige Partner-Interaktion bewerten]
\textbf{Meine Trigger-Reaktion:} \ctmmCheckBox{partner_my_calm}{Ruhig geblieben} \ctmmCheckBox{partner_my_escalated}{Eskaliert} \ctmmCheckBox{partner_my_withdrawal}{Rückzug}\\[0.2cm]
\textbf{Partner-Reaktion:} \ctmmCheckBox{partner_their_supportive}{Unterstützend} \ctmmCheckBox{partner_their_triggered}{Mit-getriggert} \ctmmCheckBox{partner_their_neutral}{Neutral}\\[0.2cm]
\textbf{Ko-Regulation gelungen:} \ctmmCheckBox{coregulation_yes}{Ja} \ctmmCheckBox{coregulation_no}{Nein} \ctmmCheckBox{coregulation_partial}{Teilweise}\\[0.2cm]
\textbf{Verwendete Safe-Words:} \ctmmTextField[8cm]{}{partner_safewords_used}\\[0.2cm]
\textbf{Verbesserungspotential:} \ctmmTextField[10cm]{}{partner_improvement}
\end{ctmmYellowBox}

\subsection*{\faCalendarCheck~Wöchentliche Diagram-Analyse}

\begin{ctmmBlueBox}[title=Muster in den Diagrammen erkennen]
\textbf{Diese Woche aufgefallene Muster:}\\
\ctmmTextArea[12cm]{3}{weekly_pattern_analysis}{}\\[0.3cm]
\textbf{Häufigste Trigger-Phase:} \ctmmCheckBox{pattern_recognize}{Erkennen} \ctmmCheckBox{pattern_prevent}{Vorbeugen} \ctmmCheckBox{pattern_react}{Reagieren} \ctmmCheckBox{pattern_learn}{Lernen}\\[0.2cm]
\textbf{Erfolgreichste Intervention:} \ctmmTextField[8cm]{}{most_effective_intervention}\\[0.2cm]
\textbf{Nächste Woche fokussieren auf:} \ctmmTextField[8cm]{}{next_week_focus}
\end{ctmmBlueBox}

\vspace{1cm}
\begin{center}
\textit{\textcolor{ctmmBlue}{\faChevronRight~Zurück zu} \ctmmRef{sec:navigation}{Navigation} | \textcolor{ctmmGreen}{\faChevronRight~Weiter zu} \ctmmRef{sec:feedback}{Selbstreflexion}}
\end{center}

% QR-Code Integration für digitale Ressourcen

\newpage
\section*{\textcolor{ctmmPurple}{\faQrcode~Digitale Ressourcen}}
\addcontentsline{toc}{section}{Digitale Ressourcen}
\label{sec:qrcode}

\begin{quote}
\textit{\textcolor{ctmmOrange}{Verbindung von analog und digital -- das CTMM-System erweitern.}}\\
\textbf{\textcolor{ctmmPurple}{Online-Tools \& Apps für das CTMM-System}}\\
Scannen Sie die QR-Codes für zusätzliche digitale Hilfsmittel und Updates.
\end{quote}

\subsection*{\textcolor{ctmmPurple}{Online-Trigger-Tagebuch}}

\begin{ctmmBlueBox}{Digitales Tracking}
Führen Sie Ihr Trigger-Tagebuch digital und erhalten Sie automatische Muster-Analysen.

\begin{center}
\begin{tabular}{c}
\textbf{QR-Code: Trigger-App}\\
\rule{3cm}{3cm}\\
\small{Hier würde ein QR-Code zur}\\
\small{CTMM-Trigger-App erscheinen}
\end{tabular}
\end{center}
\end{ctmmBlueBox}

\subsection*{\textcolor{ctmmPurple}{CTMM-Notfall-App}}

\begin{ctmmRedBox}{Sofortige Hilfe}
Schneller Zugriff auf Ihre personalisierten Safe Words und Notfallkontakte.

\begin{center}
\begin{tabular}{c}
\textbf{QR-Code: Notfall-App}\\
\rule{3cm}{3cm}\\
\small{Hier würde ein QR-Code zur}\\
\small{CTMM-Notfall-App erscheinen}
\end{tabular}
\end{center}
\end{ctmmRedBox}

\subsection*{\textcolor{ctmmPurple}{Online-Community}}

\begin{ctmmGreenBox}{Austausch \& Support}
Vernetzen Sie sich mit anderen CTMM-Anwendern in einem sicheren Forum.

\begin{center}
\begin{tabular}{c}
\textbf{QR-Code: Community}\\
\rule{3cm}{3cm}\\
\small{Hier würde ein QR-Code zum}\\
\small{CTMM-Community-Forum erscheinen}
\end{tabular}
\end{center}
\end{ctmmGreenBox}

\subsection*{\textcolor{ctmmPurple}{Video-Tutorials}}

\begin{ctmmOrangeBox}{Lernen \& Vertiefen}
Schauen Sie sich praktische Anwendungsvideos für alle CTMM-Tools an.

\begin{center}
\begin{tabular}{|c|c|}
\hline
\textbf{Video-Thema} & \textbf{QR-Code / Link} \\
\hline
\textbf{4-7-8 Atemtechnik} & \href{https://youtube.com/watch?v=YRPh_GaiL8s}{\textcolor{ctmmBlue}{\faYoutube~3 Min Anleitung}} \\
\hline
\textbf{5-4-3-2-1 Grounding} & \href{https://youtube.com/watch?v=utUVx0ayoYw}{\textcolor{ctmmBlue}{\faYoutube~Grounding-Übung}} \\
\hline
\textbf{DBT Skills Demo} & \href{https://youtube.com/watch?v=q15eTySnWxc}{\textcolor{ctmmBlue}{\faYoutube~Skills Training}} \\
\hline
\textbf{Trigger-Management} & \href{https://youtube.com/watch?v=Mz3Mi_OZYno}{\textcolor{ctmmBlue}{\faYoutube~PTSD Coping}} \\
\hline
\textbf{Paartherapie-Kommunikation} & \href{https://youtube.com/watch?v=2s9ACDMcpjA}{\textcolor{ctmmBlue}{\faYoutube~Gottman-Methode}} \\
\hline
\end{tabular}
\end{center}

\vspace{0.5cm}
\textbf{Empfohlene Kanäle:}
\begin{itemize}
    \item \textcolor{ctmmGreen}{\faYoutube~Therapy in a Nutshell} - Trauma \& DBT Skills
    \item \textcolor{ctmmBlue}{\faYoutube~Kati Morton} - Mental Health Education  
    \item \textcolor{ctmmOrange}{\faYoutube~The School of Life} - Relationship Skills
    \item \textcolor{ctmmPurple}{\faYoutube~Marsha Linehan} - Original DBT Videos
\end{itemize}
\end{ctmmOrangeBox}

\vspace{1cm}
\begin{center}
\textit{\textcolor{ctmmPurple}{\faChevronRight~Weiter zu} \ctmmRef{sec:feedback}{Selbstreflexions-System} | \textcolor{ctmmGreen}{\faHome~Zurück zu} \ctmmRef{sec:navigation}{Navigation}}
\end{center}

% Therapiekoordination - Multidisziplinäres Treatment

\newpage
\section*{\textcolor{ctmmPurple}{\faUserMd~Therapie-Koordination}}
\addcontentsline{toc}{section}{Therapie-Koordination}
\label{sec:therapiekoordination}

\begin{quote}
\textit{\textcolor{ctmmPurple}{Koordination verschiedener Therapeuten für optimale Behandlung.}}\\
\textbf{\textcolor{ctmmPurple}{Multidisziplinärer Ansatz}}\\
Komplexe Fälle erfordern verschiedene Experten: Psychotherapeut, Neurologe, DBT-Spezialist, Trauma-Experte. Dieses Modul hilft bei der Koordination.
\end{quote}

\subsection*{\textcolor{ctmmPurple}{Mein Behandlungsteam}}

\begin{ctmmPurpleBox}{Therapeuten-Übersicht}
\begin{tabularx}{\textwidth}{|X|X|X|}
\hline
\textbf{Therapeut} & \textbf{Kontakt} & \textbf{Fokus} \\
\hline
\textbf{Psychotherapeut:} & \ctmmTextField[4cm]{}{therapist_psycho_contact} & PTBS, Borderline, ADHS \\
\hline
\textbf{Neurologe:} & \ctmmTextField[4cm]{}{therapist_neuro_contact} & Epilepsie, post-OP Betreuung \\
\hline
\textbf{DBT-Spezialist:} & \ctmmTextField[4cm]{}{therapist_dbt_contact} & Skills-Training, Emotionsregulation \\
\hline
\textbf{Trauma-Experte:} & \ctmmTextField[4cm]{}{therapist_trauma_contact} & KPTBS, Dissoziation \\
\hline
\textbf{Hausarzt:} & \ctmmTextField[4cm]{}{therapist_gp_contact} & Medikation, Koordination \\
\hline
\end{tabularx}
\end{ctmmPurpleBox}

\subsection*{\textcolor{ctmmPurple}{Aktuelle Behandlungsphase}}

\begin{ctmmBlueBox}{Dreiphasen-Modell (Trauma)}
\textbf{Phase 1 - Stabilisierung:} \\
\ctmmCheckBox{phase1_active}{Aktiv} \quad \ctmmCheckBox{phase1_completed}{Abgeschlossen} \\
\textbf{Schwerpunkt:} Achtsamkeit, Skills, Sicherheit \\[0.3cm]

\textbf{Phase 2 - Traumakonfrontation:} \\
\ctmmCheckBox{phase2_prep}{Vorbereitung} \quad \ctmmCheckBox{phase2_active}{Aktiv} \quad \ctmmCheckBox{phase2_completed}{Abgeschlossen} \\
\textbf{Schwerpunkt:} Verarbeitung traumatischer Erinnerungen \\[0.3cm]

\textbf{Phase 3 - Integration:} \\
\ctmmCheckBox{phase3_prep}{Vorbereitung} \quad \ctmmCheckBox{phase3_active}{Aktiv} \\
\textbf{Schwerpunkt:} Alltags-Integration, Zukunftsplanung
\end{ctmmBlueBox}

\subsection*{\textcolor{ctmmPurple}{DBT-Module-Fortschritt}}

\begin{ctmmGreenBox}{Skills-Training Status}
\begin{tabular}{|l|c|c|c|}
\hline
\textbf{DBT-Modul} & \textbf{Begonnen} & \textbf{In Arbeit} & \textbf{Abgeschlossen} \\
\hline
Achtsamkeit & \ctmmCheckBox{dbt_mindful_start}{} & \ctmmCheckBox{dbt_mindful_prog}{} & \ctmmCheckBox{dbt_mindful_done}{} \\
\hline
Stresstoleranz & \ctmmCheckBox{dbt_stress_start}{} & \ctmmCheckBox{dbt_stress_prog}{} & \ctmmCheckBox{dbt_stress_done}{} \\
\hline
Emotionsregulation & \ctmmCheckBox{dbt_emotion_start}{} & \ctmmCheckBox{dbt_emotion_prog}{} & \ctmmCheckBox{dbt_emotion_done}{} \\
\hline
Zwischenmenschliche Fertigkeiten (ZMF) & \ctmmCheckBox{dbt_zmf_start}{} & \ctmmCheckBox{dbt_zmf_prog}{} & \ctmmCheckBox{dbt_zmf_done}{} \\
\hline
\end{tabular}
\end{ctmmGreenBox}

\subsection*{\textcolor{ctmmPurple}{Neurologische Überwachung}}

\begin{ctmmOrangeBox}{Post-OP Monitoring (Hippocampus/Amygdala)}
\textbf{Letzte Kontrolle:} \ctmmTextField[4cm]{}{neuro_last_check} \\[0.3cm]
\textbf{Epilepsie-Medikation:} \ctmmTextField[6cm]{}{neuro_medication} \\[0.3cm]
\textbf{Neuropsychologische Tests:} \\
\ctmmCheckBox{neuro_memory}{Gedächtnistest} \quad \ctmmCheckBox{neuro_emotion}{Emotionsregulation} \\
\ctmmCheckBox{neuro_cognition}{Kognitive Funktion} \quad \ctmmCheckBox{neuro_other}{Andere:} \ctmmTextField[3cm]{}{neuro_other_text}
\end{ctmmOrangeBox}

\subsection*{\textcolor{ctmmPurple}{Therapie-Synchronisation}}

\textbf{Nächste Termine koordinieren:} \\
\ctmmTextArea[12cm]{2}{next_appointments}{}\\[0.3cm]

\textbf{Informationen zwischen Therapeuten teilen:} \\
\ctmmTextArea[12cm]{2}{info_sharing}{}\\[0.3cm]

\textbf{CTMM-Tools mit Therapeuten besprechen:} \\
\ctmmCheckBox{discuss_safewords}{Safe-Words System} \quad \ctmmCheckBox{discuss_triggers}{Trigger-Management} \\
\ctmmCheckBox{discuss_monitoring}{Depression-Monitor} \quad \ctmmCheckBox{discuss_progress}{Fortschrittsmessung}

\vspace{1cm}
\begin{center}
\textit{\textcolor{ctmmGreen}{\faChevronRight~Weiter zu} \ctmmRef{sec:qrcode}{QR-Code Integration} | \textcolor{ctmmPurple}{\faChevronLeft~Zurück zu} \ctmmRef{sec:feedback}{Selbstreflexion}}
\end{center}

% Feedback und Selbstreflexions-System
\newpage
\section*{\textcolor{ctmmPurple}{\faChartLine~Selbstreflexions-System}}
\addcontentsline{toc}{section}{Selbstreflexions-System}
\label{sec:feedback}

\begin{ctmmPurpleBox}{\faSync~Kontinuierliche Verbesserung}
\textit{Dieses System hilft uns, Fortschritte zu erfassen, Erfolge zu würdigen und Verbesserungsbereiche zu identifizieren. Es wächst mit uns mit.}
\end{ctmmPurpleBox}

\subsection*{\faCalendar~Tägliche Mikro-Reflexion (2 Minuten)}
\textit{Nach dem Abend-Check-In → \ctmmRef{sec:5.1}{\faEdit~Kap.5.1}}

\textbf{Drei einfache Fragen:}\\[0.3cm]
1. \textbf{Werkzeug-Check:} Welches Tool/Skill haben wir heute benutzt?\\
\ctmmTextField[10cm]{}{daily_tool}\\[0.3cm]

2. \textbf{Wirksamkeit:} Hat es geholfen? (1-10)\\
\ctmmTextField[3cm]{}{daily_effectiveness}\\[0.3cm]

3. \textbf{Anpassung:} Eine kleine Verbesserung für morgen?\\
\ctmmTextField[10cm]{}{daily_improvement}\\[0.5cm]

\subsection*{\faCalendar~Wöchentliche Mini-Retrospektive}
\textit{Sonntag beim Depression-Monitor → \ctmmRef{sec:5.3}{\faEdit~Kap.5.3}}

\begin{ctmmYellowBox}{Wochenreflexion}
\textbf{Genutzte Tools diese Woche:}\\
\ctmmCheckBox{weekly_safewords}{Safe-Words} \quad 	extbf{Skills:} \ctmmCheckBox{weekly_skills}{Neue Skills} \quad 	extbf{Rituale:} \ctmmCheckBox{weekly_rituals}{Gemeinsame Zeit}\\[0.3cm]

\textbf{Was hat gut funktioniert?}\\
\ctmmTextArea[12cm]{2}{week_success}{}\\[0.3cm]

\textbf{Was verbessern wir nächste Woche?}\\
\ctmmTextField[10cm]{}{week_improvement}\\[0.3cm]

\textbf{Ein Erfolg dieser Woche (egal wie klein):}\\
\ctmmTextField[10cm]{}{week_win}
\end{ctmmYellowBox}

\subsection*{\faChartLine~Fortschrittsmessung}

\begin{center}
\begin{tabularx}{\textwidth}{|X|X|X|X|}
\hline
\textbf{Metrik} & \textbf{Letzter Monat} & \textbf{Dieser Monat} & \textbf{Trend} \\
\hline
Safe-Word Nutzung/Woche & \ctmmTextField[2cm]{}{metric_safeword_last} & \ctmmTextField[2cm]{}{metric_safeword_now} & \ctmmCheckBox[trend_safeword_up]{↑} \ctmmCheckBox[trend_safeword_down]{↓} \\
\hline
Krisenzeit (Minuten) & \ctmmTextField[2cm]{}{metric_crisis_last} & \ctmmTextField[2cm]{}{metric_crisis_now} & \ctmmCheckBox{trend_crisis_up}{↑} \ctmmCheckBox{trend_crisis_down}{↓} \\
\hline
Schlafqualität (1-10) & \ctmmTextField[2cm]{}{metric_sleep_last} & \ctmmTextField[2cm]{}{metric_sleep_now} & \ctmmCheckBox{trend_sleep_up}{↑} \ctmmCheckBox{trend_sleep_down}{↓} \\
\hline
Gemeinsame Aktivitäten/Woche & \ctmmTextField[2cm]{}{metric_activities_last} & \ctmmTextField[2cm]{}{metric_activities_now} & \ctmmCheckBox{trend_activities_up}{↑} \ctmmCheckBox{trend_activities_down}{↓} \\
\hline
\end{tabularx}
\end{center}

\subsection*{\faTrophy~Erfolgs-Bibliothek}
\label{sec:erfolge}

\begin{ctmmGreenBox}{\faBullseye~Beweise sammeln}
\textbf{Datum:} \ctmmTextField[3cm]{}{success_date} \quad \textbf{Kategorie:} \ctmmCheckBox[cat_micro]{Mikro} \ctmmCheckBox[cat_breakthrough]{Durchbruch} \ctmmCheckBox[cat_surprise]{Überraschung}\\[0.3cm]

\textbf{Was ist passiert? (Konkrete Details)}\\
\ctmmTextArea[12cm]{3}{success_description}{}\\[0.3cm]

\textbf{Vorher vs. Nachher:}\\
\textbf{Früher:} \ctmmTextField[5cm]{}{success_before} \textbf{Heute:} \ctmmTextField[5cm]{}{success_after}\\[0.3cm]

\textbf{Gefühl und Bedeutung:}\\
\ctmmTextArea[12cm]{2}{success_meaning}{}\\[0.3cm]

\textbf{Was war entscheidend für diesen Erfolg?}\\
\ctmmTextField[10cm]{}{success_key_factor}
\end{ctmmGreenBox}

\subsection*{\faLightbulb~System-Anpassungen}

\begin{ctmmOrangeBox}{\faWrench~Unser System entwickelt sich weiter}
\textbf{Dokument-Feedback markieren:}\\
\faThumbtack~\textbf{Unklar:} \ctmmTextField[8cm]{}{feedback_unclear}\\
\faTimes~\textbf{Funktioniert nicht:} \ctmmTextField[8cm]{}{feedback_broken}\\
\faPlus~\textbf{Fehlt:} \ctmmTextField[8cm]{}{feedback_missing}\\
\faStar~\textbf{Besonders hilfreich:} \ctmmTextField[8cm]{}{feedback_helpful}\\[0.3cm]

\textbf{Neue Idee zum Testen:}\\
\ctmmTextArea[12cm]{2}{new_idea}{}\\[0.3cm]

\textbf{Gewünschte Änderung:}\\
\ctmmTextField[10cm]{}{desired_change}
\end{ctmmOrangeBox}

\vspace{1cm}
\begin{center}
\textit{\textcolor{ctmmGreen}{\faChevronRight~Weiter zu} \ctmmRef{sec:5.1}{Arbeitsblätter} | \textcolor{ctmmBlue}{\faHome~Zurück zu} \ctmmRef{sec:navigation}{Navigation}}
\end{center}


\newpage
\part*{KAPITEL 5: ARBEITSBLÄTTER}
\addcontentsline{toc}{part}{KAPITEL 5: ARBEITSBLÄTTER}

% Arbeitsblatt 5.1: Täglicher Check-In
\newpage
\section*{\textcolor{ctmmGreen}{\faCalendar~5.1 TÄGLICHER CHECK-IN}}
\label{sec:5.1}
\addcontentsline{toc}{section}{5.1 Täglicher Check-In}

\begin{ctmmGreenBox}{\faCircle~Jeden Morgen und Abend ausfüllen}
\textit{Kurze Selbsteinschätzung für bessere Selbstwahrnehmung und Kommunikation}
\end{ctmmGreenBox}

\vspace{0.5cm}

\subsection*{\faCalendar~Grunddaten}
\textbf{Datum:} \ctmmTextField[5.5cm]{}{date} \quad \textbf{Zeit:} \ctmmTextField[2.5cm]{}{time}\\[0.2cm]
\textbf{Check:} \ctmmCheckBox{morning}{Morgen} \ctmmCheckBox{evening}{Abend}

\vspace{0.5cm}

\subsection*{\faUsers~Beide Partner}

\begin{center}
\begin{tabularx}{\textwidth}{|l|X|X|}
\hline
\textbf{Bereich} & \textbf{ER} & \textbf{SIE} \\
\hline
\textbf{Stimmung (1-10):} & \ctmmTextField[3cm]{}{er_mood} & \ctmmTextField[3cm]{}{sie\_mm \\
\hline
\textbf{Schlafstunden:} & \ctmmTextField[3cm]{}{er_sleep} & \ctmmTextField[3cm]{}{sie\_mm \\
\hline
\textbf{Medikamente:} & \ctmmCheckBox{er_meds_yes}{OK} \ctmmCheckBox{er_meds_forgotten}{Vergessen} & \ctmmCheckBox{sie_meds_yes}{OK} \ctmmCheckBox{sie_meds_forgotten}{Vergessen} \\
\hline
\end{tabularx}
\end{center}

\vspace{0.7cm}
\subsection*{\faClipboard~Tagesplanung}
\textbf{Schwierige Termine heute:} \\
\ctmmTextField[12cm]{}{appointments}

\textbf{Support-Person verfügbar:} \\
\ctmmCheckBox{support_yes}{Ja, Name:} \ctmmTextField[4cm]{}{support\_mm \\
\ctmmCheckBox{support_no}{Nein, Backup:} \ctmmTextField[4cm]{}{support\_mm

\textbf{Safe-Words erklärt:} \\
\ctmmCheckBox{safewords_explained_guests}{Allen Anwesenden (Gäste/Familie)} \\
\ctmmCheckBox{safewords_explained_not_needed}{Nicht nötig (nur wir zwei)}

\vspace{0.7cm}
\subsection*{\faMoon~Abend-Reflexion}
\textbf{Was war heute gut:} \\
\ctmmTextField[12cm]{}{evening\_mm

\textbf{Was war schwierig:} \\
\ctmmTextField[12cm]{}{evening\_mm

\textbf{Safe-Word benutzt:} \ctmmCheckBox{safeword_no}{Nein} \ctmmCheckBox{safeword_yes}{Ja, welches:} \ctmmTextField[4cm]{}{safeword\_mm

\textbf{Morgen wichtig:} \\
\ctmmTextField[12cm]{}{tomorrow\_mm

\vspace{1cm}
\begin{center}
\textit{\textcolor{ctmmPurple}{\faChevronRight~Weiter zu} \ctmmRef{sec:5.2}{Trigger-Tagebuch} | \textcolor{ctmmGreen}{\faChevronLeft~Zurück zu} \ctmmRef{sec:navigation}{Navigation}}
\end{center}

% Arbeitsblatt 5.2: Trigger-Tagebuch
\newpage
\section*{\faBolt~5.2 TRIGGER-TAGEBUCH}
\label{sec:5.2}
\addcontentsline{toc}{section}{5.2 Trigger-Tagebuch}

\begin{ctmmYellowBox}{\faLightbulb~Tooltip}
Auch kleine Trigger zählen! Jedes Ausfüllen hilft uns beide besser zu verstehen.
\end{ctmmYellowBox}

\vspace{0.5cm}
\textbf{Datum/Zeit:} \ctmmTextField[8cm]{}{trigger_datetime}
\vspace{0.5cm}

\subsection*{\faUserFriends~TRIGGER-SITUATION}
\textit{Einfach ankreuzen was passt - kombinieren ist ok!}

\textbf{Ich war zusammen mit:} \\
\ctmmCheckBox[with_partner]{Partner} \quad
\ctmmCheckBox[with_friends]{Freunden} \quad
\ctmmCheckBox[with_family]{Familie} \quad
\ctmmCheckBox[with_others]{Anderen:} \ctmmTextField[3cm]{}{with_others_text}

\vspace{0.5cm}
\textbf{Triggermoment war:} \\
\ctmmCheckBox[trigger_sound]{Geräusch (laut/plötzlich)} \quad
\ctmmCheckBox[trigger_words]{Bestimmte Wörter/Sätze} \\
\ctmmCheckBox[trigger_tone]{Tonfall/Betonung} \quad
\ctmmCheckBox[trigger_bodylanguage]{Körpersprache} \\
\ctmmCheckBox[trigger_crowd]{Menschenmenge} \quad
\ctmmCheckBox[trigger_memory]{Erinnerung/Flashback} \\
\ctmmCheckBox[trigger_change]{Planänderung} \quad
\ctmmCheckBox[trigger_overwhelm]{Reizüberflutung} \\
\ctmmCheckBox[trigger_other]{Andere:} \ctmmTextField[3cm]{}{trigger_other_text}

\vspace{0.5cm}
\textbf{Beschreibe die Situation:} \\
\textit{Schüler-Tipp: "Es war wie..." oder "Es fühlte sich an wie..."} \\
\textit{Beispiel: "Wir waren im Supermarkt, als eine Kiste umfiel. Es war wie ein Blitz durch meinen Körper..."} \\
\ctmmTextArea[12cm]{2}{situation_desc}{}

\vspace{0.5cm}
\textbf{Welche Gefühle hat das bei mir ausgelöst:} \\
\textit{Du darfst mehrere ankreuzen - Gefühle sind oft gemischt!} \\
\ctmmCheckBox[feeling_angst]{Angst} \quad
\ctmmCheckBox[feeling_wut]{Wut} \quad
\ctmmCheckBox[feeling_trauer]{Trauer} \quad
\ctmmCheckBox[feeling_verlassenheit]{Verlassenheit} \\
\ctmmCheckBox[feeling_ueberforderung]{Überforderung} \quad
\ctmmCheckBox[feeling_hilflosigkeit]{Hilflosigkeit} \quad
\ctmmCheckBox[feeling_scham]{Scham} \quad
\ctmmCheckBox[feeling_verwirrung]{Verwirrung} \\
\ctmmCheckBox[feeling_other]{Andere:} \ctmmTextField[3cm]{}{feeling_other_text}

\subsection*{\faSync~REAKTIONEN}
\textit{Ehrlich sein hilft - wir urteilen nicht!}

\textbf{ER:} \ctmmCheckBox[reaktion_er_rueckzug]{Rückzug} \quad \ctmmCheckBox[reaktion_er_shutdown]{Shutdown} \quad \ctmmCheckBox[reaktion_er_panik]{Panik} \quad \ctmmCheckBox[reaktion_er_verwirrt]{Verwirrt} \\
\textbf{SIE:} \ctmmCheckBox[reaktion_sie_vorwuerfe]{Vorwürfe} \quad \ctmmCheckBox[reaktion_sie_klammern]{Klammern} \quad \ctmmCheckBox[reaktion_sie_weinen]{Weinen} \quad \ctmmCheckBox[reaktion_sie_wut]{Wut}

\vspace{0.5cm}
\textbf{Stress-Level (1-10):} \\
\textit{1=total entspannt, 10=extremer Notfall} \\
ER: \ctmmTextField[2cm]{}{stress_er} \quad SIE: \ctmmTextField[2cm]{}{stress_sie}

\subsection*{\faTools~WAS HALF}
\ctmmCheckBox[half_safeword]{Safe-Word benutzt:} \ctmmTextField[5cm]{}{half_safeword_text} \\
\ctmmCheckBox[half_skills]{Skills angewendet:} \ctmmTextField[5cm]{}{half_skills_text} \\
\ctmmCheckBox[half_friends]{Freunde geholt:} \ctmmTextField[5cm]{}{half_friends_text} \\
\ctmmCheckBox[half_pause]{Pause gemacht:} \ctmmTextField[5cm]{}{half_pause_text} \\
\ctmmCheckBox[half_other]{Andere:} \ctmmTextField[5cm]{}{half_other_text}

\subsection*{\faChartLine~LERNPUNKT}
\textbf{Nächstes Mal besser machen:} \\
\ctmmTextArea[12cm]{2}{learning_next_time}{}

\vspace{1cm}
\begin{center}
\textit{\textcolor{ctmmPurple}{\faChevronRight~Weiter zu} \ctmmRef{sec:5.3}{Depression-Monitor} | \textcolor{ctmmPurple}{\faChevronLeft~Zurück zu} \ctmmRef{sec:5.1}{Täglicher Check-In}}
\end{center}

% Arbeitsblatt 5.3: Depression-Monitoring
\newpage
\section*{\faChartLine~5.3 DEPRESSION-MONITORING}
\label{sec:5.3}
\addcontentsline{toc}{section}{5.3 Depression-Monitoring}

\begin{ctmmYellowBox}{\faLightbulb~Tooltip}
Dieses Blatt hilft uns, Muster in Stimmung und Energie zu erkennen. Sei ehrlich, es gibt keine 'richtigen' oder 'falschen' Antworten.
\end{ctmmYellowBox}

\vspace{0.5cm}
\textit{Fülle diese Tabelle bitte täglich aus, idealerweise immer zur gleichen Zeit. Das dauert weniger als 2 Minuten.}

\begin{center}
\renewcommand{\arraystretch}{1.8}
\begin{tabularx}{\textwidth}{|l|c|c|c|c|X|}
\hline
\textbf{Tag} & \textbf{Datum} & \textbf{Stimmung} (1-10) & \textbf{Energie} (1-10) & \textbf{Selbstfürsorge?} & \textbf{Notizen / Besonderheiten} \\
\hline
\textbf{Mo} & \dailyInput & \dailyInput & \dailyInput & \dailyInput & \\
\hline
\textbf{Di} & \dailyInput & \dailyInput & \dailyInput & \dailyInput & \\
\hline
\textbf{Mi} & \dailyInput & \dailyInput & \dailyInput & \dailyInput & \\
\hline
\textbf{Do} & \dailyInput & \dailyInput & \dailyInput & \dailyInput & \\
\hline
\textbf{Fr} & \dailyInput & \dailyInput & \dailyInput & \dailyInput & \\
\hline
\textbf{Sa} & \dailyInput & \dailyInput & \dailyInput & \dailyInput & \\
\hline
\textbf{So} & \dailyInput & \dailyInput & \dailyInput & \dailyInput & \\
\hline
\hline
\textbf{Mo} & \dailyInput & \dailyInput & \dailyInput & \dailyInput & \\
\hline
\textbf{Di} & \dailyInput & \dailyInput & \dailyInput & \dailyInput & \\
\hline
\textbf{Mi} & \dailyInput & \dailyInput & \dailyInput & \dailyInput & \\
\hline
\textbf{Do} & \dailyInput & \dailyInput & \dailyInput & \dailyInput & \\
\hline
\textbf{Fr} & \dailyInput & \dailyInput & \dailyInput & \dailyInput & \\
\hline
\textbf{Sa} & \dailyInput & \dailyInput & \dailyInput & \dailyInput & \\
\hline
\textbf{So} & \dailyInput & \dailyInput & \dailyInput & \dailyInput & \\
\hline
\end{tabularx}
\end{center}

\vspace{1cm}
\subsection*{\faQuestionCircle~AUSWERTUNG NACH 2 WOCHEN}
\textit{Nimm dir einen Moment Zeit, um die Tabelle zu betrachten. Was fällt dir auf?}

\begin{ctmmBlueBox}{Reflexionsfragen}
\begin{itemize}
    \item Was fällt mir auf, wenn ich die letzten zwei Wochen betrachte? Gibt es ein Muster?
    \item Gab es Tage, an denen meine Stimmung oder Energie besonders niedrig war? Was war an diesen Tagen los? (Siehe Trigger-Tagebuch)
    \item Welche Selbstfürsorge-Aktivitäten haben mir geholfen, auch nur ein bisschen?
    \item Was ist der eine kleine Schritt, den ich nächste Woche tun kann, um meine Energie oder Stimmung zu stabilisieren?
\end{itemize}
\end{ctmmBlueBox}

\vspace{1cm}
\begin{center}
\textit{\textcolor{ctmmPurple}{\faChevronRight~Weiter zu} \ctmmRef{sec:demo-interactive}{Interaktive Demo} | \textcolor{ctmmBlue}{\faHeart~Hilfe bei} \ctmmRef{sec:depression}{Depression-Modul}}
\end{center}

% =====================================================
% CTMM Täglicher Stimmungscheck Arbeitsblatt
% Purpose: Strukturierte Selbstreflexion und Dokumentation
% Context: Neurodiverse Paare - Tägliche Routine und Mustererkennung
% Author: CTMM-Team
% Integration: % =====================================================
% CTMM Täglicher Stimmungscheck Arbeitsblatt
% Purpose: Strukturierte Selbstreflexion und Dokumentation
% Context: Neurodiverse Paare - Tägliche Routine und Mustererkennung
% Author: CTMM-Team
% Integration: % =====================================================
% CTMM Täglicher Stimmungscheck Arbeitsblatt
% Purpose: Strukturierte Selbstreflexion und Dokumentation
% Context: Neurodiverse Paare - Tägliche Routine und Mustererkennung
% Author: CTMM-Team
% Integration: \input{modules/arbeitsblatt-taeglicher-stimmungscheck}
% =====================================================

\section*{\textcolor{ctmmGreen}{\faEdit~Täglicher Stimmungscheck}}

\begin{quote}
\textbf{\textcolor{ctmmGreen}{Worum geht's hier -- für Freunde?}}\\
Dieses Arbeitsblatt unterstützt Sie bei täglicher stimmungscheck.
\end{quote}

\textit{Anwendung: Regelmäßig ausfüllen und als Teil Ihrer CTMM-Routine nutzen}

\subsection*{\textcolor{ctmmGreen}{Ausfüllbereich}}

\begin{ctmmGreenBox}[title=Täglicher Stimmungscheck Dokumentation]
\begin{itemize}
  \item \textbf{Datum:} \underline{\hspace{3cm}} \textbf{Zeit:} \underline{\hspace{2cm}}
  \item \textbf{Ausgangssituation:} \underline{\hspace{5cm}}
  \item \textbf{Angewendete Strategie:} \underline{\hspace{5cm}}
  \item \textbf{Wirksamkeit (1-10):} \underline{\hspace{5cm}}
\end{itemize}

\vspace{0.5cm}
\textbf{Notizen:}\\
\underline{\hspace{\textwidth}}\\
\underline{\hspace{\textwidth}}\\
\underline{\hspace{\textwidth}}
\end{ctmmGreenBox}

\subsection*{\textcolor{ctmmPurple}{Reflexion}}
\textbf{Was war heute anders:}\\
\underline{\hspace{\textwidth}}\\
\underline{\hspace{\textwidth}}

\subsection*{\textcolor{ctmmBlue}{CTMM-Navigation}}
\begin{itemize}
  \item \texttt{bindungsleitfaden} ← Grundlagen für sichere Anwendung
  \item \texttt{triggermanagement} ← Ergänzende Strategien und Vertiefung
\end{itemize}
% =====================================================

\section*{\textcolor{ctmmGreen}{\faEdit~Täglicher Stimmungscheck}}

\begin{quote}
\textbf{\textcolor{ctmmGreen}{Worum geht's hier -- für Freunde?}}\\
Dieses Arbeitsblatt unterstützt Sie bei täglicher stimmungscheck.
\end{quote}

\textit{Anwendung: Regelmäßig ausfüllen und als Teil Ihrer CTMM-Routine nutzen}

\subsection*{\textcolor{ctmmGreen}{Ausfüllbereich}}

\begin{ctmmGreenBox}[title=Täglicher Stimmungscheck Dokumentation]
\begin{itemize}
  \item \textbf{Datum:} \underline{\hspace{3cm}} \textbf{Zeit:} \underline{\hspace{2cm}}
  \item \textbf{Ausgangssituation:} \underline{\hspace{5cm}}
  \item \textbf{Angewendete Strategie:} \underline{\hspace{5cm}}
  \item \textbf{Wirksamkeit (1-10):} \underline{\hspace{5cm}}
\end{itemize}

\vspace{0.5cm}
\textbf{Notizen:}\\
\underline{\hspace{\textwidth}}\\
\underline{\hspace{\textwidth}}\\
\underline{\hspace{\textwidth}}
\end{ctmmGreenBox}

\subsection*{\textcolor{ctmmPurple}{Reflexion}}
\textbf{Was war heute anders:}\\
\underline{\hspace{\textwidth}}\\
\underline{\hspace{\textwidth}}

\subsection*{\textcolor{ctmmBlue}{CTMM-Navigation}}
\begin{itemize}
  \item \texttt{bindungsleitfaden} ← Grundlagen für sichere Anwendung
  \item \texttt{triggermanagement} ← Ergänzende Strategien und Vertiefung
\end{itemize}
% =====================================================

\section*{\textcolor{ctmmGreen}{\faEdit~Täglicher Stimmungscheck}}

\begin{quote}
\textbf{\textcolor{ctmmGreen}{Worum geht's hier -- für Freunde?}}\\
Dieses Arbeitsblatt unterstützt Sie bei täglicher stimmungscheck.
\end{quote}

\textit{Anwendung: Regelmäßig ausfüllen und als Teil Ihrer CTMM-Routine nutzen}

\subsection*{\textcolor{ctmmGreen}{Ausfüllbereich}}

\begin{ctmmGreenBox}[title=Täglicher Stimmungscheck Dokumentation]
\begin{itemize}
  \item \textbf{Datum:} \underline{\hspace{3cm}} \textbf{Zeit:} \underline{\hspace{2cm}}
  \item \textbf{Ausgangssituation:} \underline{\hspace{5cm}}
  \item \textbf{Angewendete Strategie:} \underline{\hspace{5cm}}
  \item \textbf{Wirksamkeit (1-10):} \underline{\hspace{5cm}}
\end{itemize}

\vspace{0.5cm}
\textbf{Notizen:}\\
\underline{\hspace{\textwidth}}\\
\underline{\hspace{\textwidth}}\\
\underline{\hspace{\textwidth}}
\end{ctmmGreenBox}

\subsection*{\textcolor{ctmmPurple}{Reflexion}}
\textbf{Was war heute anders:}\\
\underline{\hspace{\textwidth}}\\
\underline{\hspace{\textwidth}}

\subsection*{\textcolor{ctmmBlue}{CTMM-Navigation}}
\begin{itemize}
  \item \texttt{bindungsleitfaden} ← Grundlagen für sichere Anwendung
  \item \texttt{triggermanagement} ← Ergänzende Strategien und Vertiefung
\end{itemize}
% =====================================================
% CTMM Trigger-Forschungstagebuch
% Purpose: Mustererkennung, Selbstreflexion und Beziehungssicherheit
% Context: Einzelanwendung oder Paar - Trigger-Tracking
% Author: CTMM-Team
% Copilot: Use for pattern recognition, trigger analysis, relationship safety documentation
% =====================================================

\section*{\textcolor{ctmmBlue}{\faSearch~Trigger-Forschungstagebuch}}
\label{sec:trigger-forschungstagebuch}
\addcontentsline{toc}{section}{Trigger-Forschungstagebuch}

\begin{quote}
\textbf{\textcolor{ctmmBlue}{Modul zur Mustererkennung, Selbstreflexion und Beziehungssicherheit}}\\
Für Einzel- oder Paaranwendung. Wirklicher Fortschritt beginnt dort, wo ich mich ehrlich selbst beobachte -- ohne mich zu verurteilen.
\end{quote}

\subsection*{\textcolor{ctmmBlue}{Tägliches Trigger-Tracking}}

\begin{center}
\begin{tabular}{|p{1.5cm}|p{3cm}|p{2.5cm}|p{2cm}|p{1.5cm}|p{2cm}|p{1cm}|p{2cm}|}
\hline
\textbf{Datum} & \textbf{Kontext (Ort, Menschen, Situation)} & \textbf{Auslösender Trigger} & \textbf{Frühwarnzeichen} & \textbf{Reaktion} & \textbf{Skill angewendet?} & \textbf{Wirkung (1-10)} & \textbf{Reflexion (später)} \\
\hline
\ctmmTextField[1.4cm]{}{datum1} & \ctmmTextField[2.8cm]{}{kontext1} & \ctmmTextField[2.3cm]{}{trigger1} & \ctmmTextField[1.8cm]{}{warnung1} & \ctmmTextField[1.3cm]{}{reaktion1} & \ctmmTextField[1.8cm]{}{skill1} & \ctmmTextField[0.8cm]{}{wirkung1} & \ctmmTextField[1.8cm]{}{reflexion1} \\
\hline
\ctmmTextField[1.4cm]{}{datum2} & \ctmmTextField[2.8cm]{}{kontext2} & \ctmmTextField[2.3cm]{}{trigger2} & \ctmmTextField[1.8cm]{}{warnung2} & \ctmmTextField[1.3cm]{}{reaktion2} & \ctmmTextField[1.8cm]{}{skill2} & \ctmmTextField[0.8cm]{}{wirkung2} & \ctmmTextField[1.8cm]{}{reflexion2} \\
\hline
\ctmmTextField[1.4cm]{}{datum3} & \ctmmTextField[2.8cm]{}{kontext3} & \ctmmTextField[2.3cm]{}{trigger3} & \ctmmTextField[1.8cm]{}{warnung3} & \ctmmTextField[1.3cm]{}{reaktion3} & \ctmmTextField[1.8cm]{}{skill3} & \ctmmTextField[0.8cm]{}{wirkung3} & \ctmmTextField[1.8cm]{}{reflexion3} \\
\hline
\ctmmTextField[1.4cm]{}{datum4} & \ctmmTextField[2.8cm]{}{kontext4} & \ctmmTextField[2.3cm]{}{trigger4} & \ctmmTextField[1.8cm]{}{warnung4} & \ctmmTextField[1.3cm]{}{reaktion4} & \ctmmTextField[1.8cm]{}{skill4} & \ctmmTextField[0.8cm]{}{wirkung4} & \ctmmTextField[1.8cm]{}{reflexion4} \\
\hline
\ctmmTextField[1.4cm]{}{datum5} & \ctmmTextField[2.8cm]{}{kontext5} & \ctmmTextField[2.3cm]{}{trigger5} & \ctmmTextField[1.8cm]{}{warnung5} & \ctmmTextField[1.3cm]{}{reaktion5} & \ctmmTextField[1.8cm]{}{skill5} & \ctmmTextField[0.8cm]{}{wirkung5} & \ctmmTextField[1.8cm]{}{reflexion5} \\
\hline
\end{tabular}
\end{center}

\textbf{Hinweis:} „Ich erinnere mich an ..." → Schreibe auf, auch wenn es klein erscheint.

\subsection*{\textcolor{ctmmPurple}{Wöchentliche Auswertung}}

\begin{tcolorbox}[colback=ctmmPurple!5!white,colframe=ctmmPurple,title=\textcolor{white}{\textbf{Reflexionsfragen für diese Woche}}]

\textbf{1. Wiederkehrende Muster:}\\
\ctmmTextArea[12cm]{3}{muster_woche}{}

\textbf{2. Welche Situationen meide ich (bewusst oder unbewusst)?}\\
\ctmmTextArea[12cm]{3}{vermeidung}{}

\textbf{3. Was funktioniert gut?}\\
\ctmmTextArea[12cm]{3}{funktioniert_gut}{}

\textbf{4. Was braucht neue Ideen oder Veränderungen?}\\
\ctmmTextArea[12cm]{3}{neue_ideen}{}

\end{tcolorbox}

\subsection*{\textcolor{ctmmGreen}{Reflexions-Fragen (1x pro Woche oder nach Bedarf)}}

\begin{itemize}
  \item \textbf{Wann hat mein Notfallplan gut funktioniert?}\\
  \ctmmTextArea[12cm]{2}{notfallplan_gut}{}
  
  \item \textbf{Wo hätte ein anderes Ritual helfen können?}\\
  \ctmmTextArea[12cm]{2}{ritual_anders}{}
  
  \item \textbf{Wer kann mich in solchen Momenten konkret unterstützen?}\\
  \ctmmTextArea[12cm]{2}{unterstuetzung}{}
\end{itemize}

\subsection*{\textcolor{ctmmOrange}{Monatliche Trend-Analyse}}

\textbf{Häufigste Trigger diesen Monat:}\\
1. \ctmmTextField[4cm]{}{haeufig1} \quad 2. \ctmmTextField[4cm]{}{haeufig2} \quad 3. \ctmmTextField[4cm]{}{haeufig3}

\textbf{Erfolgreichste Skills:}\\
1. \ctmmTextField[4cm]{}{erfolg1} \quad 2. \ctmmTextField[4cm]{}{erfolg2} \quad 3. \ctmmTextField[4cm]{}{erfolg3}

\textbf{Verbesserung zu letztem Monat:}\\
\ctmmYesNo{verbesserung} \quad \textbf{Wie zeigt sich das:} \ctmmTextField[6cm]{}{verbesserung_wie}

\textbf{Neue Trigger entdeckt:}\\
\ctmmTextArea[12cm]{2}{neue_trigger}{}

\textbf{Support-Netzwerk erweitert:}\\
\ctmmYesNo{netzwerk} \quad \textbf{Wie:} \ctmmTextField[8cm]{}{netzwerk_wie}

\subsection*{\textcolor{ctmmBlue}{Digitale Erweiterung}}

\begin{ctmmBlueBox}[title=Tipp für digitale Nutzung]
\textbf{Digitale Version:} Auch als Excel/Google Sheets nutzbar -- mit Filterfunktion, Diagrammen, Suchfunktion. Export als .xlsx empfohlen für Langzeitanalyse.

\textbf{Empfohlene Spalten für digitale Version:}
\begin{itemize}
  \item Datum (Datumsformat für Sortierung)
  \item Trigger-Kategorie (Dropdown: Lärm, Kritik, Nähe, Distanz, Überforderung)
  \item Intensität (1-10 Skala)
  \item Dauer (in Minuten)  
  \item Erfolgreich bewältigt (Ja/Nein)
  \item Beteiligte Personen (Tags)
\end{itemize}
\end{ctmmBlueBox}

\subsection*{\textcolor{ctmmBlue}{CTMM-Navigation}}
\begin{itemize}
  \item \texttt{triggermanagement} ← Grundlagen und Strategien
  \item \texttt{notfallkarten} ← Akute Krisenintervention
  \item \texttt{selbstreflexion} ← Vertiefende Analysen
\end{itemize}

% ================================================================
% CTMM Barrierefreiheits-Modul
% Accessibility Features für neurodiverse Nutzer
% ================================================================

\section{Barrierefreiheit und Zugänglichkeit}
\label{sec:accessibility}

\begin{ctmmBlueBox}[title=CTMM Barrierefreiheits-Standards]

Dieses Dokument wurde nach den Prinzipien des \textbf{Universal Design} erstellt, um allen Nutzern, unabhängig von ihren individuellen Bedürfnissen, den bestmöglichen Zugang zu ermöglichen.

\end{ctmmBlueBox}

\subsection{Visuelle Barrierefreiheit}

\begin{ctmmGreenBox}[title=Sehfreundliche Gestaltung]

\textbf{Implementierte Features:}
\begin{itemize}
    \item \textbf{Hoher Kontrast:} Alle Farben erfüllen WCAG 2.1 AA Standards
    \item \textbf{Skalierbare Schriften:} PDF kann bis 400\% vergrößert werden
    \item \textbf{Klare Strukturierung:} Logische Überschriftenhierarchie
    \item \textbf{Farbkodierung + Text:} Keine Information nur durch Farbe
\end{itemize}

\textbf{Farbkontrast-Werte:}
\begin{tabular}{|l|l|l|}
\hline
\textbf{Element} & \textbf{Farbkombination} & \textbf{Kontrastverhältnis} \\
\hline
Standard Text & Schwarz auf Weiß & 21:1 \\
\hline
\textcolor{ctmmBlue}{Überschriften} & ctmmBlue auf Weiß & 8.2:1 \\
\hline
\textcolor{ctmmGreen}{Erfolg} & ctmmGreen auf Weiß & 7.1:1 \\
\hline
\textcolor{ctmmRed}{Warnung} & ctmmRed auf Weiß & 6.8:1 \\
\hline
\end{tabular}

\end{ctmmGreenBox}

\subsection{Kognitive Barrierefreiheit}

\begin{ctmmOrangeBox}[title=Neurodiverse-freundliche Strukturierung]

\textbf{Anpassungen für verschiedene Lernstile:}

\subsubsection{Für Menschen mit Autismus}
\begin{itemize}
    \item \textbf{Vorhersagbare Struktur:} Jedes Modul folgt demselben Aufbau
    \item \textbf{Klare Anweisungen:} Schritt-für-Schritt Erklärungen
    \item \textbf{Visuelle Hilfsmittel:} Icons und Symbole zur Orientierung
    \item \textbf{Reizarme Gestaltung:} Keine überstimulierenden Elemente
\end{itemize}

\subsubsection{Für Menschen mit ADHS}
\begin{itemize}
    \item \textbf{Kurze Abschnitte:} Maximale Textblöcke von 150 Wörtern
    \item \textbf{Hervorhebungen:} Wichtige Punkte visuell betont
    \item \textbf{Interaktive Elemente:} Checkbox und Eingabefelder
    \item \textbf{Fortschrittsanzeigen:} Seitennummern und Kapitelübersicht
\end{itemize}

\subsubsection{Für Menschen mit Dyslexie}
\begin{itemize}
    \item \textbf{Dyslexie-freundliche Schrift:} OpenDyslexic optional verfügbar
    \item \textbf{Erhöhter Zeilenabstand:} 1.5-facher Standard-Abstand
    \item \textbf{Linksbündiger Text:} Keine Blocksatz-Formatierung
    \item \textbf{Kurze Zeilen:} Maximal 70 Zeichen pro Zeile
\end{itemize}

\end{ctmmOrangeBox}

\subsection{Motorische Barrierefreiheit}

\begin{ctmmPurpleBox}[title=Eingabehilfen und Navigation]

\textbf{PDF-Formular Optimierungen:}
\begin{itemize}
    \item \textbf{Große Eingabebereiche:} Mindestens 44pt Touch-Targets
    \item \textbf{Tab-Reihenfolge:} Logische Keyboard-Navigation
    \item \textbf{Fehlertoleranz:} Undo-Funktionen in Formularen
    \item \textbf{Zeitlimits:} Keine automatischen Timeouts
\end{itemize}

\textbf{Alternative Eingabemethoden:}
\begin{itemize}
    \item \textbf{Spracheingabe:} Kompatibel mit Screen Readern
    \item \textbf{Touch-Optimierung:} Für Tablet-Nutzung geeignet
    \item \textbf{Tastatur-Navigation:} Vollständig ohne Maus bedienbar
\end{itemize}

\end{ctmmPurpleBox}

\subsection{Screen Reader Kompatibilität}

\begin{ctmmBlueBox}[title=Assistive Technologie Support]

\textbf{PDF-Accessibility Features:}
\begin{itemize}
    \item \textbf{Alt-Text:} Alle Grafiken mit Beschreibung
    \item \textbf{Heading Tags:} Strukturierte H1-H6 Hierarchie
    \item \textbf{Reading Order:} Logische Lesereihenfolge definiert
    \item \textbf{Language Tags:} Sprache für Text-to-Speech optimiert
\end{itemize}

\textbf{Getestete Screen Reader:}
\begin{itemize}
    \item \textbf{NVDA:} Vollständig kompatibel
    \item \textbf{JAWS:} Formularfelder funktional
    \item \textbf{VoiceOver:} MacOS/iOS Unterstützung
    \item \textbf{TalkBack:} Android Zugänglichkeit
\end{itemize}

\end{ctmmBlueBox}

\subsection{Anpassbare Darstellungsoptionen}

\begin{ctmmGreenBox}[title=Personalisierbare Einstellungen]

\textbf{PDF-Viewer Einstellungen:}

\subsubsection{Schriftgrößen-Anpassung}
\begin{itemize}
    \item \textbf{Standard:} 11pt Grundschrift
    \item \textbf{Groß:} 14pt für bessere Lesbarkeit
    \item \textbf{Sehr groß:} 18pt für Sehbeeinträchtigungen
    \item \textbf{Zoom:} Bis 400\% ohne Qualitätsverlust
\end{itemize}

\subsubsection{Farbschema-Optionen}
\begin{itemize}
    \item \textbf{Standard:} CTMM Farbpalette
    \item \textbf{High Contrast:} Schwarz-Weiß Darstellung
    \item \textbf{Dark Mode:} Dunkler Hintergrund verfügbar
    \item \textbf{Farbenblind-freundlich:} Alternative Markierungen
\end{itemize}

\end{ctmmGreenBox}

\subsection{Sprachliche Barrierefreiheit}

\begin{ctmmOrangeBox}[title=Verständliche Kommunikation]

\textbf{Plain Language Prinzipien:}
\begin{itemize}
    \item \textbf{Einfache Sprache:} Verzicht auf Fachtermini ohne Erklärung
    \item \textbf{Kurze Sätze:} Durchschnittlich 15-20 Wörter
    \item \textbf{Aktive Formulierungen:} Klare Handlungsanweisungen
    \item \textbf{Glossar:} Fachbegriffe erklärt
\end{itemize}

\textbf{Mehrsprachige Unterstützung:}
\begin{itemize}
    \item \textbf{Deutsch:} Vollständige Version
    \item \textbf{Einfache Sprache:} Reduzierte Komplexität
    \item \textbf{Piktogramme:} Universelle Symbole
    \item \textbf{Audio-Version:} Geplant für zukünftige Releases
\end{itemize}

\end{ctmmOrangeBox}

\subsection{Technische Kompatibilität}

\begin{ctmmPurpleBox}[title=Geräte- und Software-Unterstützung]

\textbf{Unterstützte Plattformen:}
\begin{tabular}{|l|l|l|}
\hline
\textbf{Platform} & \textbf{PDF Reader} & \textbf{Formular-Support} \\
\hline
Windows & Adobe Reader, Foxit & Vollständig \\
\hline
macOS & Preview, Adobe Reader & Vollständig \\
\hline
iOS & PDF Expert, Adobe & Teilweise \\
\hline
Android & Adobe Reader & Vollständig \\
\hline
Linux & Evince, Okular & Basis \\
\hline
\end{tabular}

\textbf{Mindestanforderungen:}
\begin{itemize}
    \item \textbf{PDF-Version:} 1.7 oder höher
    \item \textbf{JavaScript:} Für interaktive Features
    \item \textbf{Formular-Unterstützung:} AcroForms kompatibel
    \item \textbf{Unicode:} UTF-8 Zeichenkodierung
\end{itemize}

\end{ctmmPurpleBox}

\subsection{Nutzerfeedback und Verbesserungen}

\begin{ctmmRedBox}[title=\textcolor{white}{Accessibility Feedback-System}]

\textcolor{white}{\textbf{Verbesserungsvorschläge erwünscht:}}

\begin{itemize}
    \item[\textcolor{white}{•}] \textcolor{white}{Welche Barrieren sind Ihnen aufgefallen?}
    \item[\textcolor{white}{•}] \textcolor{white}{Welche assistiven Technologien nutzen Sie?}
    \item[\textcolor{white}{•}] \textcolor{white}{Welche Anpassungen wären hilfreich?}
\end{itemize}

\textcolor{white}{\textbf{Kontakt für Accessibility-Feedback:}}\\
\textcolor{white}{E-Mail: accessibility@ctmm-system.org}

\end{ctmmRedBox}

\subsection{Barrierefreiheits-Checkliste für Therapeuten}

\begin{ctmmGreenBox}[title=Implementierungs-Leitfaden]

\textbf{Vor der Nutzung mit Klienten prüfen:}

\CheckBox[name=accessibility_screen_reader,width=1em,height=1em]{} Screen Reader Kompatibilität getestet \\
\CheckBox[name=accessibility_font_size,width=1em,height=1em]{} Schriftgröße für Klient angepasst \\
\CheckBox[name=accessibility_color_contrast,width=1em,height=1em]{} Farbkontrast ausreichend \\
\CheckBox[name=accessibility_navigation,width=1em,height=1em]{} Navigation erprobt \\
\CheckBox[name=accessibility_input_methods,width=1em,height=1em]{} Alternative Eingabemethoden verfügbar \\
\CheckBox[name=accessibility_language_level,width=1em,height=1em]{} Sprachniveau angemessen \\
\CheckBox[name=accessibility_time_limits,width=1em,height=1em]{} Ausreichend Zeit eingeplant \\
\CheckBox[name=accessibility_backup_formats,width=1em,height=1em]{} Alternative Formate verfügbar \\

\textbf{Individuelle Anpassungen notiert:} \\
\TextField[name=accessibility_individual_notes,width=14cm,height=3em,multiline=true,bordercolor=ctmmGreen,backgroundcolor=ctmmGreen!5]{}

\end{ctmmGreenBox}

\newpage

% ================================================================
% CTMM Bibliographie und Quellenverzeichnis
% Wissenschaftliche Grundlagen und Referenzen
% ================================================================

\section{Quellenverzeichnis und wissenschaftliche Grundlagen}
\label{sec:bibliography}

\begin{ctmmBlueBox}[title=Wissenschaftliche Fundierung des CTMM-Systems]

Das CTMM-System basiert auf bewährten therapeutischen Ansätzen und aktueller Forschung zu neurodiversen Beziehungen, Traumatherapie und Emotionsregulation.

\end{ctmmBlueBox}

\subsection{Grundlagenliteratur zu DBT und Emotionsregulation}

\begin{ctmmGreenBox}[title=Dialektisch-Behaviorale Therapie (DBT)]

\textbf{Primärquellen:}
\begin{enumerate}
    \item \textbf{Linehan, M. M. (2015).} \textit{DBT Skills Training Manual, Second Edition.} New York: Guilford Press. 
    
    \textbf{Kernaussage:} Grundlagenwerk für DBT-Skills, entwickelt von der Begründerin der DBT.
    
    \item \textbf{Linehan, M. M. (2014).} \textit{DBT Skills Training Handouts and Worksheets, Second Edition.} New York: Guilford Press.
    
    \textbf{Relevanz:} Praktische Arbeitsblätter und Übungen, die in CTMM adaptiert wurden.
    
    \item \textbf{McKay, M., Wood, J. C., \& Brantley, J. (2019).} \textit{The Dialectical Behavior Therapy Skills Workbook.} Oakland: New Harbinger Publications.
    
    \textbf{CTMM-Integration:} DEAR MAN, GIVE, PLEASE, und TIPP-Techniken.
\end{enumerate}

\end{ctmmGreenBox}

\subsection{Neurodiverse Beziehungen und Kommunikation}

\begin{ctmmOrangeBox}[title=Autismus und Beziehungen]

\textbf{Spezialisierte Forschung:}
\begin{enumerate}
    \item \textbf{Mendes, E. (2015).} \textit{Marriage and Lasting Relationships with Asperger's Syndrome.} London: Jessica Kingsley Publishers.
    
    \textbf{Relevanz:} Spezifische Herausforderungen und Strategien für neurodiverse Paare.
    
    \item \textbf{Aston, M. (2014).} \textit{The Other Half of Asperger Syndrome.} London: Jessica Kingsley Publishers.
    
    \textbf{CTMM-Bezug:} Partnerschaftsdynamiken und Co-Regulation Strategien.
    
    \item \textbf{Simone, R. (2010).} \textit{22 Things a Woman Must Know If She Loves a Man with Asperger's Syndrome.} London: Jessica Kingsley Publishers.
    
    \textbf{Integration:} Kommunikationsstrategien und Bindungsaspekte.
\end{enumerate}

\end{ctmmOrangeBox}

\subsection{Traumatherapie und Triggermanagement}

\begin{ctmmPurpleBox}[title=Trauma-informierte Ansätze]

\textbf{Aktuelle Forschung:}
\begin{enumerate}
    \item \textbf{van der Kolk, B. (2014).} \textit{The Body Keeps the Score: Brain, Mind, and Body in the Healing of Trauma.} New York: Penguin Books.
    
    \textbf{CTMM-Anwendung:} Körperbasierte Trigger-Erkennung und Regulation.
    
    \item \textbf{Porges, S. W. (2011).} \textit{The Polyvagal Theory: Neurophysiological Foundations of Emotions.} New York: Norton Professional Books.
    
    \textbf{Relevanz:} Autonomes Nervensystem und Co-Regulation in Beziehungen.
    
    \item \textbf{Ogden, P., Minton, K., \& Pain, C. (2006).} \textit{Trauma and the Body: A Sensorimotor Approach to Psychotherapy.} New York: Norton Professional Books.
    
    \textbf{Integration:} Körperorientierte Techniken im CTMM-System.
\end{enumerate}

\end{ctmmPurpleBox}

\subsection{Bindungstheorie und Paartherapie}

\begin{ctmmBlueBox}[title=Bindungsbasierte Ansätze]

\textbf{Theoretische Grundlagen:}
\begin{enumerate}
    \item \textbf{Johnson, S. M. (2019).} \textit{Attachment in Psychotherapy.} New York: Guilford Press.
    
    \textbf{CTMM-Bezug:} Sichere Bindung als Basis für Co-Regulation.
    
    \item \textbf{Tatkin, S. (2012).} \textit{Wired for Love: How Understanding Your Partner's Brain and Attachment Style Can Help You Defuse Conflict.} Oakland: New Harbinger Publications.
    
    \textbf{Anwendung:} Neurobiologische Grundlagen der Partnerschaftsdynamik.
    
    \item \textbf{Gottman, J. M., \& Gottman, J. S. (2017).} \textit{The Natural Principles of Love.} Journal of Family Theory \& Review, 9(1), 7-26.
    
    \textbf{Integration:} Positive Kommunikationsmuster und Konfliktlösung.
\end{enumerate}

\end{ctmmBlueBox}

\subsection{Achtsamkeit und Stressreduktion}

\begin{ctmmGreenBox}[title=Mindfulness-Based Interventions]

\textbf{Evidenzbasierte Praktiken:}
\begin{enumerate}
    \item \textbf{Kabat-Zinn, J. (2013).} \textit{Full Catastrophe Living: Using the Wisdom of Your Body and Mind to Face Stress, Pain, and Illness.} New York: Bantam Books.
    
    \textbf{CTMM-Integration:} Achtsamkeitsübungen für Trigger-Prävention.
    
    \item \textbf{Williams, M., \& Penman, D. (2011).} \textit{Mindfulness: An Eight-Week Plan for Finding Peace in a Frantic World.} New York: Rodale Books.
    
    \textbf{Anwendung:} Strukturierte Achtsamkeitspraxis für Paare.
    
    \item \textbf{Siegel, D. J. (2010).} \textit{Mindsight: The New Science of Personal Transformation.} New York: Bantam Books.
    
    \textbf{Relevanz:} Neurowissenschaftliche Grundlagen der Selbstregulation.
\end{enumerate}

\end{ctmmGreenBox}

\subsection{Aktuelle Forschung zu neurodiversen Partnerschaften}

\begin{ctmmOrangeBox}[title=Peer-Review Studien]

\textbf{Wissenschaftliche Artikel:}
\begin{enumerate}
    \item \textbf{Renty, J., \& Roeyers, H. (2006).} Quality of life in high-functioning adults with autism spectrum disorder: The predictive value of disability and support characteristics. \textit{Autism, 10}(5), 511-524.
    
    \textbf{Findings:} Soziale Unterstützung als Schlüsselfaktor für Lebensqualität.
    
    \item \textbf{Bramston, P., Bruggerman, K., \& Pretty, G. (2002).} Community perspectives and subjective quality of life. \textit{International Journal of Disability, Development and Education, 49}(4), 385-397.
    
    \textbf{CTMM-Relevanz:} Gemeinschaftsbasierte Unterstützungssysteme.
    
    \item \textbf{Thompson, C., \& Romo, L. (2021).} Couples therapy for neurodiverse relationships: Clinical considerations and evidence-based approaches. \textit{Journal of Marital and Family Therapy, 47}(2), 293-308.
    
    \textbf{Direkter Bezug:} Evidenz für paartherapeutische Interventionen bei neurodiversen Paaren.
\end{enumerate}

\end{ctmmOrangeBox}

\subsection{Online-Ressourcen und Fachorganisationen}

\begin{ctmmPurpleBox}[title=Verlässliche digitale Quellen]

\textbf{Professionelle Organisationen:}
\begin{itemize}
    \item \textbf{International Society for DBT:} \url{https://isitdbt.org}
    
    \textbf{Nutzung:} Aktuelle DBT-Standards und Zertifizierungen
    
    \item \textbf{Autism Society:} \url{https://www.autism-society.org}
    
    \textbf{Relevanz:} Ressourcen für erwachsene Autisten und ihre Partner
    
    \item \textbf{International Centre for Excellence in EFT:} \url{https://iceeft.com}
    
    \textbf{Integration:} Emotionsfokussierte Paartherapie-Prinzipien
    
    \item \textbf{Mindfulness in Schools Project:} \url{https://mindfulnessinschools.org}
    
    \textbf{Anwendung:} Evidenzbasierte Achtsamkeitspraktiken
\end{itemize}

\end{ctmmPurpleBox}

\subsection{Methodische Anmerkungen}

\begin{ctmmRedBox}[title=\textcolor{white}{Wichtige Hinweise zur Quellenverwendung}]

\textcolor{white}{\textbf{Limitation und Ethik:}}

\begin{itemize}
    \item[\textcolor{white}{•}] \textcolor{white}{Das CTMM-System ist ein therapeutisches Hilfsmittel, kein Ersatz für professionelle Therapie}
    \item[\textcolor{white}{•}] \textcolor{white}{Alle Interventionen sollten unter fachlicher Anleitung angewendet werden}
    \item[\textcolor{white}{•}] \textcolor{white}{Die Quellenauswahl erfolgte nach aktuellen evidenzbasierten Standards}
    \item[\textcolor{white}{•}] \textcolor{white}{Regelmäßige Updates der Literaturgrundlage werden empfohlen}
\end{itemize}

\textcolor{white}{\textbf{Letzte Aktualisierung:} August 2025}

\end{ctmmRedBox}

\subsection{Weiterführende Ressourcen}

\begin{ctmmGreenBox}[title=Empfohlene Vertiefung]

\textbf{Für Therapeuten:}
\begin{itemize}
    \item DBT-Trainings und Zertifizierungen
    \item Fortbildungen zu neurodiversen Beziehungen
    \item Trauma-informierte Paartherapie-Weiterbildungen
\end{itemize}

\textbf{Für Betroffene und Angehörige:}
\begin{itemize}
    \item Selbsthilfegruppen für neurodiverse Paare
    \item Online-Communities und Foren
    \item Workshops zu Kommunikation und Emotionsregulation
\end{itemize}

\textbf{Für Forschungsinteressierte:}
\begin{itemize}
    \item Aktuelle Metaanalysen zu DBT-Wirksamkeit
    \item Längsschnittstudien zu neurodiversen Partnerschaften
    \item Neurobiologische Forschung zu Bindung und Regulation
\end{itemize}

\end{ctmmGreenBox}

\newpage

% =====================================================
% CTMM Form Elements Demo
% Purpose: Demonstrate interactive form capabilities
% Author: CTMM-Team
% Integration: % =====================================================
% CTMM Form Elements Demo
% Purpose: Demonstrate interactive form capabilities
% Author: CTMM-Team
% Integration: % =====================================================
% CTMM Form Elements Demo
% Purpose: Demonstrate interactive form capabilities
% Author: CTMM-Team
% Integration: \input{modules/form-demo}
% =====================================================

\section*{\textcolor{ctmmOrange}{\faCog~Form Elements Demo}}

\begin{quote}
\textbf{\textcolor{ctmmOrange}{Worum geht's hier?}}\\
Diese Demo zeigt alle verfügbaren interaktiven Formularelemente für CTMM-Arbeitsblätter.
\end{quote}

\subsection*{\textcolor{ctmmBlue}{Basis-Eingabefelder}}

\begin{ctmmBlueBox}[title=Standard-Eingaben]
\ctmmDate{demo} \quad \ctmmTime{demo}\\[0.3cm]

\textbf{Name:} \ctmmTextField[6cm]{}{demo-name}\\[0.3cm]

\textbf{Notizen:}\\
\ctmmTextArea[14cm]{3}{demo-notes}{}
\end{ctmmBlueBox}

\subsection*{\textcolor{ctmmGreen}{Stimmungs-Tracking}}

\begin{ctmmGreenBox}[title=Emotionale Bewertung]
\ctmmEmotionScale{Aktuelle Stimmung}{demo-mood}\\[0.5cm]

\ctmmStressLevel{demo}\\[0.3cm]

\textbf{Energie heute:}\\
\ctmmYesNo{demo-energy} Hoch \quad \ctmmYesNo{demo-energy} Normal \quad \ctmmYesNo{demo-energy} Niedrig
\end{ctmmGreenBox}

\subsection*{\textcolor{ctmmOrange}{Trigger-Management}}

\begin{ctmmOrangeBox}[title=Trigger-Bewertung]
\ctmmTriggerScale{demo}\\[0.5cm]

\ctmmSafeWordOptions{demo}\\[0.3cm]

\textbf{Bewältigungsstrategien verwendet:}\\
\ctmmCheckBox{demo-breathing}{Atemtechnik} \quad 
\ctmmCheckBox{demo-grounding}{Grounding} \quad 
\ctmmCheckBox{demo-pause}{Pause}\\
\ctmmCheckBox{demo-movement}{Bewegung} \quad 
\ctmmCheckBox{demo-talk}{Gespräch} \quad 
\ctmmCheckBox{demo-music}{Musik}
\end{ctmmOrangeBox}

\subsection*{\textcolor{ctmmPurple}{Wochenübersicht}}

\begin{ctmmPurpleBox}[title=7-Tage-Muster]
\ctmmWeeklyPattern{demo}\\[0.3cm]

\textbf{Auffällige Muster diese Woche:}\\
\ctmmTextArea[14cm]{2}{demo-pattern}{}
\end{ctmmPurpleBox}

\subsection*{\textcolor{ctmmRed}{Quick-Check Komponenten}}

% Kompletter Tagescheck
\ctmmDailyTracker{demo}

\vspace{0.5cm}

% Krisen-Protokoll
\ctmmCrisisForm{demo}

\subsection*{\textcolor{ctmmBlue}{CTMM-Navigation}}
\begin{itemize}
  \item \texttt{arbeitsblatt-taeglicher-stimmungscheck} ← Praktische Anwendung
  \item \texttt{interactive} ← Weitere interaktive Beispiele
\end{itemize}

% =====================================================

\section*{\textcolor{ctmmOrange}{\faCog~Form Elements Demo}}

\begin{quote}
\textbf{\textcolor{ctmmOrange}{Worum geht's hier?}}\\
Diese Demo zeigt alle verfügbaren interaktiven Formularelemente für CTMM-Arbeitsblätter.
\end{quote}

\subsection*{\textcolor{ctmmBlue}{Basis-Eingabefelder}}

\begin{ctmmBlueBox}[title=Standard-Eingaben]
\ctmmDate{demo} \quad \ctmmTime{demo}\\[0.3cm]

\textbf{Name:} \ctmmTextField[6cm]{}{demo-name}\\[0.3cm]

\textbf{Notizen:}\\
\ctmmTextArea[14cm]{3}{demo-notes}{}
\end{ctmmBlueBox}

\subsection*{\textcolor{ctmmGreen}{Stimmungs-Tracking}}

\begin{ctmmGreenBox}[title=Emotionale Bewertung]
\ctmmEmotionScale{Aktuelle Stimmung}{demo-mood}\\[0.5cm]

\ctmmStressLevel{demo}\\[0.3cm]

\textbf{Energie heute:}\\
\ctmmYesNo{demo-energy} Hoch \quad \ctmmYesNo{demo-energy} Normal \quad \ctmmYesNo{demo-energy} Niedrig
\end{ctmmGreenBox}

\subsection*{\textcolor{ctmmOrange}{Trigger-Management}}

\begin{ctmmOrangeBox}[title=Trigger-Bewertung]
\ctmmTriggerScale{demo}\\[0.5cm]

\ctmmSafeWordOptions{demo}\\[0.3cm]

\textbf{Bewältigungsstrategien verwendet:}\\
\ctmmCheckBox{demo-breathing}{Atemtechnik} \quad 
\ctmmCheckBox{demo-grounding}{Grounding} \quad 
\ctmmCheckBox{demo-pause}{Pause}\\
\ctmmCheckBox{demo-movement}{Bewegung} \quad 
\ctmmCheckBox{demo-talk}{Gespräch} \quad 
\ctmmCheckBox{demo-music}{Musik}
\end{ctmmOrangeBox}

\subsection*{\textcolor{ctmmPurple}{Wochenübersicht}}

\begin{ctmmPurpleBox}[title=7-Tage-Muster]
\ctmmWeeklyPattern{demo}\\[0.3cm]

\textbf{Auffällige Muster diese Woche:}\\
\ctmmTextArea[14cm]{2}{demo-pattern}{}
\end{ctmmPurpleBox}

\subsection*{\textcolor{ctmmRed}{Quick-Check Komponenten}}

% Kompletter Tagescheck
\ctmmDailyTracker{demo}

\vspace{0.5cm}

% Krisen-Protokoll
\ctmmCrisisForm{demo}

\subsection*{\textcolor{ctmmBlue}{CTMM-Navigation}}
\begin{itemize}
  \item \texttt{arbeitsblatt-taeglicher-stimmungscheck} ← Praktische Anwendung
  \item \texttt{interactive} ← Weitere interaktive Beispiele
\end{itemize}

% =====================================================

\section*{\textcolor{ctmmOrange}{\faCog~Form Elements Demo}}

\begin{quote}
\textbf{\textcolor{ctmmOrange}{Worum geht's hier?}}\\
Diese Demo zeigt alle verfügbaren interaktiven Formularelemente für CTMM-Arbeitsblätter.
\end{quote}

\subsection*{\textcolor{ctmmBlue}{Basis-Eingabefelder}}

\begin{ctmmBlueBox}[title=Standard-Eingaben]
\ctmmDate{demo} \quad \ctmmTime{demo}\\[0.3cm]

\textbf{Name:} \ctmmTextField[6cm]{}{demo-name}\\[0.3cm]

\textbf{Notizen:}\\
\ctmmTextArea[14cm]{3}{demo-notes}{}
\end{ctmmBlueBox}

\subsection*{\textcolor{ctmmGreen}{Stimmungs-Tracking}}

\begin{ctmmGreenBox}[title=Emotionale Bewertung]
\ctmmEmotionScale{Aktuelle Stimmung}{demo-mood}\\[0.5cm]

\ctmmStressLevel{demo}\\[0.3cm]

\textbf{Energie heute:}\\
\ctmmYesNo{demo-energy} Hoch \quad \ctmmYesNo{demo-energy} Normal \quad \ctmmYesNo{demo-energy} Niedrig
\end{ctmmGreenBox}

\subsection*{\textcolor{ctmmOrange}{Trigger-Management}}

\begin{ctmmOrangeBox}[title=Trigger-Bewertung]
\ctmmTriggerScale{demo}\\[0.5cm]

\ctmmSafeWordOptions{demo}\\[0.3cm]

\textbf{Bewältigungsstrategien verwendet:}\\
\ctmmCheckBox{demo-breathing}{Atemtechnik} \quad 
\ctmmCheckBox{demo-grounding}{Grounding} \quad 
\ctmmCheckBox{demo-pause}{Pause}\\
\ctmmCheckBox{demo-movement}{Bewegung} \quad 
\ctmmCheckBox{demo-talk}{Gespräch} \quad 
\ctmmCheckBox{demo-music}{Musik}
\end{ctmmOrangeBox}

\subsection*{\textcolor{ctmmPurple}{Wochenübersicht}}

\begin{ctmmPurpleBox}[title=7-Tage-Muster]
\ctmmWeeklyPattern{demo}\\[0.3cm]

\textbf{Auffällige Muster diese Woche:}\\
\ctmmTextArea[14cm]{2}{demo-pattern}{}
\end{ctmmPurpleBox}

\subsection*{\textcolor{ctmmRed}{Quick-Check Komponenten}}

% Kompletter Tagescheck
\ctmmDailyTracker{demo}

\vspace{0.5cm}

% Krisen-Protokoll
\ctmmCrisisForm{demo}

\subsection*{\textcolor{ctmmBlue}{CTMM-Navigation}}
\begin{itemize}
  \item \texttt{arbeitsblatt-taeglicher-stimmungscheck} ← Praktische Anwendung
  \item \texttt{interactive} ← Weitere interaktive Beispiele
\end{itemize}

% =====================================================
% CTMM Diagrams Demo Module
% Purpose: Demonstrate visual representation capabilities
% Author: CTMM-Team
% Integration: % =====================================================
% CTMM Diagrams Demo Module
% Purpose: Demonstrate visual representation capabilities
% Author: CTMM-Team
% Integration: % =====================================================
% CTMM Diagrams Demo Module
% Purpose: Demonstrate visual representation capabilities
% Author: CTMM-Team
% Integration: \input{modules/diagrams-demo-fixed}
% =====================================================

\section*{\textcolor{ctmmBlue}{\faPalette~Diagramme \& Visualisierungen}}

\begin{quote}
\textbf{\textcolor{ctmmBlue}{Worum geht's hier?}}\\
Dieses Modul zeigt die visuellen Möglichkeiten des CTMM-Systems für therapeutische Darstellungen und Diagramme.
\end{quote}

\subsection*{\textcolor{ctmmBlue}{Der CTMM-Trigger-Zyklus}}

Der Kern des CTMM-Systems als visueller Kreislauf:

\begin{center}
\begin{tcolorbox}[colback=ctmmBlue!10!white,colframe=ctmmBlue,width=10cm]
\centering
\textbf{CTMM-Trigger-Zyklus}\\[0.5cm]
\textcolor{ctmmBlue}{\textbf{ERKENNEN}} $\rightarrow$ \textcolor{ctmmGreen}{\textbf{VORBEUGEN}} $\rightarrow$ \textcolor{ctmmOrange}{\textbf{REAGIEREN}} $\rightarrow$ \textcolor{ctmmPurple}{\textbf{LERNEN}}
\end{tcolorbox}
\end{center}

\textit{Dieser Zyklus zeigt die vier Hauptphasen des Trigger-Managements: Erkennen $\rightarrow$ Vorbeugen $\rightarrow$ Reagieren $\rightarrow$ Lernen}

\subsection*{\textcolor{ctmmOrange}{Eskalationsstufen}}

Die vier Ebenen der Krisenintervention:

\begin{center}
\begin{tcolorbox}[colback=ctmmOrange!10!white,colframe=ctmmOrange,width=12cm]
\centering
\textbf{Eskalationsstufen}\\[0.5cm]
\textcolor{ctmmGreen}{\textbf{1. Grün - Stabil}} $\rightarrow$ \textcolor{ctmmOrange}{\textbf{2. Gelb - Warnung}} $\rightarrow$ \textcolor{ctmmRed}{\textbf{3. Rot - Krise}} $\rightarrow$ \textcolor{ctmmPurple}{\textbf{4. Stabilisierung}}
\end{tcolorbox}
\end{center}

\textit{Jede Stufe hat spezifische Strategien und Interventionen. Der Kreislauf zeigt sowohl Eskalation als auch den Weg zurück zur Stabilität.}

\newpage

\subsection*{\textcolor{ctmmRed}{Partner-Dynamiken}}

Systemische Sicht auf die Paarbeziehung:

\begin{center}
\begin{tcolorbox}[colback=ctmmRed!10!white,colframe=ctmmRed,width=12cm]
\centering
\textbf{Partner-Dynamiken}\\[0.5cm]
\textcolor{ctmmBlue}{\textbf{Partner A}} $\leftrightarrow$ \textcolor{ctmmGreen}{\textbf{Ko-Regulation}} $\leftrightarrow$ \textcolor{ctmmBlue}{\textbf{Partner B}}
\end{tcolorbox}
\end{center}

\textit{Das Diagramm zeigt typische Trigger-Reaktions-Muster und wie CTMM-Tools zur Ko-Regulation beitragen.}

\subsection*{\textcolor{ctmmGreen}{Weitere Diagramm-Möglichkeiten}}

\begin{ctmmGreenBox}[title=Zukünftige Erweiterungen]
Das CTMM-System kann erweitert werden um:
\begin{itemize}
  \item Timeline-Visualisierungen für Therapieverlauf
  \item Stimmungsdiagramme mit Trend-Analysen
  \item Interaktive Flowcharts für Entscheidungen
  \item Beziehungsdiagramme mit Feedback-Loops
\end{itemize}
\end{ctmmGreenBox}

\subsection*{\textcolor{ctmmBlue}{Verwendung in eigenen Modulen}}

\begin{ctmmBlueBox}[title=So nutzen Sie die Diagramme]
Die Diagramm-Befehle können in allen CTMM-Modulen verwendet werden:
\begin{itemize}
  \item Einfache Pfeil-Notation: \texttt{\$\textbackslash rightarrow\$}
  \item Bidirektionale Pfeile: \texttt{\$\textbackslash leftrightarrow\$}
  \item Farbige Boxen: \texttt{\textbackslash begin\{ctmmBlueBox\}}
  \item Zentrierte Darstellung mit \texttt{center} Umgebung
\end{itemize}
\end{ctmmBlueBox}

\subsection*{\textcolor{ctmmBlue}{CTMM-Navigation}}
\begin{itemize}
  \item \texttt{triggermanagement} $\leftarrow$ Praktische Anwendung der Zyklen
  \item \texttt{bindungsleitfaden} $\leftarrow$ Systemische Grundlagen
\end{itemize}

% =====================================================

\section*{\textcolor{ctmmBlue}{\faPalette~Diagramme \& Visualisierungen}}

\begin{quote}
\textbf{\textcolor{ctmmBlue}{Worum geht's hier?}}\\
Dieses Modul zeigt die visuellen Möglichkeiten des CTMM-Systems für therapeutische Darstellungen und Diagramme.
\end{quote}

\subsection*{\textcolor{ctmmBlue}{Der CTMM-Trigger-Zyklus}}

Der Kern des CTMM-Systems als visueller Kreislauf:

\begin{center}
\begin{tcolorbox}[colback=ctmmBlue!10!white,colframe=ctmmBlue,width=10cm]
\centering
\textbf{CTMM-Trigger-Zyklus}\\[0.5cm]
\textcolor{ctmmBlue}{\textbf{ERKENNEN}} $\rightarrow$ \textcolor{ctmmGreen}{\textbf{VORBEUGEN}} $\rightarrow$ \textcolor{ctmmOrange}{\textbf{REAGIEREN}} $\rightarrow$ \textcolor{ctmmPurple}{\textbf{LERNEN}}
\end{tcolorbox}
\end{center}

\textit{Dieser Zyklus zeigt die vier Hauptphasen des Trigger-Managements: Erkennen $\rightarrow$ Vorbeugen $\rightarrow$ Reagieren $\rightarrow$ Lernen}

\subsection*{\textcolor{ctmmOrange}{Eskalationsstufen}}

Die vier Ebenen der Krisenintervention:

\begin{center}
\begin{tcolorbox}[colback=ctmmOrange!10!white,colframe=ctmmOrange,width=12cm]
\centering
\textbf{Eskalationsstufen}\\[0.5cm]
\textcolor{ctmmGreen}{\textbf{1. Grün - Stabil}} $\rightarrow$ \textcolor{ctmmOrange}{\textbf{2. Gelb - Warnung}} $\rightarrow$ \textcolor{ctmmRed}{\textbf{3. Rot - Krise}} $\rightarrow$ \textcolor{ctmmPurple}{\textbf{4. Stabilisierung}}
\end{tcolorbox}
\end{center}

\textit{Jede Stufe hat spezifische Strategien und Interventionen. Der Kreislauf zeigt sowohl Eskalation als auch den Weg zurück zur Stabilität.}

\newpage

\subsection*{\textcolor{ctmmRed}{Partner-Dynamiken}}

Systemische Sicht auf die Paarbeziehung:

\begin{center}
\begin{tcolorbox}[colback=ctmmRed!10!white,colframe=ctmmRed,width=12cm]
\centering
\textbf{Partner-Dynamiken}\\[0.5cm]
\textcolor{ctmmBlue}{\textbf{Partner A}} $\leftrightarrow$ \textcolor{ctmmGreen}{\textbf{Ko-Regulation}} $\leftrightarrow$ \textcolor{ctmmBlue}{\textbf{Partner B}}
\end{tcolorbox}
\end{center}

\textit{Das Diagramm zeigt typische Trigger-Reaktions-Muster und wie CTMM-Tools zur Ko-Regulation beitragen.}

\subsection*{\textcolor{ctmmGreen}{Weitere Diagramm-Möglichkeiten}}

\begin{ctmmGreenBox}[title=Zukünftige Erweiterungen]
Das CTMM-System kann erweitert werden um:
\begin{itemize}
  \item Timeline-Visualisierungen für Therapieverlauf
  \item Stimmungsdiagramme mit Trend-Analysen
  \item Interaktive Flowcharts für Entscheidungen
  \item Beziehungsdiagramme mit Feedback-Loops
\end{itemize}
\end{ctmmGreenBox}

\subsection*{\textcolor{ctmmBlue}{Verwendung in eigenen Modulen}}

\begin{ctmmBlueBox}[title=So nutzen Sie die Diagramme]
Die Diagramm-Befehle können in allen CTMM-Modulen verwendet werden:
\begin{itemize}
  \item Einfache Pfeil-Notation: \texttt{\$\textbackslash rightarrow\$}
  \item Bidirektionale Pfeile: \texttt{\$\textbackslash leftrightarrow\$}
  \item Farbige Boxen: \texttt{\textbackslash begin\{ctmmBlueBox\}}
  \item Zentrierte Darstellung mit \texttt{center} Umgebung
\end{itemize}
\end{ctmmBlueBox}

\subsection*{\textcolor{ctmmBlue}{CTMM-Navigation}}
\begin{itemize}
  \item \texttt{triggermanagement} $\leftarrow$ Praktische Anwendung der Zyklen
  \item \texttt{bindungsleitfaden} $\leftarrow$ Systemische Grundlagen
\end{itemize}

% =====================================================

\section*{\textcolor{ctmmBlue}{\faPalette~Diagramme \& Visualisierungen}}

\begin{quote}
\textbf{\textcolor{ctmmBlue}{Worum geht's hier?}}\\
Dieses Modul zeigt die visuellen Möglichkeiten des CTMM-Systems für therapeutische Darstellungen und Diagramme.
\end{quote}

\subsection*{\textcolor{ctmmBlue}{Der CTMM-Trigger-Zyklus}}

Der Kern des CTMM-Systems als visueller Kreislauf:

\begin{center}
\begin{tcolorbox}[colback=ctmmBlue!10!white,colframe=ctmmBlue,width=10cm]
\centering
\textbf{CTMM-Trigger-Zyklus}\\[0.5cm]
\textcolor{ctmmBlue}{\textbf{ERKENNEN}} $\rightarrow$ \textcolor{ctmmGreen}{\textbf{VORBEUGEN}} $\rightarrow$ \textcolor{ctmmOrange}{\textbf{REAGIEREN}} $\rightarrow$ \textcolor{ctmmPurple}{\textbf{LERNEN}}
\end{tcolorbox}
\end{center}

\textit{Dieser Zyklus zeigt die vier Hauptphasen des Trigger-Managements: Erkennen $\rightarrow$ Vorbeugen $\rightarrow$ Reagieren $\rightarrow$ Lernen}

\subsection*{\textcolor{ctmmOrange}{Eskalationsstufen}}

Die vier Ebenen der Krisenintervention:

\begin{center}
\begin{tcolorbox}[colback=ctmmOrange!10!white,colframe=ctmmOrange,width=12cm]
\centering
\textbf{Eskalationsstufen}\\[0.5cm]
\textcolor{ctmmGreen}{\textbf{1. Grün - Stabil}} $\rightarrow$ \textcolor{ctmmOrange}{\textbf{2. Gelb - Warnung}} $\rightarrow$ \textcolor{ctmmRed}{\textbf{3. Rot - Krise}} $\rightarrow$ \textcolor{ctmmPurple}{\textbf{4. Stabilisierung}}
\end{tcolorbox}
\end{center}

\textit{Jede Stufe hat spezifische Strategien und Interventionen. Der Kreislauf zeigt sowohl Eskalation als auch den Weg zurück zur Stabilität.}

\newpage

\subsection*{\textcolor{ctmmRed}{Partner-Dynamiken}}

Systemische Sicht auf die Paarbeziehung:

\begin{center}
\begin{tcolorbox}[colback=ctmmRed!10!white,colframe=ctmmRed,width=12cm]
\centering
\textbf{Partner-Dynamiken}\\[0.5cm]
\textcolor{ctmmBlue}{\textbf{Partner A}} $\leftrightarrow$ \textcolor{ctmmGreen}{\textbf{Ko-Regulation}} $\leftrightarrow$ \textcolor{ctmmBlue}{\textbf{Partner B}}
\end{tcolorbox}
\end{center}

\textit{Das Diagramm zeigt typische Trigger-Reaktions-Muster und wie CTMM-Tools zur Ko-Regulation beitragen.}

\subsection*{\textcolor{ctmmGreen}{Weitere Diagramm-Möglichkeiten}}

\begin{ctmmGreenBox}[title=Zukünftige Erweiterungen]
Das CTMM-System kann erweitert werden um:
\begin{itemize}
  \item Timeline-Visualisierungen für Therapieverlauf
  \item Stimmungsdiagramme mit Trend-Analysen
  \item Interaktive Flowcharts für Entscheidungen
  \item Beziehungsdiagramme mit Feedback-Loops
\end{itemize}
\end{ctmmGreenBox}

\subsection*{\textcolor{ctmmBlue}{Verwendung in eigenen Modulen}}

\begin{ctmmBlueBox}[title=So nutzen Sie die Diagramme]
Die Diagramm-Befehle können in allen CTMM-Modulen verwendet werden:
\begin{itemize}
  \item Einfache Pfeil-Notation: \texttt{\$\textbackslash rightarrow\$}
  \item Bidirektionale Pfeile: \texttt{\$\textbackslash leftrightarrow\$}
  \item Farbige Boxen: \texttt{\textbackslash begin\{ctmmBlueBox\}}
  \item Zentrierte Darstellung mit \texttt{center} Umgebung
\end{itemize}
\end{ctmmBlueBox}

\subsection*{\textcolor{ctmmBlue}{CTMM-Navigation}}
\begin{itemize}
  \item \texttt{triggermanagement} $\leftarrow$ Praktische Anwendung der Zyklen
  \item \texttt{bindungsleitfaden} $\leftarrow$ Systemische Grundlagen
\end{itemize}

% Demo für interaktive PDF-Formulare
\newpage
\section*{\textcolor{ctmmBlue}{\faEdit~Demo: Interaktive PDF-Formulare}}
\label{sec:demo-interactive}
\addcontentsline{toc}{section}{Demo: Interaktive Formulare}

\begin{ctmmBlueBox}{Testen Sie die interaktiven Felder}
\textbf{Hinweis:} Diese Felder können direkt im PDF ausgefüllt werden. Die Eingaben werden lokal in der PDF-Datei gespeichert.
\end{ctmmBlueBox}

\vspace{1cm}

\subsection*{Textfelder}
\textbf{Name:} \ctmmTextField[6cm]{}{demo_name} \\[0.5cm]
\textbf{E-Mail:} \ctmmTextField[8cm]{}{demo_email} \\[0.5cm]
\textbf{Telefon:} \ctmmTextField[4cm]{}{demo_phone}

\vspace{1cm}

\subsection*{Checkboxen}
\textbf{Hobbys:} \\[0.3cm]
\ctmmCheckBox[hobby_reading]{Lesen} \quad
\ctmmCheckBox[hobby_music]{Musik} \quad
\ctmmCheckBox[hobby_sports]{Sport} \\[0.3cm]
\ctmmCheckBox[hobby_cooking]{Kochen} \quad
\ctmmCheckBox[hobby_travel]{Reisen} \quad
\ctmmCheckBox[hobby_gaming]{Gaming}

\vspace{1cm}

\subsection*{Mehrzeilige Textfelder}
\textbf{Beschreiben Sie Ihren Tag:} \\[0.3cm]
\ctmmTextArea[12cm]{3}{demo_day_description}{}

\vspace{0.5cm}

\textbf{Ziele für morgen:} \\[0.3cm]
\ctmmTextArea[12cm]{2}{demo_tomorrow_goals}{}

\vspace{1cm}

\begin{ctmmYellowBox}{\faInfoCircle~Funktionen}
\begin{itemize}
    \item \textbf{Ausfüllen:} Klicken Sie in die Felder und geben Sie Text ein
    \item \textbf{Checkboxen:} Klicken Sie die Kästchen an/ab
    \item \textbf{Speichern:} Strg+S speichert Ihre Eingaben in der PDF
    \item \textbf{Drucken:} Die Felder werden mit den Eingaben gedruckt
    \item \textbf{Reset:} Formular → Alle Felder löschen (im PDF-Viewer)
\end{itemize}
\end{ctmmYellowBox}

\vspace{1cm}

\begin{ctmmGreenBox}{Vorteile für das CTMM-System}
\begin{itemize}
    \item \textbf{Digitale Dokumentation:} Alle Eingaben direkt im PDF
    \item \textbf{Datenschutz:} Daten bleiben lokal bei Ihnen
    \item \textbf{Archivierung:} Gespeicherte PDFs für Verlaufsdokumentation
    \item \textbf{Therapeutische Auswertung:} LLMs können die Daten analysieren
    \item \textbf{Flexibilität:} Online ausfüllen oder ausdrucken
\end{itemize}
\end{ctmmGreenBox}


\end{document}
