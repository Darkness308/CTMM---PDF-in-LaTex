% =====================================================
% CTMM Diagrams Demo
% Purpose: Showcase all available CTMM diagrams
% Author: CTMM-Team
% Integration: % =====================================================
% CTMM Diagrams Demo
% Purpose: Showcase all available CTMM diagrams
% Author: CTMM-Team
% Integration: % =====================================================
% CTMM Diagrams Demo
% Purpose: Showcase all available CTMM diagrams
% Author: CTMM-Team
% Integration: % =====================================================
% CTMM Diagrams Demo
% Purpose: Showcase all available CTMM diagrams
% Author: CTMM-Team
% Integration: \input{modules/diagrams-demo}
% =====================================================

\section*{\textcolor{ctmmPurple}{\faChartArea~CTMM Visualisierungen}}

\begin{quote}
\textbf{\textcolor{ctmmPurple}{Worum geht's hier?}}\\
Diese Seite zeigt alle verfügbaren Diagramme zur Veranschaulichung von CTMM-Konzepten.
\end{quote}

\subsection*{\textcolor{ctmmBlue}{Der CTMM-Trigger-Zyklus}}

Der Kern des CTMM-Systems als visueller Kreislauf:

\begin{center}
\begin{tcolorbox}[colback=ctmmBlue!10!white,colframe=ctmmBlue,width=10cm]
\centering
\textbf{CTMM-Trigger-Zyklus}\\[0.5cm]
\textcolor{ctmmBlue}{\textbf{ERKENNEN}} → \textcolor{ctmmGreen}{\textbf{VORBEUGEN}} → \textcolor{ctmmOrange}{\textbf{REAGIEREN}} → \textcolor{ctmmPurple}{\textbf{LERNEN}}
\end{tcolorbox}
\end{center}

\textit{Dieser Zyklus zeigt die vier Hauptphasen des Trigger-Managements: Erkennen → Vorbeugen → Reagieren → Lernen}

\subsection*{\textcolor{ctmmOrange}{Eskalationsstufen}}

Die vier Ebenen der Krisenintervention:

\begin{center}
\begin{tcolorbox}[colback=ctmmOrange!10!white,colframe=ctmmOrange,width=12cm]
\centering
\textbf{Eskalationsstufen}\\[0.5cm]
\textcolor{ctmmGreen}{\textbf{1. Grün - Stabil}} → \textcolor{ctmmOrange}{\textbf{2. Gelb - Warnung}} → \textcolor{ctmmRed}{\textbf{3. Rot - Krise}} → \textcolor{ctmmPurple}{\textbf{4. Stabilisierung}}
\end{tcolorbox}
\end{center}

\textit{Jede Stufe hat spezifische Strategien und Interventionen. Der Kreislauf zeigt sowohl Eskalation als auch den Weg zurück zur Stabilität.}

\newpage

\subsection*{\textcolor{ctmmRed}{Partner-Dynamiken}}

Systemische Sicht auf die Paarbeziehung:

\begin{center}
\begin{tcolorbox}[colback=ctmmRed!10!white,colframe=ctmmRed,width=12cm]
\centering
\textbf{Partner-Dynamiken}\\[0.5cm]
\textcolor{ctmmBlue}{\textbf{Partner A}} $\leftrightarrow$ \textcolor{ctmmGreen}{\textbf{Ko-Regulation}} $\leftrightarrow$ \textcolor{ctmmBlue}{\textbf{Partner B}}
\end{tcolorbox}
\end{center}

\textit{Das Diagramm zeigt typische Trigger-Reaktions-Muster und wie CTMM-Tools zur Ko-Regulation beitragen.}

\subsection*{\textcolor{ctmmGreen}{Therapieverlauf Timeline}}

\textbf{Beispiel einer 4-Wochen-Behandlung:}

\begin{center}
\begin{tikzpicture}[
    node distance=2cm,
    every node/.style={font=\footnotesize}
]
    % Timeline base
    \draw[thick, ctmmGray] (0,0) -- (10,0);

    % Time markers
    \foreach \x/\label in {0/Start,2.5/Woche 1,5/Woche 2,7.5/Woche 3,10/Monat} {
        \draw (\x,-0.1) -- (\x,0.1);
        \node[below] at (\x,-0.1) {\label};
    }

    % Events
    \node[circle, fill=ctmmBlue!50, draw=ctmmBlue, minimum size=0.5cm] at (1,0.5) {};
    \node[above, text width=2cm, align=center] at (1,1) {\tiny Erstkontakt\\Anamnese};
    \draw[ctmmBlue, thick] (1,0) -- (1,0.5);

    \node[circle, fill=ctmmGreen!50, draw=ctmmGreen, minimum size=0.5cm] at (3,0.5) {};
    \node[above, text width=2cm, align=center] at (3,1) {\tiny Safe-Words\\etabliert};
    \draw[ctmmGreen, thick] (3,0) -- (3,0.5);

    \node[circle, fill=ctmmOrange!50, draw=ctmmOrange, minimum size=0.5cm] at (6,0.5) {};
    \node[above, text width=2cm, align=center] at (6,1) {\tiny Erste Krise\\bewältigt};
    \draw[ctmmOrange, thick] (6,0) -- (6,0.5);

    \node[circle, fill=ctmmPurple!50, draw=ctmmPurple, minimum size=0.5cm] at (8.5,0.5) {};
    \node[above, text width=2cm, align=center] at (8.5,1) {\tiny Reflexion\\Integration};
    \draw[ctmmPurple, thick] (8.5,0) -- (8.5,0.5);
\end{tikzpicture}
\end{center}

\subsection*{\textcolor{ctmmPurple}{Stimmungsverlauf}}

\textbf{Beispiel-Graph für Mood-Tracking:}

\begin{center}
\begin{tikzpicture}[
    scale=0.8,
    every node/.style={font=\tiny}
]
    % Axes
    \draw[->] (0,0) -- (8,0) node[right] {Zeit};
    \draw[->] (0,0) -- (0,6) node[above] {Stimmung};

    % Grid
    \foreach \x in {1,2,...,7} {
        \draw[gray!30] (\x,0) -- (\x,5);
    }
    \foreach \y in {1,2,3,4,5} {
        \draw[gray!30] (0,\y) -- (7,\y);
        \node[left] at (0,\y) {\y};
    }

    % Data line
    \draw[ctmmBlue, thick]
        (0,3) -- (1,2.5) -- (2,1.5) -- (3,2) -- (4,3.5) -- (5,4) -- (6,4.2) -- (7,4.5);
    \node[ctmmBlue, above] at (3.5,3) {\tiny Therapiestart};

    % Zones
    \fill[ctmmRed!10] (0,0) rectangle (7,2);
    \fill[ctmmOrange!10] (0,2) rectangle (7,3.5);
    \fill[ctmmGreen!10] (0,3.5) rectangle (7,5);

    \node[right, font=\tiny] at (7.1,1) {Krise};
    \node[right, font=\tiny] at (7.1,2.75) {Achtung};
    \node[right, font=\tiny] at (7.1,4.25) {Stabil};
\end{tikzpicture}
\end{center}

\subsection*{\textcolor{ctmmBlue}{Verwendung in eigenen Modulen}}

\begin{ctmmBlueBox}[title=So nutzen Sie die Diagramme]
\begin{itemize}
    \item \texttt{\textbackslash ctmmTriggerLoop} - Grundlagen-Psychoedukation
    \item \texttt{\textbackslash ctmmEskalationsStufen} - Krisenplan erklären
    \item \texttt{\textbackslash ctmmPartnerDynamics} - Paartherapie-Sessions
    \item \texttt{\textbackslash ctmmTimeline\{...\}} - Therapieverlauf dokumentieren
    \item \texttt{\textbackslash ctmmMoodGraph\{...\}} - Stimmungsdaten visualisieren
\end{itemize}
\end{ctmmBlueBox}

\subsection*{\textcolor{ctmmBlue}{CTMM-Navigation}}
\begin{itemize}
  \item \texttt{triggermanagement} ← Praktische Anwendung der Zyklen
  \item \texttt{bindungsleitfaden} ← Systemische Grundlagen
\end{itemize}

% =====================================================

\section*{\textcolor{ctmmPurple}{\faChartArea~CTMM Visualisierungen}}

\begin{quote}
\textbf{\textcolor{ctmmPurple}{Worum geht's hier?}}\\
Diese Seite zeigt alle verfügbaren Diagramme zur Veranschaulichung von CTMM-Konzepten.
\end{quote}

\subsection*{\textcolor{ctmmBlue}{Der CTMM-Trigger-Zyklus}}

Der Kern des CTMM-Systems als visueller Kreislauf:

\begin{center}
\begin{tcolorbox}[colback=ctmmBlue!10!white,colframe=ctmmBlue,width=10cm]
\centering
\textbf{CTMM-Trigger-Zyklus}\\[0.5cm]
\textcolor{ctmmBlue}{\textbf{ERKENNEN}} → \textcolor{ctmmGreen}{\textbf{VORBEUGEN}} → \textcolor{ctmmOrange}{\textbf{REAGIEREN}} → \textcolor{ctmmPurple}{\textbf{LERNEN}}
\end{tcolorbox}
\end{center}

\textit{Dieser Zyklus zeigt die vier Hauptphasen des Trigger-Managements: Erkennen → Vorbeugen → Reagieren → Lernen}

\subsection*{\textcolor{ctmmOrange}{Eskalationsstufen}}

Die vier Ebenen der Krisenintervention:

\begin{center}
\begin{tcolorbox}[colback=ctmmOrange!10!white,colframe=ctmmOrange,width=12cm]
\centering
\textbf{Eskalationsstufen}\\[0.5cm]
\textcolor{ctmmGreen}{\textbf{1. Grün - Stabil}} → \textcolor{ctmmOrange}{\textbf{2. Gelb - Warnung}} → \textcolor{ctmmRed}{\textbf{3. Rot - Krise}} → \textcolor{ctmmPurple}{\textbf{4. Stabilisierung}}
\end{tcolorbox}
\end{center}

\textit{Jede Stufe hat spezifische Strategien und Interventionen. Der Kreislauf zeigt sowohl Eskalation als auch den Weg zurück zur Stabilität.}

\newpage

\subsection*{\textcolor{ctmmRed}{Partner-Dynamiken}}

Systemische Sicht auf die Paarbeziehung:

\begin{center}
\begin{tcolorbox}[colback=ctmmRed!10!white,colframe=ctmmRed,width=12cm]
\centering
\textbf{Partner-Dynamiken}\\[0.5cm]
\textcolor{ctmmBlue}{\textbf{Partner A}} $\leftrightarrow$ \textcolor{ctmmGreen}{\textbf{Ko-Regulation}} $\leftrightarrow$ \textcolor{ctmmBlue}{\textbf{Partner B}}
\end{tcolorbox}
\end{center}

\textit{Das Diagramm zeigt typische Trigger-Reaktions-Muster und wie CTMM-Tools zur Ko-Regulation beitragen.}

\subsection*{\textcolor{ctmmGreen}{Therapieverlauf Timeline}}

\textbf{Beispiel einer 4-Wochen-Behandlung:}

\begin{center}
\begin{tikzpicture}[
    node distance=2cm,
    every node/.style={font=\footnotesize}
]
    % Timeline base
    \draw[thick, ctmmGray] (0,0) -- (10,0);

    % Time markers
    \foreach \x/\label in {0/Start,2.5/Woche 1,5/Woche 2,7.5/Woche 3,10/Monat} {
        \draw (\x,-0.1) -- (\x,0.1);
        \node[below] at (\x,-0.1) {\label};
    }

    % Events
    \node[circle, fill=ctmmBlue!50, draw=ctmmBlue, minimum size=0.5cm] at (1,0.5) {};
    \node[above, text width=2cm, align=center] at (1,1) {\tiny Erstkontakt\\Anamnese};
    \draw[ctmmBlue, thick] (1,0) -- (1,0.5);

    \node[circle, fill=ctmmGreen!50, draw=ctmmGreen, minimum size=0.5cm] at (3,0.5) {};
    \node[above, text width=2cm, align=center] at (3,1) {\tiny Safe-Words\\etabliert};
    \draw[ctmmGreen, thick] (3,0) -- (3,0.5);

    \node[circle, fill=ctmmOrange!50, draw=ctmmOrange, minimum size=0.5cm] at (6,0.5) {};
    \node[above, text width=2cm, align=center] at (6,1) {\tiny Erste Krise\\bewältigt};
    \draw[ctmmOrange, thick] (6,0) -- (6,0.5);

    \node[circle, fill=ctmmPurple!50, draw=ctmmPurple, minimum size=0.5cm] at (8.5,0.5) {};
    \node[above, text width=2cm, align=center] at (8.5,1) {\tiny Reflexion\\Integration};
    \draw[ctmmPurple, thick] (8.5,0) -- (8.5,0.5);
\end{tikzpicture}
\end{center}

\subsection*{\textcolor{ctmmPurple}{Stimmungsverlauf}}

\textbf{Beispiel-Graph für Mood-Tracking:}

\begin{center}
\begin{tikzpicture}[
    scale=0.8,
    every node/.style={font=\tiny}
]
    % Axes
    \draw[->] (0,0) -- (8,0) node[right] {Zeit};
    \draw[->] (0,0) -- (0,6) node[above] {Stimmung};

    % Grid
    \foreach \x in {1,2,...,7} {
        \draw[gray!30] (\x,0) -- (\x,5);
    }
    \foreach \y in {1,2,3,4,5} {
        \draw[gray!30] (0,\y) -- (7,\y);
        \node[left] at (0,\y) {\y};
    }

    % Data line
    \draw[ctmmBlue, thick]
        (0,3) -- (1,2.5) -- (2,1.5) -- (3,2) -- (4,3.5) -- (5,4) -- (6,4.2) -- (7,4.5);
    \node[ctmmBlue, above] at (3.5,3) {\tiny Therapiestart};

    % Zones
    \fill[ctmmRed!10] (0,0) rectangle (7,2);
    \fill[ctmmOrange!10] (0,2) rectangle (7,3.5);
    \fill[ctmmGreen!10] (0,3.5) rectangle (7,5);

    \node[right, font=\tiny] at (7.1,1) {Krise};
    \node[right, font=\tiny] at (7.1,2.75) {Achtung};
    \node[right, font=\tiny] at (7.1,4.25) {Stabil};
\end{tikzpicture}
\end{center}

\subsection*{\textcolor{ctmmBlue}{Verwendung in eigenen Modulen}}

\begin{ctmmBlueBox}[title=So nutzen Sie die Diagramme]
\begin{itemize}
    \item \texttt{\textbackslash ctmmTriggerLoop} - Grundlagen-Psychoedukation
    \item \texttt{\textbackslash ctmmEskalationsStufen} - Krisenplan erklären
    \item \texttt{\textbackslash ctmmPartnerDynamics} - Paartherapie-Sessions
    \item \texttt{\textbackslash ctmmTimeline\{...\}} - Therapieverlauf dokumentieren
    \item \texttt{\textbackslash ctmmMoodGraph\{...\}} - Stimmungsdaten visualisieren
\end{itemize}
\end{ctmmBlueBox}

\subsection*{\textcolor{ctmmBlue}{CTMM-Navigation}}
\begin{itemize}
  \item \texttt{triggermanagement} ← Praktische Anwendung der Zyklen
  \item \texttt{bindungsleitfaden} ← Systemische Grundlagen
\end{itemize}

% =====================================================

\section*{\textcolor{ctmmPurple}{\faChartArea~CTMM Visualisierungen}}

\begin{quote}
\textbf{\textcolor{ctmmPurple}{Worum geht's hier?}}\\
Diese Seite zeigt alle verfügbaren Diagramme zur Veranschaulichung von CTMM-Konzepten.
\end{quote}

\subsection*{\textcolor{ctmmBlue}{Der CTMM-Trigger-Zyklus}}

Der Kern des CTMM-Systems als visueller Kreislauf:

\begin{center}
\begin{tcolorbox}[colback=ctmmBlue!10!white,colframe=ctmmBlue,width=10cm]
\centering
\textbf{CTMM-Trigger-Zyklus}\\[0.5cm]
\textcolor{ctmmBlue}{\textbf{ERKENNEN}} → \textcolor{ctmmGreen}{\textbf{VORBEUGEN}} → \textcolor{ctmmOrange}{\textbf{REAGIEREN}} → \textcolor{ctmmPurple}{\textbf{LERNEN}}
\end{tcolorbox}
\end{center}

\textit{Dieser Zyklus zeigt die vier Hauptphasen des Trigger-Managements: Erkennen → Vorbeugen → Reagieren → Lernen}

\subsection*{\textcolor{ctmmOrange}{Eskalationsstufen}}

Die vier Ebenen der Krisenintervention:

\begin{center}
\begin{tcolorbox}[colback=ctmmOrange!10!white,colframe=ctmmOrange,width=12cm]
\centering
\textbf{Eskalationsstufen}\\[0.5cm]
\textcolor{ctmmGreen}{\textbf{1. Grün - Stabil}} → \textcolor{ctmmOrange}{\textbf{2. Gelb - Warnung}} → \textcolor{ctmmRed}{\textbf{3. Rot - Krise}} → \textcolor{ctmmPurple}{\textbf{4. Stabilisierung}}
\end{tcolorbox}
\end{center}

\textit{Jede Stufe hat spezifische Strategien und Interventionen. Der Kreislauf zeigt sowohl Eskalation als auch den Weg zurück zur Stabilität.}

\newpage

\subsection*{\textcolor{ctmmRed}{Partner-Dynamiken}}

Systemische Sicht auf die Paarbeziehung:

\begin{center}
\begin{tcolorbox}[colback=ctmmRed!10!white,colframe=ctmmRed,width=12cm]
\centering
\textbf{Partner-Dynamiken}\\[0.5cm]
\textcolor{ctmmBlue}{\textbf{Partner A}} $\leftrightarrow$ \textcolor{ctmmGreen}{\textbf{Ko-Regulation}} $\leftrightarrow$ \textcolor{ctmmBlue}{\textbf{Partner B}}
\end{tcolorbox}
\end{center}

\textit{Das Diagramm zeigt typische Trigger-Reaktions-Muster und wie CTMM-Tools zur Ko-Regulation beitragen.}

\subsection*{\textcolor{ctmmGreen}{Therapieverlauf Timeline}}

\textbf{Beispiel einer 4-Wochen-Behandlung:}

\begin{center}
\begin{tikzpicture}[
    node distance=2cm,
    every node/.style={font=\footnotesize}
]
    % Timeline base
    \draw[thick, ctmmGray] (0,0) -- (10,0);

    % Time markers
    \foreach \x/\label in {0/Start,2.5/Woche 1,5/Woche 2,7.5/Woche 3,10/Monat} {
        \draw (\x,-0.1) -- (\x,0.1);
        \node[below] at (\x,-0.1) {\label};
    }

    % Events
    \node[circle, fill=ctmmBlue!50, draw=ctmmBlue, minimum size=0.5cm] at (1,0.5) {};
    \node[above, text width=2cm, align=center] at (1,1) {\tiny Erstkontakt\\Anamnese};
    \draw[ctmmBlue, thick] (1,0) -- (1,0.5);

    \node[circle, fill=ctmmGreen!50, draw=ctmmGreen, minimum size=0.5cm] at (3,0.5) {};
    \node[above, text width=2cm, align=center] at (3,1) {\tiny Safe-Words\\etabliert};
    \draw[ctmmGreen, thick] (3,0) -- (3,0.5);

    \node[circle, fill=ctmmOrange!50, draw=ctmmOrange, minimum size=0.5cm] at (6,0.5) {};
    \node[above, text width=2cm, align=center] at (6,1) {\tiny Erste Krise\\bewältigt};
    \draw[ctmmOrange, thick] (6,0) -- (6,0.5);

    \node[circle, fill=ctmmPurple!50, draw=ctmmPurple, minimum size=0.5cm] at (8.5,0.5) {};
    \node[above, text width=2cm, align=center] at (8.5,1) {\tiny Reflexion\\Integration};
    \draw[ctmmPurple, thick] (8.5,0) -- (8.5,0.5);
\end{tikzpicture}
\end{center}

\subsection*{\textcolor{ctmmPurple}{Stimmungsverlauf}}

\textbf{Beispiel-Graph für Mood-Tracking:}

\begin{center}
\begin{tikzpicture}[
    scale=0.8,
    every node/.style={font=\tiny}
]
    % Axes
    \draw[->] (0,0) -- (8,0) node[right] {Zeit};
    \draw[->] (0,0) -- (0,6) node[above] {Stimmung};

    % Grid
    \foreach \x in {1,2,...,7} {
        \draw[gray!30] (\x,0) -- (\x,5);
    }
    \foreach \y in {1,2,3,4,5} {
        \draw[gray!30] (0,\y) -- (7,\y);
        \node[left] at (0,\y) {\y};
    }

    % Data line
    \draw[ctmmBlue, thick]
        (0,3) -- (1,2.5) -- (2,1.5) -- (3,2) -- (4,3.5) -- (5,4) -- (6,4.2) -- (7,4.5);
    \node[ctmmBlue, above] at (3.5,3) {\tiny Therapiestart};

    % Zones
    \fill[ctmmRed!10] (0,0) rectangle (7,2);
    \fill[ctmmOrange!10] (0,2) rectangle (7,3.5);
    \fill[ctmmGreen!10] (0,3.5) rectangle (7,5);

    \node[right, font=\tiny] at (7.1,1) {Krise};
    \node[right, font=\tiny] at (7.1,2.75) {Achtung};
    \node[right, font=\tiny] at (7.1,4.25) {Stabil};
\end{tikzpicture}
\end{center}

\subsection*{\textcolor{ctmmBlue}{Verwendung in eigenen Modulen}}

\begin{ctmmBlueBox}[title=So nutzen Sie die Diagramme]
\begin{itemize}
    \item \texttt{\textbackslash ctmmTriggerLoop} - Grundlagen-Psychoedukation
    \item \texttt{\textbackslash ctmmEskalationsStufen} - Krisenplan erklären
    \item \texttt{\textbackslash ctmmPartnerDynamics} - Paartherapie-Sessions
    \item \texttt{\textbackslash ctmmTimeline\{...\}} - Therapieverlauf dokumentieren
    \item \texttt{\textbackslash ctmmMoodGraph\{...\}} - Stimmungsdaten visualisieren
\end{itemize}
\end{ctmmBlueBox}

\subsection*{\textcolor{ctmmBlue}{CTMM-Navigation}}
\begin{itemize}
  \item \texttt{triggermanagement} ← Praktische Anwendung der Zyklen
  \item \texttt{bindungsleitfaden} ← Systemische Grundlagen
\end{itemize}

% =====================================================

\section*{\textcolor{ctmmPurple}{\faChartArea~CTMM Visualisierungen}}

\begin{quote}
\textbf{\textcolor{ctmmPurple}{Worum geht's hier?}}\\
Diese Seite zeigt alle verfügbaren Diagramme zur Veranschaulichung von CTMM-Konzepten.
\end{quote}

\subsection*{\textcolor{ctmmBlue}{Der CTMM-Trigger-Zyklus}}

Der Kern des CTMM-Systems als visueller Kreislauf:

\begin{center}
\begin{tcolorbox}[colback=ctmmBlue!10!white,colframe=ctmmBlue,width=10cm]
\centering
\textbf{CTMM-Trigger-Zyklus}\\[0.5cm]
\textcolor{ctmmBlue}{\textbf{ERKENNEN}} → \textcolor{ctmmGreen}{\textbf{VORBEUGEN}} → \textcolor{ctmmOrange}{\textbf{REAGIEREN}} → \textcolor{ctmmPurple}{\textbf{LERNEN}}
\end{tcolorbox}
\end{center}

\textit{Dieser Zyklus zeigt die vier Hauptphasen des Trigger-Managements: Erkennen → Vorbeugen → Reagieren → Lernen}

\subsection*{\textcolor{ctmmOrange}{Eskalationsstufen}}

Die vier Ebenen der Krisenintervention:

\begin{center}
\begin{tcolorbox}[colback=ctmmOrange!10!white,colframe=ctmmOrange,width=12cm]
\centering
\textbf{Eskalationsstufen}\\[0.5cm]
\textcolor{ctmmGreen}{\textbf{1. Grün - Stabil}} → \textcolor{ctmmOrange}{\textbf{2. Gelb - Warnung}} → \textcolor{ctmmRed}{\textbf{3. Rot - Krise}} → \textcolor{ctmmPurple}{\textbf{4. Stabilisierung}}
\end{tcolorbox}
\end{center}

\textit{Jede Stufe hat spezifische Strategien und Interventionen. Der Kreislauf zeigt sowohl Eskalation als auch den Weg zurück zur Stabilität.}

\newpage

\subsection*{\textcolor{ctmmRed}{Partner-Dynamiken}}

Systemische Sicht auf die Paarbeziehung:

\begin{center}
\begin{tcolorbox}[colback=ctmmRed!10!white,colframe=ctmmRed,width=12cm]
\centering
\textbf{Partner-Dynamiken}\\[0.5cm]
\textcolor{ctmmBlue}{\textbf{Partner A}} $\leftrightarrow$ \textcolor{ctmmGreen}{\textbf{Ko-Regulation}} $\leftrightarrow$ \textcolor{ctmmBlue}{\textbf{Partner B}}
\end{tcolorbox}
\end{center}

\textit{Das Diagramm zeigt typische Trigger-Reaktions-Muster und wie CTMM-Tools zur Ko-Regulation beitragen.}

\subsection*{\textcolor{ctmmGreen}{Therapieverlauf Timeline}}

\textbf{Beispiel einer 4-Wochen-Behandlung:}

\begin{center}
\begin{tikzpicture}[
    node distance=2cm,
    every node/.style={font=\footnotesize}
]
    % Timeline base
    \draw[thick, ctmmGray] (0,0) -- (10,0);

    % Time markers
    \foreach \x/\label in {0/Start,2.5/Woche 1,5/Woche 2,7.5/Woche 3,10/Monat} {
        \draw (\x,-0.1) -- (\x,0.1);
        \node[below] at (\x,-0.1) {\label};
    }

    % Events
    \node[circle, fill=ctmmBlue!50, draw=ctmmBlue, minimum size=0.5cm] at (1,0.5) {};
    \node[above, text width=2cm, align=center] at (1,1) {\tiny Erstkontakt\\Anamnese};
    \draw[ctmmBlue, thick] (1,0) -- (1,0.5);

    \node[circle, fill=ctmmGreen!50, draw=ctmmGreen, minimum size=0.5cm] at (3,0.5) {};
    \node[above, text width=2cm, align=center] at (3,1) {\tiny Safe-Words\\etabliert};
    \draw[ctmmGreen, thick] (3,0) -- (3,0.5);

    \node[circle, fill=ctmmOrange!50, draw=ctmmOrange, minimum size=0.5cm] at (6,0.5) {};
    \node[above, text width=2cm, align=center] at (6,1) {\tiny Erste Krise\\bewältigt};
    \draw[ctmmOrange, thick] (6,0) -- (6,0.5);

    \node[circle, fill=ctmmPurple!50, draw=ctmmPurple, minimum size=0.5cm] at (8.5,0.5) {};
    \node[above, text width=2cm, align=center] at (8.5,1) {\tiny Reflexion\\Integration};
    \draw[ctmmPurple, thick] (8.5,0) -- (8.5,0.5);
\end{tikzpicture}
\end{center}

\subsection*{\textcolor{ctmmPurple}{Stimmungsverlauf}}

\textbf{Beispiel-Graph für Mood-Tracking:}

\begin{center}
\begin{tikzpicture}[
    scale=0.8,
    every node/.style={font=\tiny}
]
    % Axes
    \draw[->] (0,0) -- (8,0) node[right] {Zeit};
    \draw[->] (0,0) -- (0,6) node[above] {Stimmung};

    % Grid
    \foreach \x in {1,2,...,7} {
        \draw[gray!30] (\x,0) -- (\x,5);
    }
    \foreach \y in {1,2,3,4,5} {
        \draw[gray!30] (0,\y) -- (7,\y);
        \node[left] at (0,\y) {\y};
    }

    % Data line
    \draw[ctmmBlue, thick]
        (0,3) -- (1,2.5) -- (2,1.5) -- (3,2) -- (4,3.5) -- (5,4) -- (6,4.2) -- (7,4.5);
    \node[ctmmBlue, above] at (3.5,3) {\tiny Therapiestart};

    % Zones
    \fill[ctmmRed!10] (0,0) rectangle (7,2);
    \fill[ctmmOrange!10] (0,2) rectangle (7,3.5);
    \fill[ctmmGreen!10] (0,3.5) rectangle (7,5);

    \node[right, font=\tiny] at (7.1,1) {Krise};
    \node[right, font=\tiny] at (7.1,2.75) {Achtung};
    \node[right, font=\tiny] at (7.1,4.25) {Stabil};
\end{tikzpicture}
\end{center}

\subsection*{\textcolor{ctmmBlue}{Verwendung in eigenen Modulen}}

\begin{ctmmBlueBox}[title=So nutzen Sie die Diagramme]
\begin{itemize}
    \item \texttt{\textbackslash ctmmTriggerLoop} - Grundlagen-Psychoedukation
    \item \texttt{\textbackslash ctmmEskalationsStufen} - Krisenplan erklären
    \item \texttt{\textbackslash ctmmPartnerDynamics} - Paartherapie-Sessions
    \item \texttt{\textbackslash ctmmTimeline\{...\}} - Therapieverlauf dokumentieren
    \item \texttt{\textbackslash ctmmMoodGraph\{...\}} - Stimmungsdaten visualisieren
\end{itemize}
\end{ctmmBlueBox}

\subsection*{\textcolor{ctmmBlue}{CTMM-Navigation}}
\begin{itemize}
  \item \texttt{triggermanagement} ← Praktische Anwendung der Zyklen
  \item \texttt{bindungsleitfaden} ← Systemische Grundlagen
\end{itemize}
