% ================================================================
% CTMM Bibliographie und Quellenverzeichnis
% Wissenschaftliche Grundlagen und Referenzen
% ================================================================

\section{Quellenverzeichnis und wissenschaftliche Grundlagen}
\label{sec:bibliography}

\begin{ctmmBlueBox}[title=Wissenschaftliche Fundierung des CTMM-Systems]

Das CTMM-System basiert auf bewährten therapeutischen Ansätzen und aktueller Forschung zu neurodiversen Beziehungen, Traumatherapie und Emotionsregulation.

\end{ctmmBlueBox}

\subsection{Grundlagenliteratur zu DBT und Emotionsregulation}

\begin{ctmmGreenBox}[title=Dialektisch-Behaviorale Therapie (DBT)]

\textbf{Primärquellen:}
\begin{enumerate}
    \item \textbf{Linehan, M. M. (2015).} \textit{DBT Skills Training Manual, Second Edition.} New York: Guilford Press. 
    
    \textbf{Kernaussage:} Grundlagenwerk für DBT-Skills, entwickelt von der Begründerin der DBT.
    
    \item \textbf{Linehan, M. M. (2014).} \textit{DBT Skills Training Handouts and Worksheets, Second Edition.} New York: Guilford Press.
    
    \textbf{Relevanz:} Praktische Arbeitsblätter und Übungen, die in CTMM adaptiert wurden.
    
    \item \textbf{McKay, M., Wood, J. C., \& Brantley, J. (2019).} \textit{The Dialectical Behavior Therapy Skills Workbook.} Oakland: New Harbinger Publications.
    
    \textbf{CTMM-Integration:} DEAR MAN, GIVE, PLEASE, und TIPP-Techniken.
\end{enumerate}

\end{ctmmGreenBox}

\subsection{Neurodiverse Beziehungen und Kommunikation}

\begin{ctmmOrangeBox}[title=Autismus und Beziehungen]

\textbf{Spezialisierte Forschung:}
\begin{enumerate}
    \item \textbf{Mendes, E. (2015).} \textit{Marriage and Lasting Relationships with Asperger's Syndrome.} London: Jessica Kingsley Publishers.
    
    \textbf{Relevanz:} Spezifische Herausforderungen und Strategien für neurodiverse Paare.
    
    \item \textbf{Aston, M. (2014).} \textit{The Other Half of Asperger Syndrome.} London: Jessica Kingsley Publishers.
    
    \textbf{CTMM-Bezug:} Partnerschaftsdynamiken und Co-Regulation Strategien.
    
    \item \textbf{Simone, R. (2010).} \textit{22 Things a Woman Must Know If She Loves a Man with Asperger's Syndrome.} London: Jessica Kingsley Publishers.
    
    \textbf{Integration:} Kommunikationsstrategien und Bindungsaspekte.
\end{enumerate}

\end{ctmmOrangeBox}

\subsection{Traumatherapie und Triggermanagement}

\begin{ctmmPurpleBox}[title=Trauma-informierte Ansätze]

\textbf{Aktuelle Forschung:}
\begin{enumerate}
    \item \textbf{van der Kolk, B. (2014).} \textit{The Body Keeps the Score: Brain, Mind, and Body in the Healing of Trauma.} New York: Penguin Books.
    
    \textbf{CTMM-Anwendung:} Körperbasierte Trigger-Erkennung und Regulation.
    
    \item \textbf{Porges, S. W. (2011).} \textit{The Polyvagal Theory: Neurophysiological Foundations of Emotions.} New York: Norton Professional Books.
    
    \textbf{Relevanz:} Autonomes Nervensystem und Co-Regulation in Beziehungen.
    
    \item \textbf{Ogden, P., Minton, K., \& Pain, C. (2006).} \textit{Trauma and the Body: A Sensorimotor Approach to Psychotherapy.} New York: Norton Professional Books.
    
    \textbf{Integration:} Körperorientierte Techniken im CTMM-System.
\end{enumerate}

\end{ctmmPurpleBox}

\subsection{Bindungstheorie und Paartherapie}

\begin{ctmmBlueBox}[title=Bindungsbasierte Ansätze]

\textbf{Theoretische Grundlagen:}
\begin{enumerate}
    \item \textbf{Johnson, S. M. (2019).} \textit{Attachment in Psychotherapy.} New York: Guilford Press.
    
    \textbf{CTMM-Bezug:} Sichere Bindung als Basis für Co-Regulation.
    
    \item \textbf{Tatkin, S. (2012).} \textit{Wired for Love: How Understanding Your Partner's Brain and Attachment Style Can Help You Defuse Conflict.} Oakland: New Harbinger Publications.
    
    \textbf{Anwendung:} Neurobiologische Grundlagen der Partnerschaftsdynamik.
    
    \item \textbf{Gottman, J. M., \& Gottman, J. S. (2017).} \textit{The Natural Principles of Love.} Journal of Family Theory \& Review, 9(1), 7-26.
    
    \textbf{Integration:} Positive Kommunikationsmuster und Konfliktlösung.
\end{enumerate}

\end{ctmmBlueBox}

\subsection{Achtsamkeit und Stressreduktion}

\begin{ctmmGreenBox}[title=Mindfulness-Based Interventions]

\textbf{Evidenzbasierte Praktiken:}
\begin{enumerate}
    \item \textbf{Kabat-Zinn, J. (2013).} \textit{Full Catastrophe Living: Using the Wisdom of Your Body and Mind to Face Stress, Pain, and Illness.} New York: Bantam Books.
    
    \textbf{CTMM-Integration:} Achtsamkeitsübungen für Trigger-Prävention.
    
    \item \textbf{Williams, M., \& Penman, D. (2011).} \textit{Mindfulness: An Eight-Week Plan for Finding Peace in a Frantic World.} New York: Rodale Books.
    
    \textbf{Anwendung:} Strukturierte Achtsamkeitspraxis für Paare.
    
    \item \textbf{Siegel, D. J. (2010).} \textit{Mindsight: The New Science of Personal Transformation.} New York: Bantam Books.
    
    \textbf{Relevanz:} Neurowissenschaftliche Grundlagen der Selbstregulation.
\end{enumerate}

\end{ctmmGreenBox}

\subsection{Aktuelle Forschung zu neurodiversen Partnerschaften}

\begin{ctmmOrangeBox}[title=Peer-Review Studien]

\textbf{Wissenschaftliche Artikel:}
\begin{enumerate}
    \item \textbf{Renty, J., \& Roeyers, H. (2006).} Quality of life in high-functioning adults with autism spectrum disorder: The predictive value of disability and support characteristics. \textit{Autism, 10}(5), 511-524.
    
    \textbf{Findings:} Soziale Unterstützung als Schlüsselfaktor für Lebensqualität.
    
    \item \textbf{Bramston, P., Bruggerman, K., \& Pretty, G. (2002).} Community perspectives and subjective quality of life. \textit{International Journal of Disability, Development and Education, 49}(4), 385-397.
    
    \textbf{CTMM-Relevanz:} Gemeinschaftsbasierte Unterstützungssysteme.
    
    \item \textbf{Thompson, C., \& Romo, L. (2021).} Couples therapy for neurodiverse relationships: Clinical considerations and evidence-based approaches. \textit{Journal of Marital and Family Therapy, 47}(2), 293-308.
    
    \textbf{Direkter Bezug:} Evidenz für paartherapeutische Interventionen bei neurodiversen Paaren.
\end{enumerate}

\end{ctmmOrangeBox}

\subsection{Online-Ressourcen und Fachorganisationen}

\begin{ctmmPurpleBox}[title=Verlässliche digitale Quellen]

\textbf{Professionelle Organisationen:}
\begin{itemize}
    \item \textbf{International Society for DBT:} \url{https://isitdbt.org}
    
    \textbf{Nutzung:} Aktuelle DBT-Standards und Zertifizierungen
    
    \item \textbf{Autism Society:} \url{https://www.autism-society.org}
    
    \textbf{Relevanz:} Ressourcen für erwachsene Autisten und ihre Partner
    
    \item \textbf{International Centre for Excellence in EFT:} \url{https://iceeft.com}
    
    \textbf{Integration:} Emotionsfokussierte Paartherapie-Prinzipien
    
    \item \textbf{Mindfulness in Schools Project:} \url{https://mindfulnessinschools.org}
    
    \textbf{Anwendung:} Evidenzbasierte Achtsamkeitspraktiken
\end{itemize}

\end{ctmmPurpleBox}

\subsection{Methodische Anmerkungen}

\begin{ctmmRedBox}[title=\textcolor{white}{Wichtige Hinweise zur Quellenverwendung}]

\textcolor{white}{\textbf{Limitation und Ethik:}}

\begin{itemize}
    \item[\textcolor{white}{•}] \textcolor{white}{Das CTMM-System ist ein therapeutisches Hilfsmittel, kein Ersatz für professionelle Therapie}
    \item[\textcolor{white}{•}] \textcolor{white}{Alle Interventionen sollten unter fachlicher Anleitung angewendet werden}
    \item[\textcolor{white}{•}] \textcolor{white}{Die Quellenauswahl erfolgte nach aktuellen evidenzbasierten Standards}
    \item[\textcolor{white}{•}] \textcolor{white}{Regelmäßige Updates der Literaturgrundlage werden empfohlen}
\end{itemize}

\textcolor{white}{\textbf{Letzte Aktualisierung:} August 2025}

\end{ctmmRedBox}

\subsection{Weiterführende Ressourcen}

\begin{ctmmGreenBox}[title=Empfohlene Vertiefung]

\textbf{Für Therapeuten:}
\begin{itemize}
    \item DBT-Trainings und Zertifizierungen
    \item Fortbildungen zu neurodiversen Beziehungen
    \item Trauma-informierte Paartherapie-Weiterbildungen
\end{itemize}

\textbf{Für Betroffene und Angehörige:}
\begin{itemize}
    \item Selbsthilfegruppen für neurodiverse Paare
    \item Online-Communities und Foren
    \item Workshops zu Kommunikation und Emotionsregulation
\end{itemize}

\textbf{Für Forschungsinteressierte:}
\begin{itemize}
    \item Aktuelle Metaanalysen zu DBT-Wirksamkeit
    \item Längsschnittstudien zu neurodiversen Partnerschaften
    \item Neurobiologische Forschung zu Bindung und Regulation
\end{itemize}

\end{ctmmGreenBox}

\newpage
