% =====================================================
% CTMM Matching-Matrix: Trigger-Reaktion-Intervention
% Purpose: Verstehen typischer Reiz-Reaktionsmuster in Beziehungen
% Context: Neurodiverse Paare - Mustererkennung und Intervention
% Author: CTMM-Team
% Copilot: Use for pattern recognition, couple dynamics, intervention planning
% =====================================================

\section*{\textcolor{ctmmPurple}{\faPuzzlePiece~Matching-Matrix: Trigger -- Reaktion -- Intervention}}

\begin{quote}
\textbf{\textcolor{ctmmPurple}{Worum geht's hier -- für Freunde?}}\\
Dieses Modul hilft, typische Reiz-Reaktionsmuster in unserer Beziehung zu verstehen. Es ist wie ein Übersetzungsblatt -- was passiert in mir, in dir, und wie können wir hilfreich reagieren?
\end{quote}

\textit{Dynamisches Tool zur Reflexion und Alltagssteuerung -- ergänzt das Trigger-Tagebuch \& die Ko-Regulation}

\subsection*{\textcolor{ctmmPurple}{Kapitelzuordnung im CTMM-System}}

\begin{itemize}
  \item \texttt{Kap. 1} → Grundlogik der Bindungsdynamik (Auslöser, Reaktion, Integration)
  \item \texttt{Kap. 2.6} → Team-Regeln, Ko-Regulation, Nähe/Distanz
  \item \texttt{Kap. 3.1 – 3.5} → Eskalationssicherung, Rückzug, Intervention
  \item \texttt{Kap. 5.2} → Trigger-Tagebuch, Matching-Auswertung, Reaktionslogik
\end{itemize}

\subsection*{\textcolor{ctmmPurple}{Farbcode \& Systemnavigation}}

\begin{center}
\begin{tabular}{|c|l|l|}
\hline
\textbf{Farbe} & \textbf{Phase} & \textbf{Modulkontext} \\
\hline
\textcolor{ctmmBlue}{$\bullet$} & Mustererkennung & \texttt{bindungsdynamik}, \texttt{Kap. 1} \\
\textcolor{ctmmOrange}{$\bullet$} & Reaktion \& Intervention & \texttt{tool\_26\_co\_regulation} \\
\textcolor{ctmmRed}{$\bullet$} & Eskalation / Rückzug & \texttt{trigger\_notfallkarten} \\
\hline
\end{tabular}
\end{center}

\subsection*{\textcolor{ctmmBlue}{Matching-Tabelle: Typische Szenarien}}

\begin{center}
\begin{tabular}{|p{3.5cm}|p{3cm}|p{3cm}|p{4.5cm}|}
\hline
\textbf{Trigger (Beispiel)} & \textbf{Reaktion ER} & \textbf{Reaktion SIE} & \textbf{Was hilft konkret (Aktion/Tool)} \\
\hline
\textbf{Vorwurf / Kritik} & Rückzug, Erstarren & Lautwerden, Rechtfertigung & 🔁 Ritualsatz, Ko-Regulation, Safe-Word \\
\hline
\textbf{Näheforderung ohne Warnung} & Reizüberflutung, Freeze & Impulsives Klammern & 🟠 Ankündigung, Rückzugsregel, Timer \\
\hline
\textbf{Kontrollverlust} & Frust, Bewegung, Fluchtimpuls & Angst, Überforderung & 🔴 Strukturzettel, Rückkehrritual, Buddy-Call \\
\hline
\textbf{Unklarheit, was los ist} & Grübeln, Vermeidung & Drängen, wiederholen & 🧠 Matching-Check-In, CTMM-Visualisierung \\
\hline
\end{tabular}
\end{center}

\subsection*{\textcolor{ctmmOrange}{Eigene Beobachtungen}}

\begin{center}
\begin{tabular}{|p{3.5cm}|p{3cm}|p{3cm}|p{4.5cm}|}
\hline
\textbf{Trigger (Kontext)} & \textbf{Reaktion ER} & \textbf{Reaktion SIE} & \textbf{Guter Umgang / Was hilft} \\
\hline
\ctmmTextArea[3.2cm]{2}{trigger_1} & \ctmmTextArea[2.7cm]{2}{reaktion_er_1} & \ctmmTextArea[2.7cm]{2}{reaktion_sie_1} & \ctmmTextArea[4.2cm]{2}{hilft_1} \\
\hline
\ctmmTextArea[3.2cm]{2}{trigger_2} & \ctmmTextArea[2.7cm]{2}{reaktion_er_2} & \ctmmTextArea[2.7cm]{2}{reaktion_sie_2} & \ctmmTextArea[4.2cm]{2}{hilft_2} \\
\hline
\ctmmTextArea[3.2cm]{2}{trigger_3} & \ctmmTextArea[2.7cm]{2}{reaktion_er_3} & \ctmmTextArea[2.7cm]{2}{reaktion_sie_3} & \ctmmTextArea[4.2cm]{2}{hilft_3} \\
\hline
\ctmmTextArea[3.2cm]{2}{trigger_4} & \ctmmTextArea[2.7cm]{2}{reaktion_er_4} & \ctmmTextArea[2.7cm]{2}{reaktion_sie_4} & \ctmmTextArea[4.2cm]{2}{hilft_4} \\
\hline
\end{tabular}
\end{center}

\subsection*{\textcolor{ctmmGreen}{Wochenauswertung -- Reflexion \& Lernfelder}}

\begin{tcolorbox}[colback=ctmmGreen!5!white,colframe=ctmmGreen,title=Welche Muster wiederholen sich? Was willst du nächste Woche ausprobieren?]

\textbf{Häufige Trigger / Wiederholungen:}\\
\ctmmTextArea[12cm]{3}{haeufige_trigger}

\textbf{Welche Strategien haben geholfen:}\\
\ctmmTextArea[12cm]{3}{hilfreiche_strategien}

\textbf{Was willst du beim nächsten Mal anders machen:}\\
\ctmmTextArea[12cm]{3}{naechstes_mal}

\end{tcolorbox}

\subsection*{\textcolor{ctmmRed}{Eskalations-Matching}}

\textbf{Wann eskalieren wir besonders schnell:}\\
\textbf{Tageszeit:} \ctmmTextField[3cm]{}{eskalZeit} \quad \textbf{Situation:} \ctmmTextField[6cm]{}{eskal_situation}\\
\textbf{Tageszeit:} \ctmmTextField[3cm]{}{eskalZeit} \quad \textbf{Situation:} \ctmmTextField[6cm]{}{eskalSituation}\\
\textbf{Vorwarnung erkennbar:} \ctmmCheckBox{eskalWarnung}{Ja} \quad \textbf{Wie:} \ctmmTextField[6cm]{}{eskalWie}

\textbf{Erfolgreichste Deeskalation:}\\
\ctmmTextArea[12cm]{2}{beste_deeskalation}

\textbf{Was funktioniert nicht:}\\
\ctmmTextArea[12cm]{2}{funktioniert_nicht}

\subsection*{\textcolor{ctmmBlue}{CTMM-Navigation}}

\begin{itemize}
  \item \texttt{trigger\_forschungstagebuch} ← individuelle Triggerdokumentation (\texttt{Kap. 5.2})
  \item \texttt{tool\_23\_triggermanagement} ← Auslöser-Logik, Eskalationszyklus (\texttt{Kap. 5.3})
  \item \texttt{tool\_26\_co\_regulation} ← Partnerintervention, Haltung in Echtzeit (\texttt{Kap. 2.6})
  \item \texttt{ritual\_workbook} ← Rückkehrrituale nach Matching-Konflikt (\texttt{Kap. 3.4})
\end{itemize}

\begin{tcolorbox}[colback=ctmmPurple!5!white,colframe=ctmmPurple,title=Verwendung]
📎 \textbf{Diese Matching-Matrix ist besonders nützlich bei:}
\begin{itemize}
  \item Paartherapie-Sitzungen
  \item Reflexion am Abend
  \item Buddy-Supervision
  \item Vorbereitung auf schwierige Situationen
\end{itemize}
\end{tcolorbox}

\subsection*{\textcolor{ctmmOrange}{Matching-Erfolg messen}}

\textbf{Diesen Monat haben wir besser gehandelt bei:}\\
\ctmmTextArea[12cm]{2}{besser_gehandelt}

\textbf{Das müssen wir noch üben:}\\
\ctmmTextArea[12cm]{2}{noch_ueben}

\textbf{Unser Matching wird:} \ctmmCheckBox{besser} \ctmmCheckBox{gleich} \ctmmCheckBox{braucht Hilfe}
