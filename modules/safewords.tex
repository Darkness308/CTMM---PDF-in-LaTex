\section*{\textcolor{ctmmRed}{\faStop~Safe-Words \& Signalsysteme}}
\label{sec:safewords}
\addcontentsline{toc}{section}{Safe-Words \& Signalsysteme}

\begin{quote}
\textbf{\textcolor{ctmmRed}{Worum geht's hier -- für Freunde?}}\\
Safe-Words sind vereinbarte Codes oder Zeichen, die sofort signalisieren: \glqq Ich kann nicht mehr\grqq{}, \glqq Ich brauch Ruhe\grqq{} oder \glqq Stopp -- das wird mir zu viel\grqq{}. Sie schützen vor Eskalation, Überforderung, Rückzug oder Missverständnissen -- ohne viele Worte.
\end{quote}

\textit{Zentraler Bestandteil der Eskalationsprävention -- mit Symbol- und Notfallsystem}

\subsection*{\textcolor{ctmmRed}{Safe-Words (Beispiele + Eigene)}}

\begin{ctmmRedBox}{Bewährte Safe-Word Beispiele}
\begin{center}
\begin{tabular}{|p{3cm}|p{5cm}|p{5cm}|}
\hline
\textbf{Safe-Word / Geste} & \textbf{Bedeutung / Wirkung} & \textbf{Wann einsetzen?} \\
\hline
\glqq Orange\grqq{} & Warnstufe -- ich werde gleich überfordert & Bei Stress, lautem Ton, innerem Rückzug \\
\hline
\glqq Kristall\grqq{} & Stopp -- bitte sofort aufhören & Bei Eskalation, Überforderung, Trigger \\
\hline
Handzeichen (offen) & Ich will reden, aber schaff's nicht & Bei Freeze, Erstarrung, Nonverbales \\
\hline
\glqq Lagerfeuer\grqq{} & 5min in den Arm nehmen, ohne zu reden & Wenn ich Nähe brauche (Angst, Frieden) \\
\hline
\end{tabular}
\end{center}
\end{ctmmRedBox}

\textbf{Eigene Safe-Words:}\\
\ctmmTextField{4cm}{Safe-Word 1:} \ctmmTextField{6cm}{Bedeutung:}\\
\ctmmTextField{4cm}{Safe-Word 2:} \ctmmTextField{6cm}{Bedeutung:}\\
\ctmmTextField{4cm}{Safe-Word 3:} \ctmmTextField{6cm}{Bedeutung:}

\subsection*{\textcolor{ctmmOrange}{Signalsysteme zur Unterstützung}}

\begin{ctmmOrangeBox}{Leise Zeichen für schwere Momente}
\begin{itemize}
  \item \textbf{Symbolischer Gegenstand} (z. B. Stofftier, Stein, Karte)
  \item \textbf{Tagesanzeiger} (Magnet, Schild, Farbe auf Tür)
  \item \textbf{Lautstärke-Code} (Musikart, Kopfhörer sichtbar = \glqq Bitte in Ruhe lassen\grqq{})
\end{itemize}

\textbf{Diese Zeichen können leise, sichtbar, intuitiv sein -- auch bei Disso oder Sprachverlust}
\end{ctmmOrangeBox}

\vspace{1cm}
\begin{center}
\textit{\textcolor{ctmmGreen}{\faChevronRight~Weiter zu} \ctmmRef{sec:interactive}{Interaktive Demo} | \textcolor{ctmmRed}{\faExclamationTriangle~Notfall zu} \ctmmRef{sec:notfallkarten}{Notfallkarten}}
\end{center}
