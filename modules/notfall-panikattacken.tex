% Notfallkarte für Panikattacken
% CTMM Notfallkarte - Automatisch generiert

\section{Notfallkarte für Panikattacken}

\begin{ctmmRedBox}{Notfallkarte}
    \textbf{Notfallkarte für Panikattacken} \\
    Diese Karte hilft Ihnen, durch eine Panikattacke zu navigieren und schnell zur Ruhe zu finden.
\end{ctmmRedBox}

\subsection{Sofortmaßnahmen}

\begin{enumerate}
    \setlength{\itemsep}{0.5em}
    \item \textbf{Langsam und tief atmen: 4 Sekunden ein, 6 Sekunden aus}
    \item \textbf{5-4-3-2-1 Grounding anwenden: 5 sehen, 4 hören, 3 fühlen, 2 riechen, 1 schmecken}
    \item \textbf{Sich selbst beruhigen: "Das ist eine Panikattacke, sie wird vorübergehen"}
    \item \textbf{Sichere Position einnehmen: Hinsetzen oder anlehnen}
    \item \textbf{Bei Bedarf Hilfe rufen: Vertrauensperson oder Notfallnummer kontaktieren}
\end{enumerate}

\subsection{Wichtige Kontakte}

\begin{tabular}{p{4cm}p{8cm}}
    \textbf{Notfallnummer:} & \ctmmTextField[6cm]{}{emergency_contact} \\[0.5cm]
    \textbf{Therapeut:} & \ctmmTextField[6cm]{}{therapist} \\[0.5cm]
    \textbf{Vertrauensperson:} & \ctmmTextField[6cm]{}{trusted_person} \\
\end{tabular}

\subsection{Persönliche Notizen}

\ctmmTextArea[15cm]{3}{Was hilft mir in der Situation}{personal_notes}

\vfill
\begin{center}
    \textcolor{ctmmRed}{\small \textbf{NOTFALLKARTE} - Generiert mit CTMM Module Generator}
\end{center}
