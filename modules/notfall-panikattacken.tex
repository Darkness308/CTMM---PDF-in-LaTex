% notfall-panikattacken.tex - CTMM Notfallkarte
% Generated by CTMM Module Generator
% Emergency intervention card for crisis situations in neurodiverse couples

\section{Notfallplan bei Panikattacken}
\label{sec:notfall-panikattacken}

\begin{ctmmRedBox}{NOTFALL - Sofortige Intervention}
Sofortige Hilfe und Protokoll für Panikattacken in der Partnerschaft
\end{ctmmRedBox}

\subsection{Wann diese Karte verwenden?}
\begin{itemize}
    \item \textcolor{ctmmRed}{\textbf{Akute Panikattacken}}
    \item \textcolor{ctmmRed}{\textbf{Schwere Trigger-Reaktionen}}
    \item \textcolor{ctmmRed}{\textbf{Emotionale Überflutung}}
    \item \textcolor{ctmmRed}{\textbf{Selbst- oder Fremdgefährdung}}
\end{itemize}

\subsection{SOFORT-PROTOKOLL (0-2 Minuten)}

\begin{center}
\begin{tcolorbox}[colback=ctmmRed!10, colframe=ctmmRed, title=\textbf{PHASE 1: SICHERHEIT}]
\Large
\textbf{1. SICHERHEIT PRÜFEN} \\
\textbf{2. RUHE BEWAHREN} \\
\textbf{3. KONTAKT HALTEN} \\
\textbf{4. HILFE HOLEN (bei Bedarf)}
\end{tcolorbox}
\end{center}

\subsection{Detaillierte Schritte}

\begin{ctmmRedBox}{Schritt 1: Sicherheit (15 Sekunden)}
\begin{itemize}
    \item Ist die Person bei Bewusstsein?
    \item Besteht Selbst- oder Fremdgefährdung?
    \item Ist ein sicherer Ort erreichbar?
\end{itemize}
\textbf{Bei Gefahr: Sofort Notruf 112!}
\end{ctmmRedBox}

\begin{ctmmOrangeBox}{Schritt 2: Beruhigung (1-2 Minuten)}
\begin{itemize}
    \item Sprechen Sie ruhig und langsam
    \item Verwenden Sie vertraute Worte oder Safewords
    \item Bieten Sie körperliche Nähe an (nur wenn erwünscht)
    \item Atmen Sie bewusst und hörbar
\end{itemize}
\end{ctmmOrangeBox}

\begin{ctmmBlueBox}{Schritt 3: Grounding (2-5 Minuten)}
\textbf{5-4-3-2-1 Technik:}
\begin{itemize}
    \item \textbf{5 Dinge} die Sie sehen können
    \item \textbf{4 Dinge} die Sie hören können  
    \item \textbf{3 Dinge} die Sie fühlen können
    \item \textbf{2 Dinge} die Sie riechen können
    \item \textbf{1 Ding} das Sie schmecken können
\end{itemize}
\end{ctmmBlueBox}

\subsection{Notfall-Kontakte}

\begin{center}
\begin{tcolorbox}[colback=ctmmYellow!20, colframe=ctmmRed, title=\textbf{WICHTIGE TELEFONNUMMERN}]
\textbf{Notruf:} 112 \\
\textbf{Therapeut*in:} \ctmmTextField[6cm]{Telefon}{therapeut_tel} \\
\textbf{Vertrauensperson:} \ctmmTextField[6cm]{Telefon}{vertrauen_tel} \\
\textbf{Krisentelefon:} 0800 111 0 111 (kostenfrei)
\end{tcolorbox}
\end{center}

\subsection{Nach der Krise}

\begin{ctmmGreenBox}{Nachbereitung (wenn Ruhe eingekehrt ist)}
\begin{enumerate}
    \item Sorgen Sie für Grundbedürfnisse (Wasser, warme Decke, etc.)
    \item Sprechen Sie nur, wenn die Person bereit ist
    \item Planen Sie den nächsten sicheren Schritt
    \item Vereinbaren Sie zeitnahen therapeutischen Termin
    \item Dokumentieren Sie den Vorfall (optional)
\end{enumerate}
\end{ctmmGreenBox}

\subsection{Präventive Maßnahmen}

\begin{ctmmPurpleBox}{Langfristige Vorbereitung}
\textbf{Besprechen Sie mit Ihrer therapeutischen Fachkraft:}
\begin{itemize}
    \item Individuelle Trigger und Warnsignale
    \item Persönliche Beruhigungsstrategien
    \item Safewords und Kommunikationssignale
    \item Medikamentöse Notfalloptionen
\end{itemize}
\end{ctmmPurpleBox}

\subsection{Persönliche Notizen}

\textbf{Individuelle Trigger:}
\ctmmTextArea[12cm]{2}{Trigger}{trigger}

\textbf{Bewährte Beruhigungsstrategien:}
\ctmmTextArea[12cm]{2}{Strategien}{strategien}

\textbf{Safewords:}
\ctmmTextField[8cm]{Safewords}{safewords}

% Notfall-Referenz
\vspace{1cm}
\begin{center}
\textcolor{ctmmRed}{\textbf{\faExclamationTriangle{} Im Zweifel: Lieber einmal zu oft Hilfe holen!}} \\
\faCompass{} \textcolor{ctmmBlue}{\textbf{Weitere Hilfe:}} 
\textcolor{ctmmGray}{Safewords | Triggermanagement | Therapeutische Unterstützung}
\end{center}