% =====================================================
% CTMM Panikattacken Notfallkarte
% Purpose: Schnelle Hilfe in Krisensituationen
% Context: Eskalationsmanagement und Sicherheit
% Author: CTMM-Team
% Integration: % CTMM Notfallkarte: Panikattacken-Protokoll
% Generiert am: 19.08.2024
% Emergency Protocol für CTMM Crisis Intervention
% 
% Diese Notfallkarte bietet sofortige Unterstützung in Krisensituationen
% und folgt den CTMM-Prinzipien für neurodiverse Paare.

\section{Notfallkarte: Panikattacken}

\begin{ctmmRedBox}{\faExclamationTriangle\space Notfall-Information}
\textbf{Krisentyp:} Akute Panikattacke - intensive Angst mit körperlichen Symptomen

\textbf{Dringlichkeit:} Hoch - sofortige Intervention erforderlich

\textbf{Zielgruppe:} Betroffene und unterstützende Partner
\end{ctmmRedBox}

\subsection{Sofortige Maßnahmen}

\begin{ctmmYellowBox}{\faHandStopO\space STOPP - Erste Hilfe}
\textbf{1. STOPP:} Stoppe alle Aktivitäten. Setze oder lege dich hin. Du bist sicher.

\textbf{2. ATMEN:} 4 Sekunden einatmen, 4 Sekunden halten, 6 Sekunden ausatmen. Wiederholen.

\textbf{3. SICHERHEIT:} Erinnere dich: "Das ist eine Panikattacke. Sie ist vorübergehend. Ich bin nicht in Gefahr."
\end{ctmmYellowBox}

\subsection{5-Schritt Notfall-Protokoll}

\begin{enumerate}
\item \textbf{CATCH:} Erkenne die Panikattacke frühzeitig: Herzrasen, Schwitzen, Atemnot, Schwindel, Übelkeit, Angst vor Kontrollverlust.

\item \textbf{TRACK:} Dokumentiere mental oder später schriftlich: Wo bin ich? Was ist passiert? Welche Symptome treten auf?

\item \textbf{MAP:} Verstehe den Zusammenhang: Ist dies ein bekannter Trigger? Welche Stressoren waren heute präsent?

\item \textbf{MATCH:} Wähle passende Bewältigungsstrategie: 5-4-3-2-1 Grounding, Atemtechnik, sicherer Ort, Unterstützung rufen.

\item \textbf{FOLLOW-UP:} Nach der Attacke: Ruhe, Selbstfürsorge, Partner informieren, bei Bedarf professionelle Hilfe kontaktieren.
\end{enumerate}

\subsection{Beruhigungstechniken}

\begin{itemize}
\item \textbf{Bauchatmung:} Hand auf Brust, Hand auf Bauch. Nur die untere Hand soll sich bewegen.
\item \textbf{5-4-3-2-1 Grounding:} 5 Dinge sehen, 4 hören, 3 berühren, 2 riechen, 1 schmecken.
\item \textbf{Positive Affirmationen:} "Das geht vorüber", "Ich bin sicher", "Ich habe das schon geschafft".
\item \textbf{Körperliche Techniken:} Kaltes Wasser ins Gesicht, Eiswürfel halten, feste Umarmung.
\item \textbf{Ablenkung:} Rückwärts von 100 in 7er-Schritten zählen, Gegenstände im Raum benennen.
\end{itemize}

\subsection{Kontakt-Informationen}

\begin{ctmmBlueBox}{Wichtige Kontakte}
\textbf{Therapeut/in:} \ctmmTextField[6cm]{Therapeut}{therapeut}

\textbf{Partner/Vertrauensperson:} \ctmmTextField[6cm]{Vertrauensperson}{partner}

\textbf{Krisentelefon:} \ctmmTextField[6cm]{0800 111 0 111 (kostenlos)}{krisentelefon}

\textbf{Notfall (112):} \textcolor{ctmmRed}{\textbf{112}} - bei Herzinfarkt-Verdacht!
\end{ctmmBlueBox}

\subsection{Für Partner und Unterstützende}

\begin{ctmmPurpleBox}{Partner-Hilfe}
\textbf{DO:}
\begin{itemize}
\item Ruhig und unterstützend bleiben
\item Fragen: "Wie kann ich dir helfen?"
\item Bei Atemtechniken anleiten
\item Physischen Kontakt anbieten (wenn gewünscht)
\item Im Hier und Jetzt halten: "Du bist bei mir, du bist sicher"
\end{itemize}

\textbf{DON'T:}
\begin{itemize}
\item Nicht sagen: "Beruhige dich" oder "Das ist nicht schlimm"
\item Nicht minimalisieren oder rationalisieren
\item Nicht ungeduldig werden
\item Nicht eigene Angst zeigen
\end{itemize}
\end{ctmmPurpleBox}

\subsection{Nachsorge-Checkliste}

\begin{ctmmGreenBox}{Nach der Panikattacke}
\ctmmCheckBox[nachsorge1]{Attacke ist vorüber - körperliche Sicherheit bestätigt}

\ctmmCheckBox[nachsorge2]{Partner/Vertrauensperson kontaktiert und informiert}

\ctmmCheckBox[nachsorge3]{Ausreichend Ruhe und Erholung eingeplant}

\ctmmCheckBox[nachsorge4]{Trigger-Analyse für Therapie dokumentiert}

\ctmmCheckBox[nachsorge5]{Bei häufigen Attacken: Therapeut/in kontaktiert}

\textbf{Erkenntnisse aus dieser Attacke:} \ctmmTextField[\textwidth]{Was hat geholfen/nicht geholfen?}{erkenntnisse}

\textbf{Nächster Therapie-Termin:} \ctmmTextField[4cm]{Datum}{naechster_termin}
\end{ctmmGreenBox}

\textcolor{ctmmRed}{\small \textbf{Wichtig:} Bei Verdacht auf Herzinfarkt (Brustschmerzen, Atemnot, Übelkeit, Schmerzen im linken Arm) sofort 112 wählen!}

\textcolor{ctmmGray}{\small Diese Notfallkarte ist Teil der CTMM-Therapiematerialien und folgt der Catch-Track-Map-Match Methodik für neurodiverse Paare.}

% =====================================================

\section*{\textcolor{ctmmRed}{\faExclamationTriangle~Panikattacken -- Notfallkarte}}

\begin{ctmmRedBox}[title=\textcolor{white}{\textbf{SOFORTMASSNAHMEN}}]
\textcolor{white}{\textbf{1. STOPP:}} \textcolor{white}{Reizunterbrechung, Ort wechseln, tief durchatmen}\\
\textcolor{white}{\textbf{2. SIGNAL:}} \textcolor{white}{Safe-Word verwenden oder vereinbartes Zeichen}\\
\textcolor{white}{\textbf{3. HILFE:}} \textcolor{white}{Support-Person kontaktieren oder Notfallkontakt}\\
\textcolor{white}{\textbf{4. SICHERHEIT:}} \textcolor{white}{Sicheren Rückzugsort aufsuchen}
\end{ctmmRedBox}

\subsection*{\textcolor{ctmmOrange}{Safe-Words & Signale}}
\begin{ctmmOrangeBox}[title=Notfall-Kommunikation]
\begin{itemize}
  \item \textbf{"ANKER"} → Sofortiger Stopp aller Aktivitäten, Beruhigung nötig
  \item \textbf{"RESET"} → Neustart des Gesprächs ohne Vorwürfe oder Schuldzuweisungen
  \item \textbf{Körpersprache:} Hand auf Herz = "Ich brauche eine Pause"
\end{itemize}
\end{ctmmOrangeBox}

\subsection*{\textcolor{ctmmGreen}{Nach der Krise}}
\begin{ctmmGreenBox}[title=Nachsorge & Reflexion]
\begin{enumerate}
  \item \textbf{Beruhigung:} 10 Minuten Ruhe, bewusst atmen
  \item \textbf{Gespräch:} Ruhiges Gespräch über Auslöser und Gefühle
  \item \textbf{Lernen:} Was können wir beim nächsten Mal anders machen?
\end{enumerate}
\end{ctmmGreenBox}

\subsection*{\textcolor{ctmmBlue}{CTMM-Navigation}}
\begin{itemize}
  \item \texttt{safewords} ← Grundlagen der Kommunikation
  \item \texttt{triggermanagement} ← Präventive Maßnahmen
\end{itemize}