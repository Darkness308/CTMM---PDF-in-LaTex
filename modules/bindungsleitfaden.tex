% CTMM Bindungsleitfaden Modul
\section{Bindungsleitfaden}
\begin{tcolorbox}[colback=ctmmBlue!5!white,colframe=ctmmBlue,title=Was ist Bindung?]
Bindung beschreibt das emotionale Band zwischen Menschen, das Sicherheit, Vertrauen und Nähe ermöglicht. Im CTMM-Kontext ist Bindung die Basis für Entwicklung und Veränderung.
\end{tcolorbox}

\subsection{Bindungstypen}
\begin{itemize}
  \item \textbf{Sichere Bindung}: Vertrauen, Nähe, Unterstützung
  \item \textbf{Unsichere Bindung}: Angst, Unsicherheit, Rückzug
  \item \textbf{Ambivalente Bindung}: Schwankend zwischen Nähe und Distanz
  \item \textbf{Desorganisierte Bindung}: Widersprüchliches Verhalten
\end{itemize}

\subsection{Bindung im Alltag}
\begin{tcolorbox}[colback=ctmmGreen!5!white,colframe=ctmmGreen,title=Bindung stärken]
\begin{itemize}
  \item Verlässlichkeit zeigen
  \item Zuhören und Verständnis signalisieren
  \item Gemeinsame Rituale pflegen
  \item Gefühle benennen und annehmen
\end{itemize}
\end{tcolorbox}

\subsection{Bindung und CTMM}
Bindung ist die Grundlage für die Arbeit mit Triggern, Mustern und Veränderungen im CTMM-System. Ein sicherer Bindungsrahmen erleichtert die Anwendung der Tools und fördert nachhaltige Entwicklung.
