% =====================================================
% CTMM Diagrams Demo Module
% Purpose: Demonstrate visual representation capabilities
% Author: CTMM-Team
% Integration: % =====================================================
% CTMM Diagrams Demo Module
% Purpose: Demonstrate visual representation capabilities
% Author: CTMM-Team
% Integration: % =====================================================
% CTMM Diagrams Demo Module
% Purpose: Demonstrate visual representation capabilities
% Author: CTMM-Team
% Integration: % =====================================================
% CTMM Diagrams Demo Module
% Purpose: Demonstrate visual representation capabilities
% Author: CTMM-Team
% Integration: \input{modules/diagrams-demo-fixed}
% =====================================================

\section*{\texorpdfstring{\textcolor{ctmmBlue}{\faPalette~Diagramme \& Visualisierungen}}{Diagramme \& Visualisierungen}}

\begin{quote}
\textbf{\textcolor{ctmmBlue}{Worum geht's hier?}}\\
Dieses Modul zeigt die visuellen Möglichkeiten des CTMM-Systems für therapeutische Darstellungen und Diagramme.
\end{quote}

\subsection*{\texorpdfstring{\textcolor{ctmmBlue}{Der CTMM-Trigger-Zyklus}}{Der CTMM-Trigger-Zyklus}}

Der Kern des CTMM-Systems als visueller Kreislauf:

\begin{center}
\begin{tcolorbox}[colback=ctmmBlue!10!white,colframe=ctmmBlue,width=10cm]
\centering
\textbf{CTMM-Trigger-Zyklus}\\[0.5cm]
\textcolor{ctmmBlue}{\textbf{ERKENNEN}} $\rightarrow$ \textcolor{ctmmGreen}{\textbf{VORBEUGEN}} $\rightarrow$ \textcolor{ctmmOrange}{\textbf{REAGIEREN}} $\rightarrow$ \textcolor{ctmmPurple}{\textbf{LERNEN}}
\end{tcolorbox}
\end{center}

\textit{Dieser Zyklus zeigt die vier Hauptphasen des Trigger-Managements: Erkennen $\rightarrow$ Vorbeugen $\rightarrow$ Reagieren $\rightarrow$ Lernen}

\subsection*{\textcolor{ctmmOrange}{Eskalationsstufen}}

Die vier Ebenen der Krisenintervention:

\begin{center}
\begin{tcolorbox}[colback=ctmmOrange!10!white,colframe=ctmmOrange,width=12cm]
\centering
\textbf{Eskalationsstufen}\\[0.5cm]
\textcolor{ctmmGreen}{\textbf{1. Grün - Stabil}} $\rightarrow$ \textcolor{ctmmOrange}{\textbf{2. Gelb - Warnung}} $\rightarrow$ \textcolor{ctmmRed}{\textbf{3. Rot - Krise}} $\rightarrow$ \textcolor{ctmmPurple}{\textbf{4. Stabilisierung}}
\end{tcolorbox}
\end{center}

\textit{Jede Stufe hat spezifische Strategien und Interventionen. Der Kreislauf zeigt sowohl Eskalation als auch den Weg zurück zur Stabilität.}

\newpage

\subsection*{\textcolor{ctmmRed}{Partner-Dynamiken}}

Systemische Sicht auf die Paarbeziehung:

\begin{center}
\begin{tcolorbox}[colback=ctmmRed!10!white,colframe=ctmmRed,width=12cm]
\centering
\textbf{Partner-Dynamiken}\\[0.5cm]
\textcolor{ctmmBlue}{\textbf{Partner A}} $\leftrightarrow$ \textcolor{ctmmGreen}{\textbf{Ko-Regulation}} $\leftrightarrow$ \textcolor{ctmmBlue}{\textbf{Partner B}}
\end{tcolorbox}
\end{center}

\textit{Das Diagramm zeigt typische Trigger-Reaktions-Muster und wie CTMM-Tools zur Ko-Regulation beitragen.}

\subsection*{\textcolor{ctmmGreen}{Weitere Diagramm-Möglichkeiten}}

\begin{ctmmGreenBox}[title=Zukünftige Erweiterungen]
Das CTMM-System kann erweitert werden um:
\begin{itemize}
  \item Timeline-Visualisierungen für Therapieverlauf
  \item Stimmungsdiagramme mit Trend-Analysen
  \item Interaktive Flowcharts für Entscheidungen
  \item Beziehungsdiagramme mit Feedback-Loops
\end{itemize}
\end{ctmmGreenBox}

\subsection*{\texorpdfstring{\textcolor{ctmmBlue}{Verwendung in eigenen Modulen}}{Verwendung in eigenen Modulen}}

\begin{ctmmBlueBox}[title=So nutzen Sie die Diagramme]
Die Diagramm-Befehle können in allen CTMM-Modulen verwendet werden:
\begin{itemize}
  \item Einfache Pfeil-Notation: \texttt{\$\textbackslash rightarrow\$}
  \item Bidirektionale Pfeile: \texttt{\$\textbackslash leftrightarrow\$}
  \item Farbige Boxen: \texttt{\textbackslash begin\{ctmmBlueBox\}}
  \item Zentrierte Darstellung mit \texttt{center} Umgebung
\end{itemize}
\end{ctmmBlueBox}

\subsection*{\texorpdfstring{\textcolor{ctmmBlue}{CTMM-Navigation}}{CTMM-Navigation}}
\begin{itemize}
  \item \texttt{triggermanagement} $\leftarrow$ Praktische Anwendung der Zyklen
  \item \texttt{bindungsleitfaden} $\leftarrow$ Systemische Grundlagen
\end{itemize}

% =====================================================

\section*{\texorpdfstring{\textcolor{ctmmBlue}{\faPalette~Diagramme \& Visualisierungen}}{Diagramme \& Visualisierungen}}

\begin{quote}
\textbf{\textcolor{ctmmBlue}{Worum geht's hier?}}\\
Dieses Modul zeigt die visuellen Möglichkeiten des CTMM-Systems für therapeutische Darstellungen und Diagramme.
\end{quote}

\subsection*{\texorpdfstring{\textcolor{ctmmBlue}{Der CTMM-Trigger-Zyklus}}{Der CTMM-Trigger-Zyklus}}

Der Kern des CTMM-Systems als visueller Kreislauf:

\begin{center}
\begin{tcolorbox}[colback=ctmmBlue!10!white,colframe=ctmmBlue,width=10cm]
\centering
\textbf{CTMM-Trigger-Zyklus}\\[0.5cm]
\textcolor{ctmmBlue}{\textbf{ERKENNEN}} $\rightarrow$ \textcolor{ctmmGreen}{\textbf{VORBEUGEN}} $\rightarrow$ \textcolor{ctmmOrange}{\textbf{REAGIEREN}} $\rightarrow$ \textcolor{ctmmPurple}{\textbf{LERNEN}}
\end{tcolorbox}
\end{center}

\textit{Dieser Zyklus zeigt die vier Hauptphasen des Trigger-Managements: Erkennen $\rightarrow$ Vorbeugen $\rightarrow$ Reagieren $\rightarrow$ Lernen}

\subsection*{\textcolor{ctmmOrange}{Eskalationsstufen}}

Die vier Ebenen der Krisenintervention:

\begin{center}
\begin{tcolorbox}[colback=ctmmOrange!10!white,colframe=ctmmOrange,width=12cm]
\centering
\textbf{Eskalationsstufen}\\[0.5cm]
\textcolor{ctmmGreen}{\textbf{1. Grün - Stabil}} $\rightarrow$ \textcolor{ctmmOrange}{\textbf{2. Gelb - Warnung}} $\rightarrow$ \textcolor{ctmmRed}{\textbf{3. Rot - Krise}} $\rightarrow$ \textcolor{ctmmPurple}{\textbf{4. Stabilisierung}}
\end{tcolorbox}
\end{center}

\textit{Jede Stufe hat spezifische Strategien und Interventionen. Der Kreislauf zeigt sowohl Eskalation als auch den Weg zurück zur Stabilität.}

\newpage

\subsection*{\textcolor{ctmmRed}{Partner-Dynamiken}}

Systemische Sicht auf die Paarbeziehung:

\begin{center}
\begin{tcolorbox}[colback=ctmmRed!10!white,colframe=ctmmRed,width=12cm]
\centering
\textbf{Partner-Dynamiken}\\[0.5cm]
\textcolor{ctmmBlue}{\textbf{Partner A}} $\leftrightarrow$ \textcolor{ctmmGreen}{\textbf{Ko-Regulation}} $\leftrightarrow$ \textcolor{ctmmBlue}{\textbf{Partner B}}
\end{tcolorbox}
\end{center}

\textit{Das Diagramm zeigt typische Trigger-Reaktions-Muster und wie CTMM-Tools zur Ko-Regulation beitragen.}

\subsection*{\textcolor{ctmmGreen}{Weitere Diagramm-Möglichkeiten}}

\begin{ctmmGreenBox}[title=Zukünftige Erweiterungen]
Das CTMM-System kann erweitert werden um:
\begin{itemize}
  \item Timeline-Visualisierungen für Therapieverlauf
  \item Stimmungsdiagramme mit Trend-Analysen
  \item Interaktive Flowcharts für Entscheidungen
  \item Beziehungsdiagramme mit Feedback-Loops
\end{itemize}
\end{ctmmGreenBox}

\subsection*{\texorpdfstring{\textcolor{ctmmBlue}{Verwendung in eigenen Modulen}}{Verwendung in eigenen Modulen}}

\begin{ctmmBlueBox}[title=So nutzen Sie die Diagramme]
Die Diagramm-Befehle können in allen CTMM-Modulen verwendet werden:
\begin{itemize}
  \item Einfache Pfeil-Notation: \texttt{\$\textbackslash rightarrow\$}
  \item Bidirektionale Pfeile: \texttt{\$\textbackslash leftrightarrow\$}
  \item Farbige Boxen: \texttt{\textbackslash begin\{ctmmBlueBox\}}
  \item Zentrierte Darstellung mit \texttt{center} Umgebung
\end{itemize}
\end{ctmmBlueBox}

\subsection*{\texorpdfstring{\textcolor{ctmmBlue}{CTMM-Navigation}}{CTMM-Navigation}}
\begin{itemize}
  \item \texttt{triggermanagement} $\leftarrow$ Praktische Anwendung der Zyklen
  \item \texttt{bindungsleitfaden} $\leftarrow$ Systemische Grundlagen
\end{itemize}

% =====================================================

\section*{\texorpdfstring{\textcolor{ctmmBlue}{\faPalette~Diagramme \& Visualisierungen}}{Diagramme \& Visualisierungen}}

\begin{quote}
\textbf{\textcolor{ctmmBlue}{Worum geht's hier?}}\\
Dieses Modul zeigt die visuellen Möglichkeiten des CTMM-Systems für therapeutische Darstellungen und Diagramme.
\end{quote}

\subsection*{\texorpdfstring{\textcolor{ctmmBlue}{Der CTMM-Trigger-Zyklus}}{Der CTMM-Trigger-Zyklus}}

Der Kern des CTMM-Systems als visueller Kreislauf:

\begin{center}
\begin{tcolorbox}[colback=ctmmBlue!10!white,colframe=ctmmBlue,width=10cm]
\centering
\textbf{CTMM-Trigger-Zyklus}\\[0.5cm]
\textcolor{ctmmBlue}{\textbf{ERKENNEN}} $\rightarrow$ \textcolor{ctmmGreen}{\textbf{VORBEUGEN}} $\rightarrow$ \textcolor{ctmmOrange}{\textbf{REAGIEREN}} $\rightarrow$ \textcolor{ctmmPurple}{\textbf{LERNEN}}
\end{tcolorbox}
\end{center}

\textit{Dieser Zyklus zeigt die vier Hauptphasen des Trigger-Managements: Erkennen $\rightarrow$ Vorbeugen $\rightarrow$ Reagieren $\rightarrow$ Lernen}

\subsection*{\textcolor{ctmmOrange}{Eskalationsstufen}}

Die vier Ebenen der Krisenintervention:

\begin{center}
\begin{tcolorbox}[colback=ctmmOrange!10!white,colframe=ctmmOrange,width=12cm]
\centering
\textbf{Eskalationsstufen}\\[0.5cm]
\textcolor{ctmmGreen}{\textbf{1. Grün - Stabil}} $\rightarrow$ \textcolor{ctmmOrange}{\textbf{2. Gelb - Warnung}} $\rightarrow$ \textcolor{ctmmRed}{\textbf{3. Rot - Krise}} $\rightarrow$ \textcolor{ctmmPurple}{\textbf{4. Stabilisierung}}
\end{tcolorbox}
\end{center}

\textit{Jede Stufe hat spezifische Strategien und Interventionen. Der Kreislauf zeigt sowohl Eskalation als auch den Weg zurück zur Stabilität.}

\newpage

\subsection*{\textcolor{ctmmRed}{Partner-Dynamiken}}

Systemische Sicht auf die Paarbeziehung:

\begin{center}
\begin{tcolorbox}[colback=ctmmRed!10!white,colframe=ctmmRed,width=12cm]
\centering
\textbf{Partner-Dynamiken}\\[0.5cm]
\textcolor{ctmmBlue}{\textbf{Partner A}} $\leftrightarrow$ \textcolor{ctmmGreen}{\textbf{Ko-Regulation}} $\leftrightarrow$ \textcolor{ctmmBlue}{\textbf{Partner B}}
\end{tcolorbox}
\end{center}

\textit{Das Diagramm zeigt typische Trigger-Reaktions-Muster und wie CTMM-Tools zur Ko-Regulation beitragen.}

\subsection*{\textcolor{ctmmGreen}{Weitere Diagramm-Möglichkeiten}}

\begin{ctmmGreenBox}[title=Zukünftige Erweiterungen]
Das CTMM-System kann erweitert werden um:
\begin{itemize}
  \item Timeline-Visualisierungen für Therapieverlauf
  \item Stimmungsdiagramme mit Trend-Analysen
  \item Interaktive Flowcharts für Entscheidungen
  \item Beziehungsdiagramme mit Feedback-Loops
\end{itemize}
\end{ctmmGreenBox}

\subsection*{\texorpdfstring{\textcolor{ctmmBlue}{Verwendung in eigenen Modulen}}{Verwendung in eigenen Modulen}}

\begin{ctmmBlueBox}[title=So nutzen Sie die Diagramme]
Die Diagramm-Befehle können in allen CTMM-Modulen verwendet werden:
\begin{itemize}
  \item Einfache Pfeil-Notation: \texttt{\$\textbackslash rightarrow\$}
  \item Bidirektionale Pfeile: \texttt{\$\textbackslash leftrightarrow\$}
  \item Farbige Boxen: \texttt{\textbackslash begin\{ctmmBlueBox\}}
  \item Zentrierte Darstellung mit \texttt{center} Umgebung
\end{itemize}
\end{ctmmBlueBox}

\subsection*{\texorpdfstring{\textcolor{ctmmBlue}{CTMM-Navigation}}{CTMM-Navigation}}
\begin{itemize}
  \item \texttt{triggermanagement} $\leftarrow$ Praktische Anwendung der Zyklen
  \item \texttt{bindungsleitfaden} $\leftarrow$ Systemische Grundlagen
\end{itemize}

% =====================================================

\section*{\texorpdfstring{\textcolor{ctmmBlue}{\faPalette~Diagramme \& Visualisierungen}}{Diagramme \& Visualisierungen}}

\begin{quote}
\textbf{\textcolor{ctmmBlue}{Worum geht's hier?}}\\
Dieses Modul zeigt die visuellen Möglichkeiten des CTMM-Systems für therapeutische Darstellungen und Diagramme.
\end{quote}

\subsection*{\texorpdfstring{\textcolor{ctmmBlue}{Der CTMM-Trigger-Zyklus}}{Der CTMM-Trigger-Zyklus}}

Der Kern des CTMM-Systems als visueller Kreislauf:

\begin{center}
\begin{tcolorbox}[colback=ctmmBlue!10!white,colframe=ctmmBlue,width=10cm]
\centering
\textbf{CTMM-Trigger-Zyklus}\\[0.5cm]
\textcolor{ctmmBlue}{\textbf{ERKENNEN}} $\rightarrow$ \textcolor{ctmmGreen}{\textbf{VORBEUGEN}} $\rightarrow$ \textcolor{ctmmOrange}{\textbf{REAGIEREN}} $\rightarrow$ \textcolor{ctmmPurple}{\textbf{LERNEN}}
\end{tcolorbox}
\end{center}

\textit{Dieser Zyklus zeigt die vier Hauptphasen des Trigger-Managements: Erkennen $\rightarrow$ Vorbeugen $\rightarrow$ Reagieren $\rightarrow$ Lernen}

\subsection*{\textcolor{ctmmOrange}{Eskalationsstufen}}

Die vier Ebenen der Krisenintervention:

\begin{center}
\begin{tcolorbox}[colback=ctmmOrange!10!white,colframe=ctmmOrange,width=12cm]
\centering
\textbf{Eskalationsstufen}\\[0.5cm]
\textcolor{ctmmGreen}{\textbf{1. Grün - Stabil}} $\rightarrow$ \textcolor{ctmmOrange}{\textbf{2. Gelb - Warnung}} $\rightarrow$ \textcolor{ctmmRed}{\textbf{3. Rot - Krise}} $\rightarrow$ \textcolor{ctmmPurple}{\textbf{4. Stabilisierung}}
\end{tcolorbox}
\end{center}

\textit{Jede Stufe hat spezifische Strategien und Interventionen. Der Kreislauf zeigt sowohl Eskalation als auch den Weg zurück zur Stabilität.}

\newpage

\subsection*{\textcolor{ctmmRed}{Partner-Dynamiken}}

Systemische Sicht auf die Paarbeziehung:

\begin{center}
\begin{tcolorbox}[colback=ctmmRed!10!white,colframe=ctmmRed,width=12cm]
\centering
\textbf{Partner-Dynamiken}\\[0.5cm]
\textcolor{ctmmBlue}{\textbf{Partner A}} $\leftrightarrow$ \textcolor{ctmmGreen}{\textbf{Ko-Regulation}} $\leftrightarrow$ \textcolor{ctmmBlue}{\textbf{Partner B}}
\end{tcolorbox}
\end{center}

\textit{Das Diagramm zeigt typische Trigger-Reaktions-Muster und wie CTMM-Tools zur Ko-Regulation beitragen.}

\subsection*{\textcolor{ctmmGreen}{Weitere Diagramm-Möglichkeiten}}

\begin{ctmmGreenBox}[title=Zukünftige Erweiterungen]
Das CTMM-System kann erweitert werden um:
\begin{itemize}
  \item Timeline-Visualisierungen für Therapieverlauf
  \item Stimmungsdiagramme mit Trend-Analysen
  \item Interaktive Flowcharts für Entscheidungen
  \item Beziehungsdiagramme mit Feedback-Loops
\end{itemize}
\end{ctmmGreenBox}

\subsection*{\texorpdfstring{\textcolor{ctmmBlue}{Verwendung in eigenen Modulen}}{Verwendung in eigenen Modulen}}

\begin{ctmmBlueBox}[title=So nutzen Sie die Diagramme]
Die Diagramm-Befehle können in allen CTMM-Modulen verwendet werden:
\begin{itemize}
  \item Einfache Pfeil-Notation: \texttt{\$\textbackslash rightarrow\$}
  \item Bidirektionale Pfeile: \texttt{\$\textbackslash leftrightarrow\$}
  \item Farbige Boxen: \texttt{\textbackslash begin\{ctmmBlueBox\}}
  \item Zentrierte Darstellung mit \texttt{center} Umgebung
\end{itemize}
\end{ctmmBlueBox}

\subsection*{\texorpdfstring{\textcolor{ctmmBlue}{CTMM-Navigation}}{CTMM-Navigation}}
\begin{itemize}
  \item \texttt{triggermanagement} $\leftarrow$ Praktische Anwendung der Zyklen
  \item \texttt{bindungsleitfaden} $\leftarrow$ Systemische Grundlagen
\end{itemize}
