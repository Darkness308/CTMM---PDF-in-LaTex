% Navigation und Querverweise für CTMM-System
\section*{\textcolor{ctmmBlue}{\faMap~Navigations-System}}
\addcontentsline{toc}{section}{Navigations-System}
\label{sec:navigation}

\begin{ctmmBlueBox}{Orientierung im CTMM-System}
\textbf{Farbkodierung für schnelle Navigation:}

\begin{itemize}
    \item \textcolor{ctmmBlue}{\textbf{BLAU - Grundlagen}} - Warum wir uns triggern
    \item \textcolor{ctmmGreen}{\textbf{GRÜN - Tägliche Tools}} - Skills und Routinen  
    \item \textcolor{ctmmRed}{\textbf{ROT - Notfall-Guide}} - Krisenintervention
    \item \textcolor{ctmmYellow}{\textbf{GELB - Support}} - Freunde und Familie
    \item \textcolor{ctmmPurple}{\textbf{LILA - Arbeitsblätter}} - Tracking und Reflexion
\end{itemize}
\end{ctmmBlueBox}

\vspace{0.5cm}

\subsection*{\faChevronRight~Schnell-Navigation}

\begin{ctmmGreenBox}{GRÜN: Tägliche Tools - Jeden Tag nutzen}
\begin{itemize}
    \item \ctmmRef{sec:5.1}{Täglicher Check-In} - Morgens und abends
    \item \ctmmRef{sec:safewords}{Safe-Words System} - Bei Überforderung
    \item \ctmmRef{sec:triggermanagement}{Trigger-Management} - Präventiv
\end{itemize}
\end{ctmmGreenBox}

\begin{ctmmRedBox}{ROT: Notfall-Protokolle - In Krisen}
\begin{itemize}
    \item \ctmmRef{sec:notfallkarten}{Notfallkarten} - Sofort verfügbar
    \item \ctmmRef{sec:5.2}{Trigger-Tagebuch} - Nach der Krise
    \item \ctmmRef{sec:5.3}{Depression-Monitor} - Wöchentlich
\end{itemize}
\end{ctmmRedBox}

\begin{ctmmPurpleBox}{LILA: Reflexion \& Wachstum - Regelmäßig}
\begin{itemize}
    \item \ctmmRef{sec:feedback}{Selbstreflexions-System} - Monatlich
    \item \ctmmRef{sec:fortschritt}{Fortschrittsmessung} - Kontinuierlich
    \item \ctmmRef{sec:erfolge}{Erfolgs-Bibliothek} - Motivation
\end{itemize}
\end{ctmmPurpleBox}

\vspace{0.5cm}

\subsection*{\faQuestionCircle~Häufige Situationen}

\begin{tabular}{p{5cm}p{8cm}}
\textbf{Situation} & \textbf{Gehe zu} \\
\hline
Überforderung spürbar & $\rightarrow$ \ctmmRef{sec:safewords}{Safe-Words} \\
Nach einem Streit & $\rightarrow$ \ctmmRef{sec:5.2}{Trigger-Tagebuch} \\
Schlechte Schlafqualität & $\rightarrow$ \ctmmRef{sec:5.3}{Depression-Monitor} \\
Erfolg feiern & $\rightarrow$ \ctmmRef{sec:erfolge}{Erfolgs-Bibliothek} \\
System anpassen & $\rightarrow$ \ctmmRef{sec:feedback}{Selbstreflexion} \\
\end{tabular}

\vspace{0.5cm}

\textit{Tipp: Nutzen Sie die Farbkodierung, um schnell das richtige Werkzeug zu finden!}
