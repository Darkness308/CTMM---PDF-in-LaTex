% CTMM Arbeitsblatt: Täglicher Stimmungscheck
% Generiert am: 19.08.2024
% CTMM Methodology: Catch-Track-Map-Match
% 
% Dieses Arbeitsblatt unterstützt die strukturierte Selbstreflexion
% im Rahmen der CTMM-Therapiemethodik für neurodiverse Paare.

\section{Täglicher Stimmungscheck}

\begin{ctmmBlueBox}{Arbeitsblatt-Übersicht}
\textbf{Zweck:} Regelmäßige Erfassung und Reflexion der täglichen emotionalen Verfassung

\textbf{CTMM-Phase:} Track - Systematische Dokumentation emotionaler Muster

\textbf{Zielgruppe:} Neurodiverse Einzelpersonen und Paare

\textbf{Bearbeitungszeit:} 5-10 Minuten täglich
\end{ctmmBlueBox}

\subsection{Anleitung}

Nimm dir jeden Tag zur gleichen Zeit (idealerweise morgens und/oder abends) einige Minuten Zeit für diese Reflexion. 
Ehrlichkeit ist wichtiger als Perfektion - dokumentiere deine tatsächlichen Gefühle, nicht die erwünschten.

\subsection{Reflexionsfragen}

\begin{enumerate}
\item Wie würde ich meine heutige Grundstimmung beschreiben? (1-10 Skala)
\item Welche Emotionen waren heute besonders präsent?
\item Gab es heute Situationen, die meine Stimmung stark beeinflusst haben?
\item Wie gut konnte ich heute mit schwierigen Gefühlen umgehen?
\item Was hat mir heute geholfen, mich besser zu fühlen?
\end{enumerate}

\subsection{Interaktive Elemente}

\begin{tabular}{|p{0.7\textwidth}|p{0.25\textwidth}|}
\hline
\textbf{Bewertungskriterium} & \textbf{Selbsteinschätzung} \\
\hline
Allgemeine Stimmung (1-10) & \ctmmTextField[3cm]{Stimmung}{grundstimmung} \\
\hline
Energielevel (1-10) & \ctmmTextField[3cm]{Energie}{energielevel} \\
\hline
Stresslevel (1-10) & \ctmmTextField[3cm]{Stress}{stresslevel} \\
\hline
Beziehungsqualität (1-10) & \ctmmTextField[3cm]{Beziehung}{beziehung} \\
\hline
\end{tabular}

\subsection{Tracking und Dokumentation}

\textbf{Datum:} \ctmmTextField[4cm]{Datum}{datum}

\textbf{Uhrzeit:} \ctmmTextField[3cm]{Uhrzeit}{uhrzeit}

\textbf{Dominante Emotion heute:} \ctmmTextField[6cm]{Emotion}{emotion}

\textbf{Trigger oder besondere Ereignisse:}

\ctmmTextArea[\textwidth]{3}{Ereignisse}{ereignisse}

\textbf{Bewältigungsstrategien verwendet:}

\ctmmCheckBox[strategie1]{Atemtechniken angewendet}
\ctmmCheckBox[strategie2]{Partner kontaktiert}
\ctmmCheckBox[strategie3]{Auszeit genommen}
\ctmmCheckBox[strategie4]{Körperliche Aktivität}
\ctmmCheckBox[strategie5]{Andere: } \ctmmTextField[6cm]{Andere Strategie}{andere_strategie}

\subsection{Nächste Schritte}

\begin{ctmmGreenBox}{Handlungsplan}
\ctmmCheckBox[schritt1]{Muster in der Stimmung erkannt}

\ctmmCheckBox[schritt2]{Bewältigungsstrategie für morgen gewählt}

\ctmmCheckBox[schritt3]{Bei Bedarf Unterstützung organisiert}

\textbf{Priorität für morgen:} \ctmmTextField[\textwidth]{Priorität}{prioritaet}
\end{ctmmGreenBox}

\textcolor{ctmmGray}{\small Dieses Arbeitsblatt ist Teil der CTMM-Therapiematerialien und unterstützt die Catch-Track-Map-Match Methodik für neurodiverse Paare.}
