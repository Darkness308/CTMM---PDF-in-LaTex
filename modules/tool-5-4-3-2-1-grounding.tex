% CTMM Tool: 5-4-3-2-1 Grounding Technique
% Generiert am: 19.08.2024
% Therapeutic Tool für CTMM Methodology
% 
% Dieses Tool bietet strukturierte Techniken zur Bewältigung
% spezifischer Herausforderungen im Rahmen der CTMM-Therapie.

\section{5-4-3-2-1 Grounding Technique}

\begin{ctmmOrangeBox}{Tool-Übersicht}
\textbf{Anwendungsbereich:} Akute Angst, Panikattacken, Dissoziation, Überforderung

\textbf{CTMM-Phase:} Catch - Frühzeitige Erkennung und Intervention bei Krisenmomenten

\textbf{Schwierigkeitsgrad:} Einfach - für alle Altersstufen und Kompetenzniveaus geeignet

\textbf{Benötigte Zeit:} 3-5 Minuten
\end{ctmmOrangeBox}

\subsection{Wann wird dieses Tool angewendet?}

Diese Technik hilft bei:
\begin{itemize}
\item Akuten Angstzuständen oder Panikattacken
\item Gefühlen der Dissoziation oder "Nicht-da-sein"
\item Überwältigung durch intensive Emotionen
\item Hypervigilanz oder übermäßiger Wachsamkeit
\item Flashbacks oder traumatischen Erinnerungen
\item Allgemeiner Unruhe und Stress
\end{itemize}

\subsection{Schritt-für-Schritt Anleitung}

\begin{enumerate}
\item \textbf{5 Dinge SEHEN:} Schaue dich um und benenne bewusst 5 Dinge, die du sehen kannst. Beschreibe sie in Gedanken (Farbe, Form, Größe).

\item \textbf{4 Dinge HÖREN:} Höre bewusst hin und identifiziere 4 verschiedene Geräusche um dich herum (Verkehr, Stimmen, Musik, Naturgeräusche).

\item \textbf{3 Dinge BERÜHREN:} Berühre 3 verschiedene Oberflächen oder Gegenstände. Konzentriere dich auf die Textur, Temperatur und Beschaffenheit.

\item \textbf{2 Dinge RIECHEN:} Identifiziere 2 verschiedene Gerüche in deiner Umgebung oder rieche bewusst an etwas (Parfüm, Essen, frische Luft).

\item \textbf{1 Ding SCHMECKEN:} Konzentriere dich auf einen Geschmack in deinem Mund oder nimm bewusst etwas zu dir (Kaugummi, Wasser, Bonbon).
\end{enumerate}

\subsection{Beispiel-Anwendung}

\begin{ctmmGrayBox}{Praxis-Beispiel}
\textbf{Situation:} Sarah spürt eine Panikattacke aufkommen während eines Termins.

\textbf{Anwendung:}
\begin{itemize}
\item \textbf{5 sehen:} "Ich sehe die blaue Wand, den schwarzen Computer, meine roten Schuhe, das grüne Buch, die weiße Tasse."
\item \textbf{4 hören:} "Ich höre Stimmen im Flur, das Summen des Computers, Verkehr draußen, meine eigene Atmung."
\item \textbf{3 berühren:} "Ich berühre die glatte Tischoberfläche, den weichen Stuhlbezug, meine warme Handfläche."
\item \textbf{2 riechen:} "Ich rieche Kaffee und das Papier der Dokumente."
\item \textbf{1 schmecken:} "Ich schmecke den Pfefferminzgeschmack meines Kaugummis."
\end{itemize}

\textbf{Ergebnis:} Sarah fühlt sich wieder geerdet und kann den Termin fortsetzen.
\end{ctmmGrayBox}

\subsection{Anpassungen für neurodiverse Paare}

\begin{itemize}
\item \textbf{ADHS:} Verwende eine verkürzte 3-2-1 Version bei Konzentrationsschwierigkeiten
\item \textbf{Autismus:} Erlaube vertraute Objekte als Grounding-Anker (Stimming-Toys, vertraute Texturen)
\item \textbf{Gemeinsame Anwendung:} Partner können sich gegenseitig durch die Schritte führen
\item \textbf{Nonverbal:} Bei nonverbalen Momenten reicht es, die Sinne stumm zu aktivieren
\end{itemize}

\subsection{Erfolgsindikatoren}

\begin{itemize}
\item Verlangsamung der Atmung und des Herzschlags
\item Reduzierte körperliche Anspannung
\item Klarerer, fokussierterer Geisteszustand
\item Stärkeres Gefühl der Verbindung zum "Hier und Jetzt"
\item Abnahme dissoziative Symptome
\item Erhöhte Handlungsfähigkeit
\end{itemize}

\subsection{Troubleshooting}

\textbf{Was tun, wenn das Tool nicht funktioniert?}

\begin{itemize}
\item \textbf{Zu überwältigt:} Beginne nur mit einer Sinneswahrnehmung und steigere langsam
\item \textbf{Keine Verbesserung:} Kombiniere mit langsamer, bewusster Atmung
\item \textbf{Ablenkung schwierig:} Sage die Wahrnehmungen laut aus oder schreibe sie auf
\item \textbf{Panik verstärkt sich:} Suche sofort professionelle Hilfe oder kontaktiere eine Vertrauensperson
\end{itemize}

\subsection{Integration in den Alltag}

\begin{ctmmGreenBox}{Praktische Umsetzung}
\ctmmCheckBox[integration1]{Tool in Krisenplan integriert}

\ctmmCheckBox[integration2]{Partner über Tool informiert und geschult}

\ctmmCheckBox[integration3]{Tägliche Übung (auch ohne Krise) geplant}

\textbf{Erinnerungshilfe:} \ctmmTextField[\textwidth]{Handy-Notiz, Kärtchen, etc.}{reminder_system}
\end{ctmmGreenBox}

\textcolor{ctmmGray}{\small Dieses Tool ist Teil der CTMM-Therapiematerialien und unterstützt die Catch-Track-Map-Match Methodik für neurodiverse Paare.}
