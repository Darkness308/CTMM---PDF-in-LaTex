% Triggermanagement - CTMM Modul (Tool 23)

\newpage
\section*{\textcolor{ctmmOrange}{\faExclamationCircle~Triggermanagement}}
\addcontentsline{toc}{section}{Triggermanagement}

\begin{quote}
\textit{\textcolor{ctmmOrange}{Trigger verstehen heißt sich selbst verstehen -- der Schlüssel zur Selbstregulation.}}\\
\textbf{\textcolor{ctmmOrange}{Was ist ein Trigger?}}\\
Ein Trigger ist ein Auslöser, der starke emotionale oder körperliche Reaktionen hervorruft. Im CTMM werden Trigger erkannt, verstanden und bearbeitet.
\end{quote}

\subsection*{\textcolor{ctmmOrange}{1. Trigger erkennen}}

\begin{ctmmOrangeBox}{Erkennungszeichen}
\begin{itemize}
  \item \textbf{Körperliche Reaktionen}: Herzklopfen, Schwitzen, Anspannung
  \item \textbf{Gedanken und Erinnerungen}: Flashbacks, Grübeln, Katastrophendenken  
  \item \textbf{Gefühle}: Angst, Wut, Traurigkeit, Hilflosigkeit
  \item \textbf{Verhaltensänderungen}: Rückzug, Aggression, Erstarrung
\end{itemize}
\end{ctmmOrangeBox}

\subsection*{\textcolor{ctmmOrange}{2. Umgang mit Triggern}}

\begin{ctmmGreenBox}{Bewältigungsstrategien}
\begin{itemize}
  \item \textbf{Bewusstes Atmen}: 4-7-8 Technik, Bauchatmung
  \item \textbf{Grounding}: 5-4-3-2-1 Methode (Sinne aktivieren)
  \item \textbf{Selbstfürsorge}: Pausen, Grenzen, Ressourcen aktivieren
  \item \textbf{Soziale Unterstützung}: Safe Words, Vertrauensperson kontaktieren
  \item \textbf{Trigger-Tagebuch}: Muster erkennen, Fortschritte dokumentieren
\end{itemize}
\end{ctmmGreenBox}

\subsection*{\textcolor{ctmmOrange}{3. CTMM-Tool 23: Triggermanagement}}

Das Tool unterstützt dabei, Trigger zu identifizieren, zu reflektieren und neue Handlungsoptionen zu entwickeln. Ziel ist es, die eigene Reaktion zu verstehen und zu steuern.

\textbf{Trigger-Analyse:}\\
\ctmmTextField[4cm]{Trigger-Situation:} \ctmmTextField[4cm]{Körperreaktion:}\\
\ctmmTextField[4cm]{Gedanken:} \ctmmTextField[4cm]{Gefühle:}\\
\ctmmTextField[8cm]{Bewältigungsstrategie:}
