% ================================================================
% CTMM Module: DBT Emotionsregulation & Interpersonelle Skills
% DEAR MAN, GIVE, und weitere DBT-Techniken für neurodiverse Paare
% ================================================================

\section{DBT-Skills für Emotionsregulation}
\label{sec:dbt-skills}

\begin{ctmmBlueBox}[title=Dialektisch-Behaviorale Therapie (DBT) Skills]

\textbf{DBT-Skills} sind bewährte Techniken zur Emotionsregulation und Verbesserung zwischenmenschlicher Beziehungen. Sie sind besonders hilfreich für neurodiverse Paare.

\vspace{0.5em}

\textbf{Die vier DBT-Module:}
\begin{itemize}
    \item \textcolor{ctmmBlue}{\textbf{Achtsamkeit:}} Bewusstsein für den gegenwärtigen Moment
    \item \textcolor{ctmmGreen}{\textbf{Emotionsregulation:}} Umgang mit intensiven Gefühlen
    \item \textcolor{ctmmOrange}{\textbf{Zwischenmenschliche Fertigkeiten:}} Kommunikation und Beziehungen
    \item \textcolor{ctmmPurple}{\textbf{Stresstoleranz:}} Krisenüberstehung ohne Verschlimmerung
\end{itemize}

\end{ctmmBlueBox}

\subsection{DEAR MAN - Effektive Kommunikation}

\begin{ctmmGreenBox}[title=DEAR MAN Technik]

\textbf{DEAR MAN} ist eine strukturierte Methode für schwierige Gespräche und Bitten.

\vspace{0.5em}

\begin{tabular}{|l|l|p{8cm}|}
\hline
\textbf{Buchstabe} & \textbf{Bedeutung} & \textbf{Beschreibung} \\
\hline
\textcolor{ctmmBlue}{\textbf{D}} & \textbf{Describe} & Beschreibe die Situation objektiv, ohne Bewertung \\
\hline
\textcolor{ctmmGreen}{\textbf{E}} & \textbf{Express} & Drücke deine Gefühle und Meinungen aus \\
\hline
\textcolor{ctmmOrange}{\textbf{A}} & \textbf{Assert} & Formuliere klar, was du möchtest \\
\hline
\textcolor{ctmmPurple}{\textbf{R}} & \textbf{Reinforce} & Erkläre die positiven Konsequenzen \\
\hline
\textcolor{ctmmRed}{\textbf{M}} & \textbf{Mindful} & Bleibe fokussiert auf dein Ziel \\
\hline
\textcolor{ctmmBlue}{\textbf{A}} & \textbf{Appear confident} & Tritt selbstbewusst auf \\
\hline
\textcolor{ctmmGreen}{\textbf{N}} & \textbf{Negotiate} & Sei bereit zu Kompromissen \\
\hline
\end{tabular}

\end{ctmmGreenBox}

\subsubsection{DEAR MAN Praxis-Übung}

\begin{ctmmOrangeBox}[title=DEAR MAN Arbeitsblatt]

\textbf{Situation, die ich ansprechen möchte:} \\
\TextField[name=dearman_situation,width=14cm,height=2em,multiline=true,bordercolor=ctmmOrange,backgroundcolor=ctmmOrange!5]{}

\vspace{0.5em}

\textbf{D - Describe (Objektive Beschreibung):} \\
\TextField[name=dearman_describe,width=14cm,height=3em,multiline=true,bordercolor=ctmmBlue,backgroundcolor=ctmmBlue!5]{}

\textbf{E - Express (Gefühle ausdrücken):} \\
\TextField[name=dearman_express,width=14cm,height=2em,multiline=true,bordercolor=ctmmGreen,backgroundcolor=ctmmGreen!5]{}

\textbf{A - Assert (Klare Bitte):} \\
\TextField[name=dearman_assert,width=14cm,height=2em,multiline=true,bordercolor=ctmmOrange,backgroundcolor=ctmmOrange!5]{}

\textbf{R - Reinforce (Positive Konsequenzen):} \\
\TextField[name=dearman_reinforce,width=14cm,height=2em,multiline=true,bordercolor=ctmmPurple,backgroundcolor=ctmmPurple!5]{}

\textbf{Reflexion nach dem Gespräch:} \\
\TextField[name=dearman_reflection,width=14cm,height=3em,multiline=true,bordercolor=ctmmGreen,backgroundcolor=ctmmGreen!5]{}

\end{ctmmOrangeBox}

\subsection{GIVE - Beziehungen stärken}

\begin{ctmmPurpleBox}[title=GIVE Technik]

\textbf{GIVE} hilft dabei, Beziehungen zu erhalten und zu stärken, während man für sich selbst einsteht.

\vspace{0.5em}

\begin{tabular}{|l|l|p{8cm}|}
\hline
\textbf{Buchstabe} & \textbf{Bedeutung} & \textbf{Beschreibung} \\
\hline
\textcolor{ctmmGreen}{\textbf{G}} & \textbf{Gentle} & Sei sanft, vermeide Angriffe und Bedrohungen \\
\hline
\textcolor{ctmmBlue}{\textbf{I}} & \textbf{Interested} & Zeige echtes Interesse an der anderen Person \\
\hline
\textcolor{ctmmOrange}{\textbf{V}} & \textbf{Validate} & Validiere die Gefühle und Sichtweise des anderen \\
\hline
\textcolor{ctmmPurple}{\textbf{E}} & \textbf{Easy manner} & Bleibe entspannt und verwende Humor wenn angemessen \\
\hline
\end{tabular}

\end{ctmmPurpleBox}

\subsubsection{GIVE Praxis-Übung}

\begin{ctmmBlueBox}[title=GIVE Reflexionsbogen]

\textbf{Letzte schwierige Unterhaltung mit Partner/in:} \\
\TextField[name=give_conversation,width=14cm,height=2em,multiline=true,bordercolor=ctmmBlue,backgroundcolor=ctmmBlue!5]{}

\vspace{0.5em}

\textbf{G - War ich gentle (sanft)?} \\
\CheckBox[name=give_gentle_yes,width=1em,height=1em]{} Ja, ich war respektvoll \\
\CheckBox[name=give_gentle_no,width=1em,height=1em]{} Nein, ich war zu hart \\
\textbf{Verbesserung:} \TextField[name=give_gentle_improve,width=10cm,height=1.5em,bordercolor=ctmmGreen,backgroundcolor=ctmmGreen!5]{}

\textbf{I - War ich interested (interessiert)?} \\
\CheckBox[name=give_interested_yes,width=1em,height=1em]{} Ja, ich habe zugehört \\
\CheckBox[name=give_interested_no,width=1em,height=1em]{} Nein, ich war zu selbstfokussiert \\
\textbf{Verbesserung:} \TextField[name=give_interested_improve,width=10cm,height=1.5em,bordercolor=ctmmBlue,backgroundcolor=ctmmBlue!5]{}

\textbf{V - Habe ich validated (validiert)?} \\
\CheckBox[name=give_validate_yes,width=1em,height=1em]{} Ja, ich habe Verständnis gezeigt \\
\CheckBox[name=give_validate_no,width=1em,height=1em]{} Nein, ich habe abgewertet \\
\textbf{Verbesserung:} \TextField[name=give_validate_improve,width=10cm,height=1.5em,bordercolor=ctmmOrange,backgroundcolor=ctmmOrange!5]{}

\textbf{E - War meine Manner easy (entspannt)?} \\
\CheckBox[name=give_easy_yes,width=1em,height=1em]{} Ja, ich war locker \\
\CheckBox[name=give_easy_no,width=1em,height=1em]{} Nein, ich war zu angespannt \\
\textbf{Verbesserung:} \TextField[name=give_easy_improve,width=10cm,height=1.5em,bordercolor=ctmmPurple,backgroundcolor=ctmmPurple!5]{}

\end{ctmmBlueBox}

\subsection{PLEASE - Emotionale Vulnerabilität reduzieren}

\begin{ctmmGreenBox}[title=PLEASE Skills für emotionale Balance]

\textbf{PLEASE} hilft dabei, die emotionale Vulnerabilität zu reduzieren und die Widerstandsfähigkeit zu erhöhen.

\vspace{0.5em}

\begin{tabular}{|l|l|p{7cm}|}
\hline
\textbf{P} & \textbf{Treat PhysicaL illness} & Körperliche Krankheiten behandeln \\
\hline
\textbf{L} & \textbf{Balance Eating} & Ausgewogene Ernährung \\
\hline
\textbf{E} & \textbf{Avoid mood-Altering substances} & Suchtmittel vermeiden \\
\hline
\textbf{A} & \textbf{Balance Sleep} & Gesunden Schlaf fördern \\
\hline
\textbf{S} & \textbf{Get Exercise} & Regelmäßig bewegen \\
\hline
\textbf{E} & \textbf{Build mastery} & Kompetenzgefühl entwickeln \\
\hline
\end{tabular}

\end{ctmmGreenBox}

\subsubsection{PLEASE Wochenplanung}

\begin{ctmmOrangeBox}[title=Wöchentlicher PLEASE Tracker]

\textbf{Woche vom:} \TextField[name=please_week_date,width=4cm,height=1em,bordercolor=ctmmOrange,backgroundcolor=ctmmOrange!5]{}

\vspace{0.5em}

\begin{tabular}{|l|c|c|c|c|c|c|c|}
\hline
\textbf{PLEASE Element} & \textbf{Mo} & \textbf{Di} & \textbf{Mi} & \textbf{Do} & \textbf{Fr} & \textbf{Sa} & \textbf{So} \\
\hline
\textbf{Körperliche Gesundheit} & 
\CheckBox[name=please_physical_mo,width=1em,height=1em]{} & 
\CheckBox[name=please_physical_di,width=1em,height=1em]{} & 
\CheckBox[name=please_physical_mi,width=1em,height=1em]{} & 
\CheckBox[name=please_physical_do,width=1em,height=1em]{} & 
\CheckBox[name=please_physical_fr,width=1em,height=1em]{} & 
\CheckBox[name=please_physical_sa,width=1em,height=1em]{} & 
\CheckBox[name=please_physical_so,width=1em,height=1em]{} \\
\hline
\textbf{Ausgewogene Ernährung} & 
\CheckBox[name=please_eating_mo,width=1em,height=1em]{} & 
\CheckBox[name=please_eating_di,width=1em,height=1em]{} & 
\CheckBox[name=please_eating_mi,width=1em,height=1em]{} & 
\CheckBox[name=please_eating_do,width=1em,height=1em]{} & 
\CheckBox[name=please_eating_fr,width=1em,height=1em]{} & 
\CheckBox[name=please_eating_sa,width=1em,height=1em]{} & 
\CheckBox[name=please_eating_so,width=1em,height=1em]{} \\
\hline
\textbf{Substanzen vermieden} & 
\CheckBox[name=please_substances_mo,width=1em,height=1em]{} & 
\CheckBox[name=please_substances_di,width=1em,height=1em]{} & 
\CheckBox[name=please_substances_mi,width=1em,height=1em]{} & 
\CheckBox[name=please_substances_do,width=1em,height=1em]{} & 
\CheckBox[name=please_substances_fr,width=1em,height=1em]{} & 
\CheckBox[name=please_substances_sa,width=1em,height=1em]{} & 
\CheckBox[name=please_substances_so,width=1em,height=1em]{} \\
\hline
\textbf{Guter Schlaf} & 
\CheckBox[name=please_sleep_mo,width=1em,height=1em]{} & 
\CheckBox[name=please_sleep_di,width=1em,height=1em]{} & 
\CheckBox[name=please_sleep_mi,width=1em,height=1em]{} & 
\CheckBox[name=please_sleep_do,width=1em,height=1em]{} & 
\CheckBox[name=please_sleep_fr,width=1em,height=1em]{} & 
\CheckBox[name=please_sleep_sa,width=1em,height=1em]{} & 
\CheckBox[name=please_sleep_so,width=1em,height=1em]{} \\
\hline
\textbf{Bewegung/Sport} & 
\CheckBox[name=please_exercise_mo,width=1em,height=1em]{} & 
\CheckBox[name=please_exercise_di,width=1em,height=1em]{} & 
\CheckBox[name=please_exercise_mi,width=1em,height=1em]{} & 
\CheckBox[name=please_exercise_do,width=1em,height=1em]{} & 
\CheckBox[name=please_exercise_fr,width=1em,height=1em]{} & 
\CheckBox[name=please_exercise_sa,width=1em,height=1em]{} & 
\CheckBox[name=please_exercise_so,width=1em,height=1em]{} \\
\hline
\textbf{Mastery Aktivität} & 
\CheckBox[name=please_mastery_mo,width=1em,height=1em]{} & 
\CheckBox[name=please_mastery_di,width=1em,height=1em]{} & 
\CheckBox[name=please_mastery_mi,width=1em,height=1em]{} & 
\CheckBox[name=please_mastery_do,width=1em,height=1em]{} & 
\CheckBox[name=please_mastery_fr,width=1em,height=1em]{} & 
\CheckBox[name=please_mastery_sa,width=1em,height=1em]{} & 
\CheckBox[name=please_mastery_so,width=1em,height=1em]{} \\
\hline
\end{tabular}

\vspace{0.5em}

\textbf{Reflexion der Woche:} \\
\TextField[name=please_weekly_reflection,width=14cm,height=3em,multiline=true,bordercolor=ctmmGreen,backgroundcolor=ctmmGreen!5]{}

\end{ctmmOrangeBox}

\subsection{TIPP - Krisenbewältigung}

\begin{ctmmRedBox}[title=\textcolor{white}{TIPP - Akute Krisenbewältigung}]

\textcolor{white}{\textbf{TIPP} sind Notfall-Skills für intensive emotionale Krisen, die sofort angewendet werden können.}

\vspace{0.5em}

\begin{tabular}{|l|p{10cm}|}
\hline
\textcolor{white}{\textbf{T}} & \textcolor{white}{\textbf{Temperature} - Körpertemperatur verändern (kaltes Wasser, Eiswürfel)} \\
\hline
\textcolor{white}{\textbf{I}} & \textcolor{white}{\textbf{Intense exercise} - Intensive Bewegung (Liegestütze, Laufen)} \\
\hline
\textcolor{white}{\textbf{P}} & \textcolor{white}{\textbf{Paced breathing} - Kontrollierte Atmung (4-7-8 Technik)} \\
\hline
\textcolor{white}{\textbf{P}} & \textcolor{white}{\textbf{Progressive muscle relaxation} - Muskelentspannung} \\
\hline
\end{tabular}

\end{ctmmRedBox}

\subsubsection{TIPP Notfall-Plan}

\begin{ctmmOrangeBox}[title=Persönlicher TIPP Notfallplan]

\textbf{Meine TIPP-Strategien für Krisen:}

\textbf{T - Temperature (Verfügbare Optionen):} \\
\CheckBox[name=tipp_cold_water,width=1em,height=1em]{} Kaltes Wasser über Handgelenke \\
\CheckBox[name=tipp_ice_cubes,width=1em,height=1em]{} Eiswürfel in den Händen halten \\
\CheckBox[name=tipp_cold_shower,width=1em,height=1em]{} Kalte Dusche \\
\textbf{Andere:} \TextField[name=tipp_temperature_other,width=8cm,height=1em,bordercolor=ctmmBlue,backgroundcolor=ctmmBlue!5]{}

\textbf{I - Intense exercise (Möglichkeiten):} \\
\CheckBox[name=tipp_pushups,width=1em,height=1em]{} Liegestütze \\
\CheckBox[name=tipp_jumping_jacks,width=1em,height=1em]{} Hampelmänner \\
\CheckBox[name=tipp_stairs,width=1em,height=1em]{} Treppensteigen \\
\textbf{Andere:} \TextField[name=tipp_exercise_other,width=8cm,height=1em,bordercolor=ctmmGreen,backgroundcolor=ctmmGreen!5]{}

\textbf{P - Paced breathing (Technik):} \\
\CheckBox[name=tipp_478_breathing,width=1em,height=1em]{} 4-7-8 Atmung \\
\CheckBox[name=tipp_box_breathing,width=1em,height=1em]{} Box-Atmung (4-4-4-4) \\
\CheckBox[name=tipp_belly_breathing,width=1em,height=1em]{} Bauchatmung \\

\textbf{P - Progressive relaxation (Bereiche):} \\
\CheckBox[name=tipp_shoulders,width=1em,height=1em]{} Schultern anspannen/entspannen \\
\CheckBox[name=tipp_hands,width=1em,height=1em]{} Hände zur Faust/öffnen \\
\CheckBox[name=tipp_face,width=1em,height=1em]{} Gesichtsmuskeln \\

\end{ctmmOrangeBox}

\subsection{Verbindung zu CTMM-System}

\begin{ctmmPurpleBox}[title=DBT-Skills im CTMM-Kontext]

\textbf{Integration der DBT-Skills in das CTMM-System:}

\begin{itemize}
    \item \textbf{CATCH:} DBT-Achtsamkeit hilft beim frühzeitigen Erkennen von Triggern
    \item \textbf{TRACK:} PLEASE-Skills verbessern die Selbstwahrnehmung
    \item \textbf{MAP:} DEAR MAN strukturiert die Kommunikation über Muster
    \item \textbf{MATCH:} GIVE und TIPP bieten konkrete Handlungsoptionen
\end{itemize}

\textbf{Querverweise:}
\begin{itemize}
    \item Siehe \hyperref[sec:triggermanagement]{Triggermanagement} für CATCH-Integration
    \item Siehe \hyperref[sec:co-regulation]{Co-Regulation} für GIVE-Anwendung
    \item Siehe \hyperref[sec:krisenprotokoll]{Krisenprotokoll} für TIPP-Einsatz
\end{itemize}

\end{ctmmPurpleBox}

\newpage
