% ================================================================
% CTMM Barrierefreiheits-Modul
% Accessibility Features für neurodiverse Nutzer
% ================================================================

\section{Barrierefreiheit und Zugänglichkeit}
\label{sec:accessibility}

\begin{ctmmBlueBox}[title=CTMM Barrierefreiheits-Standards]

Dieses Dokument wurde nach den Prinzipien des \textbf{Universal Design} erstellt, um allen Nutzern, unabhängig von ihren individuellen Bedürfnissen, den bestmöglichen Zugang zu ermöglichen.

\end{ctmmBlueBox}

\subsection{Visuelle Barrierefreiheit}

\begin{ctmmGreenBox}[title=Sehfreundliche Gestaltung]

\textbf{Implementierte Features:}
\begin{itemize}
    \item \textbf{Hoher Kontrast:} Alle Farben erfüllen WCAG 2.1 AA Standards
    \item \textbf{Skalierbare Schriften:} PDF kann bis 400\% vergrößert werden
    \item \textbf{Klare Strukturierung:} Logische Überschriftenhierarchie
    \item \textbf{Farbkodierung + Text:} Keine Information nur durch Farbe
\end{itemize}

\textbf{Farbkontrast-Werte:}
\begin{tabular}{|l|l|l|}
\hline
\textbf{Element} & \textbf{Farbkombination} & \textbf{Kontrastverhältnis} \\
\hline
Standard Text & Schwarz auf Weiß & 21:1 \\
\hline
\textcolor{ctmmBlue}{Überschriften} & ctmmBlue auf Weiß & 8.2:1 \\
\hline
\textcolor{ctmmGreen}{Erfolg} & ctmmGreen auf Weiß & 7.1:1 \\
\hline
\textcolor{ctmmRed}{Warnung} & ctmmRed auf Weiß & 6.8:1 \\
\hline
\end{tabular}

\end{ctmmGreenBox}

\subsection{Kognitive Barrierefreiheit}

\begin{ctmmOrangeBox}[title=Neurodiverse-freundliche Strukturierung]

\textbf{Anpassungen für verschiedene Lernstile:}

\subsubsection{Für Menschen mit Autismus}
\begin{itemize}
    \item \textbf{Vorhersagbare Struktur:} Jedes Modul folgt demselben Aufbau
    \item \textbf{Klare Anweisungen:} Schritt-für-Schritt Erklärungen
    \item \textbf{Visuelle Hilfsmittel:} Icons und Symbole zur Orientierung
    \item \textbf{Reizarme Gestaltung:} Keine überstimulierenden Elemente
\end{itemize}

\subsubsection{Für Menschen mit ADHS}
\begin{itemize}
    \item \textbf{Kurze Abschnitte:} Maximale Textblöcke von 150 Wörtern
    \item \textbf{Hervorhebungen:} Wichtige Punkte visuell betont
    \item \textbf{Interaktive Elemente:} Checkbox und Eingabefelder
    \item \textbf{Fortschrittsanzeigen:} Seitennummern und Kapitelübersicht
\end{itemize}

\subsubsection{Für Menschen mit Dyslexie}
\begin{itemize}
    \item \textbf{Dyslexie-freundliche Schrift:} OpenDyslexic optional verfügbar
    \item \textbf{Erhöhter Zeilenabstand:} 1.5-facher Standard-Abstand
    \item \textbf{Linksbündiger Text:} Keine Blocksatz-Formatierung
    \item \textbf{Kurze Zeilen:} Maximal 70 Zeichen pro Zeile
\end{itemize}

\end{ctmmOrangeBox}

\subsection{Motorische Barrierefreiheit}

\begin{ctmmPurpleBox}[title=Eingabehilfen und Navigation]

\textbf{PDF-Formular Optimierungen:}
\begin{itemize}
    \item \textbf{Große Eingabebereiche:} Mindestens 44pt Touch-Targets
    \item \textbf{Tab-Reihenfolge:} Logische Keyboard-Navigation
    \item \textbf{Fehlertoleranz:} Undo-Funktionen in Formularen
    \item \textbf{Zeitlimits:} Keine automatischen Timeouts
\end{itemize}

\textbf{Alternative Eingabemethoden:}
\begin{itemize}
    \item \textbf{Spracheingabe:} Kompatibel mit Screen Readern
    \item \textbf{Touch-Optimierung:} Für Tablet-Nutzung geeignet
    \item \textbf{Tastatur-Navigation:} Vollständig ohne Maus bedienbar
\end{itemize}

\end{ctmmPurpleBox}

\subsection{Screen Reader Kompatibilität}

\begin{ctmmBlueBox}[title=Assistive Technologie Support]

\textbf{PDF-Accessibility Features:}
\begin{itemize}
    \item \textbf{Alt-Text:} Alle Grafiken mit Beschreibung
    \item \textbf{Heading Tags:} Strukturierte H1-H6 Hierarchie
    \item \textbf{Reading Order:} Logische Lesereihenfolge definiert
    \item \textbf{Language Tags:} Sprache für Text-to-Speech optimiert
\end{itemize}

\textbf{Getestete Screen Reader:}
\begin{itemize}
    \item \textbf{NVDA:} Vollständig kompatibel
    \item \textbf{JAWS:} Formularfelder funktional
    \item \textbf{VoiceOver:} MacOS/iOS Unterstützung
    \item \textbf{TalkBack:} Android Zugänglichkeit
\end{itemize}

\end{ctmmBlueBox}

\subsection{Anpassbare Darstellungsoptionen}

\begin{ctmmGreenBox}[title=Personalisierbare Einstellungen]

\textbf{PDF-Viewer Einstellungen:}

\subsubsection{Schriftgrößen-Anpassung}
\begin{itemize}
    \item \textbf{Standard:} 11pt Grundschrift
    \item \textbf{Groß:} 14pt für bessere Lesbarkeit
    \item \textbf{Sehr groß:} 18pt für Sehbeeinträchtigungen
    \item \textbf{Zoom:} Bis 400\% ohne Qualitätsverlust
\end{itemize}

\subsubsection{Farbschema-Optionen}
\begin{itemize}
    \item \textbf{Standard:} CTMM Farbpalette
    \item \textbf{High Contrast:} Schwarz-Weiß Darstellung
    \item \textbf{Dark Mode:} Dunkler Hintergrund verfügbar
    \item \textbf{Farbenblind-freundlich:} Alternative Markierungen
\end{itemize}

\end{ctmmGreenBox}

\subsection{Sprachliche Barrierefreiheit}

\begin{ctmmOrangeBox}[title=Verständliche Kommunikation]

\textbf{Plain Language Prinzipien:}
\begin{itemize}
    \item \textbf{Einfache Sprache:} Verzicht auf Fachtermini ohne Erklärung
    \item \textbf{Kurze Sätze:} Durchschnittlich 15-20 Wörter
    \item \textbf{Aktive Formulierungen:} Klare Handlungsanweisungen
    \item \textbf{Glossar:} Fachbegriffe erklärt
\end{itemize}

\textbf{Mehrsprachige Unterstützung:}
\begin{itemize}
    \item \textbf{Deutsch:} Vollständige Version
    \item \textbf{Einfache Sprache:} Reduzierte Komplexität
    \item \textbf{Piktogramme:} Universelle Symbole
    \item \textbf{Audio-Version:} Geplant für zukünftige Releases
\end{itemize}

\end{ctmmOrangeBox}

\subsection{Technische Kompatibilität}

\begin{ctmmPurpleBox}[title=Geräte- und Software-Unterstützung]

\textbf{Unterstützte Plattformen:}
\begin{tabular}{|l|l|l|}
\hline
\textbf{Platform} & \textbf{PDF Reader} & \textbf{Formular-Support} \\
\hline
Windows & Adobe Reader, Foxit & Vollständig \\
\hline
macOS & Preview, Adobe Reader & Vollständig \\
\hline
iOS & PDF Expert, Adobe & Teilweise \\
\hline
Android & Adobe Reader & Vollständig \\
\hline
Linux & Evince, Okular & Basis \\
\hline
\end{tabular}

\textbf{Mindestanforderungen:}
\begin{itemize}
    \item \textbf{PDF-Version:} 1.7 oder höher
    \item \textbf{JavaScript:} Für interaktive Features
    \item \textbf{Formular-Unterstützung:} AcroForms kompatibel
    \item \textbf{Unicode:} UTF-8 Zeichenkodierung
\end{itemize}

\end{ctmmPurpleBox}

\subsection{Nutzerfeedback und Verbesserungen}

\begin{ctmmRedBox}[title=\textcolor{white}{Accessibility Feedback-System}]

\textcolor{white}{\textbf{Verbesserungsvorschläge erwünscht:}}

\begin{itemize}
    \item[\textcolor{white}{•}] \textcolor{white}{Welche Barrieren sind Ihnen aufgefallen?}
    \item[\textcolor{white}{•}] \textcolor{white}{Welche assistiven Technologien nutzen Sie?}
    \item[\textcolor{white}{•}] \textcolor{white}{Welche Anpassungen wären hilfreich?}
\end{itemize}

\textcolor{white}{\textbf{Kontakt für Accessibility-Feedback:}}\\
\textcolor{white}{E-Mail: accessibility@ctmm-system.org}

\end{ctmmRedBox}

\subsection{Barrierefreiheits-Checkliste für Therapeuten}

\begin{ctmmGreenBox}[title=Implementierungs-Leitfaden]

\textbf{Vor der Nutzung mit Klienten prüfen:}

\CheckBox[name=accessibility_screen_reader,width=1em,height=1em]{} Screen Reader Kompatibilität getestet \\
\CheckBox[name=accessibility_font_size,width=1em,height=1em]{} Schriftgröße für Klient angepasst \\
\CheckBox[name=accessibility_color_contrast,width=1em,height=1em]{} Farbkontrast ausreichend \\
\CheckBox[name=accessibility_navigation,width=1em,height=1em]{} Navigation erprobt \\
\CheckBox[name=accessibility_input_methods,width=1em,height=1em]{} Alternative Eingabemethoden verfügbar \\
\CheckBox[name=accessibility_language_level,width=1em,height=1em]{} Sprachniveau angemessen \\
\CheckBox[name=accessibility_time_limits,width=1em,height=1em]{} Ausreichend Zeit eingeplant \\
\CheckBox[name=accessibility_backup_formats,width=1em,height=1em]{} Alternative Formate verfügbar \\

\textbf{Individuelle Anpassungen notiert:} \\
\TextField[name=accessibility_individual_notes,width=14cm,height=3em,multiline=true,bordercolor=ctmmGreen,backgroundcolor=ctmmGreen!5]{}

\end{ctmmGreenBox}

\newpage
